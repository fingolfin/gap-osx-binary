% generated by GAPDoc2LaTeX from XML source (Frank Luebeck)
\documentclass[a4paper,11pt]{report}
\usepackage{a4wide}
\sloppy
\pagestyle{myheadings}
\usepackage{amssymb}
\usepackage[latin1]{inputenc}
\usepackage{makeidx}
\makeindex
\usepackage{color}
\definecolor{DarkOlive}{rgb}{0.1047,0.2412,0.0064}
\definecolor{FireBrick}{rgb}{0.5812,0.0074,0.0083}
\definecolor{RoyalBlue}{rgb}{0.0236,0.0894,0.6179}
\definecolor{RoyalGreen}{rgb}{0.0236,0.6179,0.0894}
\definecolor{RoyalRed}{rgb}{0.6179,0.0236,0.0894}
\definecolor{LightBlue}{rgb}{0.8544,0.9511,1.0000}
\definecolor{Black}{rgb}{0.0,0.0,0.0}
\definecolor{FuncColor}{rgb}{1.0,0.0,0.0}
%% strange name because of pdflatex bug:
\definecolor{Chapter }{rgb}{0.0,0.0,1.0}

\usepackage{fancyvrb}

\usepackage{pslatex}

\usepackage[pdftex=true,
        a4paper=true,bookmarks=false,pdftitle={Written with GAPDoc},
        pdfcreator={LaTeX with hyperref package / GAPDoc},
        colorlinks=true,backref=page,breaklinks=true,linkcolor=RoyalBlue,
        citecolor=RoyalGreen,filecolor=RoyalRed,
        urlcolor=RoyalRed,pagecolor=RoyalBlue]{hyperref}

% write page numbers to a .pnr log file for online help
\newwrite\pagenrlog
\immediate\openout\pagenrlog =\jobname.pnr
\immediate\write\pagenrlog{PAGENRS := [}
\newcommand{\logpage}[1]{\protect\write\pagenrlog{#1, \thepage,}}
%% were never documented, give conflicts with some additional packages


\newcommand{\GAP}{\textsf{GAP}}

\begin{document}

\logpage{[ 0, 0, 0 ]}
\begin{titlepage}
\begin{center}{\Huge \textbf{\textsf{LieAlgDB}\mbox{}}}\\[1cm]
\hypersetup{pdftitle=\textsf{LieAlgDB}}
\markright{\scriptsize \mbox{}\hfill \textsf{LieAlgDB} \hfill\mbox{}}
{\Large \textbf{A database of Lie algebras\mbox{}}}\\[1cm]
{Version 2.1\mbox{}}\\[1cm]
{November 2010\mbox{}}\\[1cm]
\mbox{}\\[2cm]
{\large \textbf{ Serena Cical{\a`o}   \mbox{}}}\\
{\large \textbf{ Willem de Graaf    \mbox{}}}\\
{\large \textbf{ Csaba Schneider    \mbox{}}}\\
\hypersetup{pdfauthor= Serena Cical{\a`o}   ;  Willem de Graaf    ;  Csaba Schneider    }
\end{center}\vfill

\mbox{}\\
{\mbox{}\\
\small \noindent \textbf{ Serena Cical{\a`o}   } --- Email: \href{mailto://cicalo@science.unitn.it} {\texttt{cicalo@science.unitn.it}}\\
 --- Address: \begin{minipage}[t]{8cm}\noindent
 Dipartimento di Matematica e Informatica\\
 Via Ospedale, 72\\
 09124 Cagliari\\
 Italy\\
 \end{minipage}
}\\
{\mbox{}\\
\small \noindent \textbf{ Willem de Graaf    } --- Email: \href{mailto://degraaf@science.unitn.it} {\texttt{degraaf@science.unitn.it}}\\
 --- Homepage: \href{http://www.science.unitn.it/~degraaf/} {\texttt{http://www.science.unitn.it/\texttt{\symbol{126}}degraaf/}}\\
 --- Address: \begin{minipage}[t]{8cm}\noindent
 Dipartimento di Matematica\\
 Via Sommarive, 14\\
 38050 Povo (Trento)\\
 Italy\\
 \end{minipage}
}\\
{\mbox{}\\
\small \noindent \textbf{ Csaba Schneider    } --- Email: \href{mailto://csaba.schneider@sztaki.hu} {\texttt{csaba.schneider@sztaki.hu}}\\
 --- Homepage: \href{http://www.sztaki.hu/~schneider} {\texttt{http://www.sztaki.hu/\texttt{\symbol{126}}schneider}}\\
 --- Address: \begin{minipage}[t]{8cm}\noindent
 Informatics Research Laboratory\\
 Computer and Automation Research Institute\\
 1518 Budapest Pf. 63.\\
 Hungary \end{minipage}
}\\
\end{titlepage}

\newpage\setcounter{page}{2}
{\small 
\section*{Abstract}
\logpage{[ 0, 0, 1 ]}
 This package provides access to the classification of the following families
of Lie algebras: 
\begin{itemize}
\item  non-solvable Lie algebras over finite fields up to dimension 6; 
\item  nilpotent Lie algebras of dimension up to 9 over \mbox{\texttt{\slshape GF(2)}}, of dimension up to 7 over \mbox{\texttt{\slshape GF(3)}} or \mbox{\texttt{\slshape GF(5)}};
\item  simple Lie algebras of dimensions between 7 and 9 over \mbox{\texttt{\slshape GF(2)}}; 
\item  the classification of solvable Lie algebras of dimension at most 4; 
\item  the classification of nilpotent Lie algebras of dimensions at most 6. 
\end{itemize}
 \mbox{}}\\[1cm]
{\small 
\section*{Copyright}
\logpage{[ 0, 0, 2 ]}
 {\copyright} 2006--2007 Willem de Graaf and Csaba Schneider \mbox{}}\\[1cm]
{\small 
\section*{Acknowledgements}
\logpage{[ 0, 0, 3 ]}
 We are grateful to Andrea Caranti, Marco Costantini, Bettina Eick, Helmut
Strade, Michael Vaughan-Lee. Without their help, interest, and encouragement,
this package would not have been made. We also acknowledge the effort of the
referees to improve the implementation and the documentation.

 Serena Cical{\a`o} would like to thank the Centro de {\a'A}lgebra da
Universidade de Lisboa for their kind hospitality during July - December 2009
and May - September 2010, when the classification of the 6-dimensional
nilpotent Lie algebras over fields of characteristic 2 was made and added to
the package. \mbox{}}\\[1cm]
\newpage

\def\contentsname{Contents\logpage{[ 0, 0, 4 ]}}

\tableofcontents
\newpage

 
\chapter{\textcolor{Chapter }{Introduction}}\logpage{[ 1, 0, 0 ]}
\hyperdef{L}{X7DFB63A97E67C0A1}{}
{
  This is the manual for the \textsf{GAP} package \textsf{LieAlgDB}, for accessing and working with several classifications of Lie algebras. 

 In the mathematical literature many classifications of Lie algebras of various
types have been published (we refer to the bibliography for a few examples).
However, working with these classifications from paper is not always easy.
This package aims at making a few classifications of small-dimensional Lie
algebras that have appeared in recent years more accessible. For each
classification that is contained in the package, functions are provided that
construct Lie algebras from that classification inside \textsf{GAP}. This allows the user to obtain easy access to the often rather complicated
data contained in a classification, and to directly interface the Lie algebras
to the functionality for Lie algebras which is already contained in the
system. 

 The package contains the following classifications: 
\begin{itemize}
\item  non-solvable Lie algebras over finite fields up to dimension 6 (from \cite{Strade}); 
\item  nilpotent Lie algebras of dimension up to 9 over \mbox{\texttt{\slshape GF(2)}}, of dimension up to 7 over \mbox{\texttt{\slshape GF(3)}} or \mbox{\texttt{\slshape GF(5)}} (from \cite{sch});
\item  simple Lie algebras of dimensions between 7 and 9 over \mbox{\texttt{\slshape GF(2)}} (from \cite{VL}); 
\item  the classification of solvable Lie algebras of dimension at most 4 (from \cite{wdg05}); 
\item  the classification of nilpotent Lie algebras of dimension at most 6 (from \cite{wdg07} and \cite{cdgs10}). 
\end{itemize}
 This manual is structured as follows. The next chapter contains a description
of the main functions of the package. The third chapter contains descriptions
of the classifications used in the package. Most of these are contained in
published papers, but for the convenience of the user they have been added
here. }

 
\chapter{\textcolor{Chapter }{The families of Lie algebras included in the database}}\label{Intro}
\logpage{[ 2, 0, 0 ]}
\hyperdef{L}{X7DD822827FAF0B46}{}
{
  Here we describe the functions that access the classifications of Lie algebras
that are stored in the package. A function below either returns a single Lie
algebra, depending on a list of parameters, or a collection. It is important
to note that two calls of the function \mbox{\texttt{\slshape NonSolvableLieAlgebra}}, \mbox{\texttt{\slshape SolvableLieAlgebra}}, or \mbox{\texttt{\slshape NilpotentLieAlgebra}} may return isomorphic Lie algebras even if the parameters are different (see
the description of the parameter list for each of the functions). If, however,
the output of a function is a collection, then the members of this collection
are pairwise non-isomorphic.

 The Lie algebras in the database are stored in the form of structure constant
tables. Usually the size of a family of Lie algebras in the database is small
enough so that the entries of the structure constant tables can be stored
without any compression. However the number of nilpotent Lie algebras with
dimension at least 7 is very large, and so the structure constant tables are
compressed as follows. If $L$ is such a Lie algebra, then we fix a basis $B=\{b_1,\ldots,b_n\}$ and consider the coefficients of the products $[b_i,b_j]$ where $j>i$. We concatenate these coefficient sequences and consider the long sequence so
obtained as a number written in base $p$. Then we convert this number to base 62 and write it down using the digits $0,\ldots,9,a\ldots,z,A\ldots,Z$. Then this string is stored in the files \mbox{\texttt{\slshape gap/nilpotent/nilpotent{\textunderscore}data*.gi}}. See the function \mbox{\texttt{\slshape ReadStringToNilpotentLieAlgebra}} in the file \mbox{\texttt{\slshape gap/nilpotent/nilpotent.gi}} for the precise details. 
\section{\textcolor{Chapter }{Non-solvable Lie algebras}}\label{nonsolv}
\logpage{[ 2, 1, 0 ]}
\hyperdef{L}{X79B7688E808E0460}{}
{
  The package contains the list of non-solvable Lie algebras over finite fields
up to dimension 6. The classification follows the one in \cite{Strade}. 

\subsection{\textcolor{Chapter }{NonSolvableLieAlgebra}}
\logpage{[ 2, 1, 1 ]}\nobreak
\hyperdef{L}{X8527F9157C850557}{}
{\noindent\textcolor{FuncColor}{$\Diamond$\ \texttt{NonSolvableLieAlgebra({\slshape F, pars})\index{NonSolvableLieAlgebra@\texttt{NonSolvableLieAlgebra}}
\label{NonSolvableLieAlgebra}
}\hfill{\scriptsize (method)}}\\


 \mbox{\texttt{\slshape F}} is an arbitrary finite field, \mbox{\texttt{\slshape pars}} is a list of parameters with length between 1 and 4. The output is a
non-solvable Lie algebra corresponding to the parameters, which is displayed
as a string that describes the algebra following \cite{Strade}. The first entry of \mbox{\texttt{\slshape pars}} is the dimension of the algebra, and the possible additional entries of \mbox{\texttt{\slshape pars}} describe the algebra if there are more algebras with dimension \mbox{\texttt{\slshape pars[1]}}. }

 The possible values of \mbox{\texttt{\slshape pars}} are as follows. 
\subsection{\textcolor{Chapter }{Dimension 1 and 2}}\logpage{[ 2, 1, 2 ]}
\hyperdef{L}{X82F0F10285A94949}{}
{
  There are no non-solvable Lie algebras with dimension less than 3, and so if \mbox{\texttt{\slshape pars[1]}} is less than 3 then \mbox{\texttt{\slshape NonSolvableLieAlgebra}} returns an error message. }

 
\subsection{\textcolor{Chapter }{Dimension 3}}\logpage{[ 2, 1, 3 ]}
\hyperdef{L}{X78C35F937D99AB14}{}
{
  There is just one non-solvable Lie algebra over an arbitrary finite field \mbox{\texttt{\slshape F}} (see Section \ref{appdim3}) which is returned by \mbox{\texttt{\slshape NonSolvableLieAlgebra( F, [3] )}}. }

 
\subsection{\textcolor{Chapter }{Dimension 4}}\logpage{[ 2, 1, 4 ]}
\hyperdef{L}{X812516C97F0A1A4C}{}
{
  If \mbox{\texttt{\slshape F}} has odd characteristic then there is a unique non-solvable Lie algebra with
dimension 4 over \mbox{\texttt{\slshape F}} and this algebra is returned by \mbox{\texttt{\slshape NonSolvableLieAlgebra( F, [4] )}}. If \mbox{\texttt{\slshape F}} has characteristic 2, then there are two distinct Lie algebras and they are
returned by \mbox{\texttt{\slshape NonSolvableLieAlgebra( F, [4,i] )}} for \mbox{\texttt{\slshape i=1, 2}}. See Section \ref{appdim4} for a description of the algebras. }

 
\subsection{\textcolor{Chapter }{Dimension 5}}\logpage{[ 2, 1, 5 ]}
\hyperdef{L}{X865565C08312F7B0}{}
{
  If \mbox{\texttt{\slshape F}} has characteristic 2 then there are 5 isomorphism classes of non-solvable Lie
algebras over \mbox{\texttt{\slshape F}} and they are described in Section \ref{appdim5char2}. The possible values of \mbox{\texttt{\slshape pars}} are as follows. 
\begin{itemize}
\item  \mbox{\texttt{\slshape [5,1]}}: the Lie algebra in \ref{appdim5char2}(1).
\item  \mbox{\texttt{\slshape [5,2,i]}}: \mbox{\texttt{\slshape i=0, 1}}; the Lie algebras in \ref{appdim5char2}(2).
\item  \mbox{\texttt{\slshape [5,3,i]}}: \mbox{\texttt{\slshape i=0, 1}}; the Lie algebras in \ref{appdim5char2}(3).
\end{itemize}
 If the characteristic of \mbox{\texttt{\slshape F}} is odd, then the list of Lie algebras is as follows (see Section \ref{appdim5charodd}). 
\begin{itemize}
\item  \mbox{\texttt{\slshape [5,1,i]}}: \mbox{\texttt{\slshape i=1, 0}}; the Lie algebras that occur in \ref{appdim5charodd}(1). 
\item  \mbox{\texttt{\slshape [5,2]}}: the Lie algebra in \ref{appdim5charodd}(2). 
\item  \mbox{\texttt{\slshape [5,3]}}: this algebra only exists if the characteristic of \mbox{\texttt{\slshape F}} is 3 or 5. In the former case the algebra is the one in \ref{appdim5charodd}(3), while in the latter it is in \ref{appdim5charodd}(4). 
\end{itemize}
 }

 
\subsection{\textcolor{Chapter }{Dimension 6}}\logpage{[ 2, 1, 6 ]}
\hyperdef{L}{X7FC5F0DB7A8CD1D0}{}
{
  The 6-dimensional non-solvable Lie algebras are described in Section \ref{appdim6}. If \mbox{\texttt{\slshape F}} has characteristic 2, then the possible values of \mbox{\texttt{\slshape pars}} is as follows. 
\begin{itemize}
\item  \mbox{\texttt{\slshape [6,1]}}: the Lie algebra in \ref{appdim6char2}(1). 
\item  \mbox{\texttt{\slshape [6,2]}}: the Lie algebra in \ref{appdim6char2}(2). 
\item  \mbox{\texttt{\slshape [6,3,i]}}: \mbox{\texttt{\slshape i=0, 1}}; the two Lie algebras \ref{appdim6char2}(3). 
\item  \mbox{\texttt{\slshape [6,4,x]}}: \mbox{\texttt{\slshape x=0, 1, 2, 3}} or \mbox{\texttt{\slshape x}} is a field element. In this case \mbox{\texttt{\slshape AllNonSolvableLieAlgebras}} returns one of the Lie algebras in \ref{appdim6char2}(4). If \mbox{\texttt{\slshape x=0, 1, 2, 3}} then the Lie algebra corresponding to the \mbox{\texttt{\slshape (x+1)}}-th matrix of \ref{appdim6char2}(4) is returned. If \mbox{\texttt{\slshape x}} is a field element, then a Lie algebra is returned which corresponds to the
5th matrix in \ref{appdim6char2}(4).
\item  \mbox{\texttt{\slshape [6,5]}}: the Lie algebra in \ref{appdim6char2}(5). 
\item \mbox{\texttt{\slshape [6,6,1], [6,6,2], [6,6,3,x], [6,6,4,x]}}: \mbox{\texttt{\slshape x}} is a field element; the Lie algebras in \ref{appdim6char2}(6). The third and fourth entries of \mbox{\texttt{\slshape pars}} determine the isomorphism type of the radical as a solvable Lie algebra. More
precisely, if the third argument \mbox{\texttt{\slshape pars[3]}} is 1 or 2 then the radical is isomorphic to \mbox{\texttt{\slshape SolvableLieAlgebra( F,[3,pars[3]] )}}. If the third argument \mbox{\texttt{\slshape pars[3]}} is 3 or 4 then the radical is isomorphic to \mbox{\texttt{\slshape SolvableLieAlgebra( F,[3,pars[3],pars[4]] )}}; see \texttt{SolvableLieAlgebra} (\ref{SolvableLieAlgebra}). 
\item  \mbox{\texttt{\slshape [6,7]}}: the Lie algebra in \ref{appdim6char2}(7). 
\item  \mbox{\texttt{\slshape [6,8]}}: the Lie algebra in \ref{appdim6char2}(8).
\end{itemize}
 If the characteristic of \mbox{\texttt{\slshape F}} is odd, then the possible values of \mbox{\texttt{\slshape pars}} are the following (see Sections \ref{appdim6charodd}, \ref{appdim6char3}, and \ref{appdim6char5}). 
\begin{itemize}
\item  \mbox{\texttt{\slshape [6,1]}}: the Lie algebra in \ref{appdim6charodd}(1).
\item  \mbox{\texttt{\slshape [6,2]}}: the Lie algebra in \ref{appdim6charodd}(2).
\item \mbox{\texttt{\slshape [6,3,1], [6,3,2], [6,3,3,x], [6,3,4,x]}}: \mbox{\texttt{\slshape x}} is a field element; the Lie algebras in \ref{appdim6charodd}(3). The third and fourth entries of \mbox{\texttt{\slshape pars}} determine the isomorphism type of the radical as a solvable Lie algebra. More
precisely, if the third argument \mbox{\texttt{\slshape pars[3]}} is 1 or 2 then the radical is isomorphic to \mbox{\texttt{\slshape SolvableLieAlgebra( F,[3,pars[3]] )}}. If the third argument \mbox{\texttt{\slshape pars[3]}} is 3 or 4 then the radical is isomorphic to \mbox{\texttt{\slshape SolvableLieAlgebra( F,[3,pars[3],pars[4]] )}}; see \texttt{SolvableLieAlgebra} (\ref{SolvableLieAlgebra}). 
\item \mbox{\texttt{\slshape [6,4]}}: the Lie algebra in \ref{appdim6charodd}(4).
\item \mbox{\texttt{\slshape [6,5]}}: the Lie algebra in \ref{appdim6charodd}(5).
\item \mbox{\texttt{\slshape [6,6]}}: the Lie algebra in \ref{appdim6charodd}(6).
\item \mbox{\texttt{\slshape [6,7]}}: the Lie algebra in \ref{appdim6charodd}(7).
\end{itemize}
 If the characteristic is 3 or 5 then there are additional families. In
characteristic 3, these families are as follows. 
\begin{itemize}
\item  \mbox{\texttt{\slshape [6,8,x]}}: \mbox{\texttt{\slshape x}} is a field element; returns one of the Lie algebras in \ref{appdim6char3}(1).
\item \mbox{\texttt{\slshape [6,9]}}: the Lie algebra in \ref{appdim6char3}(2).
\item \mbox{\texttt{\slshape [6,10]}}: the Lie algebra in \ref{appdim6char3}(3).
\item \mbox{\texttt{\slshape [6,11,i]}}: \mbox{\texttt{\slshape i=0, 1}}; one of the two Lie algebras in \ref{appdim6char3}(4). 
\item \mbox{\texttt{\slshape [6,12]}}: the first Lie algebra in \ref{appdim6char3}(5).
\item \mbox{\texttt{\slshape [6,13]}}: the second Lie algebra \ref{appdim6char3}(5).
\end{itemize}
 If the characteristic is 5, then the additional Lie algebras are the
following. 
\begin{itemize}
\item \mbox{\texttt{\slshape [6,8]}}: the Lie algebra in \ref{appdim6char5}(1).
\item \mbox{\texttt{\slshape [6,9]}}: the Lie algebra in \ref{appdim6char5}(2).
\end{itemize}
 }

 

\subsection{\textcolor{Chapter }{AllNonSolvableLieAlgebras}}
\logpage{[ 2, 1, 7 ]}\nobreak
\hyperdef{L}{X85CFDCF687153158}{}
{\noindent\textcolor{FuncColor}{$\Diamond$\ \texttt{AllNonSolvableLieAlgebras({\slshape F, dim})\index{AllNonSolvableLieAlgebras@\texttt{AllNonSolvableLieAlgebras}}
\label{AllNonSolvableLieAlgebras}
}\hfill{\scriptsize (method)}}\\


 Here \mbox{\texttt{\slshape F}} is an arbitrary finite field, and \mbox{\texttt{\slshape dim}} is at most 6. A collection is returned whose members form a complete and
irredundant list of representatives of the isomorphism types of the
non-solvable Lie algebras over \mbox{\texttt{\slshape F}} with dimension \mbox{\texttt{\slshape dim}}. In order to obtain the algebras contained in the collection, one can use the
functions \mbox{\texttt{\slshape AsList}}, \mbox{\texttt{\slshape Enumerator}}, \mbox{\texttt{\slshape Iterator}}, as illustrated by the following example. 
\begin{Verbatim}[fontsize=\small,frame=single,label=Example]
  gap> L := AllNonSolvableLieAlgebras( GF(4), 4 );
  <Collection of nonsolvable Lie algebras with dimension 4 over GF(2^2)>
  gap>  e := Enumerator( L );
  <enumerator>
  gap> for i in e do Print( Dimension( LieSolvableRadical( i )), "\n" ); od;
  0
  1
  gap> AsList( L );
  [ W(1;2), W(1;2)^{(1)}+GF(4) ]
  gap> Dimension( LieCenter( last[2] ));
  1
\end{Verbatim}
 As the output of \mbox{\texttt{\slshape AllNonSolvableLieAlgebras}} is a collection, the user can efficiently access the classification of $d$-dimensional non-solvable Lie algebras over a given field, even if the
classification contains a large number of algebras. For instance, there are
95367431640638 non-solvable Lie algebras over $GF(5^{20})$. Clearly one cannot expect to be able to handle a list containing all these
algebras; it is, however, possible to work with the collection of these Lie
algebras, as follows. 
\begin{Verbatim}[fontsize=\small,frame=single,label=Example]
  gap> L := AllNonSolvableLieAlgebras( GF(5^20), 6 );
  <Collection of nonsolvable Lie algebras with dimension 6 over GF(5^20)>
  gap> e := Enumerator( L );
  <enumerator>
  gap> Length( last );
  95367431640638
  gap> Dimension( LieDerivedSubalgebra( e[462468528345] ));
  5
\end{Verbatim}
 We note that we could not enumerate the non-solvable Lie algebras of dimension
6 over finite fields of characteristic 3, and so the function \mbox{\texttt{\slshape Enumerator}} cannot be used in that context. You can, however, use the functions \mbox{\texttt{\slshape Iterator}} and \mbox{\texttt{\slshape AsList}} as follows. 
\begin{Verbatim}[fontsize=\small,frame=single,label=Example]
  gap> L := AllNonSolvableLieAlgebras( GF(3), 6 );
  <Collection of nonsolvable Lie algebras with dimension 6 over GF(3)>
  gap>  e := Iterator( L );
  <iterator>
  gap> dims := [];;
  gap> for i in e do Add( dims, Dimension( LieSolvableRadical( i ))); od;
  gap> dims;
  [ 0, 0, 3, 3, 3, 3, 3, 3, 3, 3, 3, 3, 3, 3, 3, 3, 3, 3, 3, 3, 3, 3 ]
  gap> AsList( L );
  [ sl(2,3)+sl(2,3), sl(2,GF(9)), sl(2,3)+solv([ 1 ]), sl(2,3)+solv([ 2 ]), 
    sl(2,3)+solv([ 3, 0*Z(3) ]), sl(2,3)+solv([ 3, Z(3)^0 ]), 
    sl(2,3)+solv([ 3, Z(3) ]), sl(2,3)+solv([ 4, 0*Z(3) ]), 
    sl(2,3)+solv([ 4, Z(3) ]), sl(2,3)+solv([ 4, Z(3)^0 ]), sl(2,3):(V(1)+V(0)),
    sl(2,3):V(2), sl(2,3):H, sl(2,3):<x,y,z|[x,y]=y,[x,z]=z>, 
    sl(2,3):V(2,0*Z(3)), sl(2,3):V(2,Z(3)), W(1;1):O(1;1), W(1;1):O(1;1)*, 
    sl(2,3).H(0), sl(2,3).H(1), sl(2,3).(GF(3)+GF(3)+GF(3))(1), 
    sl(2,3).(GF(3)+GF(3)+GF(3))(2) ]
\end{Verbatim}
 }

 

\subsection{\textcolor{Chapter }{AllSimpleLieAlgebras}}
\logpage{[ 2, 1, 8 ]}\nobreak
\hyperdef{L}{X7B54426D7B43CEF8}{}
{\noindent\textcolor{FuncColor}{$\Diamond$\ \texttt{AllSimpleLieAlgebras({\slshape F, dim})\index{AllSimpleLieAlgebras@\texttt{AllSimpleLieAlgebras}}
\label{AllSimpleLieAlgebras}
}\hfill{\scriptsize (method)}}\\


 Here \mbox{\texttt{\slshape F}} is a finite field, and \mbox{\texttt{\slshape dim}} is either an integer not greater than 6, or, if \mbox{\texttt{\slshape F=GF(2)}}, then \mbox{\texttt{\slshape dim}} is not greater than 9. The output is a list of simple Lie algebras over \mbox{\texttt{\slshape F}} of dimension \mbox{\texttt{\slshape dim}}. If \mbox{\texttt{\slshape dim}} is at most 6, then the classification by Strade \cite{Strade} is used. If \mbox{\texttt{\slshape F=GF(2)}} and \mbox{\texttt{\slshape dim}} is between 7 and 9, then the Lie algebras in \cite{VL} are returned (see Section \ref{simple}). The algebras in the list are pairwise non-isomorphic. Note that the output
of this function is a list and not a collection, and the package does not
contain a function called \mbox{\texttt{\slshape SimpleLieAlgebra}}. }

 }

 
\section{\textcolor{Chapter }{Solvable and nilpotent Lie algebras}}\label{solvnilp}
\logpage{[ 2, 2, 0 ]}
\hyperdef{L}{X8697F2028731A29F}{}
{
  The package contains the classification of solvable Lie algebras of dimensions
2, 3 and 4 (taken from \cite{wdg05}), and the classification of nilpotent Lie algebras of dimensions 5 and 6
(from \cite{cdgs10}). The classifications are complemented by a function for identifying a given
Lie algebra as a member of the list. This function also returns an explicit
isomorphism. In Section \ref{listdescr} the list is given of the multiplication tables of the solvable and nilpotent
Lie algebras, corresponding to the functions in this section. 

\subsection{\textcolor{Chapter }{SolvableLieAlgebra}}
\logpage{[ 2, 2, 1 ]}\nobreak
\hyperdef{L}{X81FDF3D17E495C6A}{}
{\noindent\textcolor{FuncColor}{$\Diamond$\ \texttt{SolvableLieAlgebra({\slshape F, pars})\index{SolvableLieAlgebra@\texttt{SolvableLieAlgebra}}
\label{SolvableLieAlgebra}
}\hfill{\scriptsize (method)}}\\


 Here \mbox{\texttt{\slshape F}} is an arbitrary field, \mbox{\texttt{\slshape pars}} is a list of parameters with length between \texttt{2} and \texttt{4}. The first entry of \mbox{\texttt{\slshape pars}} is the dimension of the algebra, which has to be 2, 3, or 4. If the dimension
is 3, or 4, then the second entry of \mbox{\texttt{\slshape pars}} is the number of the Lie algebra with which it appears in the list of \cite{wdg05}. If the dimension is 2, then there are only two (isomorphism classes of)
solvable Lie algebras. In this case, if the second entry is 1, then the
abelian Lie algebra is returned, if it is 2, then the unique non-abelian
solvable Lie algebra of dimension 2 is returned. A Lie algebra in the list of \cite{wdg05} can have one or two parameters. In that case the list \mbox{\texttt{\slshape pars}} also has to contain the parameters. 
\begin{Verbatim}[fontsize=\small,frame=single,label=Example]
  gap> SolvableLieAlgebra( Rationals, [4,6,1,2] );
  <Lie algebra of dimension 4 over Rationals>
\end{Verbatim}
 }

 

\subsection{\textcolor{Chapter }{NilpotentLieAlgebra}}
\logpage{[ 2, 2, 2 ]}\nobreak
\hyperdef{L}{X7DA312A97A6242B4}{}
{\noindent\textcolor{FuncColor}{$\Diamond$\ \texttt{NilpotentLieAlgebra({\slshape F, pars})\index{NilpotentLieAlgebra@\texttt{NilpotentLieAlgebra}}
\label{NilpotentLieAlgebra}
}\hfill{\scriptsize (method)}}\\


 Here \mbox{\texttt{\slshape F}} is an arbitrary field, \mbox{\texttt{\slshape pars}} is a list of parameters with length between \texttt{2} and \texttt{3}. The first entry of \mbox{\texttt{\slshape pars}} is the dimension of the algebra, which has to be 5 or 6. The second entry of \mbox{\texttt{\slshape pars}} is the number of the Lie algebra with which it appears in the list of Section \ref{listdescr}. A Lie algebra in the list of Section \ref{listdescr} can have one parameter. In that case the list \mbox{\texttt{\slshape pars}} also has to contain the parameter. 
\begin{Verbatim}[fontsize=\small,frame=single,label=Example]
  gap> NilpotentLieAlgebra( GF(3^7), [ 6, 24, Z(3^7)^101 ] );
  <Lie algebra of dimension 6 over GF(3^7)>
\end{Verbatim}
 }

 

\subsection{\textcolor{Chapter }{AllSolvableLieAlgebras}}
\logpage{[ 2, 2, 3 ]}\nobreak
\hyperdef{L}{X8217183E7C19FA16}{}
{\noindent\textcolor{FuncColor}{$\Diamond$\ \texttt{AllSolvableLieAlgebras({\slshape F, dim})\index{AllSolvableLieAlgebras@\texttt{AllSolvableLieAlgebras}}
\label{AllSolvableLieAlgebras}
}\hfill{\scriptsize (method)}}\\


 Here \mbox{\texttt{\slshape F}} is an arbitrary finite field, and \mbox{\texttt{\slshape dim}} is at most 4. A collection of all solvable Lie algebras over \mbox{\texttt{\slshape F}} of dimension \mbox{\texttt{\slshape dim}} is returned. The output does not contain isomorphic Lie algebras. The order in
which the Lie algebras appear in the list is always the same. It is possible
to construct an enumerator from the output collection for all of the valid
choices of the parameters. See \mbox{\texttt{\slshape AllNonSolvableLieAlgebra}} for a more detailed description of usage. }

 

\subsection{\textcolor{Chapter }{AllNilpotentLieAlgebras}}
\logpage{[ 2, 2, 4 ]}\nobreak
\hyperdef{L}{X86629DF87C2AEB4E}{}
{\noindent\textcolor{FuncColor}{$\Diamond$\ \texttt{AllNilpotentLieAlgebras({\slshape F, dim})\index{AllNilpotentLieAlgebras@\texttt{AllNilpotentLieAlgebras}}
\label{AllNilpotentLieAlgebras}
}\hfill{\scriptsize (method)}}\\


 Here \mbox{\texttt{\slshape F}} is a finite field, and \mbox{\texttt{\slshape dim}} not greater than 9. Further, if \mbox{\texttt{\slshape dim=9}} or \mbox{\texttt{\slshape dim=8}}, then \mbox{\texttt{\slshape F}} must be \mbox{\texttt{\slshape GF(2)}}; if \mbox{\texttt{\slshape dim=7}} then \mbox{\texttt{\slshape F}} must be one of \mbox{\texttt{\slshape GF(2)}}, \mbox{\texttt{\slshape GF(3)}}, \mbox{\texttt{\slshape GF(5)}} and if \mbox{\texttt{\slshape dim{\ensuremath{\leq}}6}} then \mbox{\texttt{\slshape F}} can be an arbitrary finite field. A collection of all nilpotent Lie algebras
over \mbox{\texttt{\slshape F}} of dimension \mbox{\texttt{\slshape dim}} is returned. If \mbox{\texttt{\slshape dim}} is not greater than 6 then the collection of nilpotent Lie algebras is
determined by \cite{cdgs10}, otherwise the classification can be found in \cite{sch}. The output does not contain isomorphic Lie algebras. The order in which the
Lie algebras appear in the collection is always the same. It is possible to
construct an enumerator from the output collection for all of the valid
choices of the parameters. See \mbox{\texttt{\slshape AllNonSolvableLieAlgebra}} for a more detailed description of usage. }

 

\subsection{\textcolor{Chapter }{NrNilpotentLieAlgebras}}
\logpage{[ 2, 2, 5 ]}\nobreak
\hyperdef{L}{X8770350D84304182}{}
{\noindent\textcolor{FuncColor}{$\Diamond$\ \texttt{NrNilpotentLieAlgebras({\slshape F, dim})\index{NrNilpotentLieAlgebras@\texttt{NrNilpotentLieAlgebras}}
\label{NrNilpotentLieAlgebras}
}\hfill{\scriptsize (method)}}\\


 Here \mbox{\texttt{\slshape F}} is a finite field, and \mbox{\texttt{\slshape dim}} is an integer. The restrictions for \mbox{\texttt{\slshape F}} and \mbox{\texttt{\slshape dim}} are the same as in the function \mbox{\texttt{\slshape AllNilpotentLieAlgebras}}. The number of nilpotent Lie algebras over \mbox{\texttt{\slshape F}} of dimension \mbox{\texttt{\slshape dim}} is returned. }

 

\subsection{\textcolor{Chapter }{LieAlgebraIdentification}}
\logpage{[ 2, 2, 6 ]}\nobreak
\hyperdef{L}{X8068FF417F93FCD2}{}
{\noindent\textcolor{FuncColor}{$\Diamond$\ \texttt{LieAlgebraIdentification({\slshape L})\index{LieAlgebraIdentification@\texttt{LieAlgebraIdentification}}
\label{LieAlgebraIdentification}
}\hfill{\scriptsize (method)}}\\


 Here \mbox{\texttt{\slshape L}} is a solvable Lie algebra of dimension 2,3, or 4, or it is a nilpotent Lie
algebra of dimension 5 or 6. This function returns a record with three fields. 
\begin{itemize}
\item \mbox{\texttt{\slshape name}} This is a string containing the name of the Lie algebra. It starts with a
capital L if it is a solvable Lie algebra of dimension 2,3,4. It starts with a
capital N if it is a nilpotent Lie algebra of dimension 5 or 6. A name like 
\begin{verbatim}  "N6_24( GF(3^7), Z(3^7) )"
\end{verbatim}
 means that the input Lie algebra is isomorphic to the Lie algebra with number
24 in the list of 6-dimensional nilpotent Lie algebras. Furthermore the field
is given and the value of the parameters (if there are any). 
\item \mbox{\texttt{\slshape parameters}} This contains the parameters that appear in the name of the algebra.
\item \mbox{\texttt{\slshape isomorphism}} This is an isomorphism of the input Lie algebra to the Lie algebra from the
classification with the given name.
\end{itemize}
 
\begin{Verbatim}[fontsize=\small,frame=single,label=Example]
  gap> L:= SolvableLieAlgebra( Rationals, [4,14,3] );
  <Lie algebra of dimension 4 over Rationals>
  gap>  LieAlgebraIdentification( L );
  rec( name := "L4_14( Rationals, 1/3 )", parameters := [ 1/3 ],
    isomorphism := CanonicalBasis( <Lie algebra of dimension
      4 over Rationals> ) -> [ v.3, (-1)*v.2, v.1, (1/3)*v.4 ] )
\end{Verbatim}
 In the example we see that the program finds a different parameter, than the
one with which the Lie algebra was constructed. The explanation is that some
parametric classes of Lie algebras contain isomorphic Lie algebras, for
different values of the parameters. In that case the identification function
makes its own choice. }

 }

 }

 
\chapter{\textcolor{Chapter }{A description of the Lie algebras that are contained in the package}}\logpage{[ 3, 0, 0 ]}
\hyperdef{L}{X7BB72096877E0847}{}
{
 
\section{\textcolor{Chapter }{Description of the non-solvable Lie algebras}}\label{nonsolvdescr}
\logpage{[ 3, 1, 0 ]}
\hyperdef{L}{X7D10CF477EF901A7}{}
{
  In this section we list the non-solvable Lie algebras contained in the
package. Our notation follows \cite{Strade}, where a more detailed description can also be found. In particular if $L$ is a Lie algebra over $F$ then $C(L)$ denotes the center of $L$. Further, if $x_1,\ldots,x_k$ are elements of $L$, then $F<x_1,\ldots,x_k>$ denotes the linear subspace generated by $x_1,\ldots,x_k$, and we also write $Fx_1$ for $F<x_1>$ }

 
\section{\textcolor{Chapter }{Dimension 3}}\label{appdim3}
\logpage{[ 3, 2, 0 ]}
\hyperdef{L}{X78C35F937D99AB14}{}
{
  There are no non-solvable Lie algebras with dimension 1 or 2. Over an
arbitrary finite field \mbox{\texttt{\slshape F}}, there is just one isomorphism type of non-solvable Lie algebras: 
\begin{enumerate}
\item  If \mbox{\texttt{\slshape char F=2}} then the algebra is $W(1;\underline 2)^{(1)}$. 
\item  If \mbox{\texttt{\slshape char F{\textgreater}2}} then the algebra is $\mbox{sl}(2,F)$.
\end{enumerate}
 See Theorem 3.2 of \cite{Strade} for details. }

 
\section{\textcolor{Chapter }{Dimension 4}}\label{appdim4}
\logpage{[ 3, 3, 0 ]}
\hyperdef{L}{X812516C97F0A1A4C}{}
{
  Over a finite field \mbox{\texttt{\slshape F}} of characteristic 2 there are two isomorphism classes of non-solvable Lie
algebras with dimension 4, while over a finite field \mbox{\texttt{\slshape F}} of odd characteristic the number of isomorphism classes is one (see Theorem
4.1 of \cite{Strade}). The classes are as follows: 
\begin{enumerate}
\item  characteristic 2: $W(1;\underline 2)$ and $W(1;\underline 2)^{(1)}\oplus F$. 
\item  odd characteristic: $\mbox{gl}(2,F)$.
\end{enumerate}
 }

 
\section{\textcolor{Chapter }{Dimension 5}}\label{appdim5}
\logpage{[ 3, 4, 0 ]}
\hyperdef{L}{X865565C08312F7B0}{}
{
  
\subsection{\textcolor{Chapter }{Characteristic 2}}\label{appdim5char2}
\logpage{[ 3, 4, 1 ]}
\hyperdef{L}{X783B7FC180919CBC}{}
{
  Over a finite field \mbox{\texttt{\slshape F}} of characteristic 2 there are 5 isomorphism classes of non-solvable Lie
algebras with dimension 5: 
\begin{enumerate}
\item  $\mbox{Der}(W(1;\underline 2)^{(1)})$;
\item  $W(1;\underline 2)\ltimes Fu$ where $[W(1;\underline 2)^{(1)},u]=0$, $[x^{(3)}\partial,u]=\delta u$ and $\delta\in\{0,1\}$ (two algebras);
\item  $W(1;\underline 2)^{(1)}\oplus(F\left< h,u\right>)$, $[h,u]=\delta u$, where $\delta\in\{0,1\}$ (two algebras).
\end{enumerate}
 See Theorem 4.2 of \cite{Strade} for details. }

 
\subsection{\textcolor{Chapter }{Odd characteristic}}\label{appdim5charodd}
\logpage{[ 3, 4, 2 ]}
\hyperdef{L}{X80DAF658844AF393}{}
{
 Over a field $F$of odd characteristic the number of isomorphism types of 5-dimensional
non-solvable Lie algebras is $3$ if the characteristic is at least 7, and it is 4 otherwise (see Theorem 4.3 of \cite{Strade}). The classes are as follows. 
\begin{enumerate}
\item $\mbox{sl}(2,F)\oplus F<x,y>$, $[x,y]=\delta y$ where $\delta\in\{0,1\}$.
\item $\mbox{sl}(2,F)\ltimes V(1)$ where $V(1)$ is the irreducible 2-dimensional $\mbox{sl}(2,F)$-module.
\item If $\mbox{char }F=3$ then there is an additional algebra, namely the non-split extension $0\rightarrow V(1)\rightarrow L\rightarrow\mbox{sl}(2,F)\rightarrow 0$.
\item If $\mbox{char }F=5$ then there is an additional algebra: $W(1;\underline 1)$. 
\end{enumerate}
 }

 }

 
\section{\textcolor{Chapter }{Dimension 6}}\label{appdim6}
\logpage{[ 3, 5, 0 ]}
\hyperdef{L}{X7FC5F0DB7A8CD1D0}{}
{
 
\subsection{\textcolor{Chapter }{Characteristic 2}}\label{appdim6char2}
\logpage{[ 3, 5, 1 ]}
\hyperdef{L}{X783B7FC180919CBC}{}
{
 Over a field $F$ of characteristic 2, the isomorphism classes of non-solvable Lie algebras are
as follows. 
\begin{enumerate}
\item $W(1;\underline 2)^{(1)}\oplus W(1;\underline 2)^{(1)}$.
\item $W(1;\underline 2)^{(1)}\otimes F_{q^2}$ where $F=F_q$.
\item $\mbox{Der}(W(1;\underline 2)^{(1)})\ltimes Fu$, $[W(1;\underline 2),u]=0$, $[\partial^2,u]=\delta u$ where $\delta=\{0,1\}$.
\item $W(1;\underline 2)\ltimes (F<h,u>)$, $[W(1;\underline 2)^{(1)},(F<h,u>]=0$, $[h,u]=\delta u$, and if $\delta=0$, then the action of $x^{(3)}\partial$ on $F<h,u>$ is given by one of the following matrices: 
\[ \left(\begin{array}{cc} 0 & 0\\ 0 & 0\end{array}\right),\
\left(\begin{array}{cc} 0 & 1\\ 0 & 0\end{array}\right),\
\left(\begin{array}{cc} 1 & 0\\ 0 & 1\end{array}\right),\
\left(\begin{array}{cc} 1 & 1\\ 0 & 1\end{array}\right),\
\left(\begin{array}{cc} 0 & \xi\\ 1 & 1\end{array}\right)\mbox{ where }\xi\in
F^*.\]

\item the algebra is as in (4.), but $\delta=1$. Note that Theorem 5.1(3/b) of \cite{Strade} lists two such algebras but they turn out to be isomorphic. We take the one
with $[x^{(3)}\partial,h]=[x^{(3)}\partial,u]=0$. 
\item $W(1;\underline 2)^{(1)}\oplus K$ where $K$ is a 3-dimensional solvable Lie algebra.
\item $W(1;\underline 2)^{(1)}\ltimes \mathcal O(1;\underline 2)/F$.
\item the non-split extension $0\rightarrow \mathcal O(1;\underline 2)/F\rightarrow L\rightarrow
W(1;\underline 2)^{(1)}\rightarrow 0$.
\end{enumerate}
 See Theorem 5.1 of \cite{Strade}. }

 
\subsection{\textcolor{Chapter }{General odd characteristic}}\label{appdim6charodd}
\logpage{[ 3, 5, 2 ]}
\hyperdef{L}{X7CC42FD178D384FD}{}
{
 If the characteristic of the field is odd, then the 6-dimensional non-solvable
Lie algebras are described by Theorems 5.2--5.4 of \cite{Strade}. Over such a field $F$, let us define the following isomorphism classes of 6-dimensional
non-solvable Lie algebras. 
\begin{enumerate}
\item  $\mbox{sl}(2,F)\oplus\mbox{sl}(2,F) $.
\item $\mbox{sl}(2,F_{q^2})$ where $F=F_q$;
\item $\mbox{sl}(2,F)\oplus K$ where $K$ is a solvable Lie algebra with dimension 3;
\item $\mbox{sl}(2,F)\ltimes (V(0)\oplus V(1))$ where $V(i)$ is the $(i+1)$-dimensional irreducible $\mbox{sl}(2,F)$-module;
\item $\mbox{sl}(2,F)\ltimes V(2)$ where $V(2)$ is the $3$-dimensional irreducible $\mbox{sl}(2,F)$-module; 
\item $\mbox{sl}(2,F)\ltimes(V(1)\oplus C(L))\cong \mbox{sl}(2,F)\ltimes H$ where $H$ is the Heisenberg Lie algebra;
\item $\mbox{sl}(2,F)\ltimes K$ where $K=Fd\oplus K^{(1)}$, $K^{(1)}$ is 2-dimensional abelian, isomorphic, as an $\mbox{sl}(2,F)$-module, to $V(1)$, $[\mbox{sl}(2,F),d]=0$, and, for all $v\in K$, $[d,v]=v$;
\end{enumerate}
 If the characteristic of $F$ is at least 7, then these algebras form a complete and irredundant list of the
isomorphism classes of the 6-dimensional non-solvable Lie algebras. }

 
\subsection{\textcolor{Chapter }{Characteristic 3}}\label{appdim6char3}
\logpage{[ 3, 5, 3 ]}
\hyperdef{L}{X7F4B0CC87A7715A5}{}
{
 If the characteristic of the field $F$ is 3, then, besides the classes in Section \ref{appdim6charodd}, we also obtain the following isomorphism classes. 
\begin{enumerate}
\item $\mbox{sl}(2,F)\ltimes V(2,\chi)$ where $\chi$ is a 3-dimensional character of $\mbox{sl}(2,F)$. Each such character is described by a field element $\xi$ such that $T^3+T^2-\xi$ has a root in $F$; see Proposition 3.5 of \cite{Strade} for more details. 
\item $W(1;\underline 1)\ltimes\mathcal O(1;\underline 1)$ where $\mathcal O(1;\underline 1)$ is considered as an abelian Lie algebra. 
\item $W(1;\underline 1)\ltimes\mathcal O(1;\underline 1)^*$ where $\mathcal O(1;\underline 1)^*$ is the dual of $\mathcal O(1;\underline 1)$ and it is considered as an abelian Lie algebra.
\item One of the two 6-dimensional central extensions of the non-split extension $0\rightarrow V(1)\rightarrow L\rightarrow \mbox{sl}(2,F)\rightarrow 0$; see Proposition 4.5 of \cite{Strade}. We note that Proposition 4.5 of \cite{Strade} lists three such central extensions, but one of them is not a Lie algebra.
\item One of the two non-split extensions $0\rightarrow\mbox{rad } L\rightarrow L\rightarrow L/\mbox{rad } L\rightarrow 0$ with a 5-dimensional ideal; see Theorem 5.4 of \cite{Strade}.
\end{enumerate}
 We note here that \cite{Strade} lists one more non-solvable Lie algebra over a field of characteristic 3,
namely the one in Theorem 5.3(5). However, this algebra is isomorphic to the
one in Theorem 5.3(4). }

 
\subsection{\textcolor{Chapter }{Characteristic 5}}\label{appdim6char5}
\logpage{[ 3, 5, 4 ]}
\hyperdef{L}{X81DD369B7F4E033B}{}
{
 If the characteristic of the field $F$ is 5, then, besides the classes in Section \ref{appdim6charodd}, we also obtain the following isomorphism classes. 
\begin{enumerate}
\item $W(1;\underline 1)\oplus F$.
\item The non-split central extension $0\rightarrow F\rightarrow L\rightarrow W(1;\underline 1)\rightarrow 0$.
\end{enumerate}
 }

 }

 
\section{\textcolor{Chapter }{Description of the simple Lie algebras}}\label{simple}
\logpage{[ 3, 6, 0 ]}
\hyperdef{L}{X8411625F7E7DA71D}{}
{
  If \mbox{\texttt{\slshape F}} is a finite field, then, up to isomorphism, there is precisely one simple Lie
algebra with dimension 3, and another one with dimension 6; these can be
accessed by calling \mbox{\texttt{\slshape NonSolvableLieAlgebra(F,[3,1])}} and \mbox{\texttt{\slshape NonSolvableLieAlgebra(F,[6,2])}} (see \mbox{\texttt{\slshape NonSolvableLieAlgebra}} for the details). Over a field of characteristic 5, there is an additional
simple Lie algebra with dimension 5, namely \mbox{\texttt{\slshape NonSolvableLieAlgebra(F,[5,3])}}. These are the only isomorphism types of simple Lie algebras over finite
fields up to dimension 6. 

 In addition to the algebras above the package contains the simple Lie algebras
of dimension between 7 and 9 over \mbox{\texttt{\slshape GF(2)}}. These Lie algebras were determined by \cite{VL} and can be described as follows. 

 There are two isomorphism classes of 7-dimensional Lie algebras over \mbox{\texttt{\slshape GF(2)}}. In a basis $b1,\ldots,b7$ the non-trivial products in the first algebra are 
\begin{verbatim}  
  [b1,b2]=b3, [b1,b3]=b4, [b1,b4]=b5, [b1,b5]=b6
  [b1,b6]=b7, [b1,b7]=b1, [b2,b7]=b2, [b3,b6]=b2, 
  [b4,b5]=b2, [b4,b6]=b3, [b4,b7]=b4, [b6,b7]=b6;
\end{verbatim}
 and those in the second are 
\begin{verbatim}  
  [b1,b2]=b3, [b1,b3]=b1+b4, [b1,b4]=b5, [b1,b5]=b6, 
  [b1,b6]=b7, [b2,b3]=b2, [b2,b5]=b2+b4, [b2,b6]=b5, 
  [b2,b7]=b1+b4, [b3,b4]=b2+b4, [b3,b5]=b3, [b3,b6]=b1+b4+b6, 
  [b3,b7]=b5, [b4,b7]=b6, [b5,b6]=b6, [b5,b7]=b7.
\end{verbatim}
 

Over \mbox{\texttt{\slshape GF(2)}} there are two isomorphism types of simple Lie algebras with dimension 8. In
the basis $b1,\ldots,b8$ the non-trivial products for the first one are 
\begin{verbatim}  
  [b1,b3]=b5, [b1,b4]=b6, [b1,b7]=b2, [b1,b8]=b1, [b2,b3]=b7, [b2,b4]=b5+b8, 
  [b2,b5]=b2, [b2,b6]=b1, [b2,b8]=b2, [b3,b6]=b4, [b3,b8]=b3, [b4,b5]=b4, 
  [b4,b7]=b3, [b4,b8]=b4, [b5,b6]=b6, [b5,b7]=b7, [b6,b7]=b8;
\end{verbatim}
 and for the second one they are 
\begin{verbatim}  
  [b1,b2]=b3, [b1,b3]=b2+b5, [b1,b4]=b6, [b1,b5]=b2, [b1,b6]=b1+b4+b8, 
  [b1,b8]=b4, [b2,b3]=b4, [b2,b4]=b1, [b2,b5]=b6, [b2,b6]=b2+b7, 
  [b2,b7]=b2+b5, [b3,b4]=b2+b7, [b3,b5]=b1+b4+b8, [b3,b6]=b1, [b3,b7]=b2+b3, 
  [b3,b8]=b1, [b4,b5]=b3, [b4,b6]=b2+b4, [b4,b7]=b1+b4+b8, [b4,b8]=b3, 
  [b5,b6]=b1+b2+b5, [b5,b7]=b3, [b5,b8]=b2+b7, [b6,b7]=b4+b6, [b6,b8]=b2+b5, 
  [b7,b8]=b6.
\end{verbatim}
 

The non-trivial products for the unique simple Lie algebra with dimension 9
over \mbox{\texttt{\slshape GF(2)}} are as follows: 
\begin{verbatim}  
  [b1,b2]=b3, [b1,b3]=b5, [b1,b5]=b6, [b1,b6]=b7, [b1,b7]=b6+b9, 
  [b1,b9]=b2, [b2,b3]=b4, [b2,b4]=b6, [b2,b6]=b8, [b2,b8]=b6+b9, 
  [b2,b9]=b1, [b3,b4]=b7, [b3,b5]=b8, [b3,b7]=b1+b8, [b3,b8]=b2+b7, 
  [b4,b5]=b6+b9, [b4,b6]=b2+b7, [b4,b7]=b3+b6+b9, [b4,b9]=b5, 
  [b5,b6]=b1+b8, [b5,b8]=b3+b6+b9, [b5,b9]=b4, [b6,b7]=b1+b4+b8, 
  [b6,b8]=b2+b5+b7, [b7,b8]=b3+b9, [b7,b9]=b8, [b8,b9]=b7.
\end{verbatim}
 }

 
\section{\textcolor{Chapter }{Description of the solvable and nilpotent Lie algebras}}\label{listdescr}
\logpage{[ 3, 7, 0 ]}
\hyperdef{L}{X79FBD14A7959B5D2}{}
{
  In this section we list the multiplication tables of the nilpotent and
solvable Lie algebras contained in the package. Some parametric classes
contain isomorphic Lie algebras, for different values of the parameters. For
exact descriptions of these isomorphisms we refer to \cite{wdg05}, \cite{wdg07} and \cite{cdgs10}. In dimension 2 there are just two classes of solvable Lie algebras: 
\begin{itemize}
\item  $L_2^1$: The Abelian Lie algebra. 
\item  $L_2^2$: $[x_2,x_1]=x_1$. 
\end{itemize}
 We have the following solvable Lie algebras of dimension 3: 
\begin{itemize}
\item  $L_3^1$ The Abelian Lie algebra. 
\item  $L_3^2$ $[x_3,x_1]=x_1, [x_3,x_2]=x_2$. 
\item  $L_3^3(a)$ $[x_3,x_1]=x_2, [x_3,x_2]=ax_1+x_2$. 
\item  $L_3^4(a)$ $[x_3,x_1]=x_2, [x_3,x_2]=ax_1. $ 
\end{itemize}
 And the following solvable Lie algebras of dimension 4: 
\begin{itemize}
\item  $L_4^1$ The Abelian Lie algebra. 
\item  $L_4^2$ $ [x_4,x_1]=x_1, [x_4,x_2]=x_2, [x_4,x_3]=x_3.$ 
\item  $L_4^3(a)$ $[x_4,x_1]=x_1, [x_4,x_2]=x_3, [x_4,x_3]=-ax_2 +(a+1)x_3$.
\item  $L_4^4$ $[x_4,x_2]=x_3, [x_4,x_3]= x_3$.
\item  $L_4^5$ $[x_4,x_2]=x_3$.
\item  $L_4^6(a,b)$ $[x_4,x_1] = x_2, [x_4,x_2]=x_3, [x_4,x_3] = ax_1+bx_2+x_3$.
\item  $L_4^7(a,b)$ $[x_4,x_1] = x_2, [x_4,x_2]=x_3, [x_4,x_3] = ax_1+bx_2.$ 
\item  $L_4^8$ $[x_1,x_2]=x_2, [x_3,x_4]=x_4$.
\item  $L_4^9(a)$ $[x_4,x_1] = x_1+ax_2, [x_4,x_2]=x_1, [x_3,x_1]=x_1, [x_3,x_2]=x_2$.
\item  $L_4^{10}(a)$ $[x_4,x_1] = x_2, [x_4,x_2]=ax_1, [x_3,x_1]=x_1, [x_3,x_2]=x_2$ Condition on F: the characteristic of F is 2.
\item  $L_4^{11}(a,b)$ $[x_4,x_1] = x_1, [x_4,x_2] = bx_2, [x_4,x_3]=(1+b)x_3, [x_3,x_1]=x_2,
[x_3,x_2]=ax_1$. Condition on F: the characteristic of F is 2. 
\item  $L_4^{12}$ $ [x_4,x_1] = x_1, [x_4,x_2]=2x_2, [x_4,x_3] = x_3, [x_3,x_1]=x_2$.
\item  $L_4^{13}(a)$ $[x_4,x_1] = x_1+ax_3, [x_4,x_2]=x_2, [x_4,x_3] = x_1, [x_3,x_1]=x_2$.
\item  $L_4^{14}(a)$ $[x_4,x_1] = ax_3, [x_4,x_3]=x_1, [x_3,x_1]=x_2$.
\end{itemize}
 Nilpotent of dimension 5: 
\begin{itemize}
\item  $N_{5,1}$ Abelian. 
\item  $N_{5,2}$ $[x_1,x_2]=x_3$.
\item  $N_{5,3}$ $[x_1,x_2]=x_3, [x_1,x_3]=x_4$.
\item  $N_{5,4}$ $[x_1,x_2]=x_5, [x_3,x_4]=x_5$.
\item  $N_{5,5}$ $[x_1,x_2]=x_3, [x_1,x_3]=x_5, [x_2,x_4]=x_5$.
\item  $N_{5,6}$ $[x_1,x_2]=x_3, [x_1,x_3]=x_4, [x_1,x_4]=x_5, [x_2,x_3]=x_5$.
\item  $N_{5,7}$ $[x_1,x_2]=x_3, [x_1,x_3]=x_4, [x_1,x_4]=x_5$.
\item  $N_{5,8}$ $[x_1,x_2]=x_4, [x_1,x_3]=x_5$.
\item  $N_{5,9}$ $[x_1,x_2]=x_3, [x_1,x_3]=x_4, [x_2,x_3]=x_5$.
\end{itemize}
 We get nine 6-dimensional nilpotent Lie algebras denoted $N_{6,k}$ for $k=1,...,9$ that are the direct sum of $N_{5,k}$ and a 1-dimensional abelian ideal. Subsequently we get the following Lie
algebras. 
\begin{itemize}
\item  $N_{6,10}$ $[x_1,x_2]=x_3, [x_1,x_3]=x_6, [x_4,x_5]=x_6.$ 
\item  $N_{6,11}$ $[x_1,x_2]=x_3, [x_1,x_3]=x_4, [x_1,x_4]=x_6, [x_2,x_3]=x_6, [x_2,x_5]=x_6$.
\item  $N_{6,12}$ $[x_1,x_2]=x_3, [x_1,x_3]=x_4, [x_1,x_4]=x_6, [x_2,x_5]=x_6$.
\item  $N_{6,13}$ $[x_1,x_2]=x_3, [x_1,x_3]=x_5, [x_1,x_5]=x_6, [x_2,x_4]=x_5, [x_3,x_4]=x_6$.
\item  $N_{6,14}$ $[x_1,x_2]=x_3, [x_1,x_3]=x_4, [x_1,x_4]=x_5, [x_2,x_3]=x_5,
[x_2,x_5]=x_6,[x_3,x_4]=-x_6$.
\item  $N_{6,15}$ $[x_1,x_2]=x_3, [x_1,x_3]=x_4, [x_1,x_4]=x_5, [x_1,x_5]=x_6, [x_2,x_3]=x_5,
[x_2,x_4]=x_6$.
\item  $N_{6,16}$ $[x_1,x_2]=x_3, [x_1,x_3]=x_4, [x_1,x_4]=x_5, [x_2,x_5]=x_6, [x_3,x_4]=-x_6$.
\item  $N_{6,17}$ $[x_1,x_2]=x_3, [x_1,x_3]=x_4, [x_1,x_4]=x_5, [x_1,x_5]=x_6, [x_2,x_3]= x_6$.
\item  $N_{6,18}$ $[x_1,x_2]=x_3, [x_1,x_3]=x_4, [x_1,x_4]=x_5, [x_1,x_5]=x_6$.
\item  $N_{6,19}(a)$ $[x_1,x_2]=x_4, [x_1,x_3]=x_5, [x_1,x_5]=x_6, [x_2,x_4]=x_6, [x_3,x_5]=a x_6$, for $a{\ensuremath{\neq}}0$.
\item  $N_{6,20}$ $[x_1,x_2]=x_4, [x_1,x_3]=x_5, [x_1,x_5]=x_6, [x_2,x_4]=x_6$.
\item  $N_{6,21}(a)$ $[x_1,x_2]=x_3, [x_1,x_3]=x_4, [x_1,x_4]=x_6, [x_2,x_3]=x_5, [x_2,x_5]= a x_6$, for $a{\ensuremath{\neq}}0$.
\item  $N_{6,22}(a)$ $[x_1,x_2]=x_5, [x_1,x_3]=x_6, [x_2,x_4]= a x_6, [x_3,x_4]=x_5$.
\item  $N_{6,23}$ $[x_1,x_2]=x_3, [x_1,x_3]=x_5, [x_1,x_4]=x_6, [x_2,x_4]= x_5$.
\item  $N_{6,24}(a)$ $[x_1,x_2]=x_3, [x_1,x_3]=x_5, [x_1,x_4]=a x_6, [x_2,x_3]=x_6, [x_2,x_4]= x_5$.
\item  $N_{6,25}$ $[x_1,x_2]=x_3, [x_1,x_3]=x_5, [x_1,x_4]=x_6$.
\item  $N_{6,26}$ $[x_1,x_2]=x_4, [x_1,x_3]=x_5, [x_2,x_3]=x_6$.
\item  $N_{6,27}$ $[x_1,x_2]=x_3, [x_1,x_3]=x_5, [x_2,x_4]= x_6$.
\item  $N_{6,28}$ $[x_1,x_2]=x_3, [x_1,x_3]=x_4, [x_1,x_4]=x_5, [x_2,x_3]=x_6$.
\item  $N_{6,29}$ $[x_1,x_2]=x_3, [x_1,x_3]=x_5, [x_1,x_5]=x_6, [x_2,x_4]=x_5+x_6, [x_3,x_4]=x_6$, only over fields of characteristic $2$.
\item  $N_{6,30}$ $[x_1,x_2]=x_3, [x_1,x_3]=x_4, [x_1,x_4]=x_5, [x_1,x_5]=x_6, [x_2,x_3]=x_5+x_6,
[x_2,x_4]=x_6$, only over fields of characteristic $2$.
\item  $N_{6,31}(a)$ $[x_1,x_2]=x_3, [x_1,x_3]=x_4, [x_1,x_4]= x_5, [x_2,x_3]=x_5+a x_6,
[x_2,x_5]=x_6, [x_3,x_4]=x_6$, for $a{\ensuremath{\neq}}0$ and only over fields of characteristic $2$.
\item  $N_{6,32}(a)$ $[x_1,x_2]=x_3, [x_1,x_3]=x_4, [x_1,x_4]= x_5, [x_2,x_3]=a x_6, [x_2,x_5]=x_6,
[x_3,x_4]=x_6$, for $a{\ensuremath{\neq}}0$ and only over fields of characteristic $2$.
\item  $N_{6,33}$ $[x_1,x_2]=x_4, [x_1,x_3]=x_5, [x_2,x_5]=x_6, [x_3,x_4]=x_6$, only over fields of characteristic $2$.
\item  $N_{6,34}$ $[x_1,x_2]=x_3, [x_1,x_3]=x_4, [x_1,x_5]=x_6, [x_2,x_3]=x_5, [x_2,x_4]=x_6$, only over fields of characteristic $2$.
\item  $N_{6,35}(a)$ $[x_1,x_2]=x_5, [x_1,x_3]=x_6, [x_2,x_4]= a x_6, [x_3,x_4]=x_5+x_6$, only over fields of characteristic $2$.
\item  $N_{6,36}(a)$ $[x_1,x_2]=x_3, [x_1,x_3]=x_5, [x_1,x_4]= a x_6, [x_2,x_3]=x_6,
[x_2,x_4]=x_5+x_6$, only over fields of characteristic $2$.
\end{itemize}
 In \cite{cdgs10}, the Lie algebras $N_{5,k}$ are denoted by $L_{5,k}$ for all $k=1,...,9$. Similarly, the Lie algebras $N_{6,k}$ or $N_{6,k}(a)$, where $k=1,...,36$, are denoted by $L_{6,k}$ or $L_{6,k}(a)$ if $k=1,...,28$ and by $L_{6,k-28}^{(2)}$ or $L_{6,k-28}^{(2)}(a)$ if $k=29,...,36$. }

 }

 \def\bibname{References\logpage{[ "Bib", 0, 0 ]}
\hyperdef{L}{X7A6F98FD85F02BFE}{}
}

\bibliographystyle{alpha}
\bibliography{manual}

\newpage
\immediate\write\pagenrlog{["End"], \arabic{page}];}
\immediate\closeout\pagenrlog
\end{document}
