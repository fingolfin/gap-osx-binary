% generated by GAPDoc2LaTeX from XML source (Frank Luebeck)
\documentclass[a4paper,11pt]{report}

\usepackage{a4wide}
\sloppy
\pagestyle{myheadings}
\usepackage{amssymb}
\usepackage[utf8]{inputenc}
\usepackage{makeidx}
\makeindex
\usepackage{color}
\definecolor{FireBrick}{rgb}{0.5812,0.0074,0.0083}
\definecolor{RoyalBlue}{rgb}{0.0236,0.0894,0.6179}
\definecolor{RoyalGreen}{rgb}{0.0236,0.6179,0.0894}
\definecolor{RoyalRed}{rgb}{0.6179,0.0236,0.0894}
\definecolor{LightBlue}{rgb}{0.8544,0.9511,1.0000}
\definecolor{Black}{rgb}{0.0,0.0,0.0}

\definecolor{linkColor}{rgb}{0.0,0.0,0.554}
\definecolor{citeColor}{rgb}{0.0,0.0,0.554}
\definecolor{fileColor}{rgb}{0.0,0.0,0.554}
\definecolor{urlColor}{rgb}{0.0,0.0,0.554}
\definecolor{promptColor}{rgb}{0.0,0.0,0.589}
\definecolor{brkpromptColor}{rgb}{0.589,0.0,0.0}
\definecolor{gapinputColor}{rgb}{0.589,0.0,0.0}
\definecolor{gapoutputColor}{rgb}{0.0,0.0,0.0}

%%  for a long time these were red and blue by default,
%%  now black, but keep variables to overwrite
\definecolor{FuncColor}{rgb}{0.0,0.0,0.0}
%% strange name because of pdflatex bug:
\definecolor{Chapter }{rgb}{0.0,0.0,0.0}
\definecolor{DarkOlive}{rgb}{0.1047,0.2412,0.0064}


\usepackage{fancyvrb}

\usepackage{mathptmx,helvet}
\usepackage[T1]{fontenc}
\usepackage{textcomp}


\usepackage[
            pdftex=true,
            bookmarks=true,        
            a4paper=true,
            pdftitle={Written with GAPDoc},
            pdfcreator={LaTeX with hyperref package / GAPDoc},
            colorlinks=true,
            backref=page,
            breaklinks=true,
            linkcolor=linkColor,
            citecolor=citeColor,
            filecolor=fileColor,
            urlcolor=urlColor,
            pdfpagemode={UseNone}, 
           ]{hyperref}

\newcommand{\maintitlesize}{\fontsize{50}{55}\selectfont}

% write page numbers to a .pnr log file for online help
\newwrite\pagenrlog
\immediate\openout\pagenrlog =\jobname.pnr
\immediate\write\pagenrlog{PAGENRS := [}
\newcommand{\logpage}[1]{\protect\write\pagenrlog{#1, \thepage,}}
%% were never documented, give conflicts with some additional packages

\newcommand{\GAP}{\textsf{GAP}}

%% nicer description environments, allows long labels
\usepackage{enumitem}
\setdescription{style=nextline}

%% depth of toc
\setcounter{tocdepth}{1}





%% command for ColorPrompt style examples
\newcommand{\gapprompt}[1]{\color{promptColor}{\bfseries #1}}
\newcommand{\gapbrkprompt}[1]{\color{brkpromptColor}{\bfseries #1}}
\newcommand{\gapinput}[1]{\color{gapinputColor}{#1}}


\begin{document}

\logpage{[ 0, 0, 0 ]}
\begin{titlepage}
\mbox{}\vfill

\begin{center}{\maintitlesize \textbf{\textsf{4ti2Interface}\mbox{}}}\\
\vfill

\hypersetup{pdftitle=\textsf{4ti2Interface}}
\markright{\scriptsize \mbox{}\hfill \textsf{4ti2Interface} \hfill\mbox{}}
{\Huge \textbf{A link to 4ti2\mbox{}}}\\
\vfill

{\Huge Version 2013.09.20\mbox{}}\\[1cm]
{20/09/2013\mbox{}}\\[1cm]
\mbox{}\\[2cm]
{\Large \textbf{Sebastian Gutsche\\
    \mbox{}}}\\
\hypersetup{pdfauthor=Sebastian Gutsche\\
    }
\mbox{}\\[2cm]
\begin{minipage}{12cm}\noindent
 \\
\\
 This manual is best viewed as an \textsc{HTML} document. An \textsc{offline} version should be included in the documentation subfolder of the package. \\
\\
 \end{minipage}

\end{center}\vfill

\mbox{}\\
{\mbox{}\\
\small \noindent \textbf{Sebastian Gutsche\\
    }  Email: \href{mailto://gutsche@mathematik.uni-kl.de} {\texttt{gutsche@mathematik.uni-kl.de}}\\
  Homepage: \href{http://wwwb.math.rwth-aachen.de/~gutsche/} {\texttt{http://wwwb.math.rwth-aachen.de/\texttt{\symbol{126}}gutsche/}}\\
  Address: \begin{minipage}[t]{8cm}\noindent
 Department of Mathematics\\
 University of Kaiserslautern\\
 67653 Kaiserslautern\\
 Germany\\
 \end{minipage}
}\\
\end{titlepage}

\newpage\setcounter{page}{2}
{\small 
\section*{Copyright}
\logpage{[ 0, 0, 1 ]}
 This package may be distributed under the terms and conditions of the GNU
Public License Version 2. \mbox{}}\\[1cm]
\newpage

\def\contentsname{Contents\logpage{[ 0, 0, 2 ]}}

\tableofcontents
\newpage

 \index{\textsf{4ti2Interface}}     
\chapter{\textcolor{Chapter }{Introduction}}\label{Chapter_Introduction_automatically_generated_documentation_parts}
\logpage{[ 1, 0, 0 ]}
\hyperdef{L}{X7DFB63A97E67C0A1}{}
{
  
\section{\textcolor{Chapter }{What is the idea of 4ti2Interface}}\label{Chapter_Introduction_Section_What_is_the_idea_of_4ti2Interface_automatically_generated_documentation_parts}
\logpage{[ 1, 1, 0 ]}
\hyperdef{L}{X82A882D87ABF47EB}{}
{
  4ti2Interface is an GAP-Package that provides a link to the CAS 4ti2. It is
not supposed to do any work by itself, but to provide the methods in 4ti2 to
GAP. At the moment, it only capsules the groebner and hilbert method in 4ti2
but there are more to come. If you have any questions or suggestions, please
feel free to contact me, or leave an issue on \href{https://github.com/homalg-project/4ti2Interface.git} {\texttt{https://github.com/homalg-project/4ti2Interface.git}}. }

 }

   
\chapter{\textcolor{Chapter }{Installation}}\label{Chapter_Installation_automatically_generated_documentation_parts}
\logpage{[ 2, 0, 0 ]}
\hyperdef{L}{X8360C04082558A12}{}
{
  
\section{\textcolor{Chapter }{How to install this package}}\label{Chapter_Installation_Section_How_to_install_this_package_automatically_generated_documentation_parts}
\logpage{[ 2, 1, 0 ]}
\hyperdef{L}{X81A5946683F0AD7D}{}
{
  This package can only be used on a system that has 4ti2 installed. For more
information about this please visit \href{http://www.4ti2.de} {www.4ti2.de}. For installing this package, first make sure you have 4ti2 installed. Copy
it in your GAP pkg-directory. After this, the package can be loaded via
LoadPackage( "4ti2Interface" ); }

 }

   
\chapter{\textcolor{Chapter }{4ti2 functions}}\label{Chapter_4ti2_functions_automatically_generated_documentation_parts}
\logpage{[ 3, 0, 0 ]}
\hyperdef{L}{X876DE76280B7AB01}{}
{
  
\section{\textcolor{Chapter }{Groebner}}\label{Chapter_4ti2_functions_Section_Groebner_automatically_generated_documentation_parts}
\logpage{[ 3, 1, 0 ]}
\hyperdef{L}{X7C635ACB7DD200CF}{}
{
  These are wrappers of some use cases of 4ti2s groebner command. 

\subsection{\textcolor{Chapter }{4ti2Interface{\textunderscore}groebner{\textunderscore}matrix}}
\logpage{[ 3, 1, 1 ]}\nobreak
\hyperdef{L}{X7CCB80AD7BA246B0}{}
{\noindent\textcolor{FuncColor}{$\triangleright$\ \ \texttt{4ti2Interface{\textunderscore}groebner{\textunderscore}matrix({\mdseries\slshape matrix[, ordering]})\index{4ti2Interfacegroebnermatrix@\texttt{4ti2}\-\texttt{Interface{\textunderscore}groebner{\textunderscore}matrix}}
\label{4ti2Interfacegroebnermatrix}
}\hfill{\scriptsize (function)}}\\
\textbf{\indent Returns:\ }
A list of vectors



 This launches the 4ti2 groebner command with the argument as matrix input. The
output will be the the Groebner basis of the binomial ideal generated by the
left kernel of the input matrix. Note that this is different from 4ti2's
convention which takes the right kernel. It returns the output of the groebner
command as a list of lists. The second argument can be a vector to specify a
monomial ordering, in the way that x\texttt{\symbol{94}}m {\textgreater}
x\texttt{\symbol{94}}n if ordering*m {\textgreater} ordering*n }

 

\subsection{\textcolor{Chapter }{4ti2Interface{\textunderscore}groebner{\textunderscore}basis}}
\logpage{[ 3, 1, 2 ]}\nobreak
\hyperdef{L}{X86736FF783E8F6AF}{}
{\noindent\textcolor{FuncColor}{$\triangleright$\ \ \texttt{4ti2Interface{\textunderscore}groebner{\textunderscore}basis({\mdseries\slshape basis[, ordering]})\index{4ti2Interfacegroebnerbasis@\texttt{4ti2}\-\texttt{Interface{\textunderscore}groebner{\textunderscore}basis}}
\label{4ti2Interfacegroebnerbasis}
}\hfill{\scriptsize (function)}}\\
\textbf{\indent Returns:\ }
A list of vectors



 This launches the 4ti2 groebner command with the argument as matrix input. The
outpur will be the Groebner basis of the binomial ideal generated by the rows
of the input matrix. It returns the output of the groebner command as a list
of lists. The second argument is like before. }

 
\subsection{\textcolor{Chapter }{Defining ideal of toric variety}}\label{Groebner}
\logpage{[ 3, 1, 3 ]}
\hyperdef{L}{X8485878A84333E15}{}
{
  We want to compute the groebner basis of the ideal defining the affine toric
variety associated to the cone generated by the inequalities [ [ 7, -1 ], [ 0,
1 ] ], i.e. a rational normal curve. 
\begin{Verbatim}[commandchars=!@|,fontsize=\small,frame=single,label=Example]
  
  !gapprompt@gap>| !gapinput@LoadPackage( "4ti2Interface" );|
  true
  !gapprompt@gap>| !gapinput@cone := [ [ 7, -1 ], [ 0, 1 ] ];|
  [ [ 7, -1 ], [ 0, 1 ] ]
  !gapprompt@gap>| !gapinput@basis := 4ti2Interface_hilbert_inequalities( cone );;|
  !gapprompt@gap>| !gapinput@groebner := 4ti2Interface_groebner_matrix( basis );;|
  !gapprompt@gap>| !gapinput@time;|
  0
  !gapprompt@gap>| !gapinput@Length( groebner );|
  21
\end{Verbatim}
}

 }

 
\section{\textcolor{Chapter }{Hilbert}}\label{Chapter_4ti2_functions_Section_Hilbert_automatically_generated_documentation_parts}
\logpage{[ 3, 2, 0 ]}
\hyperdef{L}{X7F5D3AAB7834A607}{}
{
  These are wrappers of some use cases of 4ti2s hilbert command. 

\subsection{\textcolor{Chapter }{4ti2Interface{\textunderscore}hilbert{\textunderscore}inequalities}}
\logpage{[ 3, 2, 1 ]}\nobreak
\label{for inequalities}
\hyperdef{L}{X7DDFDF9D7DE9A29D}{}
{\noindent\textcolor{FuncColor}{$\triangleright$\ \ \texttt{4ti2Interface{\textunderscore}hilbert{\textunderscore}inequalities({\mdseries\slshape A})\index{4ti2Interfacehilbertinequalities@\texttt{4ti2}\-\texttt{Interface{\textunderscore}hilbert{\textunderscore}inequalities}}
\label{4ti2Interfacehilbertinequalities}
}\hfill{\scriptsize (function)}}\\
\noindent\textcolor{FuncColor}{$\triangleright$\ \ \texttt{4ti2Interface{\textunderscore}hilbert{\textunderscore}inequalities{\textunderscore}in{\textunderscore}positive{\textunderscore}orthant({\mdseries\slshape A})\index{4ti2Interfacehilbertinequalitiesinpositiveorthant@\texttt{4ti2}\-\texttt{Interface{\textunderscore}hilbert{\textunderscore}inequalities{\textunderscore}in{\textunderscore}positive{\textunderscore}orthant}}
\label{4ti2Interfacehilbertinequalitiesinpositiveorthant}
}\hfill{\scriptsize (function)}}\\
\textbf{\indent Returns:\ }
a list of vectors



 This function produces the hilbert basis of the cone C given by \mbox{\texttt{\mdseries\slshape A}}x {\textgreater}= 0 for all x in C. For the second function also x
{\textgreater}= 0 is assumed. }

 

\subsection{\textcolor{Chapter }{4ti2Interface{\textunderscore}hilbert{\textunderscore}equalities{\textunderscore}in{\textunderscore}positive{\textunderscore}orthant}}
\logpage{[ 3, 2, 2 ]}\nobreak
\hyperdef{L}{X7F9E586C817E3C08}{}
{\noindent\textcolor{FuncColor}{$\triangleright$\ \ \texttt{4ti2Interface{\textunderscore}hilbert{\textunderscore}equalities{\textunderscore}in{\textunderscore}positive{\textunderscore}orthant({\mdseries\slshape A})\index{4ti2Interfacehilbertequalitiesinpositiveorthant@\texttt{4ti2}\-\texttt{Interface{\textunderscore}hilbert{\textunderscore}equalities{\textunderscore}in{\textunderscore}positive{\textunderscore}orthant}}
\label{4ti2Interfacehilbertequalitiesinpositiveorthant}
}\hfill{\scriptsize (function)}}\\
\textbf{\indent Returns:\ }
a list of vectors



 This function produces the hilbert basis of the cone C given by the equations \mbox{\texttt{\mdseries\slshape A}}x = 0 in the positive orthant of the coordinate system. }

 

\subsection{\textcolor{Chapter }{4ti2Interface{\textunderscore}hilbert{\textunderscore}equalities{\textunderscore}and{\textunderscore}inequalities}}
\logpage{[ 3, 2, 3 ]}\nobreak
\label{for equalities and inequalities}
\hyperdef{L}{X80878F7E7F1DDDDE}{}
{\noindent\textcolor{FuncColor}{$\triangleright$\ \ \texttt{4ti2Interface{\textunderscore}hilbert{\textunderscore}equalities{\textunderscore}and{\textunderscore}inequalities({\mdseries\slshape A, B})\index{4ti2Interfacehilbertequalitiesandinequalities@\texttt{4ti2}\-\texttt{Interface{\textunderscore}hilbert{\textunderscore}equalities{\textunderscore}and{\textunderscore}inequalities}}
\label{4ti2Interfacehilbertequalitiesandinequalities}
}\hfill{\scriptsize (function)}}\\
\noindent\textcolor{FuncColor}{$\triangleright$\ \ \texttt{4ti2Interface{\textunderscore}hilbert{\textunderscore}equalities{\textunderscore}and{\textunderscore}inequalities{\textunderscore}in{\textunderscore}positive{\textunderscore}orthant({\mdseries\slshape A, B})\index{4ti2Interfacehilbertequalitiesandinequalitiesinpositiveorthant@\texttt{4ti2}\-\texttt{Interface{\textunderscore}hilbert{\textunderscore}equalities{\textunderscore}and{\textunderscore}inequalities{\textunderscore}in{\textunderscore}positive{\textunderscore}orthant}}
\label{4ti2Interfacehilbertequalitiesandinequalitiesinpositiveorthant}
}\hfill{\scriptsize (function)}}\\
\textbf{\indent Returns:\ }
a list of vectors



 This function produces the hilbert basis of the cone C given by the equations \mbox{\texttt{\mdseries\slshape A}}x = 0 and the inequations \mbox{\texttt{\mdseries\slshape B}}x {\textgreater}= 0. For the second function x{\textgreater}=0 is assumed. }

 
\subsection{\textcolor{Chapter }{Generators of semigroup}}\label{HilbertBasisEq}
\logpage{[ 3, 2, 4 ]}
\hyperdef{L}{X7822ED3E7DC13FAE}{}
{
  We want to compute the Hilbert basis of the cone obtained by intersecting the
positive orthant with the hyperplane given by the equation below. 
\begin{Verbatim}[commandchars=!@|,fontsize=\small,frame=single,label=Example]
  
  !gapprompt@gap>| !gapinput@LoadPackage( "4ti2Interface" );|
  true
  !gapprompt@gap>| !gapinput@gens := [ 23,25,37,49 ];|
  [ 23, 25, 37, 49 ]
  !gapprompt@gap>| !gapinput@equation := [ Concatenation( gens, -gens ) ];|
  [ [ 23, 25, 37, 49, -23, -25, -37, -49 ] ]
  !gapprompt@gap>| !gapinput@basis := 4ti2Interface_hilbert_equalities_in_positive_orthant( equation );;|
  !gapprompt@gap>| !gapinput@time;|
  12
  !gapprompt@gap>| !gapinput@Length( basis );|
  436
\end{Verbatim}
}

 
\subsection{\textcolor{Chapter }{Hilbert basis of dual cone}}\label{HilbertBasisIneq}
\logpage{[ 3, 2, 5 ]}
\hyperdef{L}{X7C9AE7868537AB0A}{}
{
  We want to compute the Hilbert basis of the cone which faces are represented
by the inequalities below. This example is taken from the toric and the
ToricVarieties package manual. In both packages it is very slow with the
internal algorithms. 
\begin{Verbatim}[commandchars=!@|,fontsize=\small,frame=single,label=Example]
  
  !gapprompt@gap>| !gapinput@LoadPackage( "4ti2Interface" );|
  true
  !gapprompt@gap>| !gapinput@inequalities := [ [1,2,3,4],[0,1,0,7],[3,1,0,2],[0,0,1,0] ];|
  [ [ 1, 2, 3, 4 ], [ 0, 1, 0, 7 ], [ 3, 1, 0, 2 ], [ 0, 0, 1, 0 ] ]
  !gapprompt@gap>| !gapinput@basis := 4ti2Interface_hilbert_inequalities( inequalities );;|
  !gapprompt@gap>| !gapinput@time;|
  0
  !gapprompt@gap>| !gapinput@Length( basis );|
  29
\end{Verbatim}
}

 }

 
\section{\textcolor{Chapter }{ZSolve}}\label{Chapter_4ti2_functions_Section_ZSolve_automatically_generated_documentation_parts}
\logpage{[ 3, 3, 0 ]}
\hyperdef{L}{X84237872798DB501}{}
{
  

\subsection{\textcolor{Chapter }{4ti2Interface{\textunderscore}zsolve{\textunderscore}equalities{\textunderscore}and{\textunderscore}inequalities}}
\logpage{[ 3, 3, 1 ]}\nobreak
\label{zsolve}
\hyperdef{L}{X82FD0D9F7B7EA6F8}{}
{\noindent\textcolor{FuncColor}{$\triangleright$\ \ \texttt{4ti2Interface{\textunderscore}zsolve{\textunderscore}equalities{\textunderscore}and{\textunderscore}inequalities({\mdseries\slshape eqs, eqs{\textunderscore}rhs, ineqs, ineqs{\textunderscore}rhs[, signs]})\index{4ti2Interfacezsolveequalitiesandinequalities@\texttt{4ti2}\-\texttt{Interface{\textunderscore}zsolve{\textunderscore}equalities{\textunderscore}and{\textunderscore}inequalities}}
\label{4ti2Interfacezsolveequalitiesandinequalities}
}\hfill{\scriptsize (function)}}\\
\noindent\textcolor{FuncColor}{$\triangleright$\ \ \texttt{4ti2Interface{\textunderscore}zsolve{\textunderscore}equalities{\textunderscore}and{\textunderscore}inequalities{\textunderscore}in{\textunderscore}positive{\textunderscore}orthant({\mdseries\slshape eqs, eqs{\textunderscore}rhs, ineqs, ineqs{\textunderscore}rhs})\index{4ti2Interfacezsolveequalitiesandinequalitiesinpositiveorthant@\texttt{4ti2}\-\texttt{Interface{\textunderscore}zsolve{\textunderscore}equalities{\textunderscore}and{\textunderscore}inequalities{\textunderscore}in{\textunderscore}positive{\textunderscore}orthant}}
\label{4ti2Interfacezsolveequalitiesandinequalitiesinpositiveorthant}
}\hfill{\scriptsize (function)}}\\
\textbf{\indent Returns:\ }
a list of three matrices



 This function produces a basis of the system \mbox{\texttt{\mdseries\slshape eqs}} = \mbox{\texttt{\mdseries\slshape eqs{\textunderscore}rhs}} and \mbox{\texttt{\mdseries\slshape ineqs}} {\textgreater}= \mbox{\texttt{\mdseries\slshape ineqs{\textunderscore}rhs}}. It outputs a list containing three matrices. The first one is a list of
points in a polytope, the second is the hilbert basis of a cone. The set of
solutions is then the minkowski sum of the polytope generated by the points in
the first list and the cone generated by the hilbert basis in the second
matrix. The third one is the free part of the solution polyhedron. The
optional argument \mbox{\texttt{\mdseries\slshape signs}} must be a list of zeros and ones which length is the number of variables. If
the ith entry is one, the ith variable must be {\textgreater}= 0. If the entry
is 0, the number is arbitraty. Default is all zero. For the second function xi
{\textgreater}= 0 for all variables is assumed. }

 }

 
\section{\textcolor{Chapter }{Graver}}\label{Chapter_4ti2_functions_Section_Graver_automatically_generated_documentation_parts}
\logpage{[ 3, 4, 0 ]}
\hyperdef{L}{X7D34D2A17EE6F480}{}
{
  

\subsection{\textcolor{Chapter }{4ti2Interface{\textunderscore}graver{\textunderscore}equalities}}
\logpage{[ 3, 4, 1 ]}\nobreak
\label{graver}
\hyperdef{L}{X84F90D9B79886CA6}{}
{\noindent\textcolor{FuncColor}{$\triangleright$\ \ \texttt{4ti2Interface{\textunderscore}graver{\textunderscore}equalities({\mdseries\slshape eqs[, signs]})\index{4ti2Interfacegraverequalities@\texttt{4ti2}\-\texttt{Interface{\textunderscore}graver{\textunderscore}equalities}}
\label{4ti2Interfacegraverequalities}
}\hfill{\scriptsize (function)}}\\
\noindent\textcolor{FuncColor}{$\triangleright$\ \ \texttt{4ti2Interface{\textunderscore}graver{\textunderscore}equalities{\textunderscore}in{\textunderscore}positive{\textunderscore}orthant({\mdseries\slshape eqs})\index{4ti2Interfacegraverequalitiesinpositiveorthant@\texttt{4ti2}\-\texttt{Interface{\textunderscore}graver{\textunderscore}equalities{\textunderscore}in{\textunderscore}positive{\textunderscore}orthant}}
\label{4ti2Interfacegraverequalitiesinpositiveorthant}
}\hfill{\scriptsize (function)}}\\
\textbf{\indent Returns:\ }
a matrix



 This calls the function graver with the equalities \mbox{\texttt{\mdseries\slshape eqs}} = 0. It outputs one list containing the graver basis of the system. the
optional argument \mbox{\texttt{\mdseries\slshape signs}} is used like in zsolve. The second command assumes x{\textunderscore}i
{\textgreater}= 0. }

 }

 }

   
\chapter{\textcolor{Chapter }{Tool functions}}\label{Chapter_Tool_functions_automatically_generated_documentation_parts}
\logpage{[ 4, 0, 0 ]}
\hyperdef{L}{X7A15CCB67FBCF3E3}{}
{
  
\section{\textcolor{Chapter }{Read and write matrix}}\label{Chapter_Tool_functions_Section_Read_and_write_matrix_automatically_generated_documentation_parts}
\logpage{[ 4, 1, 0 ]}
\hyperdef{L}{X86BDFBD07D23807E}{}
{
  

\subsection{\textcolor{Chapter }{4ti2Interface{\textunderscore}Read{\textunderscore}Matrix{\textunderscore}From{\textunderscore}File}}
\logpage{[ 4, 1, 1 ]}\nobreak
\hyperdef{L}{X7B786D5E8267CBD0}{}
{\noindent\textcolor{FuncColor}{$\triangleright$\ \ \texttt{4ti2Interface{\textunderscore}Read{\textunderscore}Matrix{\textunderscore}From{\textunderscore}File({\mdseries\slshape arg})\index{4ti2InterfaceReadMatrixFromFile@\texttt{4ti2}\-\texttt{Interface{\textunderscore}}\-\texttt{Read{\textunderscore}}\-\texttt{Matrix{\textunderscore}}\-\texttt{From{\textunderscore}}\-\texttt{File}}
\label{4ti2InterfaceReadMatrixFromFile}
}\hfill{\scriptsize (function)}}\\
\textbf{\indent Returns:\ }
a list of vectors



 The argument must be a string, representing a filename of a matrix to read.
Numbers must be seperated by whitespace, and the first two numbers must be the
number of rows and columns. The function then returns the matrix as list of
lists. }

 

\subsection{\textcolor{Chapter }{4ti2Interface{\textunderscore}Write{\textunderscore}Matrix{\textunderscore}To{\textunderscore}File}}
\logpage{[ 4, 1, 2 ]}\nobreak
\hyperdef{L}{X847592CF87F6DBEC}{}
{\noindent\textcolor{FuncColor}{$\triangleright$\ \ \texttt{4ti2Interface{\textunderscore}Write{\textunderscore}Matrix{\textunderscore}To{\textunderscore}File({\mdseries\slshape arg})\index{4ti2InterfaceWriteMatrixToFile@\texttt{4ti2}\-\texttt{Interface{\textunderscore}}\-\texttt{Write{\textunderscore}}\-\texttt{Matrix{\textunderscore}}\-\texttt{To{\textunderscore}}\-\texttt{File}}
\label{4ti2InterfaceWriteMatrixToFile}
}\hfill{\scriptsize (function)}}\\
\textbf{\indent Returns:\ }
nothing



 First argument must be a matrix, i.e. a list of list of integers. Second
argument has to be a filename. The method stores the matrix in this file,
seperated by whitespace, line by line. The content of the file, if there is
any, will be deleted. }

 

\subsection{\textcolor{Chapter }{4ti2Interface{\textunderscore}Cut{\textunderscore}Vector}}
\logpage{[ 4, 1, 3 ]}\nobreak
\hyperdef{L}{X80F4C48487375746}{}
{\noindent\textcolor{FuncColor}{$\triangleright$\ \ \texttt{4ti2Interface{\textunderscore}Cut{\textunderscore}Vector({\mdseries\slshape vec, d})\index{4ti2InterfaceCutVector@\texttt{4ti2}\-\texttt{Interface{\textunderscore}}\-\texttt{Cut{\textunderscore}}\-\texttt{Vector}}
\label{4ti2InterfaceCutVector}
}\hfill{\scriptsize (function)}}\\
\textbf{\indent Returns:\ }
a matrix



 Takes the vector \mbox{\texttt{\mdseries\slshape vec}} and produces a matrix with \mbox{\texttt{\mdseries\slshape d}} columns out of the entries of the vector. }

 }

 }

 \def\indexname{Index\logpage{[ "Ind", 0, 0 ]}
\hyperdef{L}{X83A0356F839C696F}{}
}

\cleardoublepage
\phantomsection
\addcontentsline{toc}{chapter}{Index}


\printindex

\newpage
\immediate\write\pagenrlog{["End"], \arabic{page}];}
\immediate\closeout\pagenrlog
\end{document}
