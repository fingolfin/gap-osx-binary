\Chapter{Mathematical background}

We assume that you are familiar with the theory of quasigroups and loops, for
instance with the textbook of Bruck \cite{Br} or Pflugfelder \cite{Pf}.
Nevertheless, we did include definitions and results in this manual in order to
unify the terminology and improve the intelligibility of the text. Some general
concepts of quasigroups and loops can be found in this chapter. More special
concepts are defined throughout the text as needed.

%%%%%%%%%%%%%%%%%%%%%%%%%%%%%%%%%%%%%%%%%%%%%%%%%%%%%%%%%%%%%%%%%%%%%%%%%%%%%%%
\Section{Quasigroups and loops}

A set with one binary operation (denoted $\cdot$ here) is called
<groupoid>\index{groupoid} or <magma> \index{magma}, the latter name being used
in {\GAP}. Associative groupoid is a <semigroup>\index{semigroup}.

An element $1$ of a groupoid $G$ is a <neutral element>\index{neutral element}
or an <identity element>\index{identity element} if $1\cdot x = x\cdot 1 = x$
for every $x$ in $G$. Semigroup with a neutral element is a
<monoid>\index{monoid}.

Let $G$ be a groupoid with neutral element $1$. Then an element $y$ is called a
<two-sided inverse>\index{inverse!two-sided} of $x$ in $G$ if $x\cdot y =
y\cdot x = 1$. A monoid in which every element has a two-sided inverse is
called a <group>\index{group}.

Groups can be reached in another way from groupoids, namely through quasigroups
and loops.

A <quasigroup>\index{quasigroup} $Q$ is a groupoid such that the equation
$x\cdot y=z$ has a unique solution in $Q$ whenever two of the three elements
$x$, $y$, $z$ of $Q$ are specified. Note that multiplication tables of finite
quasigroups are precisely <Latin squares>\index{Latin square}, i.e., a square
arrays with symbols arranged so that each symbol occurs in each row and in each
column exactly once. A <loop>\index{loop} $L$ is a quasigroup with a neutral
element.

Groups are clearly loops, and one can show easily that an associative
quasigroup is a group.

%%%%%%%%%%%%%%%%%%%%%%%%%%%%%%%%%%%%%%%%%%%%%%%%%%%%%%%%%%%%%%%%%%%%%%%%%%%%%%%
\Section{Translations}

Given an element $x$ of a quasigroup $Q$ we can associative two permutations of
$Q$ with it: the <left translation>\index{translation!left} $L_x:Q\to Q$
defined by $y\mapsto x\cdot y$, and the <right
translation>\index{translation!right} $R_x:Q\to Q$ defined by $y\mapsto y\cdot
x$.

Although it is possible to compose two right (left) translations, the resulting
permutation is not necessarily a right (left) translation. The set $\{L_x;x\in
Q\}$ is called the <left section>\index{section!left} of $Q$, and $\{R_x;x\in
Q\}$ is the <right section>\index{section!right} of $Q$.

Let $S_Q$ be the symmetric group on $Q$. Then the subgroup
${\rm{LMlt}}(Q)=\langle L_x|x\in Q\rangle$ of $S_Q$ generated by all left
translations is the <left multiplication group>\index{multiplication
group!left} of $Q$. Similarly, ${\rm{RMlt}}(Q)= \langle R_x|x\in Q\rangle$ is
the <right multiplication group>\index{multiplication group!right} of $Q$. The
smallest group containing both ${\rm{LMlt}}(Q)$ and ${\rm{RMlt}}(Q)$ is called
the <multiplication group>\index{multiplication group} of $Q$ and is denoted by
${\rm{Mlt}}(Q)$.

%%%%%%%%%%%%%%%%%%%%%%%%%%%%%%%%%%%%%%%%%%%%%%%%%%%%%%%%%%%%%%%%%%%%%%%%%%%%%%%
\Section{Homomorphisms and homotopisms}

Let $K$, $H$ be two quasigroups. Then a map $f:K\to H$ is a
<homomorphism>\index{homomorphism} if $f(x)\cdot f(y)=f(x\cdot y)$ for every
$x$, $y\in K$. If $f$ is also a bijection, we speak of an
<isomorphism>\index{isomorphism}, and the two quasigroups are called
<isomorphic>.

The ordered triple $(\alpha,\beta,\gamma)$ of maps $\alpha$, $\beta$,
$\gamma:K\to H$ is a <homotopism>\index{homotopism} if $\alpha(x)\cdot\beta(y)
= \gamma(x\cdot y)$ for every $x$, $y\in K$. If the three maps are bijections,
$(\alpha,\beta,\gamma)$ is an <isotopism>\index{isotopism}, and the two
quasigroups are <isotopic>.

Isotopic groups are necessarily isomorphic, but this is certainly not true for
nonassociative quasigroups or loops. In fact, every quasigroup is isotopic to a
loop, as we shall see.

Let $(K,\cdot)$, $(K,\circ)$ be two quasigroups defined on the same set $K$.
Then an isotopism $(\alpha,\beta,{\rm{id}}_K)$ is called a <principal
isotopism>\index{isotopism!principal}. An important class of principal
isotopisms is obtained as follows:

Let $(K,\cdot)$ be a quasigroup, and let $f$, $g$ be elements of $K$. Define a
new operation $\circ$ on $K$ by
$$
    x\circ y = R_g^{-1}(x)\cdot L_f^{-1}(y),
$$
where $R_g$, $L_f$ are translations. Then $(K,\circ)$ is a quasigroup isotopic
to $(K,\cdot)$, in fact a loop with neutral element $f\cdot g$. We call
$(K,\circ)$ a <principal loop isotope>\index{loop isotope!principal} of
$(K,\cdot)$.

%%%%%%%%%%%%%%%%%%%%%%%%%%%%%%%%%%%%%%%%%%%%%%%%%%%%%%%%%%%%%%%%%%%%%%%%%%%%%%%
\Section{Extensions}

Let $K$, $F$ be loops. Then a loop $Q$ is an <extension>\index{extension} of
$K$ by $F$ if $K$ is a normal subloop of $Q$ such that $Q/K$ is isomorphic to
$F$. An extension $Q$ of $K$ by $F$ is <nuclear>\index{extension!nuclear} if
$K$ is an abelian group and $K\le N(Q)$.

A map $\theta:F\times F\to K$ is a <cocycle>\index{cocycle} if $\theta(1,x) =
\theta(x,1) = 1$ for every $x\in F$.

The following theorem holds for loops $Q$, $F$ and an abelian group $K$: $Q$ is
a nuclear extension of $K$ by $F$ if and only if there is a cocycle
$\theta:F\times F\to K$ and a homomorphism $\varphi:F\to{\rm{Aut}}{Q}$ such
that $K\times F$ with multiplication $(a,x)(b,y) =
(a\varphi_x(b)\theta(x,y),xy)$ is isomorphic to $Q$.
