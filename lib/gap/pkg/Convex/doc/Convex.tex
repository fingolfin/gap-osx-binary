% generated by GAPDoc2LaTeX from XML source (Frank Luebeck)
\documentclass[a4paper,11pt]{report}

\usepackage{a4wide}
\sloppy
\pagestyle{myheadings}
\usepackage{amssymb}
\usepackage[utf8]{inputenc}
\usepackage{makeidx}
\makeindex
\usepackage{color}
\definecolor{FireBrick}{rgb}{0.5812,0.0074,0.0083}
\definecolor{RoyalBlue}{rgb}{0.0236,0.0894,0.6179}
\definecolor{RoyalGreen}{rgb}{0.0236,0.6179,0.0894}
\definecolor{RoyalRed}{rgb}{0.6179,0.0236,0.0894}
\definecolor{LightBlue}{rgb}{0.8544,0.9511,1.0000}
\definecolor{Black}{rgb}{0.0,0.0,0.0}

\definecolor{linkColor}{rgb}{0.0,0.0,0.554}
\definecolor{citeColor}{rgb}{0.0,0.0,0.554}
\definecolor{fileColor}{rgb}{0.0,0.0,0.554}
\definecolor{urlColor}{rgb}{0.0,0.0,0.554}
\definecolor{promptColor}{rgb}{0.0,0.0,0.589}
\definecolor{brkpromptColor}{rgb}{0.589,0.0,0.0}
\definecolor{gapinputColor}{rgb}{0.589,0.0,0.0}
\definecolor{gapoutputColor}{rgb}{0.0,0.0,0.0}

%%  for a long time these were red and blue by default,
%%  now black, but keep variables to overwrite
\definecolor{FuncColor}{rgb}{0.0,0.0,0.0}
%% strange name because of pdflatex bug:
\definecolor{Chapter }{rgb}{0.0,0.0,0.0}
\definecolor{DarkOlive}{rgb}{0.1047,0.2412,0.0064}


\usepackage{fancyvrb}

\usepackage{mathptmx,helvet}
\usepackage[T1]{fontenc}
\usepackage{textcomp}


\usepackage[
            pdftex=true,
            bookmarks=true,        
            a4paper=true,
            pdftitle={Written with GAPDoc},
            pdfcreator={LaTeX with hyperref package / GAPDoc},
            colorlinks=true,
            backref=page,
            breaklinks=true,
            linkcolor=linkColor,
            citecolor=citeColor,
            filecolor=fileColor,
            urlcolor=urlColor,
            pdfpagemode={UseNone}, 
           ]{hyperref}

\newcommand{\maintitlesize}{\fontsize{50}{55}\selectfont}

% write page numbers to a .pnr log file for online help
\newwrite\pagenrlog
\immediate\openout\pagenrlog =\jobname.pnr
\immediate\write\pagenrlog{PAGENRS := [}
\newcommand{\logpage}[1]{\protect\write\pagenrlog{#1, \thepage,}}
%% were never documented, give conflicts with some additional packages

\newcommand{\GAP}{\textsf{GAP}}

%% nicer description environments, allows long labels
\usepackage{enumitem}
\setdescription{style=nextline}

%% depth of toc
\setcounter{tocdepth}{1}





%% command for ColorPrompt style examples
\newcommand{\gapprompt}[1]{\color{promptColor}{\bfseries #1}}
\newcommand{\gapbrkprompt}[1]{\color{brkpromptColor}{\bfseries #1}}
\newcommand{\gapinput}[1]{\color{gapinputColor}{#1}}


\begin{document}

\logpage{[ 0, 0, 0 ]}
\begin{titlepage}
\mbox{}\vfill

\begin{center}{\maintitlesize \textbf{\textsf{Convex}\mbox{}}}\\
\vfill

\hypersetup{pdftitle=\textsf{Convex}}
\markright{\scriptsize \mbox{}\hfill \textsf{Convex} \hfill\mbox{}}
{\Huge \textbf{A \textsf{GAP} package for handling convex objects.\mbox{}}}\\
\vfill

{\Huge Version 2012.11.15\mbox{}}\\[1cm]
{August 2012\mbox{}}\\[1cm]
\mbox{}\\[2cm]
{\Large \textbf{Sebastian Gutsche\\
    \mbox{}}}\\
\hypersetup{pdfauthor=Sebastian Gutsche\\
    }
\mbox{}\\[2cm]
\begin{minipage}{12cm}\noindent
 \\
\\
 This manual is best viewed as an \textsc{HTML} document. An \textsc{offline} version should be included in the documentation subfolder of the package. \\
\\
 \end{minipage}

\end{center}\vfill

\mbox{}\\
{\mbox{}\\
\small \noindent \textbf{Sebastian Gutsche\\
    }  Email: \href{mailto://sebastian.gutsche@rwth-aachen.de} {\texttt{sebastian.gutsche@rwth-aachen.de}}\\
  Homepage: \href{http://wwwb.math.rwth-aachen.de/~gutsche} {\texttt{http://wwwb.math.rwth-aachen.de/\texttt{\symbol{126}}gutsche}}\\
  Address: \begin{minipage}[t]{8cm}\noindent
 Lehrstuhl B f{\"u}r Mathematik, RWTH Aachen, Templergraben 64, 52056 Aachen,
Germany \end{minipage}
}\\
\end{titlepage}

\newpage\setcounter{page}{2}
{\small 
\section*{Copyright}
\logpage{[ 0, 0, 1 ]}
 {\copyright} 2011-2012 by Sebastian Gutsche

 This package may be distributed under the terms and conditions of the GNU
Public License Version 2. \mbox{}}\\[1cm]
{\small 
\section*{Acknowledgements}
\logpage{[ 0, 0, 2 ]}
 \mbox{}}\\[1cm]
\newpage

\def\contentsname{Contents\logpage{[ 0, 0, 3 ]}}

\tableofcontents
\newpage

 \index{\textsf{Convex}}   
\chapter{\textcolor{Chapter }{Introduction}}\label{intro}
\logpage{[ 1, 0, 0 ]}
\hyperdef{L}{X7DFB63A97E67C0A1}{}
{
  
\section{\textcolor{Chapter }{What is the goal of the \textsf{Convex} package?}}\label{WhyToricVarieties}
\logpage{[ 1, 1, 0 ]}
\hyperdef{L}{X7B061C0C87A36AD1}{}
{
  \textsf{Convex} provides structures and algorithms for convex geometry. It can handle convex,
fans and polytopes. Not only the structures are provided, but also a
collection of algorithms to handle those objects. Basically, it provides
convex geometry to \textsf{GAP}. It is capable of communicating with the CAS polymake via the package \textsf{PolymakeInterface} and also provides several methods by itself. }

  }

   
\chapter{\textcolor{Chapter }{Installation of the \textsf{Convex} Package}}\label{install}
\logpage{[ 2, 0, 0 ]}
\hyperdef{L}{X781CA2768080E873}{}
{
  To install this package just extract the package's archive file to the \textsf{GAP} \texttt{pkg} directory.

 By default the \textsf{Convex} package is not automatically loaded by \textsf{GAP} when it is installed. You must load the package with \\
\\
 \texttt{LoadPackage}( "Convex" ); \\
\\
 before its functions become available.

 Please, send me an e-mail if you have any questions, remarks, suggestions,
etc. concerning this package. Also, I would be pleased to hear about
applications of this package and about any suggestions for new methods to add
to the package. \\
\\
\\
 Sebastian Gutsche  }

   
\chapter{\textcolor{Chapter }{Convex Objects}}\label{ConvexObject}
\logpage{[ 3, 0, 0 ]}
\hyperdef{L}{X8359268B7FDA6AEC}{}
{
  Convex objects are the main structure of \textsf{Convex}. All other structures, namely fans, cones, and polytopes are derived from
this structure. So all methods of this structure also apply to the other data
types. 
\section{\textcolor{Chapter }{Convex Objects: Category and Representations}}\label{ConvexObject:Category}
\logpage{[ 3, 1, 0 ]}
\hyperdef{L}{X82E0DD13824DC2C1}{}
{
  

\subsection{\textcolor{Chapter }{IsConvexObject}}
\logpage{[ 3, 1, 1 ]}\nobreak
\hyperdef{L}{X83ACD3DC7C1BE5F8}{}
{\noindent\textcolor{FuncColor}{$\triangleright$\ \ \texttt{IsConvexObject({\mdseries\slshape M})\index{IsConvexObject@\texttt{IsConvexObject}}
\label{IsConvexObject}
}\hfill{\scriptsize (Category)}}\\
\textbf{\indent Returns:\ }
\texttt{true} or \texttt{false}



 The \textsf{GAP} category of convex objects, the main category of this package. }

 }

 
\section{\textcolor{Chapter }{Convex objects: Properties}}\label{ConvexObject:Properties}
\logpage{[ 3, 2, 0 ]}
\hyperdef{L}{X85454292847AEBD5}{}
{
  

\subsection{\textcolor{Chapter }{IsFullDimensional}}
\logpage{[ 3, 2, 1 ]}\nobreak
\hyperdef{L}{X7A8A4EF182D275CA}{}
{\noindent\textcolor{FuncColor}{$\triangleright$\ \ \texttt{IsFullDimensional({\mdseries\slshape conv})\index{IsFullDimensional@\texttt{IsFullDimensional}}
\label{IsFullDimensional}
}\hfill{\scriptsize (property)}}\\
\textbf{\indent Returns:\ }
\texttt{true} or \texttt{false}



 Checks if the combinatorial dimension of the convex object \mbox{\texttt{\mdseries\slshape conv}} is the same as the dimension of the ambient space. }

 }

 
\section{\textcolor{Chapter }{Convex objects: Attributes}}\label{ConvexObject:Attributes}
\logpage{[ 3, 3, 0 ]}
\hyperdef{L}{X7E20C8697EA9490E}{}
{
  

\subsection{\textcolor{Chapter }{Dimension}}
\logpage{[ 3, 3, 1 ]}\nobreak
\hyperdef{L}{X7E6926C6850E7C4E}{}
{\noindent\textcolor{FuncColor}{$\triangleright$\ \ \texttt{Dimension({\mdseries\slshape conv})\index{Dimension@\texttt{Dimension}}
\label{Dimension}
}\hfill{\scriptsize (attribute)}}\\
\textbf{\indent Returns:\ }
an integer



 Returns the combinatorial dimension of the convex object \mbox{\texttt{\mdseries\slshape conv}}. This is the dimension of the smallest space i which \mbox{\texttt{\mdseries\slshape conv}} can be embedded. }

 

\subsection{\textcolor{Chapter }{AmbientSpaceDimension}}
\logpage{[ 3, 3, 2 ]}\nobreak
\hyperdef{L}{X791629C67F481601}{}
{\noindent\textcolor{FuncColor}{$\triangleright$\ \ \texttt{AmbientSpaceDimension({\mdseries\slshape conv})\index{AmbientSpaceDimension@\texttt{AmbientSpaceDimension}}
\label{AmbientSpaceDimension}
}\hfill{\scriptsize (attribute)}}\\
\textbf{\indent Returns:\ }
an integer



 Returns the dimension of the ambient space of the object \mbox{\texttt{\mdseries\slshape conv}}. }

 

\subsection{\textcolor{Chapter }{ContainingGrid}}
\logpage{[ 3, 3, 3 ]}\nobreak
\hyperdef{L}{X7C4692E0794B126E}{}
{\noindent\textcolor{FuncColor}{$\triangleright$\ \ \texttt{ContainingGrid({\mdseries\slshape conv})\index{ContainingGrid@\texttt{ContainingGrid}}
\label{ContainingGrid}
}\hfill{\scriptsize (attribute)}}\\
\textbf{\indent Returns:\ }
a homalg module



 Returns the ambient space of the object \mbox{\texttt{\mdseries\slshape conv}} as a homalg module. }

 }

 
\section{\textcolor{Chapter }{Convex objects: Methods}}\label{ConvexObject:Methods}
\logpage{[ 3, 4, 0 ]}
\hyperdef{L}{X7D7E0B658234B893}{}
{
  

\subsection{\textcolor{Chapter }{DrawObject}}
\logpage{[ 3, 4, 1 ]}\nobreak
\hyperdef{L}{X83FA826678EE4C1C}{}
{\noindent\textcolor{FuncColor}{$\triangleright$\ \ \texttt{DrawObject({\mdseries\slshape conv})\index{DrawObject@\texttt{DrawObject}}
\label{DrawObject}
}\hfill{\scriptsize (operation)}}\\
\textbf{\indent Returns:\ }
0



 Draws a nice picture of the object \mbox{\texttt{\mdseries\slshape conv}}, if your computer supports Java. As a side effect, you might not be able to
exit \textsf{GAP} anymore. }

 

\subsection{\textcolor{Chapter }{WeakPointerToExternalObject}}
\logpage{[ 3, 4, 2 ]}\nobreak
\hyperdef{L}{X807B4DE27F6BF439}{}
{\noindent\textcolor{FuncColor}{$\triangleright$\ \ \texttt{WeakPointerToExternalObject({\mdseries\slshape conv})\index{WeakPointerToExternalObject@\texttt{WeakPointerToExternalObject}}
\label{WeakPointerToExternalObject}
}\hfill{\scriptsize (operation)}}\\
\textbf{\indent Returns:\ }
a pointer



 Returns a pointer to an external object which is the basis of \mbox{\texttt{\mdseries\slshape conv}}. This method is not used any more. }

 }

  }

   
\chapter{\textcolor{Chapter }{Fan}}\label{Fan}
\logpage{[ 4, 0, 0 ]}
\hyperdef{L}{X80D0196B80DC94F3}{}
{
  
\section{\textcolor{Chapter }{Fan: Category and Representations}}\label{Fan:Category}
\logpage{[ 4, 1, 0 ]}
\hyperdef{L}{X7F4C80A1855F619C}{}
{
  

\subsection{\textcolor{Chapter }{IsFan}}
\logpage{[ 4, 1, 1 ]}\nobreak
\hyperdef{L}{X80B4C7D87A5ECDBF}{}
{\noindent\textcolor{FuncColor}{$\triangleright$\ \ \texttt{IsFan({\mdseries\slshape M})\index{IsFan@\texttt{IsFan}}
\label{IsFan}
}\hfill{\scriptsize (Category)}}\\
\textbf{\indent Returns:\ }
\texttt{true} or \texttt{false}



 The \textsf{GAP} category of a fan. Every fan is a convex object. }

 Remember: Every fan is a convex object. }

 
\section{\textcolor{Chapter }{Fan: Properties}}\label{Fan:Properties}
\logpage{[ 4, 2, 0 ]}
\hyperdef{L}{X7A83743785C9E8F1}{}
{
  

\subsection{\textcolor{Chapter }{IsComplete}}
\logpage{[ 4, 2, 1 ]}\nobreak
\hyperdef{L}{X7D689F21828A4278}{}
{\noindent\textcolor{FuncColor}{$\triangleright$\ \ \texttt{IsComplete({\mdseries\slshape fan})\index{IsComplete@\texttt{IsComplete}}
\label{IsComplete}
}\hfill{\scriptsize (property)}}\\
\textbf{\indent Returns:\ }
\texttt{true} or \texttt{false}



 Checks if the fan \mbox{\texttt{\mdseries\slshape fan}} is complete, i. e. if it's support is the whole space. }

 

\subsection{\textcolor{Chapter }{IsPointed}}
\logpage{[ 4, 2, 2 ]}\nobreak
\hyperdef{L}{X843A31A57EAB734C}{}
{\noindent\textcolor{FuncColor}{$\triangleright$\ \ \texttt{IsPointed({\mdseries\slshape fan})\index{IsPointed@\texttt{IsPointed}}
\label{IsPointed}
}\hfill{\scriptsize (property)}}\\
\textbf{\indent Returns:\ }
\texttt{true} or \texttt{false}



 Checks if the fan \mbox{\texttt{\mdseries\slshape fan}} is pointed, which means that every cone it contains is strictly convex. }

 

\subsection{\textcolor{Chapter }{IsSmooth}}
\logpage{[ 4, 2, 3 ]}\nobreak
\hyperdef{L}{X86CBF5497EC15CFC}{}
{\noindent\textcolor{FuncColor}{$\triangleright$\ \ \texttt{IsSmooth({\mdseries\slshape fan})\index{IsSmooth@\texttt{IsSmooth}}
\label{IsSmooth}
}\hfill{\scriptsize (property)}}\\
\textbf{\indent Returns:\ }
\texttt{true} or \texttt{false}



 Checks if the fan \mbox{\texttt{\mdseries\slshape fan}} is smooth, i. e. if every cone in the fan is smooth. }

 

\subsection{\textcolor{Chapter }{IsRegularFan}}
\logpage{[ 4, 2, 4 ]}\nobreak
\hyperdef{L}{X7838A553848AD380}{}
{\noindent\textcolor{FuncColor}{$\triangleright$\ \ \texttt{IsRegularFan({\mdseries\slshape fan})\index{IsRegularFan@\texttt{IsRegularFan}}
\label{IsRegularFan}
}\hfill{\scriptsize (property)}}\\
\textbf{\indent Returns:\ }
\texttt{true} or \texttt{false}



 Checks if the fan \mbox{\texttt{\mdseries\slshape fan}} is regular, i. e. if it is the normal fan of a polytope. }

 

\subsection{\textcolor{Chapter }{IsSimplicial (for a fan)}}
\logpage{[ 4, 2, 5 ]}\nobreak
\hyperdef{L}{X863CBF607A2AD000}{}
{\noindent\textcolor{FuncColor}{$\triangleright$\ \ \texttt{IsSimplicial({\mdseries\slshape fan})\index{IsSimplicial@\texttt{IsSimplicial}!for a fan}
\label{IsSimplicial:for a fan}
}\hfill{\scriptsize (property)}}\\
\textbf{\indent Returns:\ }
\texttt{true} or \texttt{false}



 Checks if the fan \mbox{\texttt{\mdseries\slshape fan}} is simplicial, i. e. if every cone in the fan is simplicial. }

 

\subsection{\textcolor{Chapter }{HasConvexSupport}}
\logpage{[ 4, 2, 6 ]}\nobreak
\hyperdef{L}{X8258DA9E820B9CF5}{}
{\noindent\textcolor{FuncColor}{$\triangleright$\ \ \texttt{HasConvexSupport({\mdseries\slshape fan})\index{HasConvexSupport@\texttt{HasConvexSupport}}
\label{HasConvexSupport}
}\hfill{\scriptsize (property)}}\\
\textbf{\indent Returns:\ }
\texttt{true} or \texttt{false}



 Checks if the fan \mbox{\texttt{\mdseries\slshape fan}} is simplicial, i. e. if every cone in the fan is simplicial. }

 }

 
\section{\textcolor{Chapter }{Fan: Attributes}}\label{Fan:Attributes}
\logpage{[ 4, 3, 0 ]}
\hyperdef{L}{X81E6FECC824A7C06}{}
{
  

\subsection{\textcolor{Chapter }{Rays}}
\logpage{[ 4, 3, 1 ]}\nobreak
\hyperdef{L}{X831FB73F86E6E4E9}{}
{\noindent\textcolor{FuncColor}{$\triangleright$\ \ \texttt{Rays({\mdseries\slshape fan})\index{Rays@\texttt{Rays}}
\label{Rays}
}\hfill{\scriptsize (attribute)}}\\
\textbf{\indent Returns:\ }
a list



 Returns the rays of the fan \mbox{\texttt{\mdseries\slshape fan}} as a list of cones. }

 

\subsection{\textcolor{Chapter }{RayGenerators}}
\logpage{[ 4, 3, 2 ]}\nobreak
\hyperdef{L}{X7CC22C4A85B6B51B}{}
{\noindent\textcolor{FuncColor}{$\triangleright$\ \ \texttt{RayGenerators({\mdseries\slshape fan})\index{RayGenerators@\texttt{RayGenerators}}
\label{RayGenerators}
}\hfill{\scriptsize (attribute)}}\\
\textbf{\indent Returns:\ }
a list



 Returns the generators rays of the fan \mbox{\texttt{\mdseries\slshape fan}} as a list of of list of integers. }

 

\subsection{\textcolor{Chapter }{RaysInMaximalCones}}
\logpage{[ 4, 3, 3 ]}\nobreak
\hyperdef{L}{X80472C677CB77C5B}{}
{\noindent\textcolor{FuncColor}{$\triangleright$\ \ \texttt{RaysInMaximalCones({\mdseries\slshape fan})\index{RaysInMaximalCones@\texttt{RaysInMaximalCones}}
\label{RaysInMaximalCones}
}\hfill{\scriptsize (attribute)}}\\
\textbf{\indent Returns:\ }
a list



 Returns a list of lists, which represent an incidence matrix for the
correspondence of the rays and the maximal cones of the fan \mbox{\texttt{\mdseries\slshape fan}}. The ith list in the result represents the ith maximal cone of \mbox{\texttt{\mdseries\slshape fan}}. In such a list, the jth entry is 1 if the jth ray is in the cone, 0
otherwise. }

 

\subsection{\textcolor{Chapter }{MaximalCones}}
\logpage{[ 4, 3, 4 ]}\nobreak
\hyperdef{L}{X8549BF0C78C9193B}{}
{\noindent\textcolor{FuncColor}{$\triangleright$\ \ \texttt{MaximalCones({\mdseries\slshape fan})\index{MaximalCones@\texttt{MaximalCones}}
\label{MaximalCones}
}\hfill{\scriptsize (attribute)}}\\
\textbf{\indent Returns:\ }
a list



 Returns the maximal cones of the fan \mbox{\texttt{\mdseries\slshape fan}} as a list of cones. }

 }

 
\section{\textcolor{Chapter }{Fan: Methods}}\label{Fan:Methods}
\logpage{[ 4, 4, 0 ]}
\hyperdef{L}{X8419F1C07A43ACDE}{}
{
  

\subsection{\textcolor{Chapter }{* (for fans)}}
\logpage{[ 4, 4, 1 ]}\nobreak
\hyperdef{L}{X846E545D78D769B8}{}
{\noindent\textcolor{FuncColor}{$\triangleright$\ \ \texttt{*({\mdseries\slshape fan1, fan2})\index{*@\texttt{*}!for fans}
\label{*:for fans}
}\hfill{\scriptsize (operation)}}\\
\textbf{\indent Returns:\ }
a fan



 Returns the product of the fans \mbox{\texttt{\mdseries\slshape fan1}} and \mbox{\texttt{\mdseries\slshape fan2}}. }

 }

 
\section{\textcolor{Chapter }{Fan: Constructors}}\label{Fan:Constructors}
\logpage{[ 4, 5, 0 ]}
\hyperdef{L}{X7C1E230383F32681}{}
{
  

\subsection{\textcolor{Chapter }{Fan (For Fans)}}
\logpage{[ 4, 5, 1 ]}\nobreak
\hyperdef{L}{X7C3F2E73846549A2}{}
{\noindent\textcolor{FuncColor}{$\triangleright$\ \ \texttt{Fan({\mdseries\slshape fan})\index{Fan@\texttt{Fan}!For Fans}
\label{Fan:For Fans}
}\hfill{\scriptsize (operation)}}\\
\textbf{\indent Returns:\ }
a fan



 Copy constructor for fans. For completeness reasons. }

 

\subsection{\textcolor{Chapter }{Fan (For a list of rays and a list of cones)}}
\logpage{[ 4, 5, 2 ]}\nobreak
\hyperdef{L}{X79EAB2B5838C6F1A}{}
{\noindent\textcolor{FuncColor}{$\triangleright$\ \ \texttt{Fan({\mdseries\slshape rays, cones})\index{Fan@\texttt{Fan}!For a list of rays and a list of cones}
\label{Fan:For a list of rays and a list of cones}
}\hfill{\scriptsize (operation)}}\\
\textbf{\indent Returns:\ }
a fan



 Constructs the fan out of the given \mbox{\texttt{\mdseries\slshape rays}} and a list of \mbox{\texttt{\mdseries\slshape cones}} given by a lists of numbers of rays. }

 }

 
\section{\textcolor{Chapter }{Fan: Examples}}\label{Fan:Examples}
\logpage{[ 4, 6, 0 ]}
\hyperdef{L}{X874C843E861EB3A6}{}
{
  
\subsection{\textcolor{Chapter }{Fan example}}\label{FanExamplePrimary}
\logpage{[ 4, 6, 1 ]}
\hyperdef{L}{X7A5BBAD884D93AD5}{}
{
  
\begin{Verbatim}[commandchars=!@B,fontsize=\small,frame=single,label=Example]
  !gapprompt@gap>B !gapinput@F := Fan( [[-1,5],[0,1],[1,0],[0,-1]],[[1,2],[2,3],[3,4],[4,1]] );B
  <A fan in |R^2>
  !gapprompt@gap>B !gapinput@RayGenerators( F );B
  [ [ -1, 5 ], [ 0, 1 ], [ 1, 0 ], [ 0, -1 ] ]
  !gapprompt@gap>B !gapinput@RaysInMaximalCones( F );B
  [ [ 1, 1, 0, 0 ], [ 0, 1, 1, 0 ], [ 0, 0, 1, 1 ], [ 1, 0, 0, 1 ] ]
  !gapprompt@gap>B !gapinput@IsRegularFan( F );B
  true
  !gapprompt@gap>B !gapinput@IsComplete( F );B
  true
  !gapprompt@gap>B !gapinput@IsSmooth( F );B
  true
  !gapprompt@gap>B !gapinput@F1 := MaximalCones( F )[ 1 ];B
  <A cone in |R^2>
  !gapprompt@gap>B !gapinput@DualCone( F1 );B
  <A cone in |R^2>
  !gapprompt@gap>B !gapinput@RayGenerators( F1 );B
  [ [ -1, 5 ], [ 0, 1 ] ]
  !gapprompt@gap>B !gapinput@F2 := StarSubdivisionOfIthMaximalCone( F, 1 );B
  <A fan in |R^2>
  !gapprompt@gap>B !gapinput@IsSmooth( F2 );B
  true
  !gapprompt@gap>B !gapinput@RayGenerators( F2 );B
  [ [ -1, 5 ], [ -1, 6 ], [ 0, -1 ], [ 0, 1 ], [ 1, 0 ] ]
\end{Verbatim}
}

 }

  }

   
\chapter{\textcolor{Chapter }{Cone}}\label{Cone}
\logpage{[ 5, 0, 0 ]}
\hyperdef{L}{X822975FC7F646FE5}{}
{
  
\section{\textcolor{Chapter }{Cone: Category and Representations}}\label{Cone:Category}
\logpage{[ 5, 1, 0 ]}
\hyperdef{L}{X7CAD43A27DB1C2E8}{}
{
  

\subsection{\textcolor{Chapter }{IsCone}}
\logpage{[ 5, 1, 1 ]}\nobreak
\hyperdef{L}{X80DFE6EA8575A9B0}{}
{\noindent\textcolor{FuncColor}{$\triangleright$\ \ \texttt{IsCone({\mdseries\slshape M})\index{IsCone@\texttt{IsCone}}
\label{IsCone}
}\hfill{\scriptsize (Category)}}\\
\textbf{\indent Returns:\ }
\texttt{true} or \texttt{false}



 The \textsf{GAP} category of a cone. }

 Remember: Every cone is a convex object. }

 
\section{\textcolor{Chapter }{Cone: Properties}}\label{Cone:Properties}
\logpage{[ 5, 2, 0 ]}
\hyperdef{L}{X82859C047B3C8F5E}{}
{
  

\subsection{\textcolor{Chapter }{IsRay}}
\logpage{[ 5, 2, 1 ]}\nobreak
\hyperdef{L}{X793B0F3E86C039BC}{}
{\noindent\textcolor{FuncColor}{$\triangleright$\ \ \texttt{IsRay({\mdseries\slshape cone})\index{IsRay@\texttt{IsRay}}
\label{IsRay}
}\hfill{\scriptsize (property)}}\\
\textbf{\indent Returns:\ }
\texttt{true} or \texttt{false}



 Checks if the cone \mbox{\texttt{\mdseries\slshape cone}} is a ray, i.e. if it has only one ray generator. }

 }

 
\section{\textcolor{Chapter }{Cone: Attributes}}\label{Cone:Attributes}
\logpage{[ 5, 3, 0 ]}
\hyperdef{L}{X79E016FF794B28D0}{}
{
  

\subsection{\textcolor{Chapter }{DualCone}}
\logpage{[ 5, 3, 1 ]}\nobreak
\hyperdef{L}{X8635EC787FEBB3FD}{}
{\noindent\textcolor{FuncColor}{$\triangleright$\ \ \texttt{DualCone({\mdseries\slshape cone})\index{DualCone@\texttt{DualCone}}
\label{DualCone}
}\hfill{\scriptsize (attribute)}}\\
\textbf{\indent Returns:\ }
a cone



 Returns the dual cone of the cone \mbox{\texttt{\mdseries\slshape cone}}. }

 

\subsection{\textcolor{Chapter }{HilbertBasis}}
\logpage{[ 5, 3, 2 ]}\nobreak
\hyperdef{L}{X7D549E567C52DCB5}{}
{\noindent\textcolor{FuncColor}{$\triangleright$\ \ \texttt{HilbertBasis({\mdseries\slshape cone})\index{HilbertBasis@\texttt{HilbertBasis}}
\label{HilbertBasis}
}\hfill{\scriptsize (attribute)}}\\
\textbf{\indent Returns:\ }
a list



 Returns a Hilbert Basis of the cone \mbox{\texttt{\mdseries\slshape cone}}. }

 

\subsection{\textcolor{Chapter }{RaysInFacets}}
\logpage{[ 5, 3, 3 ]}\nobreak
\hyperdef{L}{X840385CC7ACD01C4}{}
{\noindent\textcolor{FuncColor}{$\triangleright$\ \ \texttt{RaysInFacets({\mdseries\slshape cone})\index{RaysInFacets@\texttt{RaysInFacets}}
\label{RaysInFacets}
}\hfill{\scriptsize (attribute)}}\\
\textbf{\indent Returns:\ }
a list



 Returns an incidence matrix for the rays in the facets of the cone \mbox{\texttt{\mdseries\slshape cone}}. The ith entry of the result corresponds to the ith facet, the jth entry of
this is 1 if the jth ray is in th ith facet, 0 otherwise. }

 

\subsection{\textcolor{Chapter }{Facets}}
\logpage{[ 5, 3, 4 ]}\nobreak
\hyperdef{L}{X7AFE6D2C82F73788}{}
{\noindent\textcolor{FuncColor}{$\triangleright$\ \ \texttt{Facets({\mdseries\slshape cone})\index{Facets@\texttt{Facets}}
\label{Facets}
}\hfill{\scriptsize (attribute)}}\\
\textbf{\indent Returns:\ }
a list



 Returns a list of the facets of the cone \mbox{\texttt{\mdseries\slshape cone}} as homalg cones. }

 

\subsection{\textcolor{Chapter }{GridGeneratedByCone}}
\logpage{[ 5, 3, 5 ]}\nobreak
\hyperdef{L}{X7885EDAB80ED7705}{}
{\noindent\textcolor{FuncColor}{$\triangleright$\ \ \texttt{GridGeneratedByCone({\mdseries\slshape cone})\index{GridGeneratedByCone@\texttt{GridGeneratedByCone}}
\label{GridGeneratedByCone}
}\hfill{\scriptsize (attribute)}}\\
\textbf{\indent Returns:\ }
a homalg module



 Returns the grid generated by the lattice points of the cone \mbox{\texttt{\mdseries\slshape cone}} as a homalg module. }

 

\subsection{\textcolor{Chapter }{FactorGrid}}
\logpage{[ 5, 3, 6 ]}\nobreak
\hyperdef{L}{X7B1669747B6CBCAE}{}
{\noindent\textcolor{FuncColor}{$\triangleright$\ \ \texttt{FactorGrid({\mdseries\slshape cone})\index{FactorGrid@\texttt{FactorGrid}}
\label{FactorGrid}
}\hfill{\scriptsize (attribute)}}\\
\textbf{\indent Returns:\ }
a homalg module



 Returns the factor of the containing grid of the cone \mbox{\texttt{\mdseries\slshape cone}} and the grid generated by \mbox{\texttt{\mdseries\slshape cone}}. }

 

\subsection{\textcolor{Chapter }{GridGeneratedByOrthogonalCone}}
\logpage{[ 5, 3, 7 ]}\nobreak
\hyperdef{L}{X7FD62BD58783C1D6}{}
{\noindent\textcolor{FuncColor}{$\triangleright$\ \ \texttt{GridGeneratedByOrthogonalCone({\mdseries\slshape cone})\index{GridGeneratedByOrthogonalCone@\texttt{GridGeneratedByOrthogonalCone}}
\label{GridGeneratedByOrthogonalCone}
}\hfill{\scriptsize (attribute)}}\\
\textbf{\indent Returns:\ }
a homalg module



 Returns the grid generated by the lattice points of the orthogonal cone of the
cone \mbox{\texttt{\mdseries\slshape cone}}. }

 

\subsection{\textcolor{Chapter }{DefiningInequalities}}
\logpage{[ 5, 3, 8 ]}\nobreak
\hyperdef{L}{X7CB1A6657B3B3550}{}
{\noindent\textcolor{FuncColor}{$\triangleright$\ \ \texttt{DefiningInequalities({\mdseries\slshape cone})\index{DefiningInequalities@\texttt{DefiningInequalities}}
\label{DefiningInequalities}
}\hfill{\scriptsize (attribute)}}\\
\textbf{\indent Returns:\ }
a list



 Returns a list of the defining inequalities of the cone \mbox{\texttt{\mdseries\slshape cone}}. }

 

\subsection{\textcolor{Chapter }{IsContainedInFan}}
\logpage{[ 5, 3, 9 ]}\nobreak
\hyperdef{L}{X857893CC7BFDE0E0}{}
{\noindent\textcolor{FuncColor}{$\triangleright$\ \ \texttt{IsContainedInFan({\mdseries\slshape cone})\index{IsContainedInFan@\texttt{IsContainedInFan}}
\label{IsContainedInFan}
}\hfill{\scriptsize (attribute)}}\\
\textbf{\indent Returns:\ }
a fan



 If the cone \mbox{\texttt{\mdseries\slshape cone}} is constructed as part of a fan, this method returns the fan. }

 

\subsection{\textcolor{Chapter }{FactorGridMorphism}}
\logpage{[ 5, 3, 10 ]}\nobreak
\hyperdef{L}{X7AA3F8617E28E7BD}{}
{\noindent\textcolor{FuncColor}{$\triangleright$\ \ \texttt{FactorGridMorphism({\mdseries\slshape cone})\index{FactorGridMorphism@\texttt{FactorGridMorphism}}
\label{FactorGridMorphism}
}\hfill{\scriptsize (attribute)}}\\
\textbf{\indent Returns:\ }
a morphism



 Returns the morphism to the factor grid of the cone \mbox{\texttt{\mdseries\slshape cone}}. }

 }

 
\section{\textcolor{Chapter }{Cone: Methods}}\label{Cone:Methods}
\logpage{[ 5, 4, 0 ]}
\hyperdef{L}{X7DD2D0EA7EE584AA}{}
{
  

\subsection{\textcolor{Chapter }{IntersectionOfCones}}
\logpage{[ 5, 4, 1 ]}\nobreak
\hyperdef{L}{X803F0640808F0A4A}{}
{\noindent\textcolor{FuncColor}{$\triangleright$\ \ \texttt{IntersectionOfCones({\mdseries\slshape cone1, cone2})\index{IntersectionOfCones@\texttt{IntersectionOfCones}}
\label{IntersectionOfCones}
}\hfill{\scriptsize (operation)}}\\
\textbf{\indent Returns:\ }
a cone



 If the cones \mbox{\texttt{\mdseries\slshape cone1}} and \mbox{\texttt{\mdseries\slshape cone2}} share a face, the method returns their intersection, }

 

\subsection{\textcolor{Chapter }{Contains}}
\logpage{[ 5, 4, 2 ]}\nobreak
\hyperdef{L}{X851A362E8584EE03}{}
{\noindent\textcolor{FuncColor}{$\triangleright$\ \ \texttt{Contains({\mdseries\slshape cone1, cone2})\index{Contains@\texttt{Contains}}
\label{Contains}
}\hfill{\scriptsize (operation)}}\\
\textbf{\indent Returns:\ }
\texttt{true} or \texttt{false}



 Returns \texttt{true} if the cone \mbox{\texttt{\mdseries\slshape cone1}} contains the cone \mbox{\texttt{\mdseries\slshape cone2}}, \texttt{false} otherwise. }

 

\subsection{\textcolor{Chapter }{StarFan (for a cone)}}
\logpage{[ 5, 4, 3 ]}\nobreak
\hyperdef{L}{X7C7CF17887D7D27E}{}
{\noindent\textcolor{FuncColor}{$\triangleright$\ \ \texttt{StarFan({\mdseries\slshape cone})\index{StarFan@\texttt{StarFan}!for a cone}
\label{StarFan:for a cone}
}\hfill{\scriptsize (operation)}}\\
\textbf{\indent Returns:\ }
a fan



 Returns the star fan of the cone \mbox{\texttt{\mdseries\slshape cone}}, as described in cox, 3.2.7 }

 

\subsection{\textcolor{Chapter }{StarFan (for a cone and a fan)}}
\logpage{[ 5, 4, 4 ]}\nobreak
\hyperdef{L}{X84CFDA0883327BB0}{}
{\noindent\textcolor{FuncColor}{$\triangleright$\ \ \texttt{StarFan({\mdseries\slshape cone, fan})\index{StarFan@\texttt{StarFan}!for a cone and a fan}
\label{StarFan:for a cone and a fan}
}\hfill{\scriptsize (operation)}}\\
\textbf{\indent Returns:\ }
a fan



 Returns the star fan of the fan \mbox{\texttt{\mdseries\slshape fan}} along the cone \mbox{\texttt{\mdseries\slshape cone}}, as described in cox, 3.2.7 }

 

\subsection{\textcolor{Chapter }{StarSubdivisionOfIthMaximalCone}}
\logpage{[ 5, 4, 5 ]}\nobreak
\hyperdef{L}{X7E4D3AB37B384638}{}
{\noindent\textcolor{FuncColor}{$\triangleright$\ \ \texttt{StarSubdivisionOfIthMaximalCone({\mdseries\slshape fan, numb})\index{StarSubdivisionOfIthMaximalCone@\texttt{StarSubdivisionOfIthMaximalCone}}
\label{StarSubdivisionOfIthMaximalCone}
}\hfill{\scriptsize (operation)}}\\
\textbf{\indent Returns:\ }
a fan



 Returns the star subdivision of the fan \mbox{\texttt{\mdseries\slshape fan}} on the \mbox{\texttt{\mdseries\slshape numb}}th maximal cone as in cox, 3.3.13. }

 }

 
\section{\textcolor{Chapter }{Cone: Constructors}}\label{Cone:Constructors}
\logpage{[ 5, 5, 0 ]}
\hyperdef{L}{X7DFBB2A782DCFCCA}{}
{
  

\subsection{\textcolor{Chapter }{Cone (for a ray list)}}
\logpage{[ 5, 5, 1 ]}\nobreak
\hyperdef{L}{X8044339D7E71010B}{}
{\noindent\textcolor{FuncColor}{$\triangleright$\ \ \texttt{Cone({\mdseries\slshape cone})\index{Cone@\texttt{Cone}!for a ray list}
\label{Cone:for a ray list}
}\hfill{\scriptsize (operation)}}\\
\textbf{\indent Returns:\ }
a cone



 Returns a cone generated by the rays in \mbox{\texttt{\mdseries\slshape cone}}. }

 }

 
\section{\textcolor{Chapter }{Cone: Examples}}\label{Cone:Examples}
\logpage{[ 5, 6, 0 ]}
\hyperdef{L}{X84BE1F7279A2C49C}{}
{
  
\subsection{\textcolor{Chapter }{Cone example}}\label{ConeExamplePrimary}
\logpage{[ 5, 6, 1 ]}
\hyperdef{L}{X81EAFA247C2687D4}{}
{
  
\begin{Verbatim}[commandchars=!@E,fontsize=\small,frame=single,label=Example]
  !gapprompt@gap>E !gapinput@C := Cone([[1,2,3],[2,1,1],[1,0,0],[0,1,1]]);E
  <A cone in |R^3>
  !gapprompt@gap>E !gapinput@Length( RayGenerators( C ) );E
  3
  !gapprompt@gap>E !gapinput@IsSmooth( C );E
  true
  !gapprompt@gap>E !gapinput@Length( HilbertBasis( C ) );E
  3
  !gapprompt@gap>E !gapinput@IsSimplicial( C );E
  true
  !gapprompt@gap>E !gapinput@DC := DualCone( C );E
  <A cone in |R^3>
  !gapprompt@gap>E !gapinput@Length( HilbertBasis( DC ) );E
  3
\end{Verbatim}
}

 }

  }

   
\chapter{\textcolor{Chapter }{Polytope}}\label{Polytope}
\logpage{[ 6, 0, 0 ]}
\hyperdef{L}{X855106007DE72898}{}
{
  
\section{\textcolor{Chapter }{Polytope: Category and Representations}}\label{Polytope:Category}
\logpage{[ 6, 1, 0 ]}
\hyperdef{L}{X86EFB7F37A7256B8}{}
{
  

\subsection{\textcolor{Chapter }{IsPolytope}}
\logpage{[ 6, 1, 1 ]}\nobreak
\hyperdef{L}{X81EA74AA7B4B6DDB}{}
{\noindent\textcolor{FuncColor}{$\triangleright$\ \ \texttt{IsPolytope({\mdseries\slshape M})\index{IsPolytope@\texttt{IsPolytope}}
\label{IsPolytope}
}\hfill{\scriptsize (Category)}}\\
\textbf{\indent Returns:\ }
\texttt{true} or \texttt{false}



 The \textsf{GAP} category of a polytope. Every polytope is a convex object. }

 Remember: Every cone is a convex object. }

 
\section{\textcolor{Chapter }{Polytope: Properties}}\label{Polytope:Properties}
\logpage{[ 6, 2, 0 ]}
\hyperdef{L}{X7CBD76CF85B3DD81}{}
{
  

\subsection{\textcolor{Chapter }{IsNotEmpty}}
\logpage{[ 6, 2, 1 ]}\nobreak
\hyperdef{L}{X87705F6D7B129879}{}
{\noindent\textcolor{FuncColor}{$\triangleright$\ \ \texttt{IsNotEmpty({\mdseries\slshape poly})\index{IsNotEmpty@\texttt{IsNotEmpty}}
\label{IsNotEmpty}
}\hfill{\scriptsize (property)}}\\
\textbf{\indent Returns:\ }
\texttt{true} or \texttt{false}



 Checks if the polytope \mbox{\texttt{\mdseries\slshape poly}} is not empty. }

 

\subsection{\textcolor{Chapter }{IsLatticePolytope}}
\logpage{[ 6, 2, 2 ]}\nobreak
\hyperdef{L}{X79F588238781B2C9}{}
{\noindent\textcolor{FuncColor}{$\triangleright$\ \ \texttt{IsLatticePolytope({\mdseries\slshape poly})\index{IsLatticePolytope@\texttt{IsLatticePolytope}}
\label{IsLatticePolytope}
}\hfill{\scriptsize (property)}}\\
\textbf{\indent Returns:\ }
\texttt{true} or \texttt{false}



 Checks if the polytope \mbox{\texttt{\mdseries\slshape poly}} is a lattice polytope, i.e. all its vertices are lattice points. }

 

\subsection{\textcolor{Chapter }{IsVeryAmple}}
\logpage{[ 6, 2, 3 ]}\nobreak
\hyperdef{L}{X80A58559802BB02E}{}
{\noindent\textcolor{FuncColor}{$\triangleright$\ \ \texttt{IsVeryAmple({\mdseries\slshape poly})\index{IsVeryAmple@\texttt{IsVeryAmple}}
\label{IsVeryAmple}
}\hfill{\scriptsize (property)}}\\
\textbf{\indent Returns:\ }
\texttt{true} or \texttt{false}



 Checks if the polytope \mbox{\texttt{\mdseries\slshape poly}} is very ample. }

 

\subsection{\textcolor{Chapter }{IsNormalPolytope}}
\logpage{[ 6, 2, 4 ]}\nobreak
\hyperdef{L}{X7C3C14CB83C98EFD}{}
{\noindent\textcolor{FuncColor}{$\triangleright$\ \ \texttt{IsNormalPolytope({\mdseries\slshape poly})\index{IsNormalPolytope@\texttt{IsNormalPolytope}}
\label{IsNormalPolytope}
}\hfill{\scriptsize (property)}}\\
\textbf{\indent Returns:\ }
\texttt{true} or \texttt{false}



 Checks if the polytope \mbox{\texttt{\mdseries\slshape poly}} is normal. }

 

\subsection{\textcolor{Chapter }{IsSimplicial (for a polytope)}}
\logpage{[ 6, 2, 5 ]}\nobreak
\hyperdef{L}{X7AB9716B7DFE7CCF}{}
{\noindent\textcolor{FuncColor}{$\triangleright$\ \ \texttt{IsSimplicial({\mdseries\slshape poly})\index{IsSimplicial@\texttt{IsSimplicial}!for a polytope}
\label{IsSimplicial:for a polytope}
}\hfill{\scriptsize (property)}}\\
\textbf{\indent Returns:\ }
\texttt{true} or \texttt{false}



 Checks if the polytope \mbox{\texttt{\mdseries\slshape poly}} is simplicial. }

 

\subsection{\textcolor{Chapter }{IsSimplePolytope}}
\logpage{[ 6, 2, 6 ]}\nobreak
\hyperdef{L}{X7F0DF19F82E6DEBD}{}
{\noindent\textcolor{FuncColor}{$\triangleright$\ \ \texttt{IsSimplePolytope({\mdseries\slshape poly})\index{IsSimplePolytope@\texttt{IsSimplePolytope}}
\label{IsSimplePolytope}
}\hfill{\scriptsize (property)}}\\
\textbf{\indent Returns:\ }
\texttt{true} or \texttt{false}



 Checks if the polytope \mbox{\texttt{\mdseries\slshape poly}} is simple. }

 }

 
\section{\textcolor{Chapter }{Polytope: Attributes}}\label{Polytope:Attributes}
\logpage{[ 6, 3, 0 ]}
\hyperdef{L}{X87D8FC34790A474E}{}
{
  

\subsection{\textcolor{Chapter }{Vertices}}
\logpage{[ 6, 3, 1 ]}\nobreak
\hyperdef{L}{X79E4BB4F849AC8A1}{}
{\noindent\textcolor{FuncColor}{$\triangleright$\ \ \texttt{Vertices({\mdseries\slshape poly})\index{Vertices@\texttt{Vertices}}
\label{Vertices}
}\hfill{\scriptsize (attribute)}}\\
\textbf{\indent Returns:\ }
a list



 Returns the vertices of the polytope \mbox{\texttt{\mdseries\slshape poly}}. For reasons, the corresponding tester is HasVerticesOfPolytopes }

 

\subsection{\textcolor{Chapter }{LatticePoints}}
\logpage{[ 6, 3, 2 ]}\nobreak
\hyperdef{L}{X7FFECA277E47A55B}{}
{\noindent\textcolor{FuncColor}{$\triangleright$\ \ \texttt{LatticePoints({\mdseries\slshape poly})\index{LatticePoints@\texttt{LatticePoints}}
\label{LatticePoints}
}\hfill{\scriptsize (attribute)}}\\
\textbf{\indent Returns:\ }
a list



 Returns the lattice points of the polytope \mbox{\texttt{\mdseries\slshape poly}}. }

 

\subsection{\textcolor{Chapter }{FacetInequalities}}
\logpage{[ 6, 3, 3 ]}\nobreak
\hyperdef{L}{X78D14B178577BFB1}{}
{\noindent\textcolor{FuncColor}{$\triangleright$\ \ \texttt{FacetInequalities({\mdseries\slshape poly})\index{FacetInequalities@\texttt{FacetInequalities}}
\label{FacetInequalities}
}\hfill{\scriptsize (attribute)}}\\
\textbf{\indent Returns:\ }
a list



 Returns the facet inequalities for the polytope \mbox{\texttt{\mdseries\slshape poly}}. }

 

\subsection{\textcolor{Chapter }{VerticesInFacets}}
\logpage{[ 6, 3, 4 ]}\nobreak
\hyperdef{L}{X7E31AE1886051099}{}
{\noindent\textcolor{FuncColor}{$\triangleright$\ \ \texttt{VerticesInFacets({\mdseries\slshape poly})\index{VerticesInFacets@\texttt{VerticesInFacets}}
\label{VerticesInFacets}
}\hfill{\scriptsize (attribute)}}\\
\textbf{\indent Returns:\ }
a list



 Returns the incidence matrix of vertices and facets of the polytope \mbox{\texttt{\mdseries\slshape poly}}. }

 

\subsection{\textcolor{Chapter }{AffineCone}}
\logpage{[ 6, 3, 5 ]}\nobreak
\hyperdef{L}{X7C3748B8878B799A}{}
{\noindent\textcolor{FuncColor}{$\triangleright$\ \ \texttt{AffineCone({\mdseries\slshape poly})\index{AffineCone@\texttt{AffineCone}}
\label{AffineCone}
}\hfill{\scriptsize (attribute)}}\\
\textbf{\indent Returns:\ }
a cone



 Returns the affine cone of the polytope \mbox{\texttt{\mdseries\slshape poly}}. }

 

\subsection{\textcolor{Chapter }{NormalFan}}
\logpage{[ 6, 3, 6 ]}\nobreak
\hyperdef{L}{X7D7E33B97A7B4039}{}
{\noindent\textcolor{FuncColor}{$\triangleright$\ \ \texttt{NormalFan({\mdseries\slshape poly})\index{NormalFan@\texttt{NormalFan}}
\label{NormalFan}
}\hfill{\scriptsize (attribute)}}\\
\textbf{\indent Returns:\ }
a fan



 Returns the normal fan of the polytope \mbox{\texttt{\mdseries\slshape poly}}. }

 

\subsection{\textcolor{Chapter }{RelativeInteriorLatticePoints}}
\logpage{[ 6, 3, 7 ]}\nobreak
\hyperdef{L}{X7E82C1C483269893}{}
{\noindent\textcolor{FuncColor}{$\triangleright$\ \ \texttt{RelativeInteriorLatticePoints({\mdseries\slshape poly})\index{RelativeInteriorLatticePoints@\texttt{RelativeInteriorLatticePoints}}
\label{RelativeInteriorLatticePoints}
}\hfill{\scriptsize (attribute)}}\\
\textbf{\indent Returns:\ }
a list



 Returns the lattice points in the relative interior of the polytope \mbox{\texttt{\mdseries\slshape poly}}. }

 }

 
\section{\textcolor{Chapter }{Polytope: Methods}}\label{Polytope:Methods}
\logpage{[ 6, 4, 0 ]}
\hyperdef{L}{X82806E0786AB09E5}{}
{
  

\subsection{\textcolor{Chapter }{* (for polytopes)}}
\logpage{[ 6, 4, 1 ]}\nobreak
\hyperdef{L}{X87DA13AA8305F283}{}
{\noindent\textcolor{FuncColor}{$\triangleright$\ \ \texttt{*({\mdseries\slshape polytope1, polytope2})\index{*@\texttt{*}!for polytopes}
\label{*:for polytopes}
}\hfill{\scriptsize (operation)}}\\
\textbf{\indent Returns:\ }
a polytope



 Returns the Cartesian product of the polytopes \mbox{\texttt{\mdseries\slshape polytope1}} and \mbox{\texttt{\mdseries\slshape polytope2}}. }

 

\subsection{\textcolor{Chapter }{\#}}
\logpage{[ 6, 4, 2 ]}\nobreak
\hyperdef{L}{X8123456781234567}{}
{\noindent\textcolor{FuncColor}{$\triangleright$\ \ \texttt{\#({\mdseries\slshape polytope1, polytope2})\index{#@\texttt{\#}}
\label{#}
}\hfill{\scriptsize (operation)}}\\
\textbf{\indent Returns:\ }
a polytope



 Returns the Minkowski sum of the polytopes \mbox{\texttt{\mdseries\slshape polytope1}} and \mbox{\texttt{\mdseries\slshape polytope2}}. }

 }

 
\section{\textcolor{Chapter }{Polytope: Constructors}}\label{Polytope:Constructors}
\logpage{[ 6, 5, 0 ]}
\hyperdef{L}{X87A9DA5083C07E1E}{}
{
  

\subsection{\textcolor{Chapter }{Polytope (for lists of points)}}
\logpage{[ 6, 5, 1 ]}\nobreak
\hyperdef{L}{X86B877E378DF5E25}{}
{\noindent\textcolor{FuncColor}{$\triangleright$\ \ \texttt{Polytope({\mdseries\slshape points})\index{Polytope@\texttt{Polytope}!for lists of points}
\label{Polytope:for lists of points}
}\hfill{\scriptsize (operation)}}\\
\textbf{\indent Returns:\ }
a polytope



 Returns a polytope that is the convex hull of the points \mbox{\texttt{\mdseries\slshape points}}. }

 

\subsection{\textcolor{Chapter }{PolytopeByInequalities}}
\logpage{[ 6, 5, 2 ]}\nobreak
\hyperdef{L}{X7E8849CF87B77402}{}
{\noindent\textcolor{FuncColor}{$\triangleright$\ \ \texttt{PolytopeByInequalities({\mdseries\slshape ineqs})\index{PolytopeByInequalities@\texttt{PolytopeByInequalities}}
\label{PolytopeByInequalities}
}\hfill{\scriptsize (operation)}}\\
\textbf{\indent Returns:\ }
a polytope



 Returns a polytope defined by the inequalities \mbox{\texttt{\mdseries\slshape ineqs}}. }

 }

 
\section{\textcolor{Chapter }{Polytope: Examples}}\label{Polytope:Examples}
\logpage{[ 6, 6, 0 ]}
\hyperdef{L}{X854CE5FA7BD81060}{}
{
  
\subsection{\textcolor{Chapter }{Polytope example}}\label{PolytopeExamplePrimary}
\logpage{[ 6, 6, 1 ]}
\hyperdef{L}{X83A852617A774F16}{}
{
  
\begin{Verbatim}[commandchars=!@B,fontsize=\small,frame=single,label=Example]
  !gapprompt@gap>B !gapinput@P := Polytope( [ [ 2, 0 ], [ 0, 2 ], [ -1, -1 ] ] );B
  <A polytope in |R^2>
  !gapprompt@gap>B !gapinput@IsVeryAmple( P );B
  true
  !gapprompt@gap>B !gapinput@LatticePoints( P );B
  [ [ -1, -1 ], [ 0, 0 ], [ 0, 1 ], 
  [ 0, 2 ], [ 1, 0 ], [ 1, 1 ], [ 2, 0 ] ]
  !gapprompt@gap>B !gapinput@NFP := NormalFan( P );B
  <A complete fan in |R^2>
  !gapprompt@gap>B !gapinput@C1 := MaximalCones( NFP )[ 1 ];B
  <A cone in |R^2>
  !gapprompt@gap>B !gapinput@RayGenerators( C1 );B
  [ [ -1, -1 ], [ -1, 3 ] ]
  !gapprompt@gap>B !gapinput@IsRegularFan( NFP );B
  true
\end{Verbatim}
}

 }

  }

 \def\indexname{Index\logpage{[ "Ind", 0, 0 ]}
\hyperdef{L}{X83A0356F839C696F}{}
}

\cleardoublepage
\phantomsection
\addcontentsline{toc}{chapter}{Index}


\printindex

\newpage
\immediate\write\pagenrlog{["End"], \arabic{page}];}
\immediate\closeout\pagenrlog
\end{document}
