% generated by GAPDoc2LaTeX from XML source (Frank Luebeck)
\documentclass[a4paper,11pt]{report}

\usepackage{a4wide}
\sloppy
\pagestyle{myheadings}
\usepackage{amssymb}
\usepackage[latin1]{inputenc}
\usepackage{makeidx}
\makeindex
\usepackage{color}
\definecolor{FireBrick}{rgb}{0.5812,0.0074,0.0083}
\definecolor{RoyalBlue}{rgb}{0.0236,0.0894,0.6179}
\definecolor{RoyalGreen}{rgb}{0.0236,0.6179,0.0894}
\definecolor{RoyalRed}{rgb}{0.6179,0.0236,0.0894}
\definecolor{LightBlue}{rgb}{0.8544,0.9511,1.0000}
\definecolor{Black}{rgb}{0.0,0.0,0.0}

\definecolor{linkColor}{rgb}{0.0,0.0,0.554}
\definecolor{citeColor}{rgb}{0.0,0.0,0.554}
\definecolor{fileColor}{rgb}{0.0,0.0,0.554}
\definecolor{urlColor}{rgb}{0.0,0.0,0.554}
\definecolor{promptColor}{rgb}{0.0,0.0,0.589}
\definecolor{brkpromptColor}{rgb}{0.589,0.0,0.0}
\definecolor{gapinputColor}{rgb}{0.589,0.0,0.0}
\definecolor{gapoutputColor}{rgb}{0.0,0.0,0.0}

%%  for a long time these were red and blue by default,
%%  now black, but keep variables to overwrite
\definecolor{FuncColor}{rgb}{0.0,0.0,0.0}
%% strange name because of pdflatex bug:
\definecolor{Chapter }{rgb}{0.0,0.0,0.0}
\definecolor{DarkOlive}{rgb}{0.1047,0.2412,0.0064}


\usepackage{fancyvrb}

\usepackage{mathptmx,helvet}
\usepackage[T1]{fontenc}
\usepackage{textcomp}


\usepackage[
            pdftex=true,
            bookmarks=true,        
            a4paper=true,
            pdftitle={Written with GAPDoc},
            pdfcreator={LaTeX with hyperref package / GAPDoc},
            colorlinks=true,
            backref=page,
            breaklinks=true,
            linkcolor=linkColor,
            citecolor=citeColor,
            filecolor=fileColor,
            urlcolor=urlColor,
            pdfpagemode={UseNone}, 
           ]{hyperref}

\newcommand{\maintitlesize}{\fontsize{50}{55}\selectfont}

% write page numbers to a .pnr log file for online help
\newwrite\pagenrlog
\immediate\openout\pagenrlog =\jobname.pnr
\immediate\write\pagenrlog{PAGENRS := [}
\newcommand{\logpage}[1]{\protect\write\pagenrlog{#1, \thepage,}}
%% were never documented, give conflicts with some additional packages

\newcommand{\GAP}{\textsf{GAP}}

%% nicer description environments, allows long labels
\usepackage{enumitem}
\setdescription{style=nextline}

%% depth of toc
\setcounter{tocdepth}{1}





%% command for ColorPrompt style examples
\newcommand{\gapprompt}[1]{\color{promptColor}{\bfseries #1}}
\newcommand{\gapbrkprompt}[1]{\color{brkpromptColor}{\bfseries #1}}
\newcommand{\gapinput}[1]{\color{gapinputColor}{#1}}


\begin{document}

\logpage{[ 0, 0, 0 ]}
\begin{titlepage}
\mbox{}\vfill

\begin{center}{\maintitlesize \textbf{\textsf{SpinSym}\index{SpinSym package}\mbox{}}}\\
\vfill

\hypersetup{pdftitle=\textsf{SpinSym}\index{SpinSym package}}
\markright{\scriptsize \mbox{}\hfill \textsf{SpinSym}\index{SpinSym package} \hfill\mbox{}}
{\Huge \textbf{Brauer tables of spin-symmetric groups\mbox{}}}\\
\vfill

{\Huge Version 1.5 (2013)\mbox{}}\\[1cm]
{\mbox{}}\\[1cm]
\mbox{}\\[2cm]
{\Large \textbf{Lukas Maas \mbox{}}}\\
\hypersetup{pdfauthor=Lukas Maas }
\end{center}\vfill

\mbox{}\\
{\mbox{}\\
\small \noindent \textbf{Lukas Maas }  Email: \href{mailto://lukas.maas@iem.uni-due.de} {\texttt{lukas.maas@iem.uni-due.de}}}\\
\end{titlepage}

\newpage\setcounter{page}{2}
{\small 
\section*{Copyright}
\logpage{[ 0, 0, 1 ]}
 {\copyright} 2012 Lukas Maas 

 The \textsf{SpinSym} \textsf{GAP} package is free software: you can redistribute it and/or modify it under the
terms of the GNU General Public License as published by the Free Software
Foundation, either version 2 of the License, or (at your option) any later
version.

 This program is distributed in the hope that it will be useful, but WITHOUT
ANY WARRANTY; without even the implied warranty of MERCHANTABILITY or FITNESS
FOR A PARTICULAR PURPOSE. See the GNU General Public License for more details.
You should have received a copy of the GNU General Public License along with
this program (see the file COPYING). If not, see \href{http://www.gnu.org/licenses/} {\texttt{http://www.gnu.org/licenses/}}. \mbox{}}\\[1cm]
\newpage

\def\contentsname{Contents\logpage{[ 0, 0, 2 ]}}

\tableofcontents
\newpage

  
\chapter{\textcolor{Chapter }{Introduction}}\label{chap1}
\logpage{[ 1, 0, 0 ]}
\hyperdef{L}{X7DFB63A97E67C0A1}{}
{
  The purpose of this \textsf{GAP} package is to make a collection of $p$-modular character tables (Brauer tables) of spin-symmetric groups (and some
related groups) available in \textsf{GAP}, thereby extending Thomas Breuer's \textsf{GAP} Character Table Library \cite{ctbllib}. The \textsf{SpinSym} package is based on \cite{Maas2011} which serves as the general reference here. If you are interested in computing
with \textsf{SpinSym} I would like to refer you to \cite{Maas2011} for further references and a more thorough description of some of the topics
below. And, of course, I would like to hear from you about more or less
successful attempts in using the present functionalities.

 The term `spin-symmetric' refers to the groups  
\[ \tilde{S}_n=\langle\ z,t_1,\ldots,t_{n-1}\ :\ z^2=1,\ t_i^2=(t_it_{i+1})^3=z,\
(t_jt_k)^2=z\ \rangle \]
   and  
\[\hat{S}_n=\langle\ z,t_1,\ldots,t_{n-1}\ :\ z^2=1,\ t_i^2=(t_it_{i+1})^3=1,\
(t_jt_k)^2=z,\ zt_i=t_iz\ \rangle \]
   where the relations are imposed for all admissable $i,j,k$ with $|j-k|>1$. Provided $n\geq 4$, these groups are double covers of the symmetric group $S_n$  on $n$ letters. Although  $\tilde{S}_n$ and  $\hat{S}_n$ are non-isomorphic groups for $n\neq 6$, they are isoclinic and their representation theory is very similar. By \emph{choice}, we restrict the attention to  $\tilde{S}_n$. (However, if you are interested in character tables of  $\hat{S}_n$ then have a look at \texttt{CharacterTableIsoclinic()} in the \textsf{GAP} Reference Manual.) 

 The natural epimorphism  $\pi: \tilde{S}_n\to S_n,\ t_i\mapsto (i,i+1)$, whose kernel is generated by the central involution $z$, gives rise to the double cover  $\tilde{A}_n= A_n^{\pi^{-1}}$ of the alternating group $A_n$  as the preimage of $A_n$  under $\pi$. Irreducible faithful representations of  $\tilde{S}_n$ or  $\tilde{A}_n$ are called spin representations and a similar `spin' terminology is used for
all related faithful objects, to set them apart from the non-faithful objects
that belong esssentially to $S_n$  or $A_n$ , respectively. 
\section{\textcolor{Chapter }{The data part}}\label{chap1:The data part}
\logpage{[ 1, 1, 0 ]}
\hyperdef{L}{X7DD5FD4986099D5E}{}
{
  The package contains complete Brauer tables of  $\tilde{S}_n$ and  $\tilde{A}_n$ up to degree $n=18$ in characteristic $p=3,5,7$. Thus it includes the corresponding Brauer tables of $S_n$  and $A_n$ . Moreover, Brauer tables of $S_n$  and $A_n$  up to degree $n=19$ in characteristic $p=2$ are part of the package too. 

 Every Brauer table comes with lists of character parameters (row labels) and
class parameters (column labels), see \ref{chap2:Character parameters} and \ref{chap2:Class parameters}. I would like to mention that only some of the data is `new', large portions
date back to the work of James, Morris, Yaseen, and the Modular Atlas Project.
Detailed references are to be found in \cite{Maas2011}. The $2$-modular tables of $S_n$  and $A_n$  for $n=18,19$ were computed jointly by J{\"u}rgen M{\"u}ller and the author. 

 Please note that some of our Brauer tables differ to some extent from those
contained in the \textsf{GAP} Character Table Library \cite{ctbllib} (for example, in terms of the ordering of conjugacy classes and characters or
in terms of their parameters). Therefore it seemed appropriate to collect
these tables in their own package - so here we are. 

 I'm grateful to Thomas Breuer for supporting the idea of writing this package
and for converting my tables into the right \textsf{GAP} Character Table Library format. }

 
\section{\textcolor{Chapter }{The functions part}}\label{chap1:The functions part}
\logpage{[ 1, 2, 0 ]}
\hyperdef{L}{X844B24F878D50F92}{}
{
  Besides Brauer tables, the package provides some related functionalities such
as functions that determine class fusions of subgroup character tables and
functions that compute character tables of some Young subgroups of  $\tilde{S}_n$. }

 
\section{\textcolor{Chapter }{Installation and loading}}\label{chap1:Installation and loading}
\logpage{[ 1, 3, 0 ]}
\hyperdef{L}{X86C437C47F3988E8}{}
{
  To install this package, download the archive file \texttt{spinsym-1.5.tar.gz} and unpack it inside the \texttt{pkg} subdirectory of your \textsf{GAP} installation. It creates a subdirectory called \texttt{spinsym}. Then load the package using the \texttt{LoadPackage} command. 
\begin{Verbatim}[commandchars=!@|,fontsize=\small,frame=single,label=Example]
  
  !gapprompt@gap>| !gapinput@LoadPackage("spinsym");|
  
\end{Verbatim}
 The \textsf{SpinSym} package banner should appear on the screen. You may want to run a quick test
of the installation: 
\begin{Verbatim}[commandchars=!@|,fontsize=\small,frame=single,label=Example]
  
  !gapprompt@gap>| !gapinput@dir:= DirectoriesPackageLibrary( "spinsym", "tst" )[1];;|
  !gapprompt@gap>| !gapinput@tst:= Filename( dir, "testall.tst" );;|
  !gapprompt@gap>| !gapinput@Test( tst );|
  true
  
\end{Verbatim}
 }

 }

  
\chapter{\textcolor{Chapter }{Usage and features}}\label{chap2}
\logpage{[ 2, 0, 0 ]}
\hyperdef{L}{X8243772F820CC34E}{}
{
  
\section{\textcolor{Chapter }{Accessing the tables}}\label{chap2:Accessing the tables}
\logpage{[ 2, 1, 0 ]}
\hyperdef{L}{X7B57D3B284BE53F9}{}
{
  All Brauer tables in this package are relative to a \emph{generic} ordinary character table obtained by one of the following constructions 
\begin{description}
\item[{}] \texttt{CharacterTable( "2.Sym(n)" )}, the character table of  $\tilde{S}_n$,
\item[{}] \texttt{CharacterTable( "2.Alt(n)" )}, the character table of  $\tilde{A}_n$,
\item[{}] \texttt{CharacterTable( "Sym(n)" )}, the character table of $S_n$,
\item[{}] \texttt{CharacterTable( "Alt(n)" )}, the character table of $A_n$.
\end{description}
 Note that these are synonymous expressions for 
\begin{description}
\item[{}] \texttt{CharacterTable( "DoubleCoverSymmetric", n )},
\item[{}] \texttt{CharacterTable( "DoubleCoverAlternating", n )},
\item[{}] \texttt{CharacterTable( "Symmetric", n )},
\item[{}] \texttt{CharacterTable( "Alternating", n )},
\end{description}
 respectively. More detailed information on these tables is to be found in \cite{Noeske2002}. In this manual, we call such a character table an (ordinary) \emph{SpinSym table}. If \texttt{ordtbl} is an ordinary SpinSym table, the relative Brauer table in characteristic \texttt{p} can be accessed using the \texttt{mod}-operator (i.e. \texttt{ordtbl mod p;}). Such a Brauer table is called a ($p$-modular) \emph{SpinSym table} in the following. 
\begin{Verbatim}[commandchars=!@|,fontsize=\small,frame=single,label=Example]
  
  !gapprompt@gap>| !gapinput@ordtbl:= CharacterTable( "2.Sym(18)" );|
  CharacterTable( "2.Sym(18)" )
  !gapprompt@gap>| !gapinput@modtbl:= ordtbl mod 3;|
  BrauerTable( "2.Sym(18)", 3 )
  !gapprompt@gap>| !gapinput@OrdinaryCharacterTable(modtbl)=ordtbl;|
  true
  
\end{Verbatim}
 }

 
\section{\textcolor{Chapter }{Character parameters}}\label{chap2:Character parameters}
\logpage{[ 2, 2, 0 ]}
\hyperdef{L}{X7FFBFA7E7E31EA14}{}
{
  An ordinary SpinSym table has character parameters, that is, a list of
suitable labels corresponding to the rows of \texttt{ordtbl} and therefore the irreducible ordinary characters of the underlying group. See \texttt{CharacterParameters()} in the GAP Reference Manual. 
\subsection{\textcolor{Chapter }{Parameters of ordinary characters}}\label{subsec:characterparameters:ordinary}
\logpage{[ 2, 2, 1 ]}
\hyperdef{L}{X87BE0EFA878CA593}{}
{
  In the following, `ordinary (spin) character' is used synonymously for
`irreducible ordinary (spin) character'. It is well known that there is a
bijection between the set of ordinary characters of $S_n$  and the set $P(n)$ of all partitions of $n$. Recall that a partition of a natural number $n$ is a list of non-increasing positive integers (its \emph{parts}) that sum up to $n$. In this way, every ordinary character $\chi$ of $S_n$  has a label of the form \texttt{[1,c]} where \texttt{c} is a partition of $n$. The labels of the ordinary characters of $A_n$  are induced by Clifford theory as follows. Either the restriction $\psi=\chi|_{A_n}$  of $\chi$ to $A_n$  is an ordinary character of $A_n$, or $\psi$ decomposes as the sum of two distinct ordinary characters $\psi_1$ and $\psi_2$. 

 In the first case there is another ordinary character of $S_n$, say $\xi$ labelled by \texttt{[1,d]}, such that the restriction of $\xi$ to $A_n$  is equal to $\psi$. Moreover, the induced character of $S_n$  obtained from $\psi$ decomposes as the sum of $\chi$ and $\xi$. Then $\psi$ is labelled by \texttt{[1,c]} or \texttt{[1,d]}.

 In the second case, both $\psi_1$ and $\psi_2$ induce irreducibly up to $\chi$. Then $\psi_1$ and $\psi_2$ are labelled by \texttt{[1,[c,'+']]} and \texttt{[1,[c,'-']]}.

 
\begin{Verbatim}[commandchars=!@|,fontsize=\small,frame=single,label=Example]
  
  !gapprompt@gap>| !gapinput@ctS:= CharacterTable( "Sym(5)" );;|
  !gapprompt@gap>| !gapinput@CharacterParameters(ctS);|
  [ [ 1, [ 1, 1, 1, 1, 1 ] ], [ 1, [ 2, 1, 1, 1 ] ], [ 1, [ 2, 2, 1 ] ], 
    [ 1, [ 3, 1, 1 ] ], [ 1, [ 3, 2 ] ], [ 1, [ 4, 1 ] ], [ 1, [ 5 ] ] ]
  !gapprompt@gap>| !gapinput@ctA:= CharacterTable( "Alt(5)" );;|
  !gapprompt@gap>| !gapinput@CharacterParameters(ctA);|
  [ [ 1, [ 1, 1, 1, 1, 1 ] ], [ 1, [ 2, 1, 1, 1 ] ], [ 1, [ 2, 2, 1 ] ], 
    [ 1, [ [ 3, 1, 1 ], '+' ] ], [ 1, [ [ 3, 1, 1 ], '-' ] ] ]
  !gapprompt@gap>| !gapinput@chi:= Irr(ctS)[1];;|
  !gapprompt@gap>| !gapinput@psi:= RestrictedClassFunction(chi,ctA);;           |
  !gapprompt@gap>| !gapinput@Position(Irr(ctA),psi);    |
  1
  !gapprompt@gap>| !gapinput@xi:= Irr(ctS)[7];;|
  !gapprompt@gap>| !gapinput@RestrictedClassFunction(xi,ctA) = psi;|
  true
  !gapprompt@gap>| !gapinput@InducedClassFunction(psi,ctS) = chi + xi;|
  true
  !gapprompt@gap>| !gapinput@chi:= Irr(ctS)[4];;|
  !gapprompt@gap>| !gapinput@psi:= RestrictedClassFunction(chi,ctA);;|
  !gapprompt@gap>| !gapinput@psi1:= Irr(ctA)[4];; psi2:= Irr(ctA)[5];;|
  !gapprompt@gap>| !gapinput@psi = psi1 + psi2;|
  true
  !gapprompt@gap>| !gapinput@InducedClassFunction(psi1,ctS) = chi;              |
  true
  !gapprompt@gap>| !gapinput@InducedClassFunction(psi2,ctS) = chi;|
  true
  
\end{Verbatim}
 If $\chi$ is an ordinary character of  $\tilde{S}_n$ or  $\tilde{A}_n$, then $\chi(z)=\chi(1)$ or $\chi(z)=-\chi(1)$. If $\chi(z)=\chi(1)$, then $\chi$ is obtained by inflation (along the central subgroup generated by $z$) from an ordinary character of $S_n$ or $A_n$, respectively, whose label is given to $\chi$. Otherwise, if $\chi$ is a spin character, that is $\chi(z)=-\chi(1)$, then its label is described next. 

 The set of ordinary spin characters of  $\tilde{S}_n$ is parameterized by the subset $D(n)$ of $P(n)$ of all distinct-parts partitions of $n$ (also called bar partitions). If \texttt{c} is an even distinct-parts partition of $n$, then there is a unique ordinary spin character of  $\tilde{S}_n$ that is labelled by \texttt{[2,c]}. In contrast, if \texttt{c} is an odd distinct-parts partition of $n$, then there are two distinct ordinary spin characters of  $\tilde{S}_n$ that are labelled by \texttt{[2,[c,'+']]} and \texttt{[2,[c,'-']]}. Now the labels of the ordinary spin characters of  $\tilde{A}_n$ follow from the labels of  $\tilde{S}_n$ in the same way as those of $A_n$  follow from the labels of $S_n$  (see the beginning of this subsection \ref{subsec:characterparameters:ordinary}). 
\begin{Verbatim}[commandchars=!@|,fontsize=\small,frame=single,label=Example]
  
  !gapprompt@gap>| !gapinput@ctS:= CharacterTable( "Sym(5)" );;|
  !gapprompt@gap>| !gapinput@ct2S:= CharacterTable( "2.Sym(5)" );;|
  !gapprompt@gap>| !gapinput@ch:= CharacterParameters(ct2S);|
  [ [ 1, [ 1, 1, 1, 1, 1 ] ], [ 1, [ 2, 1, 1, 1 ] ], [ 1, [ 2, 2, 1 ] ], 
    [ 1, [ 3, 1, 1 ] ], [ 1, [ 3, 2 ] ], [ 1, [ 4, 1 ] ], [ 1, [ 5 ] ], 
    [ 2, [ [ 3, 2 ], '+' ] ], [ 2, [ [ 3, 2 ], '-' ] ], 
    [ 2, [ [ 4, 1 ], '+' ] ], [ 2, [ [ 4, 1 ], '-' ] ], [ 2, [ 5 ] ] ]
  !gapprompt@gap>| !gapinput@pos:= Positions( List(ch, x-> x[1]), 1 );;|
  !gapprompt@gap>| !gapinput@RestrictedClassFunctions( Irr(ctS), ct2S ) = Irr(ct2S){pos}; #inflation|
  true
  !gapprompt@gap>| !gapinput@ct2A:= CharacterTable( "2.Alt(5)" );;|
  !gapprompt@gap>| !gapinput@CharacterParameters(ct2A);|
  [ [ 1, [ 1, 1, 1, 1, 1 ] ], [ 1, [ 2, 1, 1, 1 ] ], [ 1, [ 2, 2, 1 ] ], 
    [ 1, [ [ 3, 1, 1 ], '+' ] ], [ 1, [ [ 3, 1, 1 ], '-' ] ], [ 2, [ 3, 2 ] ], 
    [ 2, [ 4, 1 ] ], [ 2, [ [ 5 ], '+' ] ], [ 2, [ [ 5 ], '-' ] ] ]
  
\end{Verbatim}
 }

 
\subsection{\textcolor{Chapter }{Parameters of modular characters}}\label{subsec:characterparameters:modular}
\logpage{[ 2, 2, 2 ]}
\hyperdef{L}{X85178CBC7E7039FF}{}
{
  In the following, `$p$-modular (spin) character' is used synonymously for `irreducible $p$-modular (spin) character'. The set of $p$-modular characters of $S_n$  is parameterized by the set of all $p$-regular partitions of $n$. A partition is $p$-regular if no part is repeated more than $p-1$ times. Now every $p$-modular character $\chi$ of $S_n$  has a label of the form \texttt{[1,c]} where \texttt{c} is a $p$-regular partition of $n$. 

 Again, the labels for the $p$-modular spin characters of $A_n$  follow from the labels of $S_n$. However, comparing subsection \ref{subsec:characterparameters:ordinary}, their format is slightly different. 

 If $\chi$ and $\xi$ are distinct $p$-modular characters of $S_n$  that restrict to the same $p$-modular character $\psi$ of $A_n$ , then $\psi$ is labelled by \texttt{[1,[c,'0']]} where either $\chi$ or $\xi$ is labelled by \texttt{[1,c]}. If $\chi$ is a $p$-modular character of $S_n$ whose restriction to $A_n$ decomposes as the sum of two distinct $p$-modular characters, then these are labelled by \texttt{[1,[c,'+']]} and \texttt{[1,[c,'-']]} where $\chi$ is labelled by \texttt{[1,c]}. 

 As in the ordinary case, the set of $p$-modular characters of  $\tilde{S}_n$ is the union of the subset consisting of all inflated $p$-modular characters of $S_n$  and the subset of spin characters characterized by negative integer values on
the central element $z$. The analogue statement holds for $\tilde{A}_n$. The set of $p$-modular spin characters of  $\tilde{S}_n$ is parameterized by the set of all restricted $p$-strict partitions of $n$. A partition is called $p$-strict if every repeated part is divisible by $p$, and a $p$-strict partition $\lambda$ is restricted if $\lambda_i-\lambda_{i+1}<p$ whenever $\lambda_i$ is divisible $p$, and $\lambda_i-\lambda_{i+1}\leq p$ otherwise for all parts $\lambda_i$ of $\lambda$ (where we set $\lambda_{i+1}=0$ if $\lambda_i$ is the last part). If \texttt{c} is a restricted $p$-strict partition of $n$ such that $n$ minus the number of parts not divisible by $p$ is even, then there is a unique $p$-modular spin character of  $\tilde{S}_n$ that is labelled by \texttt{[2,[c,'0']]}. Its restriction to  $\tilde{A}_n$ decomposes as the sum of two distinct $p$-modular characters which are labelled by \texttt{[2,[c,'+']]} and \texttt{[2,[c,'-']]}. If $n$ minus the number of parts of \texttt{c} that are not divisible by $p$ is odd, then there are two distinct $p$-modular spin characters of  $\tilde{S}_n$ that are labelled by \texttt{[2,[c,'+']]} and \texttt{[2,[c,'-']]}. Both of these characters restrict to the same irreducible $p$-modular spin character of  $\tilde{A}_n$ which is labelled by \texttt{[2,[c,'0']]}. 
\begin{Verbatim}[commandchars=!@|,fontsize=\small,frame=single,label=Example]
  
  !gapprompt@gap>| !gapinput@ctS:= CharacterTable( "Sym(5)" ) mod 3;;|
  !gapprompt@gap>| !gapinput@ct2S:= CharacterTable( "2.Sym(5)" ) mod 3;;|
  !gapprompt@gap>| !gapinput@ch:= CharacterParameters(ct2S);|
  [ [ 1, [ 5 ] ], [ 1, [ 4, 1 ] ], [ 1, [ 3, 2 ] ], 
    [ 1, [ 3, 1, 1 ] ], [ 1, [ 2, 2, 1 ] ], 
    [ 2, [ [ 4, 1 ], '+' ] ], [ 2, [ [ 4, 1 ], '-' ] ], 
    [ 2, [ [ 3, 2 ], '0' ] ] ]
  !gapprompt@gap>| !gapinput@pos:= Positions( List(ch, x-> x[1]), 1 );;|
  !gapprompt@gap>| !gapinput@RestrictedClassFunctions( Irr(ctS), ct2S ) = Irr(ct2S){pos}; #inflation|
  true
  !gapprompt@gap>| !gapinput@ct2A:= CharacterTable( "2.Alt(5)" ) mod 3;;|
  !gapprompt@gap>| !gapinput@CharacterParameters(ct2A);|
  [ [ 1, [ [ 5 ], '0' ] ], [ 1, [ [ 4, 1 ], '0' ] ], 
    [ 1, [ [ 3, 1, 1 ], '+' ] ], [ 1, [ [ 3, 1, 1 ], '-' ] ], 
    [ 2, [ [ 4, 1 ], '0' ] ], [ 2, [ [ 3, 2 ], '+' ] ], [ 2, [ [ 3, 2 ], '-' ] ] ]
  
\end{Verbatim}
 }

 }

 
\section{\textcolor{Chapter }{Class parameters}}\label{chap2:Class parameters}
\logpage{[ 2, 3, 0 ]}
\hyperdef{L}{X857C89397E32A4E1}{}
{
  Let \texttt{ct} be an ordinary SpinSym table. Then \texttt{ct} has a list of class parameters, that is, a list of suitable labels
corresponding to the columns of \texttt{ct} and therefore the conjugacy classes of the underlying group. See \texttt{ClassParameters()} in the GAP Reference Manual. If \texttt{bt} is a Brauer table in characteristic $p$ relative to \texttt{ct}, its class parameters are inherited from \texttt{ct} in correspondence with the $p$-regular conjugacy classes of the underlying group.

 Let $P(n)$ denote the set of partitions of $n$.

 The conjugacy classes of $S_n$  are naturally parameterized by the cycle types of their elements, and each
cycle type corresponds to a partition of $n$. Therefore a conjugacy class $C$ of $S_n$  is characterized by its \emph{type} $c\in P(n)$. The corresponding entry in the list of class parameters is \texttt{[1,c]}. Assume that $C\subset A_n$ . Then $C$ is also a conjugacy class of $A_n$  if and only if not all parts of $c$ are odd and pairwise distinct. Otherwise, $C$ splits as the union of two distinct $A_n$ -classes of the same size, $C^+$ of type $c^+$ and $C^-$ of type $c^-$. The corresponding entries in the list of class parameters are \texttt{[1,[c,'+']]} and \texttt{[1,[c,'-']]}, respectively.

 Furthermore,  $\tilde{C}=C^{\pi^{-1}}\subset\tilde{S}_n$ is either a conjugacy class of  $\tilde{S}_n$ of type $c$ with class parameter \texttt{[1,c]}, or $\tilde{C}$ splits as the union of two distinct  $\tilde{S}_n$-classes  $\tilde{C}_1$ and  $\tilde{C}_2=z\tilde{C}_1$, both of type $c$ with corresponding class parameters \texttt{[1,c]} and \texttt{[2,c]}, respectively. An analogous description applies for the conjugacy classes of  $\tilde{A}_n$. 
\begin{Verbatim}[commandchars=!@|,fontsize=\small,frame=single,label=Example]
  
  !gapprompt@gap>| !gapinput@ct:= CharacterTable( "Sym(3)" );;  |
  !gapprompt@gap>| !gapinput@ClassParameters(ct);|
  [ [ 1, [ 1, 1, 1 ] ], [ 1, [ 2, 1 ] ], [ 1, [ 3 ] ] ]
  !gapprompt@gap>| !gapinput@ct:= CharacterTable( "Alt(3)" );;  |
  !gapprompt@gap>| !gapinput@ClassParameters(ct);|
  [ [ 1, [ 1, 1, 1 ] ], [ 1, [ [ 3 ], '+' ] ], [ 1, [ [ 3 ], '-' ] ] ]
  !gapprompt@gap>| !gapinput@ct:= CharacterTable( "2.Sym(3)" );;|
  !gapprompt@gap>| !gapinput@ClassParameters(ct);|
  [ [ 1, [ 1, 1, 1 ] ], [ 2, [ 1, 1, 1 ] ], [ 1, [ 2, 1 ] ], [ 2, [ 2, 1 ] ], 
    [ 1, [ 3 ] ], [ 2, [ 3 ] ] ]
  !gapprompt@gap>| !gapinput@ct:= CharacterTable( "2.Alt(3)" );;|
  !gapprompt@gap>| !gapinput@ClassParameters(ct);|
  [ [ 1, [ 1, 1, 1 ] ], [ 2, [ 1, 1, 1 ] ], 
    [ 1, [ [ 3 ], '+' ] ], [ 2, [ [ 3 ], '+' ] ], 
    [ 1, [ [ 3 ], '-' ] ], [ 2, [ [ 3 ], '-' ] ] ]
  
\end{Verbatim}
 To each conjugacy class of  $\tilde{S}_n$ or  $\tilde{A}_n$ a certain standard representative is assigned in the following way. Let $c=[c_1,c_2,\ldots,c_m]$ be a partition of $n$. We set $d_1=0$, $d_i=c_1+\ldots +c_{i-1}$ for $i\geq 2$, and 
\[t(c_i,d_i)= t_{d_i+1}t_{d_i+2}\ldots t_{d_i+c_i-1}\]
 for $1\leq i\leq m-1$, where $t(c_i,d_i)= 1$ if $c_i=1$. The \emph{standard representative of type} $c$ is defined as 
\[t_c=t(c_1,d_1)t(c_2,d_2)\cdots t(c_{m-1},d_{m-1}).\]
  Furthermore, we define the standard representatives of type $c^+=$\texttt{[c,'+']} and $c^-=$\texttt{[c,'-']} to be $t_{c^+}=t_{c}$ and $t_{c^-}=t_1^{-1}t_c t_1 $, respectively. 

 For example, the standard representative of type $c=[7,4,3,1]\in P(15)$  is 
\[t_c=t_1t_2t_3t_4t_5t_6t_8t_9t_{10}t_{12}t_{13}.\]
 Now  $\tilde{C}$ is a conjugacy class of  $\tilde{S}_n$ or  $\tilde{A}_n$ with parameter 
\begin{description}
\item[{}] \texttt{[1,c]} if and only if  $t_c\in\tilde{C}$,
\item[{}] \texttt{[2,c]} if and only if  $zt_c\in\tilde{C}$,
\item[{}] \texttt{[1,[c,'+']]} if and only if  $t_{c^+}\in\tilde{C}$,
\item[{}] \texttt{[2,[c,'+']]} if and only if  $zt_{c^+}\in\tilde{C}$,
\item[{}] \texttt{[1,[c,'-']]} if and only if  $t_{c^-}\in\tilde{C}$,
\item[{}] \texttt{[2,[c,'-']]} if and only if  $zt_{c^-}\in\tilde{C}$.
\end{description}
 

\subsection{\textcolor{Chapter }{SpinSymStandardRepresentative}}
\logpage{[ 2, 3, 1 ]}\nobreak
\hyperdef{L}{X7B846F3782B35EF3}{}
{\noindent\textcolor{FuncColor}{$\triangleright$\ \ \texttt{SpinSymStandardRepresentative({\mdseries\slshape c, rep})\index{SpinSymStandardRepresentative@\texttt{SpinSymStandardRepresentative}}
\label{SpinSymStandardRepresentative}
}\hfill{\scriptsize (function)}}\\
\textbf{\indent Returns:\ }
the image of the standard representative of type \mbox{\texttt{\mdseries\slshape c}} under a given $\tilde{S}_n$ -representation. 



 Expecting the second entry of a class parameter of  $\tilde{S}_n$ or  $\tilde{A}_n$, say \mbox{\texttt{\mdseries\slshape c}}, the standard representative of type \mbox{\texttt{\mdseries\slshape c}} under a given representation of  $\tilde{S}_n$ is computed. The argument \mbox{\texttt{\mdseries\slshape rep}} is assumed to be a list $[t_1^R,t_2^R,\ldots,t_{n-1}^R]$ given by the images of the generators $t_1,\ldots,t_{n-1}$ of  $\tilde{S}_n$ under a (not necessarily faithful) representation $R$ of  $\tilde{S}_n$. }

 
\begin{Verbatim}[commandchars=!@|,fontsize=\small,frame=single,label=Example]
  
  !gapprompt@gap>| !gapinput@ct:= CharacterTable("2.Sym(15)") mod 5;;|
  !gapprompt@gap>| !gapinput@cl:= ClassParameters(ct)[99];|
  [ 1, [ 7, 4, 3, 1 ] ]
  !gapprompt@gap>| !gapinput@c:= cl[2];;|
  !gapprompt@gap>| !gapinput@rep:= BasicSpinRepresentationOfSymmetricGroup(15,5);;|
  !gapprompt@gap>| !gapinput@t:= SpinSymStandardRepresentative(c,rep); |
  < immutable compressed matrix 64x64 over GF(25) >
  !gapprompt@gap>| !gapinput@OrdersClassRepresentatives(ct)[99];|
  168
  !gapprompt@gap>| !gapinput@Order(t);|
  168
  !gapprompt@gap>| !gapinput@BrauerCharacterValue(t);|
  0
  
\end{Verbatim}
 

\subsection{\textcolor{Chapter }{SpinSymStandardRepresentativeImage}}
\logpage{[ 2, 3, 2 ]}\nobreak
\hyperdef{L}{X78925C167ABA9462}{}
{\noindent\textcolor{FuncColor}{$\triangleright$\ \ \texttt{SpinSymStandardRepresentativeImage({\mdseries\slshape c[, j]})\index{SpinSymStandardRepresentativeImage@\texttt{SpinSymStandardRepresentativeImage}}
\label{SpinSymStandardRepresentativeImage}
}\hfill{\scriptsize (function)}}\\
\textbf{\indent Returns:\ }
the image of the standard representative of type \mbox{\texttt{\mdseries\slshape c}} under the natural epimorphism $\pi:\tilde{S}_{\{j,\ldots,j+n-1\}}\to S_{\{j,\ldots,j+n-1\}}$ .



 Given the second entry \mbox{\texttt{\mdseries\slshape c}} of a class parameter of  $\tilde{S}_n$ or  $\tilde{A}_n$, and optionally a positive integer \mbox{\texttt{\mdseries\slshape j}}, the image of the standard representative of type \mbox{\texttt{\mdseries\slshape c}} under $\pi:\tilde{S}_{\{j,\ldots,j+n-1\}}\to S_{\{j,\ldots,j+n-1\}}$  with $t_i^\pi=(i,i+1)$ for $j\leq i\leq j+n-2$ is computed by calling \texttt{SpinSymStandardRepresentative(c,rep)} where \texttt{rep} is the list \texttt{[(j,j+1),(j+1,j+2),...,(j+n-2,j+n-1)]}. By default, \texttt{j=1}. }

 
\begin{Verbatim}[commandchars=!@|,fontsize=\small,frame=single,label=Example]
  
  !gapprompt@gap>| !gapinput@s1:= SpinSymStandardRepresentativeImage([7,4,3,1]);         |
  (1,7,6,5,4,3,2)(8,11,10,9)(12,14,13)
  !gapprompt@gap>| !gapinput@s2:= SpinSymStandardRepresentativeImage([[7,4,3,1],'-']);|
  (1,2,7,6,5,4,3)(8,11,10,9)(12,14,13)
  !gapprompt@gap>| !gapinput@s2 = s1^(1,2);|
  true
  !gapprompt@gap>| !gapinput@SpinSymStandardRepresentativeImage([7,4,3,1],3);       |
  (3,9,8,7,6,5,4)(10,13,12,11)(14,16,15)
  
\end{Verbatim}
 

\subsection{\textcolor{Chapter }{SpinSymPreimage}}
\logpage{[ 2, 3, 3 ]}\nobreak
\hyperdef{L}{X7C885CF87A06E0DF}{}
{\noindent\textcolor{FuncColor}{$\triangleright$\ \ \texttt{SpinSymPreimage({\mdseries\slshape c, rep})\index{SpinSymPreimage@\texttt{SpinSymPreimage}}
\label{SpinSymPreimage}
}\hfill{\scriptsize (function)}}\\
\textbf{\indent Returns:\ }
a (standard) lift of the element \mbox{\texttt{\mdseries\slshape c}} of  $S_n$ in  $\tilde{S}_n$ under a given $\tilde{S}_n$ -representation. 



See \cite[(5.1.12)]{Maas2011} for the definition of the lift that is returned by this function. The
permutation \mbox{\texttt{\mdseries\slshape c}} is written as a product of simple transpositions $(i,i+1)$, then these are replaced by the images of their canonical lifts $t_i$ under a given representation $R$ of $\tilde{S}_n$  (recall the beginning of Chapter \ref{chap1} for the definition of $t_i$). Here \mbox{\texttt{\mdseries\slshape rep}} is assumed to be the list $[t_1^R,t_2^R,\ldots,t_{n-1}^R]$. 

 Note that a more efficient computation may be achieved by computing and
storing a list of all necessary transpositions once and for all, before
lifting (many) elements (under a possibly large representation). }

 
\begin{Verbatim}[commandchars=!@|,fontsize=\small,frame=single,label=Example]
  
  !gapprompt@gap>| !gapinput@rep:= BasicSpinRepresentationOfSymmetricGroup(15);;|
  !gapprompt@gap>| !gapinput@c:= SpinSymStandardRepresentativeImage([5,4,3,2,1]);|
  (1,5,4,3,2)(6,9,8,7)(10,12,11)(13,14)
  !gapprompt@gap>| !gapinput@C:= SpinSymPreimage(c,rep);|
  < immutable compressed matrix 64x64 over GF(9) >
  !gapprompt@gap>| !gapinput@C = SpinSymStandardRepresentative([5,4,3,2,1],rep);|
  true
  
\end{Verbatim}
 

\subsection{\textcolor{Chapter }{SpinSymBrauerCharacter}}
\logpage{[ 2, 3, 4 ]}\nobreak
\hyperdef{L}{X8243866286DA609C}{}
{\noindent\textcolor{FuncColor}{$\triangleright$\ \ \texttt{SpinSymBrauerCharacter({\mdseries\slshape ccl, ords, rep})\index{SpinSymBrauerCharacter@\texttt{SpinSymBrauerCharacter}}
\label{SpinSymBrauerCharacter}
}\hfill{\scriptsize (function)}}\\
\textbf{\indent Returns:\ }
the Brauer character afforded by a given representation of  $\tilde{S}_n$. 



 This function is based on a simplified computation of the \textsf{GAP} attribute \texttt{BrauerCharacterValue(mat)} for an invertible matrix \texttt{mat} over a finite field whose characteristic is coprime to the order of \texttt{mat}. 

 The arguments \mbox{\texttt{\mdseries\slshape ccl}} and \mbox{\texttt{\mdseries\slshape ords}} are expected to be the values of the attributes \texttt{ClassParameters(modtbl)} and \texttt{OrdersClassRepresentatives(modtbl)} of a (possibly incomplete) $p$-modular SpinSym table \texttt{modtbl} of  $\tilde{S}_n$.

 The argument \mbox{\texttt{\mdseries\slshape rep}} is assumed to be a list $[t_1^R,t_2^R,\ldots,t_{n-1}^R]$ given by the images of the generators $t_1,\ldots,t_{n-1}$ of  $\tilde{S}_n$ under a (not necessarily faithful)  $\tilde{S}_n$-representation $R$. }

 
\begin{Verbatim}[commandchars=!@|,fontsize=\small,frame=single,label=Example]
  
  !gapprompt@gap>| !gapinput@ct:= CharacterTable("DoubleCoverSymmetric",15);;|
  !gapprompt@gap>| !gapinput@bt:= CharacterTableRegular(ct,5);;|
  !gapprompt@gap>| !gapinput@fus:= GetFusionMap(bt,ct);;|
  !gapprompt@gap>| !gapinput@ccl:= ClassParameters(ct){fus};;|
  !gapprompt@gap>| !gapinput@ords:= OrdersClassRepresentatives(bt);;|
  !gapprompt@gap>| !gapinput@rep:= BasicSpinRepresentationOfSymmetricGroup(15,5);;|
  !gapprompt@gap>| !gapinput@phi:= SpinSymBrauerCharacter(ccl,ords,rep);;|
  !gapprompt@gap>| !gapinput@phi in Irr(ct mod 5);|
  true
  
\end{Verbatim}
 

\subsection{\textcolor{Chapter }{SpinSymBasicCharacter}}
\logpage{[ 2, 3, 5 ]}\nobreak
\hyperdef{L}{X87EF85BC816651BF}{}
{\noindent\textcolor{FuncColor}{$\triangleright$\ \ \texttt{SpinSymBasicCharacter({\mdseries\slshape modtbl})\index{SpinSymBasicCharacter@\texttt{SpinSymBasicCharacter}}
\label{SpinSymBasicCharacter}
}\hfill{\scriptsize (function)}}\\
\textbf{\indent Returns:\ }
a $p$-modular basic spin character of the (possibly incomplete) $p$-modular SpinSym table \mbox{\texttt{\mdseries\slshape modtbl}} of  $\tilde{S}_n$. 



 This is just a shortcut for constructing a basic spin representation of  $\tilde{S}_n$ in characteristic $p$ and computing its Brauer character by calling \texttt{SpinSymBrauerCharacter} (\ref{SpinSymBrauerCharacter}) afterwards. }

 
\begin{Verbatim}[commandchars=!@|,fontsize=\small,frame=single,label=Example]
   
  !gapprompt@gap>| !gapinput@SetClassParameters(bt,ccl);|
  !gapprompt@gap>| !gapinput@SpinSymBasicCharacter(bt) = phi;|
  true
  
\end{Verbatim}
 }

 
\section{\textcolor{Chapter }{Young subgroups}}\label{sec:youngsubgroups}
\logpage{[ 2, 4, 0 ]}
\hyperdef{L}{X7BB8AC1D781BD6FE}{}
{
  Let $k$ and $l$ be integers greater than $1$ and set $n=k+l$. The following subgroup of  $\tilde{S}_n$, 
\[\tilde{S}_{k,l} = \langle t_1,\ldots,t_{k-1}, t_{k+1},\ldots,t_{n-1}\rangle,\]
  is called a (maximal) \emph{Young subgroup} of  $\tilde{S}_n$. Similarly, $\tilde{A}_{k,l}=\tilde{S}_{k,l}\cap\tilde{A}_{n}$  is a (maximal) Young subgroup of  $\tilde{A}_n$. Note that $(\tilde{S}_{k,l})^\pi \cong S_k\times S_{l}$  and $(\tilde{A}_{k,l})^\pi \cong A_k\times A_{l}$  but only $\tilde{A}_{k,l}\cong (\tilde{A}_k\times\tilde{A}_{l})/\langle(z,z)\rangle$  is a central product. In between $\tilde{A}_{k,l}$  and $\tilde{S}_{k,l}$  there are further central products $\tilde{S}_k\circ\tilde{A}_{l}\cong
(\tilde{S}_k\times\tilde{A}_{l})/\langle(z,z)\rangle$  and $\tilde{A}_k\circ\tilde{S}_{l}\cong
(\tilde{A}_k\times\tilde{S}_{l})/\langle(z,z)\rangle$  which are $\pi$-preimages of $S_k\times A_{l}$  and $A_k\times S_{l}$ , respectively. See \cite[Section 5.2]{Maas2011}. 

\subsection{\textcolor{Chapter }{SpinSymCharacterTableOfMaximalYoungSubgroup}}
\logpage{[ 2, 4, 1 ]}\nobreak
\hyperdef{L}{X798B9B1482CBDDC0}{}
{\noindent\textcolor{FuncColor}{$\triangleright$\ \ \texttt{SpinSymCharacterTableOfMaximalYoungSubgroup({\mdseries\slshape k, l, type})\index{SpinSymCharacterTableOfMaximalYoungSubgroup@\texttt{Spin}\-\texttt{Sym}\-\texttt{Character}\-\texttt{Table}\-\texttt{Of}\-\texttt{Maximal}\-\texttt{Young}\-\texttt{Subgroup}}
\label{SpinSymCharacterTableOfMaximalYoungSubgroup}
}\hfill{\scriptsize (function)}}\\
\textbf{\indent Returns:\ }
the ordinary character table of a maximal Young subgroup depending on \mbox{\texttt{\mdseries\slshape type}}.



 For integers \mbox{\texttt{\mdseries\slshape k}} and \mbox{\texttt{\mdseries\slshape l}} greater than $1$ the function returns the ordinary character table of $\tilde{A}_{k,l}$, $\tilde{A}_k\circ\tilde{S}_l$, $\tilde{S}_k\circ\tilde{A}_l$, or $\tilde{S}_{k,l}$ depending on the string \mbox{\texttt{\mdseries\slshape type}} being \texttt{"Alternating"}, \texttt{"AlternatingSymmetric"}, \texttt{"SymmetricAlternating"}, or \texttt{"Symmetric"}, respectively.

 If \mbox{\texttt{\mdseries\slshape type}} is \texttt{"Symmetric"} then the output is computed by means of Clifford's theory from the character
tables of $\tilde{S}_k\circ\tilde{A}_l$, $\tilde{A}_{k,l}$, and $\tilde{A}_k\circ\tilde{S}_l$ (see \cite[Section 5.2]{Maas2011}). These `ingredients' are computed and then stored in the attribute \texttt{SpinSymIngredients} so they can be accessed during the construction (and for the construction of a
relative Brauer table too, see \texttt{SpinSymBrauerTableOfMaximalYoungSubgroup} (\ref{SpinSymBrauerTableOfMaximalYoungSubgroup})). 

 The construction of the character tables of \mbox{\texttt{\mdseries\slshape type}} \texttt{"Alternating"}, \texttt{"AlternatingSymmetric"}, or \texttt{"SymmetricAlternating"} is straightforward and may be accomplished by first construcing a direct
product, for example, the character table of $\tilde{S}_k\times\tilde{A}_{l}$, followed by the construction of the character table of the factor group mod $\langle(z,z)\rangle$.

 However, we use a faster method that builds up the table from scratch, using
the appropriate component tables as ingredients (for example, the generic
character tables of  $\tilde{S}_k$ and  $\tilde{A}_l$). In this way we can easily build up a suitable list of class parameters that
are needed to determine the class fusion in the construction of \mbox{\texttt{\mdseries\slshape type}} \texttt{"Symmetric"}. }

 
\begin{Verbatim}[commandchars=!@|,fontsize=\small,frame=single,label=Example]
  
  !gapprompt@gap>| !gapinput@2AA:= SpinSymCharacterTableOfMaximalYoungSubgroup(8,5,"Alternating"); |
  CharacterTable( "2.(Alt(8)xAlt(5))" )
  !gapprompt@gap>| !gapinput@SpinSymCharacterTableOfMaximalYoungSubgroup(8,5,"AlternatingSymmetric");|
  CharacterTable( "2.(Alt(8)xSym(5))" )
  !gapprompt@gap>| !gapinput@SpinSymCharacterTableOfMaximalYoungSubgroup(8,5,"SymmetricAlternating");     |
  CharacterTable( "2.(Sym(8)xAlt(5))" )
  !gapprompt@gap>| !gapinput@2SS:= SpinSymCharacterTableOfMaximalYoungSubgroup(8,5,"Symmetric");           |
  CharacterTable( "2.(Sym(8)xSym(5))" )
  
\end{Verbatim}
 

\subsection{\textcolor{Chapter }{SpinSymBrauerTableOfMaximalYoungSubgroup}}
\logpage{[ 2, 4, 2 ]}\nobreak
\hyperdef{L}{X849A37657F53BAE3}{}
{\noindent\textcolor{FuncColor}{$\triangleright$\ \ \texttt{SpinSymBrauerTableOfMaximalYoungSubgroup({\mdseries\slshape ordtbl, p})\index{SpinSymBrauerTableOfMaximalYoungSubgroup@\texttt{Spin}\-\texttt{Sym}\-\texttt{Brauer}\-\texttt{Table}\-\texttt{Of}\-\texttt{Maximal}\-\texttt{Young}\-\texttt{Subgroup}}
\label{SpinSymBrauerTableOfMaximalYoungSubgroup}
}\hfill{\scriptsize (function)}}\\
\textbf{\indent Returns:\ }
the \mbox{\texttt{\mdseries\slshape p}}-modular character table of the ordinary character table \mbox{\texttt{\mdseries\slshape ordtbl}} returned by the function \texttt{SpinSymCharacterTableOfMaximalYoungSubgroup} (\ref{SpinSymCharacterTableOfMaximalYoungSubgroup}). 



 If the rational prime \mbox{\texttt{\mdseries\slshape p}} is odd, then the construction of the irreducible Brauer characters is really
the same as in the ordinary case but it depends on the \mbox{\texttt{\mdseries\slshape p}}-modular tables of of \mbox{\texttt{\mdseries\slshape ordtbl}}'s `ingredients'. If some Brauer table that is necessary for the construction
is not available then \texttt{fail} is returned. 

 Alternatively, the \texttt{mod}-operator may be used.

 For \mbox{\texttt{\mdseries\slshape p}} $=2$ the Brauer table is essentially constructed as a direct product by standard \textsf{GAP} methods written by Thomas Breuer. }

 We call a character table returned by \texttt{SpinSymCharacterTableOfMaximalYoungSubgroup} (\ref{SpinSymCharacterTableOfMaximalYoungSubgroup}) or \texttt{SpinSymBrauerTableOfMaximalYoungSubgroup} (\ref{SpinSymBrauerTableOfMaximalYoungSubgroup}) a SpinSym table too. It has lists of class and character parameters whose
format is explained in \cite[Sections 5.2, 5.3]{Maas2011}. }

 
\begin{Verbatim}[commandchars=!@|,fontsize=\small,frame=single,label=Example]
  
  !gapprompt@gap>| !gapinput@SpinSymBrauerTableOfMaximalYoungSubgroup(2AA,3);|
  BrauerTable( "2.(Alt(8)xAlt(5))", 3 )
  !gapprompt@gap>| !gapinput@2SS mod 5;|
  BrauerTable( "2.(Sym(8)xSym(5))", 5 )
  !gapprompt@gap>| !gapinput@ct:= 2SS mod 2;  |
  BrauerTable( "2.(Sym(8)xSym(5))", 2 )
  !gapprompt@gap>| !gapinput@ct1:= CharacterTable("Sym(8)") mod 2;;|
  !gapprompt@gap>| !gapinput@ct2:= CharacterTable("Sym(5)") mod 2;;     |
  !gapprompt@gap>| !gapinput@Irr(ct1*ct2) = Irr(ct);|
  true
  
\end{Verbatim}
 
\section{\textcolor{Chapter }{Class Fusions}}\label{sec:classfusions}
\logpage{[ 2, 5, 0 ]}
\hyperdef{L}{X7906869F7F190E76}{}
{
  The following functions determine class fusion maps between SpinSym tables by
means of their class parameters. Such `default' class fusion maps allow to
induce characters from various subgroups of  $\tilde{S}_n$ or  $\tilde{A}_n$ consistently. 

\subsection{\textcolor{Chapter }{SpinSymClassFusion}}
\logpage{[ 2, 5, 1 ]}\nobreak
\hyperdef{L}{X7C4BE18086E671CD}{}
{\noindent\textcolor{FuncColor}{$\triangleright$\ \ \texttt{SpinSymClassFusion({\mdseries\slshape ctSource, ctDest})\index{SpinSymClassFusion@\texttt{SpinSymClassFusion}}
\label{SpinSymClassFusion}
}\hfill{\scriptsize (function)}}\\
\textbf{\indent Returns:\ }
a fusion map from the SpinSym table \mbox{\texttt{\mdseries\slshape ctSource}} to the SpinSym table \mbox{\texttt{\mdseries\slshape ctDest}}. This map is stored if there is no other fusion map from \mbox{\texttt{\mdseries\slshape ctSource}} to \mbox{\texttt{\mdseries\slshape ctDest}} stored yet.



 The possible input tables are expected to be either ordinary or $p$-modular SpinSym tables of the following pairs of groups \begin{center}
\begin{tabular}{ccc}Source&
$\to$&
Dest\\
$\tilde{A}_n$ &
&
$\tilde{S}_n$ \\
$\tilde{S}_k$ &
&
$\tilde{S}_n$ \\
$\tilde{A}_k$ &
&
$\tilde{A}_n$ \\
$\tilde{S}_{n-2}$ &
&
$\tilde{A}_n$ \\
$\tilde{S}_{k,l}$ &
&
$\tilde{S}_{k+l}$ \\
$\tilde{S}_k\circ\tilde{A}_l$ &
&
$\tilde{S}_{k,l}$ \\
$\tilde{A}_k\circ\tilde{S}_l$ &
&
$\tilde{S}_{k,l}$ \\
$\tilde{A}_{k,l}$ &
&
$\tilde{S}_k\circ\tilde{A}_l$ \\
$\tilde{A}_{k,l}$ &
&
$\tilde{A}_k\circ\tilde{S}_l$ \\
$\tilde{A}_{k,l}$ &
&
$\tilde{A}_{k+l}$ \\
\end{tabular}\\[2mm]
\end{center}

 The appropriate function (see the descriptions below) is called to determine
the fusion map \texttt{fus}. If \texttt{GetFusionMap(ctSource, ctDest)} fails, then \texttt{fus} is stored by calling \texttt{StoreFusion(ctSource, fus, ctDest)}. }

 
\begin{Verbatim}[commandchars=!@|,fontsize=\small,frame=single,label=Example]
  
  !gapprompt@gap>| !gapinput@ctD:= CharacterTable("2.Sym(18)");;                                  |
  !gapprompt@gap>| !gapinput@ctS:= SpinSymCharacterTableOfMaximalYoungSubgroup(10,8,"Symmetric");;|
  !gapprompt@gap>| !gapinput@GetFusionMap(ctS,ctD);|
  fail
  !gapprompt@gap>| !gapinput@SpinSymClassFusion(ctS,ctD);;|
  #I SpinSymClassFusion: stored fusion map from 2.(Sym(10)xSym(8)) to 2.Sym(18)
  !gapprompt@gap>| !gapinput@GetFusionMap(ctS,ctD) <> fail;|
  true
  
\end{Verbatim}
 

\subsection{\textcolor{Chapter }{SpinSymClassFusion2Ain2S}}
\logpage{[ 2, 5, 2 ]}\nobreak
\hyperdef{L}{X857FB2A17904A9CE}{}
{\noindent\textcolor{FuncColor}{$\triangleright$\ \ \texttt{SpinSymClassFusion2Ain2S({\mdseries\slshape cclSource, cclDest})\index{SpinSymClassFusion2Ain2S@\texttt{SpinSymClassFusion2Ain2S}}
\label{SpinSymClassFusion2Ain2S}
}\hfill{\scriptsize (function)}}\\
\textbf{\indent Returns:\ }
a fusion map between the SpinSym tables of $\tilde{A}_n$  and $\tilde{S}_n$.



 Given lists of class parameters \mbox{\texttt{\mdseries\slshape cclSource}} and \mbox{\texttt{\mdseries\slshape cclDest}} of (ordinary or $p$-modular) SpinSym tables of $\tilde{A}_n$  and $\tilde{S}_n$, respectively, a corresponding class fusion map is determined. See \cite[(5.4.1)]{Maas2011}. }

 

\subsection{\textcolor{Chapter }{SpinSymClassFusion2Sin2S}}
\logpage{[ 2, 5, 3 ]}\nobreak
\hyperdef{L}{X84DDBEEF7BA42414}{}
{\noindent\textcolor{FuncColor}{$\triangleright$\ \ \texttt{SpinSymClassFusion2Sin2S({\mdseries\slshape cclSource, cclDest})\index{SpinSymClassFusion2Sin2S@\texttt{SpinSymClassFusion2Sin2S}}
\label{SpinSymClassFusion2Sin2S}
}\hfill{\scriptsize (function)}}\\
\textbf{\indent Returns:\ }
a fusion map between the SpinSym tables of $\tilde{S}_k$  and $\tilde{S}_n$  for $k\leq n$.



 Given lists of class parameters \mbox{\texttt{\mdseries\slshape cclSource}} and \mbox{\texttt{\mdseries\slshape cclDest}} of (ordinary or $p$-modular) SpinSym tables of $\tilde{S}_k$  and $\tilde{S}_n$  for $k\leq n$, respectively, a corresponding class fusion map is determined. See \cite[(5.4.2)]{Maas2011}. }

 
\begin{Verbatim}[commandchars=!@|,fontsize=\small,frame=single,label=Example]
  
  !gapprompt@gap>| !gapinput@ctD:= CharacterTable("2.Sym(18)");;|
  !gapprompt@gap>| !gapinput@ctS:= CharacterTable("2.Sym(6)");;|
  !gapprompt@gap>| !gapinput@cclD:= ClassParameters(ctD);;|
  !gapprompt@gap>| !gapinput@cclS:= ClassParameters(ctS);;|
  !gapprompt@gap>| !gapinput@fus:= SpinSymClassFusion2Sin2S(cclS,cclD);;|
  !gapprompt@gap>| !gapinput@StoreFusion(ctS,fus,ctD);|
  
\end{Verbatim}
 

\subsection{\textcolor{Chapter }{SpinSymClassFusion2Ain2A}}
\logpage{[ 2, 5, 4 ]}\nobreak
\hyperdef{L}{X7A4425B58361A2FD}{}
{\noindent\textcolor{FuncColor}{$\triangleright$\ \ \texttt{SpinSymClassFusion2Ain2A({\mdseries\slshape cclSource, cclDest})\index{SpinSymClassFusion2Ain2A@\texttt{SpinSymClassFusion2Ain2A}}
\label{SpinSymClassFusion2Ain2A}
}\hfill{\scriptsize (function)}}\\
\textbf{\indent Returns:\ }
a fusion map between the SpinSym tables of $\tilde{A}_k$  and $\tilde{A}_n$  for $k\leq n$.



 Given lists of class parameters \mbox{\texttt{\mdseries\slshape cclSource}} and \mbox{\texttt{\mdseries\slshape cclDest}} of (ordinary or $p$-modular) SpinSym tables of $\tilde{A}_k$  and $\tilde{A}_n$  for $k\leq n$, respectively, a corresponding class fusion map is determined. See \cite[(5.4.3)]{Maas2011}. }

 

\subsection{\textcolor{Chapter }{SpinSymClassFusion2Sin2A}}
\logpage{[ 2, 5, 5 ]}\nobreak
\hyperdef{L}{X7BE629FB81C12F27}{}
{\noindent\textcolor{FuncColor}{$\triangleright$\ \ \texttt{SpinSymClassFusion2Sin2A({\mdseries\slshape cclSource, cclDest})\index{SpinSymClassFusion2Sin2A@\texttt{SpinSymClassFusion2Sin2A}}
\label{SpinSymClassFusion2Sin2A}
}\hfill{\scriptsize (function)}}\\
\textbf{\indent Returns:\ }
a fusion map between the SpinSym tables of $\tilde{S}_{n-2}$  and $\tilde{A}_n$. 



 Given lists of class parameters \mbox{\texttt{\mdseries\slshape cclSource}} and \mbox{\texttt{\mdseries\slshape cclDest}} of (ordinary or $p$-modular) SpinSym tables of $\tilde{S}_{n-2}$  and $\tilde{A}_n$, respectively, a corresponding class fusion map with respect to the embedding
of $\langle t_1t_{n-2},\ldots,t_{n-3}t_{n-1}\rangle\cong\tilde{S}_{n-2}$  in $\tilde{A}_n$  is determined. See \cite[(5.4.4)]{Maas2011}. }

 

\subsection{\textcolor{Chapter }{SpinSymClassFusion2SSin2S}}
\logpage{[ 2, 5, 6 ]}\nobreak
\hyperdef{L}{X8310E6BC7CC1A2F5}{}
{\noindent\textcolor{FuncColor}{$\triangleright$\ \ \texttt{SpinSymClassFusion2SSin2S({\mdseries\slshape cclSource, cclDest})\index{SpinSymClassFusion2SSin2S@\texttt{SpinSymClassFusion2SSin2S}}
\label{SpinSymClassFusion2SSin2S}
}\hfill{\scriptsize (function)}}\\
\textbf{\indent Returns:\ }
a fusion map between the SpinSym tables of $\tilde{S}_{k,l}$  and $\tilde{S}_{k+l}$. 



 Given lists of class parameters \mbox{\texttt{\mdseries\slshape cclSource}} and \mbox{\texttt{\mdseries\slshape cclDest}} of (ordinary or $p$-modular) SpinSym tables of $\tilde{S}_{k,l}$  and $\tilde{S}_{k+l}$, respectively, a corresponding class fusion map is determined by means of \cite[(5.1.6)]{Maas2011}. }

 
\begin{Verbatim}[commandchars=!@|,fontsize=\small,frame=single,label=Example]
  
  !gapprompt@gap>| !gapinput@ctD:= CharacterTable("2.Sym(18)");;                                  |
  !gapprompt@gap>| !gapinput@ctS:= SpinSymCharacterTableOfMaximalYoungSubgroup(10,8,"Symmetric");;|
  !gapprompt@gap>| !gapinput@cclD:= ClassParameters(ctD);;|
  !gapprompt@gap>| !gapinput@cclS:= ClassParameters(ctS);;|
  !gapprompt@gap>| !gapinput@fus:= SpinSymClassFusion2SSin2S(cclS,cclD);;|
  !gapprompt@gap>| !gapinput@StoreFusion(ctS,fus,ctD);|
  
\end{Verbatim}
 

\subsection{\textcolor{Chapter }{SpinSymClassFusion2SAin2SS}}
\logpage{[ 2, 5, 7 ]}\nobreak
\hyperdef{L}{X7854E1E785A0378A}{}
{\noindent\textcolor{FuncColor}{$\triangleright$\ \ \texttt{SpinSymClassFusion2SAin2SS({\mdseries\slshape cclSource, cclDest})\index{SpinSymClassFusion2SAin2SS@\texttt{SpinSymClassFusion2SAin2SS}}
\label{SpinSymClassFusion2SAin2SS}
}\hfill{\scriptsize (function)}}\\
\textbf{\indent Returns:\ }
a fusion map between the SpinSym tables of $\tilde{S}_k\circ\tilde{A}_l$  and $\tilde{S}_{k,l}$. 



 Given lists of class parameters \mbox{\texttt{\mdseries\slshape cclSource}} and \mbox{\texttt{\mdseries\slshape cclDest}} of (ordinary or $p$-modular) SpinSym tables of $\tilde{S}_k\circ\tilde{A}_l$  and $\tilde{S}_{k,l}$, respectively, a corresponding class fusion map is determined. See \cite[(5.4.6)]{Maas2011}. }

 

\subsection{\textcolor{Chapter }{SpinSymClassFusion2ASin2SS}}
\logpage{[ 2, 5, 8 ]}\nobreak
\hyperdef{L}{X83B113007DF2BB69}{}
{\noindent\textcolor{FuncColor}{$\triangleright$\ \ \texttt{SpinSymClassFusion2ASin2SS({\mdseries\slshape cclSource, cclDest})\index{SpinSymClassFusion2ASin2SS@\texttt{SpinSymClassFusion2ASin2SS}}
\label{SpinSymClassFusion2ASin2SS}
}\hfill{\scriptsize (function)}}\\
\textbf{\indent Returns:\ }
a fusion map between the SpinSym tables of $\tilde{A}_k\circ\tilde{S}_l$  and $\tilde{S}_{k,l}$. 



 Given lists of class parameters \mbox{\texttt{\mdseries\slshape cclSource}} and \mbox{\texttt{\mdseries\slshape cclDest}} of (ordinary or $p$-modular) SpinSym tables of $\tilde{A}_k\circ\tilde{S}_l$  and $\tilde{S}_{k,l}$, respectively, a corresponding class fusion map is determined analogously to \cite[(5.4.6)]{Maas2011}. }

 

\subsection{\textcolor{Chapter }{SpinSymClassFusion2AAin2SA}}
\logpage{[ 2, 5, 9 ]}\nobreak
\hyperdef{L}{X786B6E2D7CF13B51}{}
{\noindent\textcolor{FuncColor}{$\triangleright$\ \ \texttt{SpinSymClassFusion2AAin2SA({\mdseries\slshape cclSource, cclDest})\index{SpinSymClassFusion2AAin2SA@\texttt{SpinSymClassFusion2AAin2SA}}
\label{SpinSymClassFusion2AAin2SA}
}\hfill{\scriptsize (function)}}\\
\textbf{\indent Returns:\ }
a fusion map between the SpinSym tables of $\tilde{A}_{k,l}$  and $\tilde{S}_k\circ\tilde{A}_l$. 



 Given lists of class parameters \mbox{\texttt{\mdseries\slshape cclSource}} and \mbox{\texttt{\mdseries\slshape cclDest}} of (ordinary or $p$-modular) SpinSym tables of $\tilde{A}_{k,l}$  and $\tilde{S}_k\circ\tilde{A}_l$, respectively, a corresponding class fusion map is determined. See \cite[(5.4.7)]{Maas2011}. }

 

\subsection{\textcolor{Chapter }{SpinSymClassFusion2AAin2AS}}
\logpage{[ 2, 5, 10 ]}\nobreak
\hyperdef{L}{X80DFBE347B87AA43}{}
{\noindent\textcolor{FuncColor}{$\triangleright$\ \ \texttt{SpinSymClassFusion2AAin2AS({\mdseries\slshape cclSource, cclDest})\index{SpinSymClassFusion2AAin2AS@\texttt{SpinSymClassFusion2AAin2AS}}
\label{SpinSymClassFusion2AAin2AS}
}\hfill{\scriptsize (function)}}\\
\textbf{\indent Returns:\ }
a fusion map between the SpinSym tables of $\tilde{A}_{k,l}$  and $\tilde{A}_k\circ\tilde{S}_l$. 



 Given lists of class parameters \mbox{\texttt{\mdseries\slshape cclSource}} and \mbox{\texttt{\mdseries\slshape cclDest}} of (ordinary or $p$-modular) SpinSym tables of $\tilde{A}_{k,l}$  and $\tilde{A}_k\circ\tilde{S}_l$, respectively, a corresponding class fusion map is determined analogously to \cite[(5.4.7)]{Maas2011}. }

 

\subsection{\textcolor{Chapter }{SpinSymClassFusion2AAin2A}}
\logpage{[ 2, 5, 11 ]}\nobreak
\hyperdef{L}{X796897DF78143E05}{}
{\noindent\textcolor{FuncColor}{$\triangleright$\ \ \texttt{SpinSymClassFusion2AAin2A({\mdseries\slshape cclSource, cclDest})\index{SpinSymClassFusion2AAin2A@\texttt{SpinSymClassFusion2AAin2A}}
\label{SpinSymClassFusion2AAin2A}
}\hfill{\scriptsize (function)}}\\
\textbf{\indent Returns:\ }
a fusion map between the SpinSym tables of $\tilde{A}_{k,l}$  and $\tilde{A}_{k+l}$. 



 Given lists of class parameters \mbox{\texttt{\mdseries\slshape cclSource}} and \mbox{\texttt{\mdseries\slshape cclDest}} of (ordinary or $p$-modular) SpinSym tables of $\tilde{A}_{k,l}$ and $\tilde{A}_{k+l}$, respectively, a corresponding class fusion map is determined. See \cite[(5.4.8)]{Maas2011}. }

 }

 }

 \def\bibname{References\logpage{[ "Bib", 0, 0 ]}
\hyperdef{L}{X7A6F98FD85F02BFE}{}
}

\bibliographystyle{plain}
\bibliography{biblio.xml}

\addcontentsline{toc}{chapter}{References}

\def\indexname{Index\logpage{[ "Ind", 0, 0 ]}
\hyperdef{L}{X83A0356F839C696F}{}
}

\cleardoublepage
\phantomsection
\addcontentsline{toc}{chapter}{Index}


\printindex

\newpage
\immediate\write\pagenrlog{["End"], \arabic{page}];}
\immediate\closeout\pagenrlog
\end{document}
