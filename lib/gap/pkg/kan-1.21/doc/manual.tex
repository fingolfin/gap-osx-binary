% generated by GAPDoc2LaTeX from XML source (Frank Luebeck)
\documentclass[a4paper,11pt]{report}

\usepackage{a4wide}
\sloppy
\pagestyle{myheadings}
\usepackage{amssymb}
\usepackage[latin1]{inputenc}
\usepackage{makeidx}
\makeindex
\usepackage{color}
\definecolor{FireBrick}{rgb}{0.5812,0.0074,0.0083}
\definecolor{RoyalBlue}{rgb}{0.0236,0.0894,0.6179}
\definecolor{RoyalGreen}{rgb}{0.0236,0.6179,0.0894}
\definecolor{RoyalRed}{rgb}{0.6179,0.0236,0.0894}
\definecolor{LightBlue}{rgb}{0.8544,0.9511,1.0000}
\definecolor{Black}{rgb}{0.0,0.0,0.0}

\definecolor{linkColor}{rgb}{0.0,0.0,0.554}
\definecolor{citeColor}{rgb}{0.0,0.0,0.554}
\definecolor{fileColor}{rgb}{0.0,0.0,0.554}
\definecolor{urlColor}{rgb}{0.0,0.0,0.554}
\definecolor{promptColor}{rgb}{0.0,0.0,0.589}
\definecolor{brkpromptColor}{rgb}{0.589,0.0,0.0}
\definecolor{gapinputColor}{rgb}{0.589,0.0,0.0}
\definecolor{gapoutputColor}{rgb}{0.0,0.0,0.0}

%%  for a long time these were red and blue by default,
%%  now black, but keep variables to overwrite
\definecolor{FuncColor}{rgb}{0.0,0.0,0.0}
%% strange name because of pdflatex bug:
\definecolor{Chapter }{rgb}{0.0,0.0,0.0}
\definecolor{DarkOlive}{rgb}{0.1047,0.2412,0.0064}


\usepackage{fancyvrb}

\usepackage{mathptmx,helvet}
\usepackage[T1]{fontenc}
\usepackage{textcomp}


\usepackage[
            pdftex=true,
            bookmarks=true,        
            a4paper=true,
            pdftitle={Written with GAPDoc},
            pdfcreator={LaTeX with hyperref package / GAPDoc},
            colorlinks=true,
            backref=page,
            breaklinks=true,
            linkcolor=linkColor,
            citecolor=citeColor,
            filecolor=fileColor,
            urlcolor=urlColor,
            pdfpagemode={UseNone}, 
           ]{hyperref}

\newcommand{\maintitlesize}{\fontsize{50}{55}\selectfont}

% write page numbers to a .pnr log file for online help
\newwrite\pagenrlog
\immediate\openout\pagenrlog =\jobname.pnr
\immediate\write\pagenrlog{PAGENRS := [}
\newcommand{\logpage}[1]{\protect\write\pagenrlog{#1, \thepage,}}
%% were never documented, give conflicts with some additional packages

\newcommand{\GAP}{\textsf{GAP}}

%% nicer description environments, allows long labels
\usepackage{enumitem}
\setdescription{style=nextline}

%% depth of toc
\setcounter{tocdepth}{1}





%% command for ColorPrompt style examples
\newcommand{\gapprompt}[1]{\color{promptColor}{\bfseries #1}}
\newcommand{\gapbrkprompt}[1]{\color{brkpromptColor}{\bfseries #1}}
\newcommand{\gapinput}[1]{\color{gapinputColor}{#1}}


\begin{document}

\logpage{[ 0, 0, 0 ]}
\begin{titlepage}
\mbox{}\vfill

\begin{center}{\maintitlesize \textbf{\textsf{Kan}\mbox{}}}\\
\vfill

\hypersetup{pdftitle=\textsf{Kan}}
\markright{\scriptsize \mbox{}\hfill \textsf{Kan} \hfill\mbox{}}
{\Huge \textbf{A package for Induced Category Actions\mbox{}}}\\
\vfill

{\Huge Version 1.21\mbox{}}\\[1cm]
{02/06/2015\mbox{}}\\[1cm]
\mbox{}\\[2cm]
{\Large \textbf{ Anne Heyworth \mbox{}}}\\
{\Large \textbf{ Chris Wensley    \mbox{}}}\\
\hypersetup{pdfauthor= Anne Heyworth ;  Chris Wensley    }
\end{center}\vfill

\mbox{}\\
{\mbox{}\\
\small \noindent \textbf{ Chris Wensley    }  Email: \href{mailto://c.d.wensley@bangor.ac.uk} {\texttt{c.d.wensley@bangor.ac.uk}}\\
  Homepage: \href{http://pages.bangor.ac.uk/~mas023/} {\texttt{http://pages.bangor.ac.uk/\texttt{\symbol{126}}mas023/}}\\
  Address: \begin{minipage}[t]{8cm}\noindent
 School of Computer Science, Bangor University,\\
 Dean Street, Bangor, Gwynedd, LL57 1UT, U.K. \end{minipage}
}\\
\end{titlepage}

\newpage\setcounter{page}{2}
{\small 
\section*{Abstract}
\logpage{[ 0, 0, 1 ]}
 The \textsf{Kan} package was originally implemented in 1996 using the \textsf{GAP} 3 language, to compute induced actions of categories, when the first author
was studying for a Ph.D. in Bangor. 

 This reduced version only provides functions for the computation of normal
forms of representatives of double cosets of finitely presented groups. 

 \textsf{Kan} became an accepted \textsf{GAP} package in May 2015. 

 Bug reports, suggestions and comments are, of course, welcome. Please contact
the second author at \href{mailto://c.d.wensley@bangor.ac.uk} {\texttt{c.d.wensley@bangor.ac.uk}}. \mbox{}}\\[1cm]
{\small 
\section*{Copyright}
\logpage{[ 0, 0, 2 ]}
 \index{License} {\copyright} 1996-2015 Anne Heyworth and Chris Wensley 

 \textsf{kan} is free software; you can redistribute it and/or modify it under the terms of
the \href{http://www.fsf.org/licenses/gpl.html} {GNU General Public License} as published by the Free Software Foundation; either version 2 of the License,
or (at your option) any later version. \mbox{}}\\[1cm]
{\small 
\section*{Acknowledgements}
\logpage{[ 0, 0, 3 ]}
 This documentation was prepared with the \textsf{GAPDoc} package of Frank L{\"u}beck and Max Neunh{\"o}ffer. \mbox{}}\\[1cm]
\newpage

\def\contentsname{Contents\logpage{[ 0, 0, 4 ]}}

\tableofcontents
\newpage

         
\chapter{\textcolor{Chapter }{Introduction}}\label{intro}
\logpage{[ 1, 0, 0 ]}
\hyperdef{L}{X7DFB63A97E67C0A1}{}
{
  The \textsf{Kan} package started out as part of Anne Heyworth's thesis \cite{anne-thesis}, and was designed to compute induced actions of categories (see also \cite{BrHe}). 

 This version of \textsf{Kan} only provides functions for the computation of normal forms of representatives
of double cosets of finitely presented groups, and is made available in
support of the paper \cite{BrGhHeWe}. Existing methods for computing double cosets in \textsf{GAP} are described in \cite{SteveL}. 

 The package is loaded into \textsf{GAP} with the command 
\begin{Verbatim}[commandchars=!@|,fontsize=\small,frame=single,label=Example]
  
  !gapprompt@gap>| !gapinput@LoadPackage( "kan" ); |
  
\end{Verbatim}
 The current version is 1.21 for \textsf{GAP} 4.7, released on 2nd June 2015. 

 The package may be obtained as a compressed tar file \texttt{kan-1.21.tar.gz} by ftp from one of the sites with a \textsf{GAP} 4 archive, or from the Bangor Mathematics website pages using the URL: \href{http://pages.bangor.ac.uk/~mas023/chda/kan/kan121.html} {\texttt{http://pages.bangor.ac.uk/\texttt{\symbol{126}}mas023/chda/kan/kan121.html}} 

 Some of the functions in the \textsf{automata} package are used to compute word acceptors and regular expressions for the
languages they accept. 

 The \textsf{kbmag} package is also used to compute a word acceptor of a group \texttt{G} when \texttt{G} has no finite rewriting system. If \textsf{kbmag} is not available (the user is not on a UNIX system, or the C-programs have not
been compiled), the file \texttt{dckbmag.gi} will not be read, so methods for the functions detailed in section 2.4.1 will
not be available. The information parameter \texttt{InfoKan} takes default value \texttt{0}. When raised to a higher value, additional information is printed out. 

 Once the package is loaded, it is possible to check the installation is
correct by running a test file of the manual examples with the following
command. (The test file itself is \texttt{tst/fulltest.tst} or \texttt{tst/parttest.tst}, depending whether or not \textsf{kbmag} is available. The 'all' in the log below indicates that the full test has been
run.) 
\begin{Verbatim}[commandchars=!@|,fontsize=\small,frame=single,label=Example]
  
  !gapprompt@gap>| !gapinput@ReadPackage( "kan", "tst/testall.g" );|
  + Testing all example commands in the Kan manual
  + GAP4stones: 6
  true
  
\end{Verbatim}
 

 Please send bug reports, suggestions and other comments to the second author. }

         
\chapter{\textcolor{Chapter }{Double Coset Rewriting Systems}}\label{chap-dcrws}
\logpage{[ 2, 0, 0 ]}
\hyperdef{L}{X7F197C8D835A6F45}{}
{
  The \textsf{Kan} package provides functions for the computation of normal forms for double
coset representatives of finitely presented groups. The first version of the
package was released to support the paper \cite{BrGhHeWe}, which describes the algorithms used in this package. 
\section{\textcolor{Chapter }{Rewriting Systems}}\logpage{[ 2, 1, 0 ]}
\hyperdef{L}{X7CA8FCFD81AA1890}{}
{
 

\subsection{\textcolor{Chapter }{KnuthBendixRewritingSystem}}
\logpage{[ 2, 1, 1 ]}\nobreak
\hyperdef{L}{X87A3823483E4FF86}{}
{\noindent\textcolor{FuncColor}{$\triangleright$\ \ \texttt{KnuthBendixRewritingSystem({\mdseries\slshape grp, gensorder, ordering, alph})\index{KnuthBendixRewritingSystem@\texttt{KnuthBendixRewritingSystem}}
\label{KnuthBendixRewritingSystem}
}\hfill{\scriptsize (operation)}}\\
\noindent\textcolor{FuncColor}{$\triangleright$\ \ \texttt{ReducedConfluentRewritingSystem({\mdseries\slshape grp, gensorder, ordering, limit})\index{ReducedConfluentRewritingSystem@\texttt{ReducedConfluentRewritingSystem}}
\label{ReducedConfluentRewritingSystem}
}\hfill{\scriptsize (operation)}}\\
\noindent\textcolor{FuncColor}{$\triangleright$\ \ \texttt{DisplayRwsRules({\mdseries\slshape rws})\index{DisplayRwsRules@\texttt{DisplayRwsRules}}
\label{DisplayRwsRules}
}\hfill{\scriptsize (operation)}}\\


 Methods for \texttt{KnuthBendixRewritingSystem} and \texttt{ReducedConfluentRewritingSystem} are supplied which apply to a finitely presented group. These start by calling \texttt{IsomorphismFpMonoid} and then work with the resulting monoid. The parameter \texttt{gensorder} will normally be \texttt{"shortlex"} or \texttt{"wreath"}, while \texttt{ordering} is an integer list for reordering the generators, and \texttt{alph} is an alphabet string used when printing words. A \emph{partial} rewriting system may be obtained by giving a \texttt{limit} to the number of rules calculated. As usual, $A,B$ denote the inverses of $a,b$. 

 In the example the generators are by default ordered $[A,a,B,b]$, so the list \texttt{L1} is used to specify the order \texttt{[a,A,b,B]} to be used with the shortlex ordering. Specifying a limit \texttt{0} means that no limit is prescribed. }

 
\begin{Verbatim}[commandchars=!@|,fontsize=\small,frame=single,label=Example]
  
  !gapprompt@gap>| !gapinput@G1 := FreeGroup( 2 );;|
  !gapprompt@gap>| !gapinput@L1 := [2,1,4,3];;|
  !gapprompt@gap>| !gapinput@order := "shortlex";;|
  !gapprompt@gap>| !gapinput@alph1 := "AaBb";;|
  !gapprompt@gap>| !gapinput@rws1 := ReducedConfluentRewritingSystem( G1, L1, order, 0, alph1 );|
  Rewriting System for Monoid( [ f1, f1^-1, f2, f2^-1 ], ... ) with rules
  [ [ f1*f1^-1, <identity ...> ], [ f1^-1*f1, <identity ...> ],
    [ f2*f2^-1, <identity ...> ], [ f2^-1*f2, <identity ...> ] ]
  !gapprompt@gap>| !gapinput@DisplayRwsRules( rws1 );;|
  [ [ Aa, id ], [ aA, id ], [ Bb, id ], [ bB, id ] ]
  
\end{Verbatim}
 }

 
\section{\textcolor{Chapter }{Example 1 -- free product of two cyclic groups}}\logpage{[ 2, 2, 0 ]}
\hyperdef{L}{X846F0B22859B601A}{}
{
 \index{example -- free product} 

\subsection{\textcolor{Chapter }{DoubleCosetRewritingSystem}}
\logpage{[ 2, 2, 1 ]}\nobreak
\hyperdef{L}{X825D1F4D85DE122D}{}
{\noindent\textcolor{FuncColor}{$\triangleright$\ \ \texttt{DoubleCosetRewritingSystem({\mdseries\slshape grp, genH, genK, rws})\index{DoubleCosetRewritingSystem@\texttt{DoubleCosetRewritingSystem}}
\label{DoubleCosetRewritingSystem}
}\hfill{\scriptsize (function)}}\\
\noindent\textcolor{FuncColor}{$\triangleright$\ \ \texttt{IsDoubleCosetRewritingSystem({\mdseries\slshape dcrws})\index{IsDoubleCosetRewritingSystem@\texttt{IsDoubleCosetRewritingSystem}}
\label{IsDoubleCosetRewritingSystem}
}\hfill{\scriptsize (property)}}\\


 A \emph{double coset rewriting system} for the double cosets $H \backslash G / K$ requires as data a finitely presented group $G =$\texttt{grp}; generators \texttt{genH}, \texttt{genK} for subgroups $H, K$; and a rewriting system \texttt{rws} for $G$. 

 A simple example is given by taking $G$ to be the free group on two generators $a,b$, a cyclic subgroup $H$ with generator $a^6$, and a second cyclic subgroup $K$ with generator $a^4$. (Similar examples using different powers of $a$ are easily constructed.) Since \texttt{gcd(6,4)=2}, we have $Ha^2K=HK$, so the double cosets have representatives $[HK, HaK, Ha^iba^jK, Ha^ibwba^jK]$ where $i \in [0..5]$, $j \in [0..3]$, and $w$ is any word in $a,b$. 

 In the example the free group $G$ is converted to a four generator monoid with relations defining the inverse of
each generator, \texttt{[[Aa,id],[aA,id],[Bb,id],[bB,id]]}, where \texttt{id} is the empty word. The initial rules for the double coset rewriting system
comprise those of $G$ plus those given by the generators of $H,K$, which are $[[Ha^6,H],[a^4K,K]]$. In the complete rewrite system new rules involving $H$ or $K$ may arise, and there may also be rules involving both $H$ and $K$. 

 For this example, 
\begin{itemize}
\item  there are two $H$-rules, $[[Ha^4,HA^2],[HA^3,Ha^3]]$, 
\item  there are two $K$-rules, $[[a^3K,AK],[A^2K,a^2K]]$, 
\item  and there are two $H$-$K$-rules, $[[Ha^2K,HK],[HAK,HaK]]$. 
\end{itemize}
 Here is how the computation may be performed. }

 
\begin{Verbatim}[commandchars=!@|,fontsize=\small,frame=single,label=Example]
  
  !gapprompt@gap>| !gapinput@genG1 := GeneratorsOfGroup( G1 );;|
  !gapprompt@gap>| !gapinput@genH1 := [ genG1[1]^6 ];;|
  !gapprompt@gap>| !gapinput@genK1 := [ genG1[1]^4 ];;|
  !gapprompt@gap>| !gapinput@dcrws1 := DoubleCosetRewritingSystem( G1, genH1, genK1, rws1 );;|
  !gapprompt@gap>| !gapinput@IsDoubleCosetRewritingSystem( dcrws1 );|
  true
  !gapprompt@gap>| !gapinput@DisplayRwsRules( dcrws1 );;|
  G-rules:
  [ [ Aa, id ], [ aA, id ], [ Bb, id ], [ bB, id ] ]
  H-rules:
  [ [ Haaaa, HAA ],
    [ HAAA, Haaa ] ]
  K-rules:
  [ [ aaaK, AK ],
    [ AAK, aaK ] ]
  H-K-rules:
  [ [ HaaK, HK ],
    [ HAK, HaK ] ]
  
\end{Verbatim}
 

\subsection{\textcolor{Chapter }{WordAcceptorOfReducedRws}}
\logpage{[ 2, 2, 2 ]}\nobreak
\hyperdef{L}{X83FF05087E7B133A}{}
{\noindent\textcolor{FuncColor}{$\triangleright$\ \ \texttt{WordAcceptorOfReducedRws({\mdseries\slshape rws})\index{WordAcceptorOfReducedRws@\texttt{WordAcceptorOfReducedRws}}
\label{WordAcceptorOfReducedRws}
}\hfill{\scriptsize (attribute)}}\\
\noindent\textcolor{FuncColor}{$\triangleright$\ \ \texttt{WordAcceptorOfDoubleCosetRws({\mdseries\slshape rws})\index{WordAcceptorOfDoubleCosetRws@\texttt{WordAcceptorOfDoubleCosetRws}}
\label{WordAcceptorOfDoubleCosetRws}
}\hfill{\scriptsize (attribute)}}\\
\noindent\textcolor{FuncColor}{$\triangleright$\ \ \texttt{IsWordAcceptorOfDoubleCosetRws({\mdseries\slshape aut})\index{IsWordAcceptorOfDoubleCosetRws@\texttt{IsWordAcceptorOfDoubleCosetRws}}
\label{IsWordAcceptorOfDoubleCosetRws}
}\hfill{\scriptsize (property)}}\\


 Using functions from the \textsf{automata} package, we may 
\begin{itemize}
\item  compute a \emph{word acceptor} for the rewriting system of $G$; 
\item  compute a \emph{word acceptor} for the double coset rewriting system; 
\item  test a list of words to see whether they are recognised by the automaton; 
\item  obtain a rational expression for the language of the automaton. 
\end{itemize}
 }

 

 
\begin{Verbatim}[commandchars=!@|,fontsize=\small,frame=single,label=Example]
  
  !gapprompt@gap>| !gapinput@waG1 := WordAcceptorOfReducedRws( rws1 );|
  Automaton("nondet",6,"aAbB",[ [ [ 1 ], [ 4 ], [ 1 ], [ 4 ], [ 4 ], [ 4 ] ], [ \
  [ 1 ], [ 3 ], [ 3 ], [ 1 ], [ 3 ], [ 3 ] ], [ [ 1 ], [ 6 ], [ 6 ], [ 6 ], [ 1 \
  ], [ 6 ] ], [ [ 1 ], [ 5 ], [ 5 ], [ 5 ], [ 5 ], [ 1 ] ] ],[ 2 ],[ 1 ]);;
  !gapprompt@gap>| !gapinput@wadc1 := WordAcceptorOfDoubleCosetRws( dcrws1 );|
  < deterministic automaton on 6 letters with 15 states >
  !gapprompt@gap>| !gapinput@Print( wadc1 );|
  Automaton("det",15,"HKaAbB",[ [ 2, 2, 2, 6, 2, 2, 2, 2, 2, 2, 2, 2, 2, 2, 2 ],\
   [ 2, 2, 1, 2, 1, 1, 2, 1, 1, 2, 2, 1, 1, 2, 2 ], [ 2, 2, 13, 2, 10, 5, 2, 13,\
   2, 7, 11, 11, 12, 2, 2 ], [ 2, 2, 9, 2, 2, 14, 2, 9, 15, 2, 2, 2, 2, 7, 15 ],\
   [ 2, 2, 2, 2, 8, 8, 8, 8, 8, 8, 8, 8, 8, 8, 8 ], [ 2, 2, 3, 2, 3, 3, 3, 2, 3,\
   3, 3, 3, 3, 3, 3 ] ],[ 4 ],[ 1 ]);;
  !gapprompt@gap>| !gapinput@words1 := [ "HK","HaK","HbK","HAK","HaaK","HbbK","HabK","HbaK","HbaabK"];;|
  !gapprompt@gap>| !gapinput@valid1 := List( words1, w -> IsRecognizedByAutomaton( wadc1, w ) );|
  [ true, true, true, false, false, true, true, true, true ]
  !gapprompt@gap>| !gapinput@lang1 := FAtoRatExp( wadc1 );|
  ((H(aaaUAA)BUH(a(aBUB)UABUB))(a(a(aa*BUB)UB)UA(AA*BUB)UB)*(a(a(aa*bUb)Ub)UA(AA\
  *bUb))UH(aaaUAA)bUH(a(abUb)UAbUb))((a(a(aa*BUB)UB)UA(AA*BUB))(a(a(aa*BUB)UB)UA\
  (AA*BUB)UB)*(a(a(aa*bUb)Ub)UA(AA*bUb))Ua(a(aa*bUb)Ub)UA(AA*bUb)Ub)*((a(a(aa*BU\
  B)UB)UA(AA*BUB))(a(a(aa*BUB)UB)UA(AA*BUB)UB)*(a(aKUK)UAKUK)Ua(aKUK)UAKUK)U(H(a\
  aaUAA)BUH(a(aBUB)UABUB))(a(a(aa*BUB)UB)UA(AA*BUB)UB)*(a(aKUK)UAKUK)UH(aKUK)
  
\end{Verbatim}
 }

 
\section{\textcolor{Chapter }{Example 2 -- the trefoil group}}\logpage{[ 2, 3, 0 ]}
\hyperdef{L}{X838CC08E7E92EBC0}{}
{
 \index{example -- trefoil group} \index{trefoil group} 

\subsection{\textcolor{Chapter }{PartialDoubleCosetRewritingSystem}}
\logpage{[ 2, 3, 1 ]}\nobreak
\hyperdef{L}{X83DE506B828F4B0D}{}
{\noindent\textcolor{FuncColor}{$\triangleright$\ \ \texttt{PartialDoubleCosetRewritingSystem({\mdseries\slshape grp, Hgens, Kgens, rws, limit})\index{PartialDoubleCosetRewritingSystem@\texttt{PartialDoubleCosetRewritingSystem}}
\label{PartialDoubleCosetRewritingSystem}
}\hfill{\scriptsize (operation)}}\\
\noindent\textcolor{FuncColor}{$\triangleright$\ \ \texttt{WordAcceptorOfPartialDoubleCosetRws({\mdseries\slshape grp, prws})\index{WordAcceptorOfPartialDoubleCosetRws@\texttt{WordAcceptorOfPartialDoubleCosetRws}}
\label{WordAcceptorOfPartialDoubleCosetRws}
}\hfill{\scriptsize (attribute)}}\\


 It may happen that, even when $G$ has a finite rewriting system, the double coset rewriting system is infinite.
This is the case with the trefoil group $T$ with generators $[x,y]$ and relator $[x^3 = y^2]$ when the wreath product ordering is used with $X > x > Y > y$. The group itself has a rewriting system with just 6 rules. }

 
\begin{Verbatim}[commandchars=!@|,fontsize=\small,frame=single,label=Example]
  
  !gapprompt@gap>| !gapinput@FT := FreeGroup( 2 );;|
  !gapprompt@gap>| !gapinput@relsT := [ FT.1^3*FT.2^-2 ];;|
  !gapprompt@gap>| !gapinput@T := FT/relsT;;|
  !gapprompt@gap>| !gapinput@genT := GeneratorsOfGroup( T );;|
  !gapprompt@gap>| !gapinput@x := genT[1];  y := genT[2];|
  !gapprompt@gap>| !gapinput@alphT := "XxYy";;|
  !gapprompt@gap>| !gapinput@ordT := [4,3,2,1];;|
  !gapprompt@gap>| !gapinput@orderT := "wreath";;|
  !gapprompt@gap>| !gapinput@rwsT := ReducedConfluentRewritingSystem( T, ordT, orderT, 0, alphT );|
  !gapprompt@gap>| !gapinput@DisplayRwsRules( rwsT );;|
  [ [ Yy, id ], [ yY, id ], [ xxx, yy ], [ yyx, xyy ], [ X, xxYY ], [ Yx, yxYY ]\
   ]
  !gapprompt@gap>| !gapinput@accT := WordAcceptorOfReducedRws( rwsT );|
  < deterministic automaton on 4 letters with 7 states >
  !gapprompt@gap>| !gapinput@Print( accT );|
  Automaton("nondet",7,"yYxX",[ [ [ 1 ], [ 4 ], [ 1 ], [ 4, 7 ], [ 4 ], [ 4 ], [\
   4, 7 ] ], [ [ 1 ], [ 3 ], [ 3 ], [ 1 ], [ 3 ], [ 3 ], [ 1 ] ], [ [ 1 ], [ 5 ]\
  , [ 1 ], [ 5 ], [ 5, 6 ], [ 1 ], [ 1 ] ], [ [ 1 ], [ 1 ], [ 1 ], [ 1 ], [ 1 ],\
   [ 1 ], [ 1 ] ] ],[ 2 ],[ 1 ]);;
  !gapprompt@gap>| !gapinput@langT := FAtoRatExp( accT );|
  (xx*(xyUy)Uy)(xx*(xyUy)Uyy*yUy)*(xx*((xYUY)Y*(yUxUX)Ux(xUX)UX)(yUYUxUX)*U(yy*(\
  YUxUX)UYUX)(yUYUxUX)*)Uxx*((xYUY)Y*(yUxUX)Ux(xUX)UX)(yUYUxUX)*U(YY*(yUxUX)UX)(\
  yUYUxUX)*(yxUx)((xyUy)x)*
  !gapprompt@gap>| !gapinput@r := RationalExpression( "((xyUy)y*UxY*UY*)Uyy*UY*" ); |
  (xyUy)y*UxY*UY*Uyy*UY*
  !gapprompt@gap>| !gapinput@AreEqualLang( langT, r );|
  The given languages are not over the same alphabet
  false
  
\end{Verbatim}
 In a previous version of this package the expression \texttt{r}, which does not involve the letter \texttt{X}, was returned as the language \texttt{langT}. This discrepancy should be investigated. 

 Taking subgroups $H$, $K$ to be generated by $x$ and $y$ respectively, the double coset rewriting system has an infinite number of $H$-rules. It turns out that only a finite number of these are needed to produce
the required automaton. The function \texttt{PartialDoubleCosetRewritingSystem} allows a limit to be specified on the number of rules to be computed. In the
listing below a limit of 20 is used, but in fact 10 is sufficient. 
\begin{Verbatim}[commandchars=!@|,fontsize=\small,frame=single,label=Example]
  
  !gapprompt@gap>| !gapinput@prwsT := PartialDoubleCosetRewritingSystem( T, [x], [y], rwsT, 20 );;|
  #I WARNING: reached supplied limit 20 on number of rules
  !gapprompt@gap>| !gapinput@DisplayRwsRules( prwsT );|
  G-rules:
  [ [ X, xxYY ], [ Yx, yxYY ], [ Yy, id ], [ yY, id ], [ xxx, yy ], [ yyx, xyy ]\
   ]
  H-rules:
  [ [ Hx, H ],
    [ HY, Hy ],
    [ Hyy, H ],
    [ Hyxyy, Hyx ],
    [ HyxY, Hyxy ],
    [ Hyxyxyy, Hyxyx ],
    [ Hyxxyy, Hyxx ],
    [ HyxxY, Hyxxy ],
    [ HyxyxY, Hyxyxy ],
    [ Hyxyxyxyy, Hyxyxyx ],
    [ Hyxyxxyy, Hyxyxx ],
    [ Hyxxyxyy, Hyxxyx ],
    [ HyxxyxYY, Hyxxyx ] ]
  K-rules:
  [ [ YK, K ],
    [ yK, K ] ]
  
\end{Verbatim}
 This list of partial rules is then used by a modified word acceptor function. 
\begin{Verbatim}[commandchars=!@|,fontsize=\small,frame=single,label=Example]
  
  !gapprompt@gap>| !gapinput@paccT := WordAcceptorOfPartialDoubleCosetRws( T, prwsT );;|
  < deterministic automaton on 6 letters with 6 states >
  !gapprompt@gap>| !gapinput@Print( paccT, "\n" );|
  Automaton("det",6,"HKyYxX",[ [ 2, 2, 2, 6, 2, 2 ], [ 2, 2, 1, 2, 2, 1 ], [ 2, \
  2, 5, 2, 2, 5 ], [ 2, 2, 2, 2, 2, 2 ], [ 2, 2, 6, 2, 3, 2 ], [ 2, 2, 2, 2, 2, \
  2 ] ],[ 4 ],[ 1 ]);;
  !gapprompt@gap>| !gapinput@plangT := FAtoRatExp( paccT );|
  H(yx(yx)*x)*(yx(yx)*KUK)
  !gapprompt@gap>| !gapinput@wordsT := ["HK", "HxK", "HyK", "HYK", "HyxK", "HyxxK", "HyyH", "HyxYK"];;|
  !gapprompt@gap>| !gapinput@validT := List( wordsT, w -> IsRecognizedByAutomaton( paccT, w ) );|
  [ true, false, false, false, true, true, false, false ]
  
\end{Verbatim}
 }

 
\section{\textcolor{Chapter }{Example 3 -- an infinite rewriting system}}\logpage{[ 2, 4, 0 ]}
\hyperdef{L}{X86DACE357868DC1B}{}
{
 \index{example -- infinite rws} 

\subsection{\textcolor{Chapter }{KBMagRewritingSystem}}
\logpage{[ 2, 4, 1 ]}\nobreak
\hyperdef{L}{X8722C57284F51940}{}
{\noindent\textcolor{FuncColor}{$\triangleright$\ \ \texttt{KBMagRewritingSystem({\mdseries\slshape fpgrp})\index{KBMagRewritingSystem@\texttt{KBMagRewritingSystem}}
\label{KBMagRewritingSystem}
}\hfill{\scriptsize (attribute)}}\\
\noindent\textcolor{FuncColor}{$\triangleright$\ \ \texttt{KBMagWordAcceptor({\mdseries\slshape fpgrp})\index{KBMagWordAcceptor@\texttt{KBMagWordAcceptor}}
\label{KBMagWordAcceptor}
}\hfill{\scriptsize (attribute)}}\\
\noindent\textcolor{FuncColor}{$\triangleright$\ \ \texttt{KBMagFSAtoAutomataDFA({\mdseries\slshape fsa, alph})\index{KBMagFSAtoAutomataDFA@\texttt{KBMagFSAtoAutomataDFA}}
\label{KBMagFSAtoAutomataDFA}
}\hfill{\scriptsize (operation)}}\\
\noindent\textcolor{FuncColor}{$\triangleright$\ \ \texttt{WordAcceptorByKBMag({\mdseries\slshape grp, alph})\index{WordAcceptorByKBMag@\texttt{WordAcceptorByKBMag}}
\label{WordAcceptorByKBMag}
}\hfill{\scriptsize (operation)}}\\
\noindent\textcolor{FuncColor}{$\triangleright$\ \ \texttt{WordAcceptorByKBMagOfDoubleCosetRws({\mdseries\slshape grp, dcrws})\index{WordAcceptorByKBMagOfDoubleCosetRws@\texttt{WordAcceptorByKBMagOfDoubleCosetRws}}
\label{WordAcceptorByKBMagOfDoubleCosetRws}
}\hfill{\scriptsize (operation)}}\\


 When the group $G$ has an infinite rewriting system, the double coset rewriting system will also
be infinite. In this case we may use the function \texttt{KBMagWordAcceptor} which calls \textsf{KBMAG} to compute a word acceptor for $G$, and \texttt{KBMagFSAtoAutomataDFA} to convert this to a deterministic automaton as used by the \textsf{automata} package. The resulting \texttt{dfa} forms part of the double coset automaton, together with sufficient $H$-rules, $K$-rules and $H$-$K$-rules found by the function \texttt{PartialDoubleCosetRewritingSystem}. (Note that these five attributes and operations will not be available if the \textsf{kbmag} package has not been loaded.) 

 In the following example we take a two generator group $G3$ with relators $[a^3,b^3,(a*b)^3]$, the normal forms of whose elements are some of the strings with $a$ or $a^{-1}$ alternating with $b$ or $b^{-1}$. The automatic structure computed by \textsf{KBMAG} has a word acceptor with 17 states. }

 
\begin{Verbatim}[commandchars=!|C,fontsize=\small,frame=single,label=Example]
  
  !gapprompt|gap>C !gapinput|F3 := FreeGroup("a","b");;C
  !gapprompt|gap>C !gapinput|rels3 := [ F3.1^3, F3.2^3, (F3.1*F3.2)^3 ];;C
  !gapprompt|gap>C !gapinput|G3 := F3/rels3;;C
  !gapprompt|gap>C !gapinput|alph3 := "AaBb";;C
  !gapprompt|gap>C !gapinput|waG3 := WordAcceptorByKBMag( G3, alph3 );;C
  !gapprompt|gap>C !gapinput|Print( waG3, "\n");C
  Automaton("det",18,"aAbB",[ [ 2, 18, 18, 8, 10, 12, 13, 18, 18, 18, 18, 18, 18\
  , 8, 17, 12, 18, 18 ], [ 3, 18, 18, 9, 11, 9, 12, 18, 18, 18, 18, 18, 18, 11, \
  18, 11, 18, 18 ], [ 4, 6, 6, 18, 18, 18, 18, 18, 6, 12, 16, 18, 12, 18, 18, 18\
  , 12, 18 ], [ 5, 5, 7, 18, 18, 18, 18, 14, 15, 5, 18, 18, 7, 18, 18, 18, 15, 1\
  8 ] ],[ 1 ],[ 1 .. 17 ]);;
  !gapprompt|gap>C !gapinput|langG3 := FAtoRatExp( waG3 );C
  ((abUAb)AUbA)(bA)*(b(aU@)UB(aB)*(a(bU@)U@)U@)U(abUAb)(aU@)U((aBUB)(aB)*AUba(Ba\
  )*BA)(bA)*(b(aU@)U@)U(aBUB)(aB)*(a(bU@)U@)Uba(Ba)*(BU@)UbUaUA(B(aB)*(a(bU@)UAU\
  @)U@)U@
  
\end{Verbatim}
 

\subsection{\textcolor{Chapter }{DCrules}}
\logpage{[ 2, 4, 2 ]}\nobreak
\hyperdef{L}{X815C08FD87D014B5}{}
{\noindent\textcolor{FuncColor}{$\triangleright$\ \ \texttt{DCrules({\mdseries\slshape dcrws})\index{DCrules@\texttt{DCrules}}
\label{DCrules}
}\hfill{\scriptsize (operation)}}\\
\noindent\textcolor{FuncColor}{$\triangleright$\ \ \texttt{Hrules({\mdseries\slshape dcrws})\index{Hrules@\texttt{Hrules}}
\label{Hrules}
}\hfill{\scriptsize (attribute)}}\\
\noindent\textcolor{FuncColor}{$\triangleright$\ \ \texttt{Krules({\mdseries\slshape dcrws})\index{Krules@\texttt{Krules}}
\label{Krules}
}\hfill{\scriptsize (attribute)}}\\
\noindent\textcolor{FuncColor}{$\triangleright$\ \ \texttt{HKrules({\mdseries\slshape dcrws})\index{HKrules@\texttt{HKrules}}
\label{HKrules}
}\hfill{\scriptsize (attribute)}}\\


 We now take $H$ to be generated by $ab$ and $K$ to be generated by $ba$. If we specify a limits of 50, 75, 100, 200 for the number of rules in a
partial double coset rewrite system, we obtain lists of $H$-rules, $K$-rules and $H$-$K$-rules of increasing length. The numbers of states in the resulting automata
also increase. We may deduce by hand (but not computationally -- see \cite{BrGhHeWe}) three infinite sets of rules and a limit for the automata. }

 
\begin{Verbatim}[commandchars=@|J,fontsize=\small,frame=single,label=Example]
  
  @gapprompt|gap>J @gapinput|lim := 100;;J
  @gapprompt|gap>J @gapinput|genG3 := GeneratorsOfGroup( G3 );;J
  @gapprompt|gap>J @gapinput|a := genG3[1];;  b := genG3[2];; J
  @gapprompt|gap>J @gapinput|gH3 := [ a*b ];;  gK3 := [ b*a ];;J
  @gapprompt|gap>J @gapinput|rwsG3 := KnuthBendixRewritingSystem( G3, "shortlex", [2,1,4,3], alph3 );;J
  @gapprompt|gap>J @gapinput|dcrws3 := PartialDoubleCosetRewritingSystem( G3, gH3, gK3, rwsG3, lim );;J
  #I using PartialDoubleCosetRewritingSystem with limit 100
  #I  51 rules, and 1039 pairs
  #I WARNING: reached supplied limit 100 on number of rules
  @gapprompt|gap>J @gapinput|Print( Length( Rules( dcrws3 ) ), " rules found.\n" );J
  101 rules found.
  @gapprompt|gap>J @gapinput|dcaut3 := WordAcceptorByKBMagOfDoubleCosetRws( G3, dcrws3 );;J
  @gapprompt|gap>J @gapinput|Print( "Double Coset Minimalized automaton:\n", dcaut3 );J
  Double Coset Minimalized automaton:
  Automaton("det",44,"HKaAbB",[ [ 2, 2, 2, 5, 2, 2, 2, 2, 2, 2, 2, 2, 2, 2, 2, 2\
  , 2, 2, 2, 2, 2, 2, 2, 2, 2, 2, 2, 2, 2, 2, 2, 2, 2, 2, 2, 2, 2, 2, 2, 2, 2, 2\
  , 2, 2 ], [ 2, 2, 1, 2, 1, 2, 1, 1, 2, 1, 2, 1, 2, 1, 2, 1, 2, 1, 2, 1, 2, 1, \
  2, 2, 2, 1, 2, 1, 1, 2, 1, 2, 1, 2, 1, 2, 1, 2, 1, 2, 1, 2, 2, 1 ], [ 2, 2, 2,\
   2, 3, 24, 2, 2, 2, 2, 2, 2, 2, 2, 2, 2, 2, 2, 2, 2, 43, 2, 43, 2, 27, 2, 2, 2\
  , 2, 2, 2, 2, 2, 2, 2, 2, 2, 2, 2, 2, 2, 2, 2, 2 ], [ 2, 2, 2, 2, 44, 3, 29, 2\
  , 8, 2, 10, 2, 12, 2, 14, 2, 16, 2, 18, 2, 20, 2, 22, 2, 2, 2, 2, 26, 2, 29, 2\
  , 31, 2, 33, 2, 35, 2, 37, 2, 39, 2, 41, 2, 2 ], [ 2, 2, 2, 2, 21, 2, 2, 28, 2\
  , 9, 2, 11, 2, 13, 2, 15, 2, 17, 2, 19, 2, 42, 2, 3, 2, 28, 3, 2, 7, 2, 30, 2,\
   32, 2, 34, 2, 36, 2, 38, 2, 40, 2, 2, 28 ], [ 2, 2, 2, 2, 2, 2, 2, 2, 2, 2, 2\
  , 2, 2, 2, 2, 2, 2, 2, 2, 2, 2, 2, 2, 6, 2, 25, 25, 2, 2, 2, 2, 2, 2, 2, 2, 2,\
   2, 2, 2, 2, 2, 2, 23, 6 ] ],[ 4 ],[ 1 ]);;
  @gapprompt|gap>J @gapinput|dclang3 := FAtoRatExp( dcaut3 );;J
  @gapprompt|gap>J @gapinput|Print( "Double Coset language of acceptor:\n", dclang3, "\n" );J
  Double Coset language of acceptor:
  (HbAbAbAbAbAbAbAbUHAb)(Ab)*(A(Ba(Ba)*bKUK)UK)UHbAbA(bA(bA(bA(bA(bAKUK)UK)UK)UK\
  )UK)UH(A(B(aB)*(abUA)KUK)UaKUb(a(Ba)*BA(bA(bA(bA(bA(bA(bA(bA(bA)*(bKUK)UK)UK)U\
  K)UK)UK)UK)UK)UAK)UK)
  @gapprompt|gap>J @gapinput|ok := DCrules(dcrws3);;J
  @gapprompt|gap>J @gapinput|alph3e := dcrws3!.alphabet;;J
  @gapprompt|gap>J @gapinput|Print("H-rules:\n");  DisplayAsString( Hrules(dcrws3), alph3e, true );J
  H-rules:
  [ HB, Ha ]
  [ HaB, Hb ]
  [ Hab, H ]
  [ HbAB, HAba ]
  [ HbAbAB, HAbAba ]
  [ HbAbAbAB, HAbAbAba ]
  [ HbAbAbAbAB, HAbAbAbAba ]
  [ HbAbAbAbAbAB, HAbAbAbAbAba ]
  [ HbAbAbAbAbAbAB, HAbAbAbAbAbAba ]
  [ HbAbAbAbAbAbAbAB, HAbAbAbAbAbAbAba ]
  @gapprompt|gap>J @gapinput|Print("K-rules:\n");  DisplayAsString( Krules(dcrws3), alph3e, true );;J
  K-rules:
  [ BK, aK ]
  [ BaK, bK ]
  [ baK, K ]
  [ BAbK, abAK ]
  [ BAbAbK, abAbAK ]
  [ BAbAbAbK, abAbAbAK ]
  [ BAbAbAbAbK, abAbAbAbAK ]
  [ BAbAbAbAbAbK, abAbAbAbAbAK ]
  [ BAbAbAbAbAbAbK, abAbAbAbAbAbAK ]
  [ BAbAbAbAbAbAbAbK, abAbAbAbAbAbAbAK ]
  @gapprompt|gap>J @gapinput|Print("HK-rules:\n");  DisplayAsString( HKrules(dcrws3), alph3e, true );;J
  HK-rules:
  [ HbK, HAK ]
  [ HbAbK, HAbAK ]
  [ HbAbAbK, HAbAbAK ]
  [ HbAbAbAbK, HAbAbAbAK ]
  [ HbAbAbAbAbK, HAbAbAbAbAK ]
  [ HbAbAbAbAbAbK, HAbAbAbAbAbAK ]
  [ HbAbAbAbAbAbAbK, HAbAbAbAbAbAbAK ]
  
\end{Verbatim}
 

\subsection{\textcolor{Chapter }{NextWord}}
\logpage{[ 2, 4, 3 ]}\nobreak
\hyperdef{L}{X7AF0265982B42E47}{}
{\noindent\textcolor{FuncColor}{$\triangleright$\ \ \texttt{NextWord({\mdseries\slshape dcrws, word})\index{NextWord@\texttt{NextWord}}
\label{NextWord}
}\hfill{\scriptsize (operation)}}\\
\noindent\textcolor{FuncColor}{$\triangleright$\ \ \texttt{WordToString({\mdseries\slshape word, alph})\index{WordToString@\texttt{WordToString}}
\label{WordToString}
}\hfill{\scriptsize (operation)}}\\
\noindent\textcolor{FuncColor}{$\triangleright$\ \ \texttt{DisplayAsString({\mdseries\slshape word, alph})\index{DisplayAsString@\texttt{DisplayAsString}}
\label{DisplayAsString}
}\hfill{\scriptsize (operation)}}\\
\noindent\textcolor{FuncColor}{$\triangleright$\ \ \texttt{IdentityDoubleCoset({\mdseries\slshape dcrws})\index{IdentityDoubleCoset@\texttt{IdentityDoubleCoset}}
\label{IdentityDoubleCoset}
}\hfill{\scriptsize (operation)}}\\


 These functions may be used to find normal forms of increasing length for
double coset representatives. }

 
\begin{Verbatim}[commandchars=!@|,fontsize=\small,frame=single,label=Example]
  
  !gapprompt@gap>| !gapinput@len := 30;;|
  !gapprompt@gap>| !gapinput@L3 := 0*[1..len];;|
  !gapprompt@gap>| !gapinput@L3[1] := IdentityDoubleCoset( dcrws3 );;|
  !gapprompt@gap>| !gapinput@for i in [2..len] do|
  !gapprompt@gap>| !gapinput@    L3[i] := NextWord( dcrws3, L3[i-1] );|
  !gapprompt@gap>| !gapinput@od;|
  !gapprompt@gap>| !gapinput@## List of 30 normal forms for double cosets:|
  !gapprompt@gap>| !gapinput@DisplayAsString( L3, alph3e, true );|
  [ HK, HAK, HaK, HAbK, HbAK, HABAK, HAbAK, HABabK, HAbAbK, HbAbAK, HbaBAK, HABa\
  BAK, HAbAbAK, HABaBabK, HAbABabK, HAbAbAbK, HbAbAbAK, HbaBAbAK, HbaBaBAK, HABa\
  BaBAK, HAbAbAbAK, HABaBaBabK, HAbABaBabK, HAbAbABabK, HAbAbAbAbK, HbAbAbAbAK, \
  HbaBAbAbAK, HbaBaBAbAK, HbaBaBaBAK, HABaBaBaBAK ]
  !gapprompt@gap>| !gapinput@w := NextWord( dcrws3, L3[30] );|
  m1*(m3*m6)^4*m3*m2
  !gapprompt@gap>| !gapinput@s := WordToString( w, alph3e );|
  "HAbAbAbAbAK"
  
\end{Verbatim}
 }

 }

         
\chapter{\textcolor{Chapter }{Development History}}\label{chap-history}
\logpage{[ 3, 0, 0 ]}
\hyperdef{L}{X810C43BC7F63C4B4}{}
{
  
\section{\textcolor{Chapter }{Versions of the package}}\logpage{[ 3, 1, 0 ]}
\hyperdef{L}{X8192EA4C7B7CC5CD}{}
{
 The first version of the package, written for \textsf{GAP} 3, formed part of Anne Heyworth's thesis \cite{anne-thesis} in 1999, but was not made generally available. 

 Version 0.91 was prepared to run under \textsf{GAP} 4.4.6, in July 2005. 

 Version 0.94 differed in two significant ways. 
\begin{itemize}
\item  The manual was produced using the \textsf{GAPDoc} package. 
\item  The test file \texttt{kan/tst/kan{\textunderscore}manual.tst} set the \texttt{AssertionLevel} to \texttt{0} to avoid recursion in the \textsf{Automata} package. 
\end{itemize}
 

 Version 0.95, of 9th October 2007, just fixed file protections and added a \texttt{CHANGES} file. 

 Version 0.96 was required because the \textsf{Kan} website moved with the rest of the Mathematics website at Bangor. 

 Version 0.97, of November 18th 2008, deleted temporary fixes which were no
longer needed once version 1.12 of \textsf{Automata} became available. 

 Version 1.01, of August 2011, included minor changes required for \textsf{GAP} 4.5. In particular, the test file \texttt{kan{\textunderscore}manual.tst} was replaced by the pair \texttt{fulltest.tst} and \texttt{parttest.tst}. 

 Version 1.11, of December 2014 was required when the \textsf{Kan} website moved yet again. At the same time a bitbucket repository for the
package was started. 

 \textsf{Kan} became an accepted \textsf{GAP} package in May 2015. 

 The latest version is 1.21, released on 2nd June 2015. }

 
\section{\textcolor{Chapter }{What needs doing next?}}\logpage{[ 3, 2, 0 ]}
\hyperdef{L}{X83D1530487593182}{}
{
 There are too many items to list here, but some of the most important are as
follows. 
\begin{itemize}
\item  Implement iterators and enumerators for double cosets. 
\item  At present the methods for \texttt{DoubleCosetsNC} and \texttt{RightCosetsNC} in this package return automata, rather than lists of cosets or coset
enumerators. This needs to be fixed. 
\item  Provide methods for operations such as \texttt{DoubleCosetRepsAndSizes}. 
\item  Convert the rest of the original \textsf{GAP} 3 version of \textsf{Kan} to \textsf{GAP} 4. 
\end{itemize}
 

\subsection{\textcolor{Chapter }{DoubleCosetsAutomaton}}
\logpage{[ 3, 2, 1 ]}\nobreak
\hyperdef{L}{X852CC057809CE3EE}{}
{\noindent\textcolor{FuncColor}{$\triangleright$\ \ \texttt{DoubleCosetsAutomaton({\mdseries\slshape G, U, V})\index{DoubleCosetsAutomaton@\texttt{DoubleCosetsAutomaton}}
\label{DoubleCosetsAutomaton}
}\hfill{\scriptsize (operation)}}\\
\noindent\textcolor{FuncColor}{$\triangleright$\ \ \texttt{RightCosetsAutomaton({\mdseries\slshape G, V})\index{RightCosetsAutomaton@\texttt{RightCosetsAutomaton}}
\label{RightCosetsAutomaton}
}\hfill{\scriptsize (operation)}}\\


 Alternative methods for \texttt{DoubleCosetsNC(G,U,V)} and \texttt{RightCosetsNC(G,V)} \emph{should be} provided in the cases where the group \texttt{G} has a rewriting system or is known to be infinite. At present the functions \texttt{RightCosetsAutomaton} and \texttt{DoubleCosetsAutomaton} return minimized automata, and \texttt{Iterators} for these are not yet available. }

 
\begin{Verbatim}[commandchars=!@|,fontsize=\small,frame=single,label=Example]
  
  !gapprompt@gap>| !gapinput@F := FreeGroup(2);;|
  !gapprompt@gap>| !gapinput@rels := [ F.2^2, (F.1*F.2)^2 ];;|
  !gapprompt@gap>| !gapinput@G4 := F/rels;;|
  !gapprompt@gap>| !gapinput@genG4 := GeneratorsOfGroup( G4 );;|
  !gapprompt@gap>| !gapinput@a := genG4[1];  b := genG4[2];;|
  !gapprompt@gap>| !gapinput@U := Subgroup( G4, [a^2] );;|
  !gapprompt@gap>| !gapinput@V := Subgroup( G4, [b] );;|
  !gapprompt@gap>| !gapinput@dc4 := DoubleCosetsAutomaton( G4, U, V );;|
  !gapprompt@gap>| !gapinput@Print( dc4 );|
  Automaton("det",5,"HKaAbB",[ [ 2, 2, 2, 5, 2 ], [ 2, 2, 1, 2, 1 ], [ 2, 2, 2, \
  2, 3 ], [ 2, 2, 2, 2, 2 ], [ 2, 2, 2, 2, 2 ], [ 2, 2, 2, 2, 2 ] ],[ 4 ],[ 1 ])\
  ;;
  !gapprompt@gap>| !gapinput@rc4 := RightCosetsAutomaton( G4, V );;|
  !gapprompt@gap>| !gapinput@Print( rc4 );|
  Automaton("det",6,"HKaAbB",[ [ 2, 2, 2, 6, 2, 2 ], [ 2, 2, 1, 2, 1, 1 ], [ 2, \
  2, 3, 2, 2, 3 ], [ 2, 2, 2, 2, 5, 5 ], [ 2, 2, 2, 2, 2, 2 ], [ 2, 2, 2, 2, 2, \
  2 ] ],[ 4 ],[ 1 ]);;
  
\end{Verbatim}
 }

 }

 \def\bibname{References\logpage{[ "Bib", 0, 0 ]}
\hyperdef{L}{X7A6F98FD85F02BFE}{}
}

\bibliographystyle{alpha}
\bibliography{kanbib.xml}

\addcontentsline{toc}{chapter}{References}

\def\indexname{Index\logpage{[ "Ind", 0, 0 ]}
\hyperdef{L}{X83A0356F839C696F}{}
}

\cleardoublepage
\phantomsection
\addcontentsline{toc}{chapter}{Index}


\printindex

\newpage
\immediate\write\pagenrlog{["End"], \arabic{page}];}
\immediate\closeout\pagenrlog
\end{document}
