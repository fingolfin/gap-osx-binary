% generated by GAPDoc2LaTeX from XML source (Frank Luebeck)
\documentclass[a4paper,11pt]{report}

\usepackage{a4wide}
\sloppy
\pagestyle{myheadings}
\usepackage{amssymb}
\usepackage[latin1]{inputenc}
\usepackage{makeidx}
\makeindex
\usepackage{color}
\definecolor{FireBrick}{rgb}{0.5812,0.0074,0.0083}
\definecolor{RoyalBlue}{rgb}{0.0236,0.0894,0.6179}
\definecolor{RoyalGreen}{rgb}{0.0236,0.6179,0.0894}
\definecolor{RoyalRed}{rgb}{0.6179,0.0236,0.0894}
\definecolor{LightBlue}{rgb}{0.8544,0.9511,1.0000}
\definecolor{Black}{rgb}{0.0,0.0,0.0}

\definecolor{linkColor}{rgb}{0.0,0.0,0.554}
\definecolor{citeColor}{rgb}{0.0,0.0,0.554}
\definecolor{fileColor}{rgb}{0.0,0.0,0.554}
\definecolor{urlColor}{rgb}{0.0,0.0,0.554}
\definecolor{promptColor}{rgb}{0.0,0.0,0.589}
\definecolor{brkpromptColor}{rgb}{0.589,0.0,0.0}
\definecolor{gapinputColor}{rgb}{0.589,0.0,0.0}
\definecolor{gapoutputColor}{rgb}{0.0,0.0,0.0}

%%  for a long time these were red and blue by default,
%%  now black, but keep variables to overwrite
\definecolor{FuncColor}{rgb}{0.0,0.0,0.0}
%% strange name because of pdflatex bug:
\definecolor{Chapter }{rgb}{0.0,0.0,0.0}
\definecolor{DarkOlive}{rgb}{0.1047,0.2412,0.0064}


\usepackage{fancyvrb}

\usepackage{mathptmx,helvet}
\usepackage[T1]{fontenc}
\usepackage{textcomp}


\usepackage[
            pdftex=true,
            bookmarks=true,        
            a4paper=true,
            pdftitle={Written with GAPDoc},
            pdfcreator={LaTeX with hyperref package / GAPDoc},
            colorlinks=true,
            backref=page,
            breaklinks=true,
            linkcolor=linkColor,
            citecolor=citeColor,
            filecolor=fileColor,
            urlcolor=urlColor,
            pdfpagemode={UseNone}, 
           ]{hyperref}

\newcommand{\maintitlesize}{\fontsize{36}{38}\selectfont}

% write page numbers to a .pnr log file for online help
\newwrite\pagenrlog
\immediate\openout\pagenrlog =\jobname.pnr
\immediate\write\pagenrlog{PAGENRS := [}
\newcommand{\logpage}[1]{\protect\write\pagenrlog{#1, \thepage,}}
%% were never documented, give conflicts with some additional packages

\newcommand{\GAP}{\textsf{GAP}}

%% nicer description environments, allows long labels
\usepackage{enumitem}
\setdescription{style=nextline}

%% depth of toc
\setcounter{tocdepth}{1}




\makeatletter 
\renewcommand*\l@section{\@dottedtocline{1}{1.5em}{2.8em}}
\renewcommand*\l@subsection{\@dottedtocline{2}{3.8em}{4.0em}}
\def\@pnumwidth{2.1em}
\makeatother


%% command for ColorPrompt style examples
\newcommand{\gapprompt}[1]{\color{promptColor}{\bfseries #1}}
\newcommand{\gapbrkprompt}[1]{\color{brkpromptColor}{\bfseries #1}}
\newcommand{\gapinput}[1]{\color{gapinputColor}{#1}}


\begin{document}

\logpage{[ 0, 0, 0 ]}
\begin{titlepage}
\mbox{}\vfill

\begin{center}{\maintitlesize \textbf{GAP - Changes from Earlier Versions\mbox{}}}\\
\vfill

\hypersetup{pdftitle=GAP - Changes from Earlier Versions}
\markright{\scriptsize \mbox{}\hfill GAP - Changes from Earlier Versions \hfill\mbox{}}
{\Huge Release 4.7.8, 09-Jun-2015\mbox{}}\\[1cm]
\mbox{}\\[2cm]
{\Large \textbf{The GAP Group   \mbox{}}}\\
\hypersetup{pdfauthor=The GAP Group   }
\end{center}\vfill

\mbox{}\\
{\mbox{}\\
\small \noindent \textbf{The GAP Group   }  Email: \href{mailto://support@gap-system.org} {\texttt{support@gap-system.org}}\\
  Homepage: \href{http://www.gap-system.org} {\texttt{http://www.gap-system.org}}}\\
\end{titlepage}

\newpage\setcounter{page}{2}
{\small 
\section*{Copyright}
\logpage{[ 0, 0, 1 ]}
 Copyright {\copyright} (1987-2015) for the core part of the \textsf{GAP} system by the \textsf{GAP} Group. 

 Most parts of this distribution, including the core part of the \textsf{GAP} system are distributed under the terms of the GNU General Public License, see \href{http://www.gnu.org/licenses/gpl.html} {\texttt{http://www.gnu.org/licenses/gpl.html}} or the file \texttt{GPL} in the \texttt{etc} directory of the \textsf{GAP} installation. 

 More detailed information about copyright and licenses of parts of this
distribution can be found in Section  (\textbf{Reference: Copyright and License}) of the \textsf{GAP} reference manual. 

 \textsf{GAP} is developed over a long time and has many authors and contributors. More
detailed information can be found in Section  (\textbf{Reference: Authors and Maintainers}) of the \textsf{GAP} reference manual. \mbox{}}\\[1cm]
\newpage

\def\contentsname{Contents\logpage{[ 0, 0, 2 ]}}

\tableofcontents
\newpage

 
\chapter{\textcolor{Chapter }{Preface}}\label{Preface}
\logpage{[ 1, 0, 0 ]}
\hyperdef{L}{X874E1D45845007FE}{}
{
  This is one of the three main \textsf{GAP} books. It describes most essential changes from previous \textsf{GAP} releases. 

 In addition to this manual, there is the \emph{\textsf{GAP} Tutorial}  and the \emph{\textsf{GAP} Reference Manual}  containing detailed documentation of the mathematical functionality of \textsf{GAP}. 

 A lot of the functionality of the system and a number of contributed
extensions are provided as \textsf{GAP} packages, and each of these has its own manual. New versions of packages are
released independently of \textsf{GAP} releases, and changes between package versions may be described in their
documentation. 

 }

    
\chapter{\textcolor{Chapter }{Changes between \textsf{GAP} 4.6 and \textsf{GAP} 4.7}}\label{ChangesGAP46toGAP47}
\logpage{[ 2, 0, 0 ]}
\hyperdef{L}{X871220D17E7EE651}{}
{
  This chapter contains an overview of most important changes introduced in \textsf{GAP} 4.7.2 release (the first public release of \textsf{GAP} 4.7). It also contains information about subsequent update releases for \textsf{GAP} 4.7. 
\section{\textcolor{Chapter }{\textsf{GAP} 4.7.2 (December 2013)}}\label{fix472}
\logpage{[ 2, 1, 0 ]}
\hyperdef{L}{X78594AB2871B379E}{}
{
  
\subsection{\textcolor{Chapter }{Changes in the core \textsf{GAP} system introduced in \textsf{GAP} 4.7}}\label{Changes in the core GAP47 system}
\logpage{[ 2, 1, 1 ]}
\hyperdef{L}{X8346E4677FA78976}{}
{
  Improved and extended functionality: 
\begin{itemize}
\item  The methods for computing conjugacy classes of permutation groups have been
rewritten from scratch to enable potential use for groups in different
representations. As a byproduct the resulting code is (sometimes notably)
faster. It also now is possible to calculate canonical conjugacy class
representatives in permutation groups, which can be beneficial when
calculating character tables. 
\item  The methods for determining (conjugacy classes of) subgroups in non-solvable
groups have been substantially improved in speed and scope for groups with
multiple nonabelian composition factors. 
\item  There is a new method for calculating the maximal subgroups of a permutation
group (with chief factors of width less or equal 5) without calculating the
whole subgroup lattice. 
\item  If available, information from the table of marks library is used to speed up
subgroup calculations in almost simple factor groups. 
\item  The broader availability of maximal subgroups is used to improve the
calculation of double cosets. 
\item  To illustrate the improvements listed above, one could try, for example 
\begin{Verbatim}[commandchars=!@|,fontsize=\small,frame=single,label=Example]
  g:=WreathProduct(MathieuGroup(11),Group((1,2)));
  Length(ConjugacyClassesSubgroups(g));
\end{Verbatim}
 and 
\begin{Verbatim}[commandchars=!@|,fontsize=\small,frame=single,label=Example]
  g:=SemidirectProduct(GL(3,5),GF(5)^3);
  g:=Image(IsomorphismPermGroup(g));
  MaximalSubgroupClassReps(g);
\end{Verbatim}
 
\item  Computing the exponent of a finite group $G$ could be extremely slow. This was due to a slow default method being used,
which computed all conjugacy classes of elements in order to compute the
exponent. We now instead compute Sylow subgroups $P_1$, ..., $P_k$ of $G$ and use the easily verified equality $exp(G) = exp(P_1) x ... x exp(P_k)$. This is usually at least as fast and in many cases orders of magnitude
faster. 
\begin{Verbatim}[commandchars=!@|,fontsize=\small,frame=single,label=Example]
  !gapprompt@gap>| !gapinput@G:=SmallGroup(2^7*9,33);;|
  !gapprompt@gap>| !gapinput@H:=DirectProduct(G, ElementaryAbelianGroup(2^10));;|
  !gapprompt@gap>| !gapinput@Exponent(H); # should take at most a few milliseconds|
  72
  !gapprompt@gap>| !gapinput@K := PerfectGroup(2688,3);;|
  !gapprompt@gap>| !gapinput@Exponent(K); # should take at most a few seconds|
  168
\end{Verbatim}
 
\item   The functionality in \textsf{GAP} for transformations and transformation semigroups has been rewritten and
extended. Partial permutations and inverse semigroups have been newly
implemented. The documentation for transformations and transformation
semigroups has been improved. Transformations and partial permutations are
implemented in the \textsf{GAP} kernel. Methods for calculating attributes of transformations and partial
permutations, and taking products, and so are also implemented in the kernel.
The new implementations are largely backwards compatible; some exceptions are
given below. 

 The degree of a transformation \texttt{f} is usually defined as the largest positive integer where \texttt{f} is defined. In previous versions of \textsf{GAP}, transformations were only defined on positive integers less than their
degree, it was only possible to multiply transformations of equal degree, and
a transformation did not act on any point exceeding its degree. Starting with \textsf{GAP} 4.7, transformations behave more like permutations, in that they fix
unspecified points and it is possible to multiply arbitrary transformations. 
\begin{itemize}
\item  in the display of a transformation, the trailing fixed points are no longer
printed. More precisely, in the display of a transformation \texttt{f} if \texttt{n} is the largest value such that \texttt{n\texttt{\symbol{94}}f{\textless}{\textgreater}n} or \texttt{i\texttt{\symbol{94}}f=n} for some \texttt{i{\textless}{\textgreater}n}, then the values exceeding \texttt{n} are not printed. 
\item  the display for semigroups of transformations now includes more information,
for example \texttt{{\textless}transformation semigroup on 10 pts with 10 generators{\textgreater}} and \texttt{{\textless}inverse partial perm semigroup on 10 pts with 10
generators{\textgreater}}. 
\item  transformations which define a permutation can be inverted, and groups of
transformations can be created. 
\end{itemize}
 Further information regarding transformations and partial permutations, can be
found in the relevant chapters of the reference manual. 

 The code for Rees matrix semigroups has been completely rewritten to fix the
numerous bugs in the previous versions. The display of a Rees matrix semigroup
has also been improved to include the numbers of rows and columns, and the
underlying semigroup. Again the new implementations should be backwards
compatible with the exception that the display is different. 

 The code for magmas with a zero adjoined has been improved so that it is
possible to access more information about the original magma. The display has
also been changed to indicate that the created magma is a magma with zero
adjoined (incorporating the display of the underlying magma). Elements of a
magma with zero are also printed so that it is clear that they belong to a
magma with zero. 

 If a semigroup is created by generators in the category
IsMultiplicativeElementWithOneCollection and CanEasilyCompareElements, then it
is now checked if the One of the generators is given as a generator. In this
case, the semigroup is created as a monoid. 
\item  Added a new operation \texttt{GrowthFunctionOfGroup} (\textbf{Reference: GrowthFunctionOfGroup}) that gives sizes of distance spheres in the Cayley graph of a group. 
\item   A new group constructor \texttt{FreeAbelianGroup} (\textbf{Reference: FreeAbelianGroup}) for free abelian groups has been added. By default, it creates suitable fp
groups. Though free abelian groups groups do not offer much functionality
right now, in the future other implementations may be provided, e.g. by the \textsf{Polycyclic} package. 
\item   The message about halving the pool size at startup is only shown when \texttt{-D} command line option is used (see  (\textbf{Reference: Command Line Options})). [Suggested by Volker Braun] 
\item   An info class called \texttt{InfoObsolete} (\textbf{Reference: InfoObsolete}) with the default level 0 is introduced. Setting it to 1 will trigger warnings
at runtime if an obsolete variable declared with \texttt{DeclareObsoleteSynonym} is used. This is recommended for testing \textsf{GAP} distribution and packages. 
\item  The \textsf{GAP} help system now recognises some common different spelling patterns (for
example, -ise/-ize, -isation/-ization, solvable/soluble) and searches for all
possible spelling options even when the synonyms are not declared. 
\item   Added new function \texttt{Cite} (\textbf{Reference: Cite}) which produces citation samples for \textsf{GAP} and packages. 
\item   It is now possible to compile \textsf{GAP} with user-supplied \texttt{CFLAGS} which now will not be overwritten by \textsf{GAP} default settings. [Suggested by Jeroen Demeyer] 
\end{itemize}
 Fixed bugs: 
\begin{itemize}
\item   \texttt{Union} (\textbf{Reference: Union}) had $O(n^3)$ behaviour when given many ranges (e.g. it could take 10 seconds to find a
union of 1000 1-element sets). The new implementation reduces that to $O(n log n)$ (and 4ms for the 10 second example), at the cost of not merging ranges as well
as before in some rare cases. 
\item   \texttt{IsLatticeOrderBinaryRelation} only checked the existence of upper bounds but not the uniqueness of the least
upper bound (and dually for lower bounds), so in some cases it could return
the wrong answer. [Reported by Attila Egri-Nagy] 
\item   \texttt{LowIndexSubgroupsFpGroup} (\textbf{Reference: LowIndexSubgroupsFpGroup}) triggered a break loop if the list of generators of the 2nd argument contained
the identity element of the group. [Reported by Ignat Soroko] 
\item   Fixed regression in heuristics used by \texttt{NaturalHomomorphismByNormalSubgroup} (\textbf{Reference: NaturalHomomorphismByNormalSubgroup}) that could produce a permutation representation of an unreasonably large
degree. [Reported by Izumi Miyamoto] 
\item   Fixed inconsistent behaviour of \texttt{QuotientMod( Integers, r, s, m )} in the case where \mbox{\texttt{\mdseries\slshape s}} and \mbox{\texttt{\mdseries\slshape m}} are not coprime. This fix also corrects the division behaviour of \texttt{ZmodnZ} objects, see \texttt{QuotientMod} (\textbf{Reference: QuotientMod}) and \texttt{ZmodnZ} (\textbf{Reference: ZmodnZ}). [Reported by Mark Dickinson] 
\item   Fixed an oversight in the loading process causing \texttt{OnQuit} (\textbf{Reference: OnQuit}) not resetting the options stack after exiting the break loop. 
\item   Empty strings were treated slightly differently than other strings in the \textsf{GAP} kernel, for historical reasons. This resulted in various inconsistencies. For
example, \texttt{IsStringRep("")} returned true, but a method installed for arguments of type \texttt{IsStringRep} would NOT be invoked when called with an empty string. 

 We remove this special case in the \textsf{GAP} kernel (which dates back the very early days of \textsf{GAP}{\nobreakspace}4 in 1996). This uncovered one issue in the kernel function \texttt{POSITION{\textunderscore}SUBSTRING} (when calling it with an empty string as second argument), which was also
fixed. 
\item   The parser for floating point numbers contained a bug that could cause \textsf{GAP} to crash or to get into a state where the only action left to the user was to
exit \textsf{GAP} via Ctrl-D. For example, entering four dots with spaces between them on the \textsf{GAP} prompt and then pressing the return key caused \textsf{GAP} to exit. 

 The reason was (ironically) an error check in the innards of the float parser
code which invoked the \textsf{GAP} \texttt{Error()} function at a point where it should not have. 
\item  Removing the last character in a string was supposed to overwrite the old
removed character in memory with a zero byte, but failed to do so due to an
off-by-one error. For most \textsf{GAP} operations, this has no visible effect, except for those which directly
operate on the underlying memory representation of strings. For example, when
trying to use such a string to reference a record entry, a (strange) error
could be triggered. 
\item  \texttt{ViewString} (\textbf{Reference: ViewString}) and \texttt{DisplayString} (\textbf{Reference: DisplayString}) are now handling strings, characters and immediate FFEs in a consistent
manner. 
\item   Multiple fixes to the build process for less common Debian platforms (arm,
ia64, mips, sparc, GNU/Hurd). [Suggested by Bill Allombert] 
\item   Fixes for several regressions in the \texttt{gac} script. [Suggested by Bill Allombert] 
\end{itemize}
 Changed functionality: 
\begin{itemize}
\item   It is not possible now to call \texttt{WreathProduct} (\textbf{Reference: WreathProduct}) with 2nd argument \mbox{\texttt{\mdseries\slshape H}} not being a permutation group, without using the 3rd argument specifying the
permutation representation. This is an incompatible change but it will produce
an error instead of a wrong result. The former behaviour of \texttt{WreathProduct} (\textbf{Reference: WreathProduct}) may now be achieved by using \texttt{StandardWreathProduct} (\textbf{Reference: StandardWreathProduct}) which returns the wreath product for the (right regular) permutation action of \mbox{\texttt{\mdseries\slshape H}} on its elements. 
\item   The function \texttt{ViewLength} to specify the maximal number of lines that are printed in \texttt{ViewObj} (\textbf{Reference: ViewObj}) became obsolete, since there was already a user preference \texttt{ViewLength} to specify this. The value of this preference is also accessible in \texttt{GAPInfo.ViewLength}. 
\end{itemize}
 }

 
\subsection{\textcolor{Chapter }{New and updated packages since \textsf{GAP} 4.6.5}}\label{New and updated packages since GAP 4.6.5}
\logpage{[ 2, 1, 2 ]}
\hyperdef{L}{X7EF0C705829B5D2B}{}
{
  \index{Packages, new} At the time of the release of \textsf{GAP} 4.6.5 there were 107 packages redistributed with \textsf{GAP}. The first public release of \textsf{GAP}{\nobreakspace}4.7 contains 114 packages. 

 One of essential changes is that the \textsf{Citrus} package by J.Mitchell has been renamed to \textsf{Semigroups}. The package has been completely overhauled, the performance has been
improved, and the code has been generalized so that in the future the same
code can be used to compute with other types of semigroups. 

 Furthermore, new packages that have been added to the redistribution since the
release of \textsf{GAP} 4.6.5 are: 
\begin{itemize}
\item  \textsf{4ti2interface} package by Sebastian Gutsche, providing an interface to \textsf{4ti2}, a software package for algebraic, geometric and combinatorial problems on
linear spaces (\href{http://www.4ti2.de} {\texttt{http://www.4ti2.de}}). 
\item  \textsf{CoReLG} by Heiko Dietrich, Paolo Faccin and Willem de Graaf for calculations in real
semisimple Lie algebras. 
\item  \textsf{IntPic} package by Manuel Delgado, aimed at providing a simple way of getting a
pictorial view of sets of integers. The main goal of the package is producing \textsf{Tikz} code for arrays of integers. The code produced is to be included in a {\LaTeX} file, which can then be processed. Some of the integers are emphasized by
using different colors for the cells containing them. 
\item  \textsf{LieRing} by Serena Cicalo and Willem de Graaf for constructing finitely-presented Lie
rings and calculating the Lazard correspondence. The package also provides a
database of small $n$-Engel Lie rings. 
\item  \textsf{LiePRing} package by Michael Vaughan-Lee and Bettina Eick, introducing a new
datastructure for nilpotent Lie rings of prime-power order. This allows to
define such Lie rings for specific primes as well as for symbolic primes and
other symbolic parameters. The package also includes a database of nilpotent
Lie rings of order at most $p^7$ for all primes $p > 3$. 
\item  \textsf{ModIsom} by Bettina Eick, which contains various methods for computing with nilpotent
associative algebras. In particular, it contains a method to determine the
automorphism group and to test isomorphisms of such algebras over finite
fields and of modular group algebras of finite $p$-groups. Further, it contains a nilpotent quotient algorithm for finitely
presented associative algebras and a method to determine Kurosh algebras. 
\item  \textsf{SLA} by Willem de Graaf for computations with simple Lie algebras. The main topics
of the package are nilpotent orbits, theta-groups and semisimple subalgebras. 
\end{itemize}
 \index{Packages, upgraded} Furthermore, some packages have been upgraded substantially since the \textsf{GAP}{\nobreakspace}4.6.5 release: 
\begin{itemize}
\item  \textsf{ANUPQ} package by Greg Gamble, Werner Nickel and Eamonn O'Brien has been updated
after Max Horn joined it as a maintainer. As a result, it is now much easier
to install and use it with the current \textsf{GAP} release. 
\item  \textsf{Wedderga} package by Osnel Broche Cristo, Allen Herman, Alexander Konovalov, Aurora
Olivieri, Gabriela Olteanu, {\a'A}ngel del R{\a'\i}o and Inneke Van Gelder has
been extended to include functions for calculating local and global Schur
indices of ordinary irreducible characters of finite groups, cyclotomic
algebras over abelian number fields, and rational quaternion algebras
(contribution by Allen Herman). 
\end{itemize}
 }

 }

 
\section{\textcolor{Chapter }{\textsf{GAP} 4.7.3 (February 2014)}}\label{fix473}
\logpage{[ 2, 2, 0 ]}
\hyperdef{L}{X79782A077CCDDD27}{}
{
  Fixed bugs which could lead to incorrect results: 
\begin{itemize}
\item  Incorrect result returned by \texttt{AutomorphismGroup(PSp(4,2\texttt{\symbol{94}}n))}. [Reported by Anvita] 
\item  The \texttt{Order} (\textbf{Reference: Order}) method for group homomorphisms newly introduced in \textsf{GAP}{\nobreakspace}4.7 had a bug that caused it to sometimes return incorrect
results. [Reported by Benjamin Sambale] 
\end{itemize}
 Fixed bugs that could lead to break loops: 
\begin{itemize}
\item  Several bugs were fixed and missing methods were introduced in the new code
for transformations, partial permutations and semigroups that was first
included in \textsf{GAP}{\nobreakspace}4.7. Some minor corrections were made in the documentation for
transformations. 
\item  Break loop in \texttt{IsomorphismFpMonoid} when prefixes in generators names were longer than one letter. [Reported by
Dmytro Savchuk and Yevgen Muntyan] 
\item  Break loop while displaying the result of \texttt{MagmaWithInversesByMultiplicationTable} (\textbf{Reference: MagmaWithInversesByMultiplicationTable}). [Reported by Grahame Erskine] 
\end{itemize}
 Improved functionality: 
\begin{itemize}
\item  Better detection of UTF-8 terminal encoding on some systems. [Suggested by
Andries Brouwer] 
\end{itemize}
 }

 
\section{\textcolor{Chapter }{\textsf{GAP} 4.7.4 (February 2014)}}\label{fix474}
\logpage{[ 2, 3, 0 ]}
\hyperdef{L}{X80C83AF67E5E6C7F}{}
{
  This release was prepared immediately after \textsf{GAP} 4.7.3 to revert the fix of the error handling for the single quote at the end
of an input line, contained in \textsf{GAP} 4.7.3. It happened that (only on Windows) the fix caused error messages in one
of the packages. }

 
\section{\textcolor{Chapter }{\textsf{GAP} 4.7.5 (May 2014)}}\label{fix475}
\logpage{[ 2, 4, 0 ]}
\hyperdef{L}{X864C03527D410BF8}{}
{
  Fixed bugs which could lead to incorrect results: 
\begin{itemize}
\item  \texttt{InstallValue} (\textbf{Reference: InstallValue}) cannot handle immediate values, characters or booleans for technical reasons.
A check for such values was introduced to trigger an error message and prevent
incorrect results caused by this. [Reported by Sebastian Gutsche] 
\item  \texttt{KnowsDictionary} (\textbf{Reference: KnowsDictionary}) and \texttt{LookupDictionary} (\textbf{Reference: LookupDictionary}) methods for \texttt{IsListLookupDictionary} were using \texttt{PositionFirstComponent} (\textbf{Reference: PositionFirstComponent}); the latter is only valid on sorted lists, but in \texttt{IsListLookupDictionary} the underlying list is NOT sorted in general, leading to bogus results. 
\end{itemize}
 Other fixed bugs: 
\begin{itemize}
\item  A bug in \texttt{DirectProductElementsFamily} which used \texttt{CanEasilyCompareElements} (\textbf{Reference: CanEasilyCompareElements}) instead of \texttt{CanEasilySortElements} (\textbf{Reference: CanEasilySortElements}). 
\item  Fixed wrong \texttt{Infolevel} message that caused a break loop for some automorphism group computations. 
\item  Fixed an error that sometimes caused a break loop in \texttt{HallSubgroup} (\textbf{Reference: HallSubgroup}). [Reported by Benjamin Sambale] 
\item  Fixed a rare error in computation of conjugacy classes of a finite group by
homomorphic images, providing fallback to a default algorithm. 
\item  Fixed an error in the calculation of Frattini subgroup in the case of the
trivial radical. 
\item  Several minor bugs were fixed in the documentation, kernel, and library code
for transformations. 
\item  Fixed errors in \texttt{NumberPerfectGroups} (\textbf{Reference: NumberPerfectGroups}) and \texttt{NumberPerfectLibraryGroups} (\textbf{Reference: NumberPerfectLibraryGroups}) not being aware that there are no perfect groups of odd order. 
\item  Restored the ability to build \textsf{GAP} on OS X 10.4 and 10.5 which was accidentally broken in the previous \textsf{GAP} release by using the build option not supported by these versions. 
\item  Fixed some problems for ia64 and sparc architectures. [Reported by Bill
Allombert and Volker Braun] 
\end{itemize}
 New package added for the redistribution with \textsf{GAP}: 
\begin{itemize}
\item  \textsf{permut} package by A.Ballester-Bolinches, E.Cosme-Ll{\a'o}pez, and R.Esteban-Romero to
deal with permutability in finite groups. 
\end{itemize}
 }

 
\section{\textcolor{Chapter }{\textsf{GAP} 4.7.6 (November 2014)}}\label{fix476}
\logpage{[ 2, 5, 0 ]}
\hyperdef{L}{X79E45BD2790B76BB}{}
{
  Fixed bugs which could lead to incorrect results: 
\begin{itemize}
\item  A bug that may cause \texttt{ShortestVectors} (\textbf{Reference: ShortestVectors}) to return an incomplete list. [Reported by Florian Beye] 
\item  A bug that may lead to incorrect results and infinite loops when \textsf{GAP} is compiled without GMP support using gcc 4.9. 
\item  A bug that may cause \texttt{OrthogonalEmbeddings} (\textbf{Reference: OrthogonalEmbeddings}) to return an incomplete result. [Reported by Benjamin Sambale] 
\end{itemize}
 Fixed bugs that could lead to break loops: 
\begin{itemize}
\item  \texttt{ClosureGroup} (\textbf{Reference: ClosureGroup}) should be used instead of \texttt{ClosureSubgroup} (\textbf{Reference: ClosureSubgroup}) in case there is no parent group, otherwise some calculations such as e.g. \texttt{NormalSubgroups} (\textbf{Reference: NormalSubgroups}) may fail. [Reported by Dmitrii Pasechnik] 
\item  Fixed a line in the code that used a hard-coded identity permutation, not a
generic identity element of a group. [Reported by Toshio Sumi] 
\item  Fixed a problem in the new code for calculating maximal subgroups that caused
a break loop for some groups from the transitive groups library. [Reported by
Petr Savicky] 
\item  Fixed a problem in \texttt{ClosureSubgroup} (\textbf{Reference: ClosureSubgroup}) not accepting some groups without \texttt{Parent} (\textbf{Reference: Parent}). [Reported by Inneke van Gelder] 
\end{itemize}
 Other fixed bugs: 
\begin{itemize}
\item  Eliminated a number of compiler warnings detected with some newer versions of \textsf{C} compilers. 
\item  Some minor bugs in the transformation and partial permutation code and
documentation were resolved. 
\end{itemize}
 }

 
\section{\textcolor{Chapter }{\textsf{GAP} 4.7.7 (February 2015)}}\label{fix477}
\logpage{[ 2, 6, 0 ]}
\hyperdef{L}{X86AF9F587D98F31F}{}
{
  New features: 
\begin{itemize}
\item  Introduced some arithmetic operations for infinity and negative infinity, see  \textbf{Reference: infinity}. 
\item  Introduced new property \texttt{IsGeneratorsOfSemigroup} (\textbf{Reference: IsGeneratorsOfSemigroup}) which reflects wheter the list or collection generates a semigroup. 
\end{itemize}
 Fixed bugs which could lead to incorrect results: 
\begin{itemize}
\item  Fixed a bug in \texttt{Union} (\textbf{Reference: Union}) (actually, in the internal library function \texttt{JoinRanges}) caused by downward running ranges. [Reported by Matt Fayers] 
\item  Fixed a bug where recursive records might be printed with the wrong component
name, coming from component names being ordered differently in two different
pieces of code. [Reported by Thomas Breuer] 
\item  The usage of \texttt{abs} in \texttt{src/gmpints.c} was replaced by \texttt{AbsInt}. The former is defined to operate on 32-bit integers even if \textsf{GAP} is compiled in 64-bit mode. That lead to truncating \textsf{GAP} integers and caused a crash in \texttt{RemInt} (\textbf{Reference: RemInt}), reported by Willem De Graaf and Heiko Dietrich. Using \texttt{AbsInt} fixes the crash, and ensures the correct behaviour on 32-bit and 64-bit
builds. 
\end{itemize}
 Fixed bugs that could lead to break loops: 
\begin{itemize}
\item  A problem with \texttt{ProbabilityShapes} (\textbf{Reference: ProbabilityShapes}) not setting frequencies list for small degrees. [Reported by Daniel
B{\l}a{\.z}ewicz and independently by Mathieu Gagne] 
\item  An error when generating a free monoid of rank infinity. [Reported by Nick
Loughlin] 
\item  Several bugs with the code for Rees matrix semigroups not handling trivial
cases properly. 
\item  A bug in \texttt{IsomorphismTypeInfoFiniteSimpleGroup} (\textbf{Reference: IsomorphismTypeInfoFiniteSimpleGroup}) affecting one particular group due to a misformatting in a routine that
translates between the Chevalley type and the name used in the table (in this
case, \texttt{"T"} was used instead of \texttt{["T"]}). [Reported by Petr Savicky] 
\end{itemize}
 Other fixed bugs: 
\begin{itemize}
\item  The \texttt{Basis} (\textbf{Reference: Basis}) method for full homomorphism spaces of linear mappings did not set basis
vectors which could be obtained by \texttt{GeneratorsOfLeftModule} (\textbf{Reference: GeneratorsOfLeftModule}). 
\item  A problem with \texttt{GaloisType} (\textbf{Reference: GaloisType}) entering an infinite loop in the routine for approximating a root. [Reported
by Daniel B{\l}a{\.z}ewicz] 
\item  Fixed the crash when \textsf{GAP} is called when the environment variables \texttt{HOME} or \texttt{PATH} are unset. [Reported by Bill Allombert] 
\end{itemize}
 Furthermore, new packages that have been added to the redistribution since the
release of \textsf{GAP} 4.7.6 are: 
\begin{itemize}
\item  \textsf{json} package by Christopher Jefferson, providing a mapping between the \textsf{JSON} markup language and \textsf{GAP} 
\item  \textsf{SglPPow} package by Bettina Eick and Michael Vaughan-Lee, providing the database of $p$-groups of order $p^7$ for $p > 11$, and of order $3^8$. 
\end{itemize}
 }

 
\section{\textcolor{Chapter }{\textsf{GAP} 4.7.8 (June 2015)}}\label{fix478}
\logpage{[ 2, 7, 0 ]}
\hyperdef{L}{X7EB1879587C27876}{}
{
  }

 Fixed bugs which could lead to incorrect results: 
\begin{itemize}
\item  Added two groups of degree 1575 which were missing in the library of first
primitive groups. [Reported by Gordon Royle] 
\item  Fixed the error in the code for algebra module elements in packed
representation caused by the use of \texttt{Objectify} (\textbf{Reference: Objectify}) with the type of the given object instead of \texttt{ObjByExtRep} (\textbf{Reference: ObjByExtRep}) as recommended in  (\textbf{Reference: Further Improvements in Implementing Residue Class Rings}). The problem was that after calculating \texttt{u+v} where one of the summands was known to be zero, this knowledge was wrongly
passed to the sum via the type. [Reported by Istvan Szollosi] 
\item  Fixed a bug in \texttt{PowerMod} (\textbf{Reference: PowerMod}) causing wrong results for univariate Laurent polynomials when the two
polynomial arguments are stored with the same non-zero shift. [Reported by Max
Horn] 
\end{itemize}
 Furthermore, new packages that have been added to the redistribution since the
release of \textsf{GAP} 4.7.7 are: 
\begin{itemize}
\item  \textsf{PatternClass} by Michael Albert, Ruth Hoffmann and Steve Linton, allowing to explore the
permutation pattern classes build by token passing networks. Amongst other
things, it can compute the basis of a permutation pattern class, create
automata from token passing networks and check if the deterministic automaton
is a possible representative of a token passing network. 
\item  \textsf{QPA} by Edward Green and {\O}yvind Solberg, providing data structures and
algorithms for computations with finite dimensional quotients of path
algebras, and with finitely generated modules over such algebras. It
implements data structures for quivers, quotients of path algebras, and
modules, homomorphisms and complexes of modules over quotients of path
algebras. 
\end{itemize}
 }

    
\chapter{\textcolor{Chapter }{Changes between \textsf{GAP} 4.5 and \textsf{GAP} 4.6}}\label{ChangesGAP45toGAP46}
\logpage{[ 3, 0, 0 ]}
\hyperdef{L}{X8175857F79C8AD8E}{}
{
  This chapter lists most important changes between \textsf{GAP} 4.5.7 and \textsf{GAP} 4.6.2 (i.e. between the last release of \textsf{GAP} 4.5 and the first public release of \textsf{GAP} 4.6). It also contains information about subsequent update releases for \textsf{GAP} 4.6. 
\section{\textcolor{Chapter }{\textsf{GAP} 4.6.2 (February 2013)}}\label{fix462}
\logpage{[ 3, 1, 0 ]}
\hyperdef{L}{X837BB2577D478E55}{}
{
  
\subsection{\textcolor{Chapter }{Changes in the core \textsf{GAP} system introduced in \textsf{GAP} 4.6}}\label{Changes in the core GAP46 system}
\logpage{[ 3, 1, 1 ]}
\hyperdef{L}{X8436976E7E7ED84B}{}
{
  Improved and extended functionality: 
\begin{itemize}
\item  It is now possible to declare a name as an operation with two or more
arguments (possibly several times) and \emph{THEN} declare it as an attribute. Previously this was only possible in the other
order. This should make the system more independent of the order in which
packages are loaded. 
\item  Words in fp groups are now printed in factorised form if possible and not too
time-consuming, i.e. \texttt{a*b*a*b*a*b} will be printed as \texttt{(a*b)\texttt{\symbol{94}}3}. 
\item  Added methods to calculate Hall subgroups in nonsolvable groups. 
\item  Added a generic method for \texttt{IsPSolvable} (\textbf{Reference: IsPSolvable}) and a better generic method for \texttt{IsPNilpotent} (\textbf{Reference: IsPNilpotent}) for groups. 
\item  Improvements to action homomorphisms: image of an element can use existing
stabiliser chain of the image group (to reduce the number of images to
compute), preimages under linear/projective action homomorphisms use linear
algebra to avoid factorisation. 
\item  To improve efficiency, additional code was added to make sure that the \texttt{HomePcgs} of a permutation group is in \texttt{IsPcgsPermGroupRep} representation in more cases. 
\item  Added an operation \texttt{SortBy} (\textbf{Reference: SortBy}) with arguments being a function \mbox{\texttt{\mdseries\slshape f}} of one argument and a list \mbox{\texttt{\mdseries\slshape l}} to be sorted in such a way that \texttt{\mbox{\texttt{\mdseries\slshape l}}(\mbox{\texttt{\mdseries\slshape f}}[i]) {\textless}= \mbox{\texttt{\mdseries\slshape l}}(\mbox{\texttt{\mdseries\slshape f}}[i+1])}. 
\item  Added a kernel function \texttt{MEET{\textunderscore}BLIST} which returns \texttt{true} if the two boolean lists have \texttt{true} in any common position and \texttt{false} otherwise. This is useful for representing subsets of a fixed set by boolean
lists. 
\item  When assigning to a position in a compressed FFE vector \textsf{GAP} now checks to see if the value being assigned can be converted into an
internal FFE element if it isn't one already. This uses new attribute \texttt{AsInternalFFE} (\textbf{Reference: AsInternalFFE}), for which methods are installed for internal FFEs, Conway FFEs and ZmodpZ
objects. 
\item  Replaced \texttt{ViewObj} (\textbf{Reference: ViewObj}) method for fields by \texttt{ViewString} (\textbf{Reference: ViewString}) method to improve the way how polynomial rings over algebraic extenstions of
fields are displayed. 
\item  Made the info string (optional 2nd argument to \texttt{InstallImmediateMethod} (\textbf{Reference: InstallImmediateMethod})) behave similarly to the info string in \texttt{InstallMethod} (\textbf{Reference: InstallMethod}). In particular, \texttt{TraceImmediateMethods} (\textbf{Reference: TraceImmediateMethods}) now always prints the name of the operation. 
\item  Syntax errors such as \texttt{Unbind(x,1)} had the unhelpful property that \texttt{x} got unbound before the syntax error was reported. A specific check was added
to catch this and similar cases a little earlier. 
\item  Allow a \texttt{GAPARGS} parameter to the top-level \textsf{GAP} \texttt{Makefile} to pass extra arguments to the \textsf{GAP} used for manual building. 
\item  Added an attribute \texttt{UnderlyingRingElement} (\textbf{Reference: UnderlyingRingElement}) for Lie objects. 
\item  The function \texttt{PrimeDivisors} (\textbf{Reference: PrimeDivisors}) now became an attribute. [suggested by Mohamed Barakat] 
\item  Added an operation \texttt{DistancePerms} (\textbf{Reference: DistancePerms}) with a kernel method for internal permutations and a generic method. 
\item  Added a method for \texttt{Subfields} (\textbf{Reference: Subfields}) to support large finite fields. [reported by Inneke van Gelder] 
\end{itemize}
 Fixed bugs which could lead to crashes: 
\begin{itemize}
\item  The extremely old \texttt{DEBUG{\textunderscore}DEADSONS{\textunderscore}BAGS} compile-time option has not worked correctly for many years and indeed crashes \textsf{GAP}. The type of bug it is there to detect has not arisen in many years and we
have certainly not used this option, so it has been removed. [Reported by
Volker Braun] 
\end{itemize}
 Other fixed bugs: 
\begin{itemize}
\item  Scanning of floating point literals collided with iterated use of integers as
record field elements in expressions like \texttt{r.1.2}. 
\item  Fixed two potential problems in \texttt{NorSerPermPcgs}, one corrupting some internal data and one possibly mixing up different pcgs. 
\item  Fixed a performance problem with \texttt{NiceMonomorphism} (\textbf{Reference: NiceMonomorphism}). [reported by John Bamberg] 
\item  Fixed a bug in \texttt{ReadCSV} (\textbf{Reference: ReadCSV}) that caused some \texttt{.csv} files being parsed incorrectly. 
\end{itemize}
 No longer supported: 
\begin{itemize}
\item  The file \texttt{lib/consistency.g}, which contained three undocumented auxiliary functions, has been removed
from the library. In addition, the global record \texttt{Revision} is now deprecated, so there is no need to bind its components in \textsf{GAP} packages. 
\end{itemize}
 }

 
\subsection{\textcolor{Chapter }{New and updated packages since \textsf{GAP} 4.5.4}}\label{New and updated packages since GAP 4.5.4}
\logpage{[ 3, 1, 2 ]}
\hyperdef{L}{X79A4DFE97A230E6D}{}
{
  At the time of the release of \textsf{GAP} 4.5 there were 99 packages redistributed with \textsf{GAP}. The first public release of \textsf{GAP}{\nobreakspace}4.6 contains 106 packages. 

 The new packages that have been added to the redistribution since the release
of \textsf{GAP} 4.5.4 are: 
\begin{itemize}
\item  \textsf{AutoDoc} package by S. Gutsche, providing tools for automated generation of \textsf{GAPDoc} manuals. 
\item  \textsf{Congruence} package by A. Konovalov, which provides functions to construct various
canonical congruence subgroups in $SL_2({\ensuremath{\mathbb Z}})$, and also intersections of a finite number of such subgroups, implements the
algorithm for generating Farey symbols for congruence subgroups and uses it to
produce a system of independent generators for these subgroups. 
\item  \textsf{Convex} package by S. Gutsche, which provides structures and algorithms for convex
geometry. 
\item  \textsf{Float} package by L. Bartholdi, which extends \textsf{GAP} floating-point capabilities by providing new floating-point handlers for
high-precision real, interval and complex arithmetic using MPFR, MPFI, MPC or
CXSC external libraries. It also contains a very high-performance
implementation of the LLL (Lenstra-Lenstra-Lov{\a'a}sz) lattice reduction
algorithm via the external library FPLLL. 
\item  \textsf{PolymakeInterface} package by T. Baechler and S. Gutsche, providing a link to the callable
library of the \textsf{polymake} system (\href{http://www.polymake.org} {\texttt{http://www.polymake.org}}). 
\item  \textsf{ToolsForHomalg} package by M. Barakat, S. Gutsche and M. Lange-Hegermann, which provides some
auxiliary functionality for the \textsf{homalg} project (\href{http://homalg.math.rwth-aachen.de/} {\texttt{http://homalg.math.rwth-aachen.de/}}). 
\item  \textsf{ToricVarieties} package by S. Gutsche, which provides data structures to handle toric
varieties by their commutative algebra structure and by their combinatorics. 
\end{itemize}
 Furthermore, some packages have been upgraded substantially since the \textsf{GAP}{\nobreakspace}4.5.4 release: 
\begin{itemize}
\item  Starting from 2.x.x, the functionality for iterated monodromy groups has been
moved from the \textsf{FR} package by L. Bartholdi to a separate package IMG (currently undeposited,
available from \href{https://github.com/laurentbartholdi/img} {\texttt{https://github.com/laurentbartholdi/img}}). This completely removes the dependency of \textsf{FR} on external library modules, and should make its installation much easier. 
\end{itemize}
 }

 }

 
\section{\textcolor{Chapter }{\textsf{GAP} 4.6.3 (March 2013)}}\label{fix463}
\logpage{[ 3, 2, 0 ]}
\hyperdef{L}{X7C2A78667F08A971}{}
{
  Improved functionality: 
\begin{itemize}
\item  Several changes were made to \texttt{IdentityMat} (\textbf{Reference: IdentityMat}) and \texttt{NullMat} (\textbf{Reference: NullMat}). First off, the documentation was changed to properly state that these
functions support arbitrary rings, and not just fields. Also, more usage
examples were added to the manual. 

 For \texttt{NullMat}, it is now also always possible to specify a ring element instead of a ring,
and this is documented. This matches existing \texttt{IdentityMat} behavior, and partially worked before (undocumented), but in some cases could
run into error or infinite recursion. 

 In the other direction, if a finite field element, \texttt{IdentityMat} now really creates a matrix over the smallest field containing that element.
Previously, a matrix over the prime field was created instead, contrary to the
documentation. 

 Furthermore, \texttt{IdentityMat} over small finite fields is now substantially faster when creating matrices of
large dimension (say a thousand or so). 

 Finally, \texttt{MutableIdentityMat} (\textbf{Reference: MutableIdentityMat}) and \texttt{MutableNullMat} (\textbf{Reference: MutableNullMat}) were explicitly declared obsolete (and may be removed in \textsf{GAP} 4.7). They actually were deprecated since \textsf{GAP} 4.1, and their use discouraged by the manual. Code using them should switch to \texttt{IdentityMat} (\textbf{Reference: IdentityMat}) respectively \texttt{NullMat} (\textbf{Reference: NullMat}). 
\item  Two new \texttt{PerfectResiduum} (\textbf{Reference: PerfectResiduum}) methods were added for solvable and perfect groups, handling these cases
optimally. Moreover, the existing generic method was improved by changing it
to use \texttt{DerivedSeriesOfGroup} (\textbf{Reference: DerivedSeriesOfGroup}). Previously, it would always compute the derived series from scratch and then
throw away the result. 
\item  A new \texttt{MinimalGeneratingSet} (\textbf{Reference: MinimalGeneratingSet}) method for groups handled by a nice monomorphisms was added, similar to the
existing \texttt{SmallGeneratingSet} (\textbf{Reference: SmallGeneratingSet}) method. This is useful if the nice monomorphism is already mapping into a pc
or pcp group. 
\item  Added a special method for \texttt{DerivedSubgroup} (\textbf{Reference: DerivedSubgroup}) if the group is known to be abelian. 
\end{itemize}
 Fixed bugs: 
\begin{itemize}
\item  Fixed a bug in \texttt{PowerModInt} (\textbf{Reference: PowerModInt}) computing $r^e$ mod $m$ in a special case when $e=0$ and $m=0$. [Reported by Ignat Soroko] 
\item  \texttt{CoefficientsQadic} (\textbf{Reference: CoefficientsQadic}) now better checks its arguments to avoid an infinite loop when being asked for
a \mbox{\texttt{\mdseries\slshape q}}-adic representation for $q=1$. [Reported by Ignat Soroko] 
\item  Methods for \texttt{SylowSubgroupOp} (see \texttt{SylowSubgroup} (\textbf{Reference: SylowSubgroup})) for symmetric and alternating group did not always set \texttt{IsPGroup} (\textbf{Reference: IsPGroup}) and \texttt{PrimePGroup} (\textbf{Reference: PrimePGroup}) for the returned Sylow subgroup. 
\item  Display of matrices consisting of Conway field elements (which are displayed
as polynomials) did not print constant 1 terms. 
\item  Added an extra check and a better error message in the method to access \emph{natural} generators of domains using the \texttt{.} operator (see \texttt{GeneratorsOfDomain} (\textbf{Reference: GeneratorsOfDomain})). 
\item  Trying to solve the word problem in an fp group where one or more generators
has a name of more than one alphabetic character led to a break loop. 
\item  Provided the default method for \texttt{AbsoluteIrreducibleModules} (\textbf{Reference: AbsoluteIrreducibleModules}) as a temporary workaround for the problem which may cause returning wrong
results or producing an error when being called for a non-prime field. 
\item  A bug in the \textsf{GAP} kernel caused \texttt{RNamObj} to error out when called with a string that had the \texttt{IsSSortedList} (\textbf{Reference: IsSSortedList}) property set (regardless of whether it was set to \texttt{true} or \texttt{false}). This in turn would lead to strange (and inappropriate) errors when using
such a string to access entries of a record. 
\item  \textsf{GAP} can store vectors over small finite fields (size at most 256) in a special
internal data representation where each entry of the vector uses exactly one
byte. Due to an off-by-one bug, the case of a field with exactly 256 elements
was not handled properly. As a result, \textsf{GAP} failed to convert a vector to the special data representation, which in some
situations could lead to a crash. The off-by-one error was fixed and now
vectors over $GF(256)$ work as expected. 
\item  A bug in the code for accessing sublist via the \texttt{list\texttt{\symbol{123}}poss\texttt{\symbol{125}}} syntax could lead to \textsf{GAP} crashing. Specifically, if the list was a compressed vector over a finite
field, and the sublist syntax was nested, as in \texttt{vec\texttt{\symbol{123}}poss1\texttt{\symbol{125}}\texttt{\symbol{123}}poss2\texttt{\symbol{125}}}. This now correctly triggers an error instead of crashing. 
\end{itemize}
 New packages added for the redistribution with \textsf{GAP}: 
\begin{itemize}
\item  \textsf{SpinSym} package by L. Maas, which contains Brauer tables of Schur covers of symmetric
and alternating groups and provides some related functionalities. 
\end{itemize}
 }

 
\section{\textcolor{Chapter }{\textsf{GAP} 4.6.4 (April 2013)}}\label{fix464}
\logpage{[ 3, 3, 0 ]}
\hyperdef{L}{X82859C4083F0D7B9}{}
{
  New functionality: 
\begin{itemize}
\item   New command line option \texttt{-O} was introduced to disable loading obsolete variables. This option may be used,
for example, to check that they are not used in a \textsf{GAP} package or one's own \textsf{GAP} code. For further details see  \textbf{Reference: options} and  \textbf{Reference: Replaced and Removed Command Names}. 
\end{itemize}
 Fixed bugs which could lead to incorrect results: 
\begin{itemize}
\item   Fixed the bug in \texttt{NewmanInfinityCriterion} (\textbf{Reference: NewmanInfinityCriterion}) which may cause returning \texttt{true} instead of \texttt{false}. [Reported by Lev Glebsky] 
\end{itemize}
 Fixed bugs which could lead to crashes: 
\begin{itemize}
\item   Fixed the kernel method for \texttt{Remove} (\textbf{Reference: Remove}) which did not raise an error in case of empty lists, but corrupted the object.
The error message in a library method is also improved. [Reported by Roberto
R{\a`a}dina] 
\end{itemize}
 Fixed bugs that could lead to break loops: 
\begin{itemize}
\item   Fixed requirements in a method to multiply a list and an algebraic element.
[Reported by Sebastian Gutsche] 
\item   Fixed a bug in \texttt{NaturalCharacter} (\textbf{Reference: NaturalCharacter (for a group)}) entering a break loop when being called on a homomorphism whose image is not a
permutation group. [Reported by Sebastian Gutsche] 
\item   Fixed a bug in \texttt{ExponentsConjugateLayer} (\textbf{Reference: ExponentsConjugateLayer}) which occured, for example, in some calls of \texttt{SubgroupsSolvableGroup} (\textbf{Reference: SubgroupsSolvableGroup}) [Reported by Ramon Esteban-Romero] 
\item   Fixed a problem with displaying function fields, e.g. \texttt{Field(Indeterminate(Rationals,"x"))}. [Reported by Jan Willem Knopper] 
\item   Fixed two bugs in the code for \texttt{NaturalHomomorphismByIdeal} (\textbf{Reference: NaturalHomomorphismByIdeal}) for polynomial rings. [Reported by Martin Leuner] 
\item   Added missing method for \texttt{String} (\textbf{Reference: String}) for \texttt{-infinity}. 
\item   Fixed the bug with \texttt{ONanScottType} (\textbf{Reference: ONanScottType}) not recognising product action properly in some cases. 
\item   The method for \texttt{SlotUsagePattern} (\textbf{Reference: SlotUsagePattern}) for straight line programs had a bug which triggered an error, if the straight
line program contained unnecessary steps. 
\end{itemize}
 }

 
\section{\textcolor{Chapter }{\textsf{GAP} 4.6.5 (July 2013)}}\label{fix465}
\logpage{[ 3, 4, 0 ]}
\hyperdef{L}{X7F7F29B27BF76872}{}
{
  Improved functionality: 
\begin{itemize}
\item   \texttt{TraceMethods} (\textbf{Reference: TraceMethods (for operations)}) and \texttt{UntraceMethods} (\textbf{Reference: UntraceMethods (for operations)}) now better check their arguments and provide a sensible error message if being
called without arguments. Also, both variants of calling them are now
documented. 
\item   Library methods for \texttt{Sortex} (\textbf{Reference: Sortex}) are now replaced by faster ones using the kernel \texttt{SortParallel} (\textbf{Reference: SortParallel}) functionality instead of making expensive zipped lists. 
\end{itemize}
 Fixed bugs which could lead to incorrect results: 
\begin{itemize}
\item   \texttt{IntHexString} (\textbf{Reference: IntHexString}) wrongly produced a large integer when there were too many leading zeros.
[Reported by Joe Bohanon] 
\end{itemize}
 Fixed bugs that could lead to break loops: 
\begin{itemize}
\item   A bug that may occur in some cases while calling \texttt{TransitiveIdentification} (\textbf{Reference: TransitiveIdentification}). [Reported by Izumi Miyamoto] 
\item   The new code for semidirect products of permutation groups, introduced in \textsf{GAP} 4.6, had a bug which was causing problems for \texttt{Projection} (\textbf{Reference: Projection}). [Reported by Graham Ellis] 
\end{itemize}
 }

 }

    
\chapter{\textcolor{Chapter }{Changes between \textsf{GAP} 4.4 and \textsf{GAP} 4.5}}\label{ChangesGAP44toGAP45}
\logpage{[ 4, 0, 0 ]}
\hyperdef{L}{X837AADDA7EA9A541}{}
{
  This chapter lists most important changes between \textsf{GAP} 4.4.12 and the first public release of \textsf{GAP} 4.5. It also contains information about subsequent update releases for \textsf{GAP} 4.5. It is not meant to serve as a complete account on all improvements;
instead, it should be viewed as an introduction to \textsf{GAP} 4.5, accompanying its release announcement.  
\section{\textcolor{Chapter }{Changes in the core \textsf{GAP} system introduced in \textsf{GAP} 4.5}}\label{Changes in the core GAP45 system}
\logpage{[ 4, 1, 0 ]}
\hyperdef{L}{X7DA602757C152B0C}{}
{
  In this section we list most important new features and bugfixes in the core
system introduced in \textsf{GAP} 4.5. For the list of changes in the interface between the core system and
packages as well as for an overview of new and updated packages, see Section \ref{Packages in GAP45}.  
\subsection{\textcolor{Chapter }{Improved functionality}}\label{Improved mathematical functionality}
\logpage{[ 4, 1, 1 ]}
\hyperdef{L}{X84FC44718386FA1A}{}
{
   Performance improvements: 
\begin{itemize}
\item  \index{\textsf{GMP support}} The \textsf{GAP} kernel now uses \textsf{GMP} (GNU multiple precision arithmetic library, \href{http://gmplib.org/} {\texttt{http://gmplib.org/}}) for faster large integer arithmetic. 
\item  Improved performance for records with large number of components. 
\item  Speedup of hash tables implementation at the \textsf{GAP} library level. 
\item  \texttt{MemoryUsage} (\textbf{Reference: MemoryUsage}) is now much more efficient, in particular for large objects. 
\item  Speedups in the computation of low index subgroups, Tietze transformations,
calculating high powers of matrices over finite fields, \texttt{Factorial} (\textbf{Reference: Factorial}), etc. 
\end{itemize}
 New and improved kernel functionality: 
\begin{itemize}
\item  By default, the \textsf{GAP} kernel compiles with the \textsf{GMP} and \textsf{readline} libraries. The \textsf{GMP} library is supplied with \textsf{GAP} and we recommend that you use the version we supply. There are some problems
with some other versions. It is also possible to compile the \textsf{GAP} kernel with the system \textsf{GMP} if your system has it. The \textsf{readline} library must be installed on your system in advance to be used with \textsf{GAP}. 
\item  \index{Floats} Floating point literals are now supported in the \textsf{GAP} language, so that, floating point numbers can be entered in \textsf{GAP} expressions in a natural way. Support for floats is now properly documented,
see  (\textbf{Reference: Floats}). \textsf{GAP} has an interface using which packages may add new floating point
implementations and integrate them with the parser. In particular, we expect
that there will soon be a package that implements arbitrary precision floating
point arithmetic. 
\item  The Mersenne twister random number generator has been made independent of
endianness, so that random seeds can now be transferred between architectures.
See  (\textbf{Reference: GlobalMersenneTwister}) for details. 
\item  Defaults for \texttt{-m} and \texttt{-o} options have been increased. Changes have been made to the way that \textsf{GAP} obtains memory from the Operating System, to make \textsf{GAP} more compatible with C libraries. A new \texttt{-s} option has been introduced to control or turn off the new behaviour, see  (\textbf{Reference: Command Line Options}). 
\item  The filename and lines from which a function was read can now be recovered
using \texttt{FilenameFunc} (\textbf{Reference: FilenameFunc}), \texttt{StartlineFunc} (\textbf{Reference: StartlineFunc}) and \texttt{EndlineFunc} (\textbf{Reference: EndlineFunc}). This allows you, for example, to implement a function such as \texttt{PageSource} (\textbf{Reference: PageSource}) to show the file containing the source code of a function or a method in a
pager, see \texttt{Pager} (\textbf{Reference: Pager}). 
\item  \texttt{CallFuncList} (\textbf{Reference: CallFuncList}) was made into an operation so that it can be used to define behaviour of a
non-function when called as a function. 
\item  Improvements to the cyclotomic number arithmetic for fields with large
conductors. 
\item  Better and more flexible viewing of some large objects. 
\item  Opportunity to interrupt some long kernel computations, e.g. multiplication of
compressed matrices, intercepting \texttt{Ctrl-C} in designated places in the kernel code by means of a special kernel function
for that purpose. 
\item  \texttt{ELM{\textunderscore}LIST} now allows you to install methods where the second argument is NOT a positive
integer. 
\item  Kernel function \texttt{DirectoryContents} (\textbf{Reference: DirectoryContents}) to get the list of names of files and subdirectories in a directory. 
\item  Kernel functions for Kronecker product of compressed matrices, see \texttt{KroneckerProduct} (\textbf{Reference: KroneckerProduct}). 
\end{itemize}
 New and improved library functionality: \index{Data libraries} 
\begin{itemize}
\item  Extensions of data libraries: 
\begin{itemize}
\item  Functions and iterators are now available to create and enumerate simple
groups by their order up to isomorphism: \texttt{SimpleGroup} (\textbf{Reference: SimpleGroup}), \texttt{SmallSimpleGroup} (\textbf{Reference: SmallSimpleGroup}), \texttt{SimpleGroupsIterator} (\textbf{Reference: SimpleGroupsIterator}) and \texttt{AllSmallNonabelianSimpleGroups} (\textbf{Reference: AllSmallNonabelianSimpleGroups}). 
\item  See also packages \textsf{CTblLib}, \textsf{IRREDSOL} and \textsf{Smallsemi} listed in Section \ref{New and updated packages since GAP 4.4.12}. 
\end{itemize}
 
\item  Many more methods are now available for the built-in floating point numbers,
see  (\textbf{Reference: Floats}). 
\item  The bound for the proper primality test in \texttt{IsPrimeInt} (\textbf{Reference: IsPrimeInt}) increased up to $10^{18}$. 
\item  Improved code for determining transversal and double coset representatives in
large groups. 
\item  Improvements in \texttt{Normalizer} (\textbf{Reference: Normalizer}) for $S_n$. 
\item  Smith normal form of a matrix may be computed over arbitrary euclidean rings,
see \texttt{NormalFormIntMat} (\textbf{Reference: NormalFormIntMat}). 
\item  Improved algorithms to determine the subgroup lattice of a group, as well as
the function \texttt{DotFileLatticeSubgroups} (\textbf{Reference: DotFileLatticeSubgroups}) to save the lattice structure in \texttt{.dot} file to view it e.g. with \textsf{GraphViz}. 
\item  Special teaching mode which simplifies some output and provides more basic
functionality, see  (\textbf{Reference: Teaching Mode}). 
\item  Functionality specific for use in undergraduate abstract algebra courses, e.g.
checksums ( (\textbf{Reference: Check Digits})); string/integer list conversion; rings of small orders; the function \texttt{SetNameObject} (\textbf{Reference: SetNameObject}) to set display names for objects for more informative examples, e.g.
constructing groups from ``named'' objects, such as, for example, \texttt{R90} for a 90-degree rotation).  
\item  Functions \texttt{DirectoryDesktop} (\textbf{Reference: DirectoryDesktop}) and \texttt{DirectoryHome} (\textbf{Reference: DirectoryHome}) which provide uniform access to default directories under Windows, Mac OS X
and Unix. 
\item  Improved methods for hashing when computing orbits. 
\item  Functionality to call external binaries under Windows. 
\item  Symplectic groups over residue class rings, see \texttt{SymplecticGroup} (\textbf{Reference: SymplecticGroup}). 
\item  Basic version of the simplex algorithm for matrices. 
\item  New functions, operations and attributes: \texttt{PrimeDivisors} (\textbf{Reference: PrimeDivisors}), \texttt{Shuffle} (\textbf{Reference: Shuffle}) for lists, \texttt{IteratorOfPartitions} (\textbf{Reference: IteratorOfPartitions}), \texttt{IteratorOfCombinations} (\textbf{Reference: IteratorOfCombinations}), \texttt{EnumeratorOfCombinations} (\textbf{Reference: EnumeratorOfCombinations}) and others. 
\item  The behaviour of \texttt{Info} (\textbf{Reference: Info}) statements can now be configured per info class, this applies to the way the
arguments are printed and to the output stream, see  (\textbf{Reference: Info Functions}). 
\item  New function \texttt{Test} (\textbf{Reference: Test}) which is a more flexible and informative substitute of \texttt{ReadTest} (\textbf{Reference: ReadTest}) operation. 
\item  \texttt{ConnectGroupAndCharacterTable} is replaced by more robust function \texttt{CharacterTableWithStoredGroup} (\textbf{Reference: CharacterTableWithStoredGroup}). 
\end{itemize}
 Many problems in \textsf{GAP} have have been fixed, among them the following: 
\begin{itemize}
\item  Polynomial factorisation over rationals could miss factors of degree greater
than $deg(f)/2$ if they have very small coefficients, while the cofactor has large
coefficients. 
\item  \texttt{IntermediateSubgroups} (\textbf{Reference: IntermediateSubgroups}) called on a group and a normal subgroup did not properly calculate maximal
inclusion relationships. 
\item  \texttt{CentreOfCharacter} (\textbf{Reference: CentreOfCharacter}) and \texttt{ClassPositionsOfCentre} (\textbf{Reference: ClassPositionsOfCentre (for a character)}) called for a group character could return a perhaps too large result. 
\item  \texttt{Trace} (\textbf{Reference: Traces of field elements and matrices}) called for an element of a finite field that was created with \texttt{AlgebraicExtension} (\textbf{Reference: AlgebraicExtension}) ran into an error. 
\item  \texttt{IrreducibleRepresentationsDixon} (\textbf{Reference: IrreducibleRepresentationsDixon}) did not accept a list with one character as a second argument. 
\item  Composing a homomorphism from a permutation group to a finitely presented
group with another homomorphism could give wrong results. 
\item  For certain arguments, the function \texttt{EU} (\textbf{Reference: EU}) returned wrong results. 
\item  In the table of marks of cyclic groups, \texttt{NormalizersTom} (\textbf{Reference: NormalizersTom}) value was wrong. 
\item  The function \texttt{PermChars} (\textbf{Reference: PermChars}) returned a perhaps wrong result when the second argument was a positive
integer (not a record) and the trivial character of the character table given
as the first argument was not the first in the list of irreducibles. 
\item  \textsf{GAP} crashed when the intersection of ranges became empty. 
\item  \texttt{IsPSL}, and in turn \texttt{StructureDescription} (\textbf{Reference: StructureDescription}), erroneously recognised non-PSL groups of the right order as PSL. 
\item  The semidirect product method for pcgs computable groups sometimes tried to
use finite presentations which were not polycyclic. This usually happened when
the groups were not pc groups, and there was a very low risk of getting a
wrong result. 
\item  The membership test for a group of finite field elements ran into an error if
the zero element of the field was given as the first argument. 
\item  Constant polynomials were not recognised as univariate in any variable. 
\item  The kernel recursion depth counter was not reset properly when running into
many break loops. 
\item  \textsf{GAP} did not behave well when printing of a (large) object was interrupted with \texttt{Ctrl-C}. Now the object is no longer corrupted and the indentation level is reset. 
\end{itemize}
 Potentially incompatible changes: 
\begin{itemize}
\item  The zero polynomial now has degree \texttt{-infinity}, see \texttt{DegreeOfLaurentPolynomial} (\textbf{Reference: DegreeOfLaurentPolynomial}). 
\item  Multiple unary \texttt{+} or \texttt{-} signs are no longer allowed (to avoid confusion with increment/decrement
operators from other programming languages). 
\item  Due to changes to improve the performance of records with large number of
components, the ordering of record components in \texttt{View}'ed records has changed. 
\item  Due to improvements for vectors over finite fields, certain objects have more
limitations on changing their base field. For example, one can not create a
compressed matrix over $GF(2)$ and then assign an element of $GF(4)$ to one of its entries. 
\end{itemize}
 \index{Completion files (withdrawn)} \index{\textsf{GAP}{\nobreakspace}3 compatibility mode (withdrawn)} No longer supported: 
\begin{itemize}
\item  Completion files mechanism. 
\item  \textsf{GAP} 3 compatibility mode. 
\end{itemize}
 \index{\textsf{GAP} compiler (no longer recommended)} In addition, we no longer recommend using the \textsf{GAP} compiler \texttt{gac} to compile \textsf{GAP} code to \textsf{C}, and may withdraw it in future releases. Compiling \textsf{GAP} code only ever gave a substantial speedup for rather specific types of
calculation, and much more benefit can usually be achieved quite easily by
writing a small number of key functions in \textsf{C} and loading them into the kernel as described in \texttt{LoadDynamicModule} (\textbf{Reference: LoadDynamicModule}). The \texttt{gac} script will remain available as a convenient way of compiling such kernel
modules from \textsf{C}. 

 Also, the following functions and operations were made obsolete: \texttt{AffineOperation}, \texttt{AffineOperationLayer}, \texttt{FactorCosetOperation}, \texttt{DisplayRevision}, \texttt{ProductPol}, \texttt{TeXObj}, \texttt{LaTeXObj}. }

  
\subsection{\textcolor{Chapter }{Changes in distribution formats}}\label{GAP45 Distribution}
\logpage{[ 4, 1, 2 ]}
\hyperdef{L}{X7E24E9DE7CADD1B7}{}
{
  \index{tools@\texttt{tools} archive} The \textsf{GAP} 4.5 source distribution has the form of a single archive containing the core
system and the most recent ``stable'' versions of all currently redistributed packages. There are no optional
archives to download: the \textsf{TomLib} package now contains all its tables of marks in one archive; we do not provide
separate versions of manuals for Internet Explorer, and the former \texttt{tools} archive is now included as an archive in the \texttt{etc} directory. To unpack and install the archive, user the script \texttt{etc/install-tools.sh}. 

 \index{Bugfixes and packages archives (withdrawn)} We no longer distribute separate bugfix archives when the core \textsf{GAP} system changes, or updated packages archives when a redistributed package is
updated. Instead, the single \textsf{GAP} source distribution archive will be labelled by the version of the core \textsf{GAP} system and also by a timestamp. This archive contains the core system and the
stable versions of the relevant packages on that date. To upgrade, you simply
replace the whole directory containing the \textsf{GAP} installation, and rebuild binaries for the \textsf{GAP} kernel and packages. For new versions of packages, we will also continue to
redistribute individual package archives so it will be possible to update a
single package without changing the rest of the \textsf{GAP} installation. 

 Furthermore, by default \textsf{GAP} will now automatically read a user-specific \textsf{GAP} root directory (unless \textsf{GAP} is called with the \texttt{-r} option). All user settings can be made in that directory, so there will be no
risk of them being lost during an update (see Section \ref{GAP45 User interface} below for more details). Private packages can also be installed in this
directory for the same reason. 

 There are some changes in archive formats used for the distribution: we
continue to provide \texttt{.tar.gz}, \texttt{.tar.bz2} and \texttt{-win.zip} archives. We have added \texttt{.zip}, and stopped providing \texttt{.zoo} archives. We no longer provide GAP binaries for Mac OS 9 (Classic) any more.
For installations from source on Mac OS X one may follow the instructions for
UNIX. 

 \index{\textsf{GAP} binary distributions} With the release of \textsf{GAP} 4.5, we also encourage more users to take advantage of the increasingly mature
binary distributions which are now available. These include: 
\begin{itemize}
\item  The binary \texttt{rsync} distribution for \textsf{GAP} on Linux PCs with i686 or x86{\textunderscore}64 compatible processors
provided by Frank L{\"u}beck, see \href{http://www.math.rwth-aachen.de/~Frank.Luebeck/gap/rsync} {\texttt{http://www.math.rwth-aachen.de/\texttt{\symbol{126}}Frank.Luebeck/gap/rsync}}. 
\item  \textsf{BOB}, a tool for Linux and Mac OS X to download and build \textsf{GAP} and its packages from source provided by M. Neunh{\"o}ffer: \href{http://www-groups.mcs.st-and.ac.uk/~neunhoef/Computer/Software/Gap/bob.html} {\texttt{http://www-groups.mcs.st-and.ac.uk/\texttt{\symbol{126}}neunhoef/Computer/Software/Gap/bob.html}}. 
\item  The \textsf{GAP} installer for Windows provided by Alexander Konovalov: \href{http://www.gap-system.org/ukrgap/wininst/} {\texttt{http://www.gap-system.org/ukrgap/wininst/}}. 
\end{itemize}
 In the near future, we also hope to have a binary distribution for Mac OS X.  

 Internally, we now have infrastructure to support more robust and frequent
releases, and an improved system to fetch and test new versions of the
increasingly large number of packages. The \textsf{Example} package documents technical requirements for packages, many of which are
checked automatically by our systems. This will allow us to check the
compatibility of packages with the system and with other packages more
thoroughly before publishing them on the \textsf{GAP} website. }

  
\subsection{\textcolor{Chapter }{Improvements to the user interface}}\label{GAP45 User interface}
\logpage{[ 4, 1, 3 ]}
\hyperdef{L}{X80AB8641792E74C9}{}
{
  \index{\textsf{readline} support} \index{User interface customisation} By default, \textsf{GAP} now uses the \textsf{readline} library for command line editing. It provides such advantages as working with
unicode terminals, nicer handling of long input lines, improved TAB-completion
and flexible configuration. For further details, see  (\textbf{Reference: Editing using the readline library}). 

 We have extended facilities for user interface customisation. By default \textsf{GAP} automatically scans a user specific \textsf{GAP} root directory (unless \textsf{GAP} is called with the \texttt{-r} option). The name of this user specific directory depends on the operating
system and is contained in \texttt{GAPInfo.UserGapRoot}. This directory can be used to tell \textsf{GAP} about personal preferences, to load some additional code, to install
additional packages, or to overwrite some \textsf{GAP} files, see  (\textbf{Reference: GAP Root Directories}). Instead of a single \texttt{.gaprc} file we now use more flexible setup based on two files: \texttt{gap.ini} which is read early in the startup process, and \texttt{gaprc} which is read after the startup process, but before the first input file given
on the command line. These files may be located in the user specific \textsf{GAP} root directory \texttt{GAPInfo.UserGapRoot} which by default is the first \textsf{GAP} root directory, see  (\textbf{Reference: The gap.ini and gaprc files}). For compatibility, the \texttt{.gaprc} file is still read if the directory \texttt{GAPInfo.UserGapRoot} does not exist. See  (\textbf{Reference: The former .gaprc file}) for the instructions how to migrate your old setup. 

 Furthermore, there are functions to deal with user preferences, for example,
to specify how \textsf{GAP}'s online help is shown or whether the coloured prompt should be used. Calls
to set user preferences may appear in the user's \texttt{gap.ini} file, as explained in  (\textbf{Reference: Configuring User preferences}). 

 In the Windows version, we include a new shell which uses the \textsf{mintty} terminal in addition to the two previously used shells (Windows command line
and \textsf{RXVT}). The \textsf{mintty} shell is now recommended. It supports Unicode encoding and has flexible
configurations options. Also, \textsf{GAP} under Windows now starts in the \texttt{\%HOMEDRIVE\%\%HOMEPATH\%} directory, which is the user's home directory. Besides this, a larger
workspace is now permitted without a need to modify the Windows registry. 

 Other changes in the user interface include: 
\begin{itemize}
\item  the command line history is now implemented at the \textsf{GAP} level, it can be stored on quitting a \textsf{GAP} session and reread when starting a new session, see  (\textbf{Reference: The command line history}). 
\item  \texttt{SetPrintFormattingStatus("stdout",false);} may be used to switch off the automatic line breaking in terminal output, see \texttt{SetPrintFormattingStatus} (\textbf{Reference: SetPrintFormattingStatus}). 
\item  \textsf{GAP} supports terminals with up to 4096 columns (extendable at compile time). 
\item  Directories in \texttt{-l} command-line option can now be specified starting with \texttt{\texttt{\symbol{126}}/}, see  (\textbf{Reference: Command Line Options}). 
\item  Large integers are now displayed by a short string showing the first and last
few digits, and the threshold to trigger this behaviour is user configurable
(call \texttt{UserPreference("MaxBitsIntView")} to see the default value). 
\item  The \textsf{GAP} banner has been made more compact and informative. 
\item  \texttt{SetHelpViewer} (\textbf{Reference: SetHelpViewer}) now supports the Google Chrome browser. 
\item  Multiple matches in the \textsf{GAP} online help are displayed via a function from the \textsf{Browse} package, which is loaded in the default configuration. This feature can be
replaced by the known pager using the command 
\begin{verbatim}  
  SetUserPreference( "browse", "SelectHelpMatches", false );
\end{verbatim}
 
\end{itemize}
 }

  
\subsection{\textcolor{Chapter }{Better documentation}}\label{GAP45 Documentation}
\logpage{[ 4, 1, 4 ]}
\hyperdef{L}{X81318FC0873923C3}{}
{
  \index{\textsf{MathJax support}} The main \textsf{GAP} manuals have been converted to the \textsf{GAPDoc} format provided by the \textsf{GAPDoc} package by Frank L{\"u}beck and Max Neunh{\"o}ffer (\href{http://www.math.rwth-aachen.de/~Frank.Luebeck/GAPDoc} {\texttt{http://www.math.rwth-aachen.de/\texttt{\symbol{126}}Frank.Luebeck/GAPDoc}}). This documentation format is already used by many packages and is now
recommended for all \textsf{GAP} documentation. 

 Besides improvements to the documentation layout in all formats (text, PDF and
HTML), the new \textsf{GAP} manuals incorporate a large number of corrections, clarifications, additions
and updated examples. 

 We now provide two HTML versions of the manual, one of them with \textsf{MathJax} (\href{http://www.mathjax.org} {\texttt{http://www.mathjax.org}}) support for better display of mathematical symbols. Also, there are two PDF
versions of the manual - a coloured and a monochrome one. 

 Several separate manuals now became parts of the \textsf{GAP} Reference manual. Thus, now there are three main \textsf{GAP} manual books: 
\begin{itemize}
\item  \emph{\textsf{GAP} Tutorial}  

 
\item  \emph{\textsf{GAP} Reference manual}  
\item  \emph{\textsf{GAP} - Changes from Earlier Versions} (this manual) 
\end{itemize}
 Note that there is no index file combining these three manuals. Instead of
that, please use the \textsf{GAP} help system which will search all of these and about 100 package manuals. }

 }

  
\section{\textcolor{Chapter }{Packages in \textsf{GAP} 4.5}}\label{Packages in GAP45}
\logpage{[ 4, 2, 0 ]}
\hyperdef{L}{X85D796968453CFBF}{}
{
  Here we list most important changes affecting packages and present new or
essentially changed packages. For the changes in the core \textsf{GAP} system, see Section \ref{Changes in the core GAP45 system}.  
\subsection{\textcolor{Chapter }{Interface between the core system and packages}}\label{Interface between the core system and packages}
\logpage{[ 4, 2, 1 ]}
\hyperdef{L}{X7E8089C881AB6BA1}{}
{
  \index{Namespaces} The package loading mechanism has been improved. The most important new
feature is that all dependencies are evaluated in advance and then used to
determine the order in which package files are read. This allows \textsf{GAP} to handle cyclic dependencies as well as situations where package A requires
package B to be loaded completely before any file of package A is read. To
avoid distortions of the order in which packages will be loaded, package
authors are strongly discouraged from calling \texttt{LoadPackage} (\textbf{Reference: LoadPackage}) and \texttt{TestPackageAvailability} (\textbf{Reference: TestPackageAvailability}) in a package code in order to determine whether some other package will be
loaded before or together with the current package - instead, one should use \texttt{IsPackageMarkedForLoading} (\textbf{Reference: IsPackageMarkedForLoading}). In addition, there is now a better error management if package loading fails
for packages that use the new functionality to log package loading messages
(see \texttt{DisplayPackageLoadingLog} (\textbf{Reference: DisplayPackageLoadingLog}) and the rest of the Chapter  (\textbf{Reference: Using GAP Packages}) which documents how to \emph{use} \textsf{GAP} packages), and package authors are very much encouraged to use these logging
facilities. 

 In \textsf{GAP} 4.4 certain packages were marked as \emph{autoloaded} and would be loaded, if present, when \textsf{GAP} started up. In \textsf{GAP} 4.5, this notion is divided into three. Certain packages are recorded as \emph{needed} by the \textsf{GAP} system and others as \emph{suggested}, in the same way that packages may \emph{need} or \emph{suggest} other packages. If a needed package is not loadable, \textsf{GAP} will not start. Currently only \textsf{GAPDoc} is needed. If a suggested package is loadable, it will be loaded. Typically
these are packages which install better methods for Operations and Objects
already present in \textsf{GAP}. Finally, the user preferences mechanism can be used to specify additional
packages that should be loaded if possible. By default this includes most
packages that were autoloaded in \textsf{GAP}{\nobreakspace}4.4.12, see \texttt{ShowUserPreferences} (\textbf{Reference: ShowUserPreferences}). 

 \textsf{GAP} packages may now use local \emph{namespaces} to avoid name clashes for global variables introduced in other packages or in
the \textsf{GAP} library, see  (\textbf{Reference: Namespaces for GAP packages}). 

 All guidance on how to \emph{develop} a \textsf{GAP} package has been consolidated in the \textsf{Example} package which also contains a checklist for upgrading a \textsf{GAP} package to \textsf{GAP}{\nobreakspace}4.5, see  (\textbf{Example: Guidelines for Writing a GAP Package}). 

 }

 
\subsection{\textcolor{Chapter }{New and updated packages since \textsf{GAP} 4.4.12}}\label{New and updated packages since GAP 4.4.12}
\logpage{[ 4, 2, 2 ]}
\hyperdef{L}{X81A313117C30EC28}{}
{
  At the time of the release of \textsf{GAP} 4.4.12 there were 75 packages redistributed with \textsf{GAP} (including the \textsf{TomLib} which was distributed in the core \textsf{GAP} archive). The first public release of \textsf{GAP}{\nobreakspace}4.5 contains precisely 99 packages. 

 The new packages that have been added to the redistribution since the release
of \textsf{GAP} 4.4.12 are: 
\begin{itemize}
\item  \textsf{Citrus} package by J.D. Mitchell for computations with transformation semigroups and
monoids (this package is a replacement of the \textsf{Monoid} package). 
\item  \textsf{cvec} package by M. Neunh{\"o}ffer, providing an implementation of compact vectors
over finite fields. 
\item  \textsf{fwtree} package by B. Eick and T. Rossmann for computing trees related to some pro-$p$-groups of finite width. 
\item  \textsf{GBNP} package by A.M. Cohen and J.W. Knopper, providing algorithms for computing
Grobner bases of noncommutative polynomials over fields with respect to the ``total degree first then lexicographical'' ordering. 
\item  \textsf{genss} package by M. Neunh{\"o}ffer and F. Noeske, implementing the randomised
Schreier-Sims algorithm to compute a stabiliser chain and a base and a strong
generating set for arbitrary finite groups. 
\item  \textsf{HAPprime} package by P. Smith, extending the \textsf{HAP} package with an implementation of memory-efficient algorithms for the
calculation of resolutions of small prime-power groups. 
\item  \textsf{hecke} package by D. Traytel, providing functions for calculating decomposition
matrices of Hecke algebras of the symmetric groups and $q$-Schur algebras (this package is a port of the \textsf{GAP}{\nobreakspace}3 package \textsf{Specht 2.4} to \textsf{GAP}{\nobreakspace}4). 
\item  \textsf{Homalg} project by M. Barakat, S. Gutsche, M. Lange-Hegermann et al., containing the
following packages for the homological algebra: \textsf{homalg}, \textsf{ExamplesForHomalg}, \textsf{Gauss}, \textsf{GaussForHomalg}, \textsf{GradedModules}, \textsf{GradedRingForHomalg}, \textsf{HomalgToCAS}, \textsf{IO{\textunderscore}ForHomalg}, \textsf{LocalizeRingForHomalg}, \textsf{MatricesForHomalg}, \textsf{Modules}, \textsf{RingsForHomalg} and \textsf{SCO} (see \href{http://homalg.math.rwth-aachen.de/} {\texttt{http://homalg.math.rwth-aachen.de/}}). 
\item  \textsf{MapClass} package by A. James, K. Magaard and S. Shpectorov to calculate the mapping
class group orbits for a given finite group. 
\item  \textsf{recogbase} package by M. Neunh{\"o}ffer and A. Seress, providing a framework to implement
group recognition methods in a generic way (suitable, in particular, for
permutation groups, matrix groups, projective groups and black box groups). 
\item  \textsf{recog} package by M. Neunh{\"o}ffer, A. Seress, N. Ankaralioglu, P. Brooksbank, F.
Celler, S. Howe, M. Law, S. Linton, G. Malle, A. Niemeyer, E. O'Brien and C.M.
Roney-Dougal, extending the \textsf{recogbase} package and provides a collection of methods for the constructive recognition
of groups (mostly intended for permutation groups, matrix groups and
projective groups). 
\item  \textsf{SCSCP} package by A. Konovalov and S. Linton, implementing the Symbolic Computation
Software Composability Protocol (\textsf{SCSCP}, see \href{http://www.symbolic-computation.org/scscp} {\texttt{http://www.symbolic-computation.org/scscp}}) for \textsf{GAP}, which provides interfaces to link a \textsf{GAP} instance with another copy of \textsf{GAP} or other \textsf{SCSCP}-compliant system running locally or remotely. 
\item  \textsf{simpcomp} package by F. Effenberger and J. Spreer for working with simplicial complexes. 
\item  \textsf{Smallsemi} package by A. Distler and J.D. Mitchell, containing the data library of all
semigroups with at most 8 elements as well as various information about them. 
\item  \textsf{SymbCompCC} package by D. Feichtenschlager for computations with parametrised
presentations for finite $p$-groups of fixed coclass. 
\end{itemize}
 Furthermore, some packages have been upgraded substantially since the \textsf{GAP}{\nobreakspace}4.4.12 release: 
\begin{itemize}
\item  \textsf{Alnuth} package by B. Assmann, A. Distler and B. Eick uses an interface to PARI/GP
system instead of the interface to KANT (thanks to B. Allombert for the GP
code for the new interface and help with the transition) and now also works
under Windows. 
\item  \textsf{CTblLib} package (the \textsf{GAP} Character Table Library) by T. Breuer has been extended by many new character
tables, a few bugs have been fixed, and new features have been added, for
example concerning the relation to \textsf{GAP}'s group libraries, better search facilities, and interactive overviews. For
details, see the package manual. 
\item  \textsf{DESIGN} package by L.H. Soicher: 
\begin{itemize}
\item  The functions \texttt{PointBlockIncidenceMatrix}, \texttt{ConcurrenceMatrix}, and \texttt{InformationMatrix} compute matrices associated with block designs. 
\item  The function \texttt{BlockDesignEfficiency} computes certain statistical efficiency measures of a $1-(v,k,r)$ design, using exact algebraic computation. 
\end{itemize}
 
\item  \textsf{Example} package by W. Nickel, G. Gamble and A. Konovalov has a more detailed and
up-to-date guidance on developing a \textsf{GAP} package, see  (\textbf{Example: Guidelines for Writing a GAP Package}). 
\item  \textsf{FR} package by L. Bartholdi now uses floating-point numbers to compute
approximations of rational maps given by their group-theoretical description. 
\item  The \textsf{GAPDoc} package by F. L{\"u}beck and M. Neunh{\"o}ffer provides various improvements,
for example: 
\begin{itemize}
\item  The layout of the text version of the manuals can be configured quite freely,
several standard ``themes'' are provided. The display is now adjusted to the current screen width. 
\item  Some details of the layout of the HTML version of the manuals can now be
configured by the user. All manuals are available with and without MathJax
support for display of mathematical formulae. 
\item  The text and HTML versions of manuals make more use of unicode characters (but
the text version is also still reasonably good on terminals with latin1 or
ASCII encoding). 
\item  The PDF version of the manuals uses better fonts. 
\item  Of course, there are various improvements for authors of manuals as well, for
example new functions \texttt{ExtractExamples} (\textbf{GAPDoc: ExtractExamples}) and \texttt{RunExamples} (\textbf{GAPDoc: RunExamples}) for automatic testing and correcting of manual examples. 
\end{itemize}
 
\item  \textsf{Gpd} package by E.J. Moore and C.D. Wensley has been substantially rewritten. The
main extensions provide functions for: 
\begin{itemize}
\item  Subgroupoids of a direct product with complete graph groupoid, specified by a
root group and choice of rays. 
\item  Automorphisms of finite groupoids - by object permutations; by root group
automorphisms; and by ray images. 
\item  The automorphism group of a finite groupoid together with an isomorphism to a
quotient of permutation groups. 
\item  Homogeneous groupoids (unions of isomorphic groupoids) and their morphisms, in
particular homogeneous discrete groupoids: the latter are used in constructing
crossed modules of groupoids in the \textsf{XMod} package. 
\end{itemize}
 
\item  \textsf{GRAPE} package by L.H. Soicher: 
\begin{itemize}
\item  With much help from A. Hulpke, the interface between \textsf{GRAPE} and \texttt{dreadnaut} is now done entirely in \textsf{GAP} code. 
\item  A 32-bit \texttt{nauty/dreadnaut} binary for Windows (XP and later) is included with \textsf{GRAPE}, so now \textsf{GRAPE} provides full functionality under Windows, with no installation necessary. 
\item  Graphs with ordered partitions of their vertices into ``colour-classes'' are now handled by the graph automorphism group and isomorphism testing
functions. An automorphism of a graph with colour-classes is an automorphism
of the graph which additionally preserves the list of colour-classes
(classwise), and an isomorphism from one graph with colour-classes to a second
is a graph isomorphism from the first graph to the second which additionally
maps the first list of colour-classes to the second (classwise). 
\item  The \textsf{GAP} code and old standalone programs for the undocumented functions \texttt{Enum} and \texttt{EnumColadj} have been removed as their functionality can now largely be handled by current
documented \textsf{GAP} and \textsf{GRAPE} functions. 
\end{itemize}
 
\item  \textsf{IO} package by M. Neunh{\"o}ffer: 
\begin{itemize}
\item  New build system to allow for more flexibility regarding the use of compiler
options and adjusting to \textsf{GAP}{\nobreakspace}4.5. 
\item  New functions to access time like \texttt{IO{\textunderscore}gettimeofday}, \texttt{IO{\textunderscore}gmtime} and \texttt{IO{\textunderscore}localtime}. 
\item  Some parallel skeletons built on \texttt{fork} like: \texttt{ParListByFork}, \texttt{ParMapReduceByFork}, \texttt{ParTakeFirstResultByFork} and \texttt{ParWorkerFarmByFork}. 
\item  \texttt{IOHub} objects for automatic I/O multiplexing. 
\item  New functions \texttt{IO{\textunderscore}gethostbyname} and \texttt{IO{\textunderscore}getsockname}. 
\end{itemize}
 
\item  \textsf{IRREDSOL} package by B. H{\"o}fling now covers all irreducible soluble subgroups of $GL(n,q)$ for $q^n < 1000000$ and primitive soluble permutation groups of degree $< 1000000$ (previously, the bound was $65536$). It also has faster group recognition and adds a few omissions for $GL(3,8)$ and $GL(6,5)$. 
\item  \textsf{ParGAP} package by G. Cooperman is now compiled using a system-wide MPI implementation
by default to facilitate running it on proper clusters. There is also an
option to build it with the \textsf{MPINU} library which is still supplied with the package (thanks to P. Smith for
upgrading \textsf{ParGAP} build process). 
\item  \textsf{OpenMath} package by M. Costantini, A. Konovalov, M. Nicosia and A. Solomon now supports
much more OpenMath symbols to facilitate communication by the remote procedure
call protocol implemented in the \textsf{SCSCP} package. Also, a third-party external library to support binary OpenMath
encoding has been replaced by a proper implementation made entirely in \textsf{GAP}. 
\item  \textsf{Orb} package by J. M{\"u}ller, M. Neunh{\"o}ffer and F. Noeske: 

 There have been numerous improvements to this package: 
\begin{itemize}
\item  A new fast implementation of AVL trees (balanced binary trees) in C. 
\item  New interface to hash table functionality and implementation in C for speedup. 
\item  Some new hash functions for various object types like transformations. 
\item  New function \texttt{ORB{\textunderscore}EstimateOrbitSize} using the birthday paradox. 
\item  Improved functionality for product replacer objects. 
\item  New ``tree hash tables''. 
\item  New functionality to compute weak and strong orbits for semigroups and
monoids. 
\item  \texttt{OrbitGraph} for Orb orbits. 
\item  Fast C kernel methods for the following functions: 

 \texttt{PermLeftQuoTransformationNC}, \texttt{MappingPermSetSet}, \texttt{MappingPermListList}, \texttt{ImageSetOfTransformation}, and \texttt{KernelOfTransformation}. 
\item  New build system to allow for more flexibility regarding the use of compiler
options and to adjust to \textsf{GAP}{\nobreakspace}4.5. 
\end{itemize}
 
\item  \textsf{RCWA} package by S. Kohl among the new features and other improvements has the
following: 
\begin{itemize}
\item  A database of all 52394 groups generated by 3 class transpositions of ${\ensuremath{\mathbb Z}}$ which interchange residue classes with modulus less than or equal to 6. This
database contains the orders and the moduli of all of these groups. Also it
provides information on what is known about which of these groups are equal
and how their finite and infinite orbits on ${\ensuremath{\mathbb Z}}$ look like. 
\item  More routines for investigating the action of an rcwa group on ${\ensuremath{\mathbb Z}}$. Examples are a routine which attempts to find out whether a given rcwa group
acts transitively on the set of nonnegative integers in its support and a
routine which looks for finite orbits on the set of all residue classes of ${\ensuremath{\mathbb Z}}$. 
\item  Ability to deal with rcwa permutations of ${\ensuremath{\mathbb Z}}^2$. 
\item  Important methods have been made more efficient in terms of runtime and memory
consumption. 
\item  The output has been improved. For example, rcwa permutations are now \texttt{Display}'ed in ASCII text resembling {\LaTeX} output. 
\end{itemize}
 
\item  The \textsf{XGAP} package by F. Celler and M. Neunh{\"o}ffer can now be used on 64-bit
architectures (thanks to N. Eldredge and M. Horn for sending patches).
Furthermore, there is now an export to XFig option (thanks to Russ Woodroofe
for this patch). The help system in \textsf{XGAP} has been adjusted to \textsf{GAP}{\nobreakspace}4.5. 
\item  \index{Packages under Windows} Additionally, some packages with kernel modules or external binaries are now
available in Windows. The \texttt{-win.zip} archive and the \textsf{GAP} installer for Windows include working versions of the following packages: \textsf{Browse}, \textsf{cvec}, \textsf{EDIM}, \textsf{GRAPE}, \textsf{IO} and \textsf{orb}, which were previously unavailable for Windows users. 
\end{itemize}
 Finally, the following packages are withdrawn: 
\begin{itemize}
\item  \textsf{IF} package by M. Costantini is unmaintained and no longer usable. More advanced
functionality for interfaces to other computer algebra systems is now
available in the \textsf{SCSCP} package by A. Konovalov and S. Linton. 
\item  \textsf{Monoid} package by J. Mitchell is superseded by the \textsf{Citrus} package by the same author. 
\item  \textsf{NQL} package by R. Hartung has been withdrawn by the author. 
\end{itemize}
 }

 }

 
\section{\textcolor{Chapter }{\textsf{GAP} 4.5.5 (July 2012)}}\label{fix455}
\logpage{[ 4, 3, 0 ]}
\hyperdef{L}{X83D3FC7781921387}{}
{
  Fixed bugs which could lead to crashes: 
\begin{itemize}
\item  For small primes (compact fields) \texttt{ZmodnZObj(r,p)} now returns the corresponding FFE to avoid crashes when compacting matrices.
[Reported by Ignat Soroko] 
\end{itemize}
 Other fixed bugs: 
\begin{itemize}
\item  Fixed a bug in \texttt{CommutatorSubgroup} (\textbf{Reference: CommutatorSubgroup}) for fp groups causing infinite recursion, which could, for example, be
triggered by computing automorphism groups. 
\item  Previously, the list of factors of a polynomial was mutable, and hence could
be accidentally corrupted by callers. Now the list of irreducible factors is
stored immutable. To deal with implicit reliance on old code, always a shallow
copy is returned. [reported by Jakob Kroeker] 
\item  Computing high powers of matrices ran into an error for matrices in the format
of the \textsf{cvec} package. Now the library function also works with these matrices. [reported by
Klaus Lux] 
\item  The pseudo tty code which is responsible for spawning subprocesses has been
partially rewritten to allow more than 128 subprocesses on certain systems.
This mechanism is for example used by \textsf{ANUPQ} and \textsf{nq} packages to compute group quotients via an external program. Previously, on
Mac OS X this could be done precisely 128 times, and then an error would
occur. That is, one could e.g. compute 128 nilpotent quotients, and then had
to restart \textsf{GAP} to compute more. This also affected other systems, such as OpenBSD, where it
now also works correctly. 
\item  On Mac OS X, using \textsf{GAP} compiled against GNU readline 6.2, pasting text into the terminal session
would result in this text appearing very slowly, with a 0.1 sec delay between
each ``keystroke''. This is not the case with versions 6.1 and older, and has been reported to
the GNU readline team. In the meantime, we work around this issue in most
situations by setting \texttt{rl{\textunderscore}event{\textunderscore}hook} only if \texttt{OnCharReadHookActive} is set. 
\item  \texttt{ShowUserPreferences} (\textbf{Reference: ShowUserPreferences}) ran into a break loop in case of several undeclared user preferences.
[Reported by James Mitchell] 
\item  \textsf{GAP} did not start correctly if the user preference \texttt{"InfoPackageLoadingLevel"} was set to a number $ >= 3$. The reason is that \texttt{PrintFormattedString} was called before it was installed. The current fix is a temporary solution. 
\item  The \texttt{"hints"} member of \texttt{TypOutputFile} used to contain 3*100 entries, yet \texttt{addLineBreakHint} would write entries with index up to and including 3*99+3=300, leading to a
buffer overflow. This would end up overwriting the \texttt{"stream"} member with -1. Fixed by incrementing the size of \texttt{"hints"} to 301. [Reported by Jakob Kroeker] 
\item  The function \texttt{IsDocumentedWord} tested the given word against strings obtained by splitting help matches at
non-letter characters. This way, variable names containing underscores or
digits were erroneously not regarded as documented, and certain substrings of
these names were erroneously regarded as documented. 
\item  On Windows, an error occurred if one tried to use the default Windows browser
as a help viewer (see \texttt{SetHelpViewer} (\textbf{Reference: SetHelpViewer})). Now the browser opens the top of the correspoding manual chapter. The
current fix is a temporary solution since the problem remains with the
positioning at the required manual section. 
\end{itemize}
 Improved functionality: 
\begin{itemize}
\item  \texttt{WriteGapIniFile} (\textbf{Reference: WriteGapIniFile}) on Windows now produces the \texttt{gap.ini} file with Windows style line breaks. Also, an info message is now printed if
an existing \texttt{gap.ini} file was moved to a backup file \texttt{gap.ini.bak}. 
\item  The \textsf{CTblLib} and \textsf{TomLib} packages are removed from the list of suggested packages of the core part of \textsf{GAP}. Instead they are added to the default list of the user preference \texttt{"PackagesToLoad"}. This way it is possible to configure \textsf{GAP} to not load these packages via changing the default value of \texttt{"PackagesToLoad"}. 
\item  The conjugacy test in $S_n$ for intransitive subgroups was improved. This deals with inefficiency issue in
the case reported by Stefan Kohl. 
\item  Added \texttt{InstallAndCallPostRestore} to \texttt{lib/system.g} and call it in \texttt{lib/init.g} instead of \texttt{CallAndInstallPostRestore} for the function that reads the files listed in \textsf{GAP} command line. This fixes the problem reported by Yevgen Muntyan when \texttt{SaveWorkspace} (\textbf{Reference: SaveWorkspace}) was used in a file listed in \textsf{GAP} command line (before, according to the documentation, \texttt{SaveWorkspace} (\textbf{Reference: SaveWorkspace}) was only allowed at the main \textsf{GAP} prompt). 
\item  There is now a new user preference \texttt{PackagesToIgnore}, see \texttt{SetUserPreference} (\textbf{Reference: SetUserPreference}). It contains a list of names of packages that shall be regarded as not
available at all in the current session, both for autoloading and for later
calls of \texttt{LoadPackage} (\textbf{Reference: LoadPackage}). This preference is useful for testing purposes if one wants to run some code
without loading certain packages. 
\end{itemize}
 }

 
\section{\textcolor{Chapter }{\textsf{GAP} 4.5.6 (September 2012)}}\label{fix456}
\logpage{[ 4, 4, 0 ]}
\hyperdef{L}{X79311F9185A267DB}{}
{
  Improved functionality: 
\begin{itemize}
\item  The argument of \texttt{SaveWorkspace} (\textbf{Reference: SaveWorkspace}) can now start with \texttt{\texttt{\symbol{126}}/} which is expanded to the users home directory. 
\item  Added the method for \texttt{Iterator} (\textbf{Reference: Iterator}) for \texttt{PositiveIntegers} (\textbf{Reference: PositiveIntegers}). [Suggested by Attila Egri-Nagy]. 
\item  Changed kernel tables such that list access functionality for \texttt{T{\textunderscore}SINGULAR} objects can be installed by methods at the \textsf{GAP} level. 
\item  In case of saved history, ``UP'' arrow after starting \textsf{GAP} yields last stored line. The user preference \texttt{HistoryMaxLines} is now used when storing and saving history (see \texttt{SetUserPreference} (\textbf{Reference: SetUserPreference})). 
\end{itemize}
 Fixed bugs which could lead to crashes: 
\begin{itemize}
\item  A crash occuring during garbage collection following a call to \texttt{AClosVec} for a \texttt{GF(2)} code. [Reported by Volker Braun] 
\item  A crash when parsing certain syntactically invalid code. [Reported by multiple
users] 
\item  Fixed and improved command line editing without readline support. Fixed a
segfault which could be triggered by a combination of ``UP'' and ``DOWN'' arrows. [Reported by James Mitchell] 
\item  Fixed a bug in the kernel code for floats that caused a crash on SPARC Solaris
in 32-bit mode. [Reported by Volker Braun] 
\end{itemize}
 Other fixed bugs: 
\begin{itemize}
\item  Very large (more than 1024 digit) integers were not being coded correctly in
function bodies unless the integer limb size was 16 bits. [Reported by Stefan
Kohl] 
\item  An old variable was used in assertion, causing errors in a debugging
compilation. [Reported by Volker Braun] 
\item  The environment variable \texttt{PAGER} is now correctly interpreted when it contains the full path to the pager
program. Furthermore, if the external pager \texttt{less} is found from the environment it is made sure that the option \texttt{-r} is used (same for \texttt{more -f}). [Reported by Benjamin Lorenz] 
\item  Fixed a bug in \texttt{PermliftSeries}. [Reported by Aiichi Yamasaki] 
\item  Fixed discarder function in lattice computation to distinguish general and
zuppo discarder. [Reported by Leonard Soicher] 
\item  The \texttt{GL} (\textbf{Reference: GL (for dimension and a ring)}) and \texttt{SL} (\textbf{Reference: SL (for dimension and a ring)}) constructors did not correctly handle \texttt{GL(filter,dim,ring)}. 
\item  The names of two primitive groups of degree 64 were incorrect. 
\item  The \texttt{\texttt{\symbol{92}}in} (\textbf{Reference: \texttt{\symbol{92}}in operation for testing membership}) method for groups handled by a nice monomorphism sometimes could produce an
error in situations where it should return false. This only happened when
using \texttt{SeedFaithfulAction} to influence how \texttt{NiceMonomorphism} (\textbf{Reference: NiceMonomorphism}) builds the nice monomorphims for a matrix groups. 
\item  Wrong \texttt{PrintObj} (\textbf{Reference: PrintObj}) method was removed to fix delegations accordingly to  (\textbf{Reference: View and Print}). 
\item  Fixed a method for \texttt{Coefficients} (\textbf{Reference: Coefficients}) which, after Gaussian elimination, did not check that the coefficients
actually lie in the left-acting-domain of the vector space. This could lead to
a wrong answer in a vector space membership test. [Reported by Kevin Watkins] 
\end{itemize}
 Improved documentation: 
\begin{itemize}
\item  Removed outdated statements from the documentation of \texttt{StructureDescription} (\textbf{Reference: StructureDescription}) which now non-ambiguosly states that \texttt{StructureDescription} is not an isomorphism invariant: non-isomorphic groups can have the same
string value, and two isomorphic groups in different representations can
produce different strings. 
\item  \textsf{GAP} now allows overloading of a loaded help book by another one. In this case,
only a warning is printed and no error is raised. This makes sense if a book
of a not loaded package is loaded in a workspace and then \textsf{GAP} is started with a root path that contains a newer version. [Reported by
Sebastian Gutsche] 
\item  Provided a better description of user preferences mechanism ( (\textbf{Reference: Configuring User preferences})) and a hint to familiarise with them using \texttt{WriteGapIniFile} (\textbf{Reference: WriteGapIniFile}) function to create a file which contains descriptions of all known user
preferences and also sets those user preferences which currently do not have
their default value. One can then edit that file to customize (further) the
user preferences for future \textsf{GAP} sessions. 
\end{itemize}
 New packages added for the redistribution with \textsf{GAP}: 
\begin{itemize}
\item  \textsf{AutoDoc} package by S. Gutsche, providing tools for automated generation of \textsf{GAPDoc} manuals. 
\item  \textsf{Convex} package by S. Gutsche, which provides structures and algorithms for convex
geometry. 
\item  \textsf{PolymakeInterface} package by T. Baechler and S. Gutsche, providing a link to the callable
library of the \textsf{polymake} system (\href{http://www.polymake.org} {\texttt{http://www.polymake.org}}). 
\item  \textsf{ToolsForHomalg} package by M. Barakat, S. Gutsche and M. Lange-Hegermann, which provides some
auxiliary functionality for the \textsf{homalg} project (\href{http://homalg.math.rwth-aachen.de/} {\texttt{http://homalg.math.rwth-aachen.de/}}). 
\end{itemize}
 }

 
\section{\textcolor{Chapter }{\textsf{GAP} 4.5.7 (December 2012)}}\label{fix457}
\logpage{[ 4, 5, 0 ]}
\hyperdef{L}{X81528B5F78B996D2}{}
{
  Fixed bugs which could lead to crashes: 
\begin{itemize}
\item Closing with \texttt{LogInputTo} (or \texttt{LogOutputTo}) a logfile opened with \texttt{LogTo} (\textbf{Reference: LogTo}) left the data structures corrupted, resulting in a crash. 
\item  On 32-bit systems we can have long integers \texttt{n} such that \texttt{Log2Int(n)} is not an immediate integer. In such cases \texttt{Log2Int} gave wrong or corrupted results which in turn could crash \textsf{GAP}, e.g., in \texttt{ViewObj(n)}. 
\item  Some patterns of use of \texttt{UpEnv} (\textbf{Reference: UpEnv}) and \texttt{DownEnv} (\textbf{Reference: DownEnv}) were leading to a segfault. 
\end{itemize}
 Other fixed bugs: 
\begin{itemize}
\item  Viewing of long negative integers was broken, because it went into a break
loop. 
\item  Division by zero in \texttt{ZmodnZ} (\textbf{Reference: ZmodnZ}) ($n$ not prime) produced invalid objects. [Reported by Mark Dickinson] 
\item  Fixed a bug in determining multiplicative inverse for a zero polynomial. 
\item  Fixed a bug causing infinite recursion in \texttt{NaturalHomomorphismByNormalSubgroup} (\textbf{Reference: NaturalHomomorphismByNormalSubgroup}). 
\item  A workaround was added to deal with a package method creating pcgs for
permutation groups for which the entry \texttt{permpcgsNormalSteps} is missing. 
\item  For a semigroup of associative words that is not the full semigroup of all
associative words, the methods for \texttt{Size} (\textbf{Reference: Size}) and \texttt{IsTrivial} (\textbf{Reference: IsTrivial}) called one another causing infinite recursion. 
\item  The 64-bit version of the \texttt{gac} script produced wrong ({\textgreater}= 2\texttt{\symbol{94}}31) CRC values
because of an integer conversion problem. 
\item  It was not possible to compile \textsf{GAP} on some systems where \texttt{HAVE{\textunderscore}SELECT} detects as false. 
\item  Numbers in memory options on the command line exceeding
2\texttt{\symbol{94}}32 could not be parsed correctly, even on 64-bit systems.
[Reported by Volker Braun] 
\end{itemize}
 New packages added for the redistribution with \textsf{GAP}: 
\begin{itemize}
\item  \textsf{Float} package by L. Bartholdi, which extends \textsf{GAP} floating-point capabilities by providing new floating-point handlers for
high-precision real, interval and complex arithmetic using MPFR, MPFI, MPC or
CXSC external libraries. It also contains a very high-performance
implementation of the LLL (Lenstra-Lenstra-Lov{\a'a}sz) lattice reduction
algorithm via the external library FPLLL. 
\item  \textsf{ToricVarieties} package by S. Gutsche, which provides data structures to handle toric
varieties by their commutative algebra structure and by their combinatorics. 
\end{itemize}
 }

 }

    
\chapter{\textcolor{Chapter }{Overview of updates for \textsf{GAP} 4.4}}\label{HistoryOfGAP44}
\logpage{[ 5, 0, 0 ]}
\hyperdef{L}{X78AE465E7AA00F25}{}
{
  This chapter lists changes in the main system (excluding packages) that have
been corrected or added in bugfixes and updates for \textsf{GAP} 4.4.  
\section{\textcolor{Chapter }{\textsf{GAP} 4.4 Bugfix 2 (April 2004)}}\label{fix442}
\logpage{[ 5, 1, 0 ]}
\hyperdef{L}{X82AB84287ED2B8FF}{}
{
  Fixed bugs which could lead to crashes: 
\begin{itemize}
\item  A crash when incorrect types of arguments are passed to \texttt{FileString}. 
\end{itemize}
 Other fixed bugs: 
\begin{itemize}
\item A bug in \texttt{DerivedSubgroupTom} and \texttt{DerivedSubgroupsTom}. 
\item An error in the inversion of certain \texttt{ZmodnZObj} elements. 
\item A wrong display string of the numerator in rational functions returned by \texttt{MolienSeriesWithGivenDenominator} (in the case that the constant term of this numerator is zero). 
\end{itemize}
 }

  
\section{\textcolor{Chapter }{\textsf{GAP} 4.4 Bugfix 3 (May 2004)}}\label{fix443}
\logpage{[ 5, 2, 0 ]}
\hyperdef{L}{X8208E7F578754467}{}
{
  Fixed bugs which could produce wrong results: 
\begin{itemize}
\item Incorrect setting of system variables (e.g., home directory and command line
options) after loading a workspace. 
\item Wrong handling of integer literals within functions or loops on 64-bit
architectures (only integers in the range from $2^{28}$ to $2^{60}$). 
\end{itemize}
 Fixed bugs which could lead to crashes: 
\begin{itemize}
\item A problem in the installation of the multiplication routine for matrices that
claimed to be applicable for more general list multiplications. 
\item A problem when computing weight distributions of codes with weights
{\textgreater} $2^{28}$. 
\end{itemize}
 Other fixed bugs: 
\begin{itemize}
\item Problems with the online help with some manual sections. 
\item Problems of the online help on Windows systems. 
\item A problem in \texttt{GQuotients} when mapping from a finitely presented group which has a free direct factor. 
\item A bug in the function \texttt{DisplayRevision}. 
\item The trivial finitely presented group on no generators was not recognized as
finitely presented. 
\item A problem with \texttt{Process}. 
\item A problem when intersecting subgroups of finitely presented groups that are
represented in ``quotient representation'' with the quotient not apermutation group. 
\item A bug in the generic \texttt{Intersection2} method for vector spaces, in the case that both spaces are trivial. 
\item Enable ReeGroup(q) for $q = 3$. 
\end{itemize}
 }

  
\section{\textcolor{Chapter }{\textsf{GAP} 4.4 Bugfix 4 (December 2004)}}\label{fix444}
\logpage{[ 5, 3, 0 ]}
\hyperdef{L}{X87A965F97E6E6BD0}{}
{
  Fixed bugs which could produce wrong results: 
\begin{itemize}
\item  An error in the \texttt{Order} method for matrices over cyclotomic fields which caused this method to return \texttt{infinity} for matrices of finite order in certain cases. 
\item  Representations computed by \texttt{IrreducibleRepresentations} in characteristic 0 erraneously claimed to be faithful. 
\item A primitive representation of degree 574 for PSL(2,41) has been missing in the
classification on which the \textsf{GAP} library was built. 
\item  A bug in \texttt{Append} for compressed vectors over GF(2): if the length of the result is 1 mod 32 (or
64) the last entry was forgotten to copy. 
\item  A problem with the Ree group Ree(3) of size 1512 claiming to be simple. 
\item  An error in the membership test for groups GU(n,q) and SU(n,q) for non-prime $q$. 
\item  An error in the kernel code for ranges which caused e.g. \texttt{-1 in [1..2]} to return \texttt{true}. 
\item  An error recording boolean lists in saved workspaces. 
\item  A problem in the selection function for primitive and transitive groups if no
degree is given. 
\item  \texttt{ReducedConfluentRewritingSystem} returning a cached result that might not conform to the ordering specified. 
\end{itemize}
 Other fixed bugs: 
\begin{itemize}
\item  A problem with the function \texttt{SuggestUpdates} to check for the most recent version of packages available. 
\item  A problem that caused \texttt{MatrixOfAction} to produce an error when the algebra module was constructed as a direct sum. 
\item  Problems with computing $n$-th power maps of character tables, where n is negative and the table does not
yet store its irreducible characters. 
\item  Element conjugacy in large-base permutation groups sometimes was unnecessarily
inefficient. 
\item  A missing method for getting the letter representation of an associate word in
straight line program representation. 
\item  A problem with the construction of vector space bases where the given list of
basis vectors is itself an object that was returned by \texttt{Basis}. 
\item  A problem of \texttt{AbelianInvariantsMultiplier} insisting that a result of \texttt{IsomorphismFpGroup} is known to be surjective. 
\item  An error in the routine for \texttt{Resultant} if one of the polynomials has degree zero. 
\end{itemize}
 }

  
\section{\textcolor{Chapter }{\textsf{GAP} 4.4 Update 5 (May 2005)}}\label{fix445}
\logpage{[ 5, 4, 0 ]}
\hyperdef{L}{X80221EDA877CC106}{}
{
  Fixed bugs which could produce wrong results: 
\begin{itemize}
\item  \texttt{GroupWithGenerators} (\textbf{Reference: GroupWithGenerators}) returned a meaningless group object instead of signaling an error when it was
called with an empty list of generators. 
\item  When computing preimages under an embedding into a direct product of
permutation groups, if the element was not in the image of the embedding then
a permutation had been returned instead of \texttt{fail}. 
\item  Two problems with \texttt{PowerMod} (\textbf{Reference: PowerMod}) for polynomials. [Reported by Jack Schmidt] 
\item  Some methods for computing the sum of ideals returned the first summand
instead of the sum. [Reported by Alexander Konovalov] 
\item  Wrong result in \texttt{Intersection} (\textbf{Reference: Intersection}) for pc groups. 
\item  The function \texttt{CompareVersionNumbers} (\textbf{Reference: CompareVersionNumbers}) erroneously ignored leading non-digit characters. 

 A new feature in the corrected version is an optional third argument \texttt{"equal"}, which causes the function to return \texttt{true} only if the first two arguments describe equal version numbers; documentation
is available in the ext-manual. This new feature is used in \texttt{LoadPackage} (\textbf{Reference: LoadPackage}), now one can require a specific version of a package. 

 The library code still contained parts of the handling of completion files for
packages, which does not work and therefore had already been removed from the
documentation. This code has now been removed. 

 Now a new component \texttt{PreloadFile} is supported in \texttt{The PackageInfo.g File} (\textbf{Reference: The PackageInfo.g File}) files; if it is bound then the file in question is read immediately before the
package or its documentation is loaded. 
\item  The result of \texttt{String} (\textbf{Reference: String}) for strings not in \texttt{IsStringRep} (\textbf{Reference: IsStringRep}) that occur as list entries or record components was erroneously missing the
double quotes around the strings. 
\item  A bug which caused \texttt{InducedPcgs} (\textbf{Reference: InducedPcgs}) to return a pcgs which is not induced wrt. the parent pcgs of \texttt{pcgs}. This may cause unpredictable behaviour, e.{\nobreakspace}g. when \texttt{SiftedPcElement} is used subsequently. [Reported by Alexander Konovalov] 
\item  Fixed a bug in \texttt{SmallGroupsInformation(512)}. 
\item  \texttt{PowerModCoeffs} (\textbf{Reference: PowerModCoeffs}) with exponent 1 for compressed vectors did not reduce (a copy of) the input
vector before returning it. [Reported by Frank L{\"u}beck] 
\item  Sorting a mutable non-plain list (e.g., a compressed matrix over fields of
order {\textless} 257) could potentially destroy that object. [Reported by
Alexander Hulpke] 
\item  Under rare circumstances computing the closure of a permutation group by a
normalizing element could produce a corrupted stabilizer chain. (The
underlying algorithm uses random elements, probability of failure was below 1
percent). [Reported by Thomas Breuer] 
\end{itemize}
 Fixed bugs which could lead to crashes: 
\begin{itemize}
\item  Some code and comments in the \textsf{GAP} kernel assumed that there is no garbage collection during the core printing
function \texttt{Pr}, which is not correct. This could cause \textsf{GAP} in rare cases to crash during printing permutations, cyclotomics or strings
with zero bytes. [Reported by Warwick Harvey] 
\end{itemize}
 Other fixed bugs: 
\begin{itemize}
\item  A rare problem with the choice of prime in the Dixon-Schneider algorithm for
computing the character table of a group. [Reported by Jack Schmidt] 
\item  \texttt{DirectProduct} (\textbf{Reference: DirectProduct}) for trivial permutation groups returned a strange object. 
\item  A problem with \texttt{PolynomialReduction} (\textbf{Reference: PolynomialReduction}) running into an infinite loop. 
\item  Adding linear mappings with different image domains was not possible.
[Reported by Pasha Zusmanovich] 
\item  Multiplying group ring elements with rationals was not possible. [Reported by
Laurent Bartholdi] 
\item  \texttt{Random} (\textbf{Reference: Random Sources}) now works for finite fields of size larger than $2^{28}$. [Reported by Jack Schmidt] 
\item  Univariate polynomial creators did modify the coefficient list passed.
[Reported by J{\"u}rgen M{\"u}ller] 
\item  Fixed \texttt{IntHexString} (\textbf{Reference: IntHexString}) to accept arguments not in \texttt{IsStringRep}; the argument is now first converted if necessary. [Reported by Kenn
Heinrich] 
\item  The library code for stabilizer chains contained quite some explicit
references to the identity \texttt{()}. This is unfortunate if one works with permutation groups, the elements of
which are not plain permutations but objects which carry additional
information like a memory, how they were obtained from the group generators.
For such cases it is much cleaner to use the \texttt{One(...)} operation instead of \texttt{()}, such that the library code can be used for a richer class of group objects.
This fix contains only rather trivial changes \texttt{()} to \texttt{One(...)} which were carefully checked by me. The tests for permutation groups all run
without a problem. However, it is relatively difficult to provide test code
for this particular change, since the "improvement" only shows up when one
generates new group objects. This is for example done in the package \textsf{recog} which is in preparation. [Reported by Akos Seress and Max Neunh{\"o}ffer] 
\item  Using \texttt{\texttt{\symbol{123}}\texttt{\symbol{125}}} to select elements of a known inhomogenous dense list produced a list that
might falsely claim to be known inhomogenous, which could lead to a segfault
if the list typing code tried to mark it homogenous, since the code intended
to catch such errors also had a bug. [Reported by Steve Linton] 
\item  The record for the generic iterator construction of subspaces domains of
non-row spaces was not complete. 
\item  When a workspace has been created without packages(\texttt{-A} option) and is loaded into a \textsf{GAP} session without packages (same option) then an error message is printed. 
\item  So far the functions \texttt{IsPrimeInt} (\textbf{Reference: IsPrimeInt}) and \texttt{IsProbablyPrimeInt} (\textbf{Reference: IsProbablyPrimeInt}) are essentially the same except that \texttt{IsPrimeInt} issues an additional warning when (non-proven) probable primes are considered
as primes. 

 These warnings now print the probable primes in question as well; if a
probable prime is used several times then the warning is also printed several
times; there is no longer a warning for some known large primes; the warnings
can be switched off. See \texttt{IsPrimeInt} (\textbf{Reference: IsPrimeInt}) for more details. 

 If we get a reasonable primality test in \textsf{GAP} we will change the definition of \texttt{IsPrimeInt} to do a proper test. 
\item  Corrected some names of primitive groups in degree 26. [Reported by Robert F.
Bailey] 
\end{itemize}
 New or improved functionality: 
\begin{itemize}
\item  Several changes for \texttt{ConwayPolynomial} (\textbf{Reference: ConwayPolynomial}): 
\begin{itemize}
\item  many new pre-computed polynomials 
\item  put data in several separate files (only read when needed) 
\item  added info on origins of pre-computed polynomials 
\item  improved performance of \texttt{ConwayPolynomial} (\textbf{Reference: ConwayPolynomial}) and \texttt{IsPrimitivePolynomial} (\textbf{Reference: IsPrimitivePolynomial}) for p {\textless} 256 
\item  improved documentation of \texttt{ConwayPolynomial} 
\item  added and documented new functions \texttt{IsCheapConwayPolynomial} (\textbf{Reference: IsCheapConwayPolynomial}) and \texttt{RandomPrimitivePolynomial} (\textbf{Reference: RandomPrimitivePolynomial}) 
\end{itemize}
 
\item  Added method for \texttt{NormalBase} (\textbf{Reference: NormalBase}) for extensions of finite fields. 
\item  Added more help viewers for the HTML version of the documentation (firefox,
mozilla, konqueror, w3m, safari). 
\item  New function \texttt{ColorPrompt} (\textbf{Reference: ColorPrompt}). (Users of former versions of a \texttt{colorprompt.g} file: Now you just need a \texttt{ColorPrompt(true);} in your \texttt{.gaprc} file.) 
\item  Specialised kernel functions to support \textsf{GUAVA} 2.0. \textsf{GAP} will only load \textsf{GUAVA} in version at least 2.002 after this update. 
\item  Now there is a kernel function \texttt{CYC{\textunderscore}LIST} for converting a list of rationals into a cyclotomic, without arithmetics
overhead. 
\item  New functions \texttt{ContinuedFractionExpansionOfRoot} (\textbf{Reference: ContinuedFractionExpansionOfRoot}) and \texttt{ContinuedFractionApproximationOfRoot} (\textbf{Reference: ContinuedFractionApproximationOfRoot}) for computing continued fraction expansions and continued fraction
approximations of real roots of polynomials with integer coefficients. 
\item  A method for computing structure descriptions for finite groups, available via \texttt{StructureDescription} (\textbf{Reference: StructureDescription}). 
\item  This change contains the new, extended version of the \textsf{SmallGroups} package. For example, the groups of orders $p^4$, $p^5$, $p^6$ for arbitrary primes $p$, the groups of square-free order and the groups of cube-free order at most
50000 are included now. For more detailed information see the announcement of
the extended package. 
\item  The function \texttt{ShowPackageVariables} gives an overview of the global variables in a package. It is thought as a
utility for package authors and referees. (It uses the new function \texttt{IsDocumentedVariable}.) 
\item  The mechanisms for testing \textsf{GAP} has been improved: 
\begin{itemize}
\item  The information whether a test file belongs to the list in \texttt{tst/testall.g} is now stored in the test file itself.
\item  Some targets for testing have been added to the \texttt{Makefile} in the \textsf{GAP} root directory, the output of the tests goes to the new directory \texttt{dev/log}.
\item  Utility functions for testing are in the new file \texttt{tst/testutil.g}. Now the loops over (some or all) files \texttt{tst/*.tst} can be performed with a function call, and the file \texttt{tst/testall.g} can be created automatically; the file \texttt{tst/testfull.g} is now obsolete. The remormalization of the scaling factors can now be done
using a \textsf{GAP} function, so the file \texttt{tst/renorm.g} is obsolete.
\item  Now the functions \texttt{START{\textunderscore}TEST} and \texttt{STOP{\textunderscore}TEST} use components in \texttt{GAPInfo} instead of own globals, and the random number generator is always reset in \texttt{START{\textunderscore}TEST}.
\item  \texttt{GAPInfo.SystemInformation} now takes two arguments, now one can use it easier in the tests.
\end{itemize}
 
\item  \texttt{MultiplicationTable} (\textbf{Reference: MultiplicationTable}) is now an attribute, and the construction of a magma, monoid, etc. from
multiplication tables has been unified. 
\end{itemize}
 }

  
\section{\textcolor{Chapter }{\textsf{GAP} 4.4 Update 6 (September 2005)}}\label{fix446}
\logpage{[ 5, 5, 0 ]}
\hyperdef{L}{X80F9D3E07DD59BD9}{}
{
  Attribution of bugfixes and improved functionalities to those who reported or
provided these, respectively, is still fairly incomplete and inconsistent with
this update. We apologise for this fact and will discuss until the next update
how to improve this feature. 

 Fixed bugs which could produce wrong results: 
\begin{itemize}
\item  The perfect group library does not contain any information on the trivial
group, so the trivial group must be handled specially. \texttt{PerfectGroup} (\textbf{Reference: PerfectGroup}) and \texttt{NrPerfectLibraryGroups} were changed to indicate that the trivial group is not part of the library. 
\item  The descriptions of \texttt{PerfectGroup(734832,3)} and \texttt{PerfectGroup(864000,3)} were corrected in the \texttt{Finite Perfect Groups} (\textbf{Reference: Finite Perfect Groups}) library of perfect groups. 
\item  The functions \texttt{EpimorphismSchurCover} (\textbf{Reference: EpimorphismSchurCover}) and \texttt{AbelianInvariantsMultiplier} (\textbf{Reference: AbelianInvariantsMultiplier}) may have produced wrong results without warning [Reported by Colin Ingalls].
These problems are fixed. However, the methods currently used can be expected
to be slower than the ones used before; we hope to fix this in the next
version of \textsf{GAP}. 
\item  \texttt{DerivedSubgroup} (\textbf{Reference: DerivedSubgroup}) and \texttt{CommutatorSubgroup} (\textbf{Reference: CommutatorSubgroup}) for permutation groups sometimes returned groups with an incorrect stabilizer
chain due to a missing verification step after a random Schreier Sims. 
\item  \texttt{NaturalHomomorphismByNormalSubgroup} (\textbf{Reference: NaturalHomomorphismByNormalSubgroup}) for FpGroups did unnecessary rewrites. 
\item  The alternating group $A_3$ incorrectly claimed to be not simple. 
\item  \texttt{ExponentSyllable} (\textbf{Reference: ExponentSyllable}) for straight line program elements gave a wrong result. 
\item  \texttt{PrimePGroup} (\textbf{Reference: PrimePGroup}) is defined to return \texttt{fail} for trivial groups, but if the group was constructed as a factor or subgroup
of a known $p$-group, the value of $p$ was retained. 
\item  The functions \texttt{TestPackageAvailability} (\textbf{Reference: TestPackageAvailability}) and \texttt{LoadPackage} (\textbf{Reference: LoadPackage}) did not work correctly when one asked for a particular version of the package,
via a version number starting with the character \texttt{=}, in the sense that a version with a larger version number was loaded if it
was available. [Reported by Burkhard H{\"o}fling] 
\item  The generator names constructed by \texttt{AlgebraByStructureConstants} (\textbf{Reference: AlgebraByStructureConstants}) were nonsense. 
\item  The undocumented function (but recently advertised on gap-dev) \texttt{COPY{\textunderscore}LIST{\textunderscore}ENTRIES} did not handle overlapping source and destination areas correctly in some
cases. 
\item  The elements in a free magma ring have the filter \texttt{IsAssociativeElement} (\textbf{Reference: IsAssociativeElement}) set whenever the elements in the underlying magma and in the coefficients ring
have this filter set. [Reported by Randy Cone] 
\item  The function \texttt{InstallValue} (\textbf{Reference: InstallValue}) must not be used for objects in the filter \texttt{IsFamily} because these objects are compared via \texttt{IsIdenticalObj} (\textbf{Reference: IsIdenticalObj}). [Reported by Max Neunh{\"o}ffer] 
\end{itemize}
 Fixed bugs which could lead to crashes: 
\begin{itemize}
\item  Problem in composition series for permutation groups for non-Frobenius groups
with regular point stabilizer. 
\item  After lots of computations with compressed GF(2) vectors \textsf{GAP} occasionally crashed. The reason were three missing \texttt{CHANGED{\textunderscore}BAG}s in \texttt{SemiEchelonPListGF2Vecs}. They were missing, because a garbage collection could be triggered during
the computation such that newly created bags could become ``old''. It is not possible to provide test code because the error condition cannot
easily be reproduced. [Reported by Klaus Lux] 
\item  Minor bug that crashed \textsf{GAP}: The type of \texttt{IMPLICATIONS} could not be determined in a fresh session. [Reported by Marco Costantini] 
\item  \texttt{Assert} (\textbf{Reference: Assert}) caused an infinite loop if called as the first line of a function called from
another function. 
\end{itemize}
 Other fixed bugs: 
\begin{itemize}
\item  Wrong choice of prime in Dixon-Schneider if prime is bigger than group order
(if group has large exponent). 
\item  Groebner basis code ran into problems when comparing monomial orderings. 
\item  When testing for conjugacy of a primitive group to an imprimitive group,\textsf{GAP} runs into an error in EARNS calculation. [Reported by John Jones] 
\item  The centre of a magma is commonly defined to be the set of elements that
commute and associate with all elements. The previous definition left out ``associate'' and caused problems with extending the functionality to nonassociative loops.
[Reported by Petr Vojtechovsky] 
\item  New kernel methods for taking the intersection and difference between sets of
substantially different sizes give a big performance increase. 
\item  The commands \texttt{IsNaturalSymmetricGroup} (\textbf{Reference: IsNaturalSymmetricGroup}) and \texttt{IsNaturalAlternatingGroup} (\textbf{Reference: IsNaturalAlternatingGroup}) are faster and should run much less often into inefficient tests. 
\item  The perfect group library, see \texttt{Finite Perfect Groups} (\textbf{Reference: Finite Perfect Groups}), is split into several files which are loaded and unloaded to keep memory
usage down. The global variable \texttt{PERFSELECT} is a blist which indicates which orders are currently loaded. An off-by-one
error wrongly added the last order of the previous file into the list of valid
orders when a new file was loaded. A subsequent access to this order raises an
error. 
\item  Up to now, the method installed for testing the membership of rationals in the
field of rationals via \texttt{IsRat} (\textbf{Reference: IsRat}) was not called; instead a more general method was used that called \texttt{Conductor} (\textbf{Reference: Conductor (for a cyclotomic)}) and thus was much slower. Now the special method has been ranked up by
changing the requirements in the method installation. 
\item  Fixed a bug in \texttt{APPEND{\textunderscore}VEC8BIT}, which was triggered in the following situation: Let \texttt{e} be the number of field elements stored in one byte. If a compressed
8bit-vector \texttt{v} had length not divisible by \texttt{e} and another compressed 8-bit vector \texttt{w} was appended, such that the sum of the lengths became divisible by \texttt{e}, then one 0 byte too much was written, which destroyed the \texttt{TNUM} of the next \textsf{GAP} object in memory. [Reported by Klaus Lux] 
\item  \texttt{PermutationCycle} (\textbf{Reference: PermutationCycle}) returned \texttt{fail} if the cycle was not a contiguous subset of the specified domain. [Reported by
Luc Teirlinck] 
\item  Now \texttt{Inverse} (\textbf{Reference: Inverse}) correctly returns \texttt{fail} for zeros in finite fields (and does no longer enter a break loop). 
\item  Up to now, \texttt{CharacterDegrees} (\textbf{Reference: CharacterDegrees}) ignored the attribute \texttt{Irr} (\textbf{Reference: Irr}) if the argument was a group that knew that it was solvable. 
\item  The function \texttt{Debug} now prints a meaningful message if the user tries to debug an operation. Also,
the help file for \texttt{vi} is now available in the case of several \textsf{GAP} root directories. 
\item  It is no longer possible to create corrupt objects via ranges of length
{\textgreater}$2^{28}$, resp. {\textgreater}$2^{60}$ (depending on the architecture). The limitation concerning the arguments of
ranges is documented. [Reported by Stefan Kohl] 
\item  Now \texttt{IsElementaryAbelian} (\textbf{Reference: IsElementaryAbelian}) and \texttt{ClassPositionsOfMinimalNormalSubgroups} (\textbf{Reference: ClassPositionsOfMinimalNormalSubgroups}) are available for ordinary character tables. Now the operation \texttt{CharacterTableIsoclinic} (\textbf{Reference: CharacterTableIsoclinic}) is an attribute, and there is another new attribute \texttt{SourceOfIsoclinicTable} (\textbf{Reference: SourceOfIsoclinicTable}) that points back to the original table; this is used for computing the Brauer
tables of those tables in the character table library that are computed using \texttt{CharacterTableIsoclinic}. Now \texttt{ClassPositionsOfDerivedSubgroup} (\textbf{Reference: ClassPositionsOfDerivedSubgroup}) avoids calling \texttt{Irr} (\textbf{Reference: Irr}), since \texttt{LinearCharacters} (\textbf{Reference: LinearCharacters}) is sufficient. Now \texttt{ClassPositionsOfElementaryAbelianSeries} (\textbf{Reference: ClassPositionsOfElementaryAbelianSeries}) works also for the table of the trivial group. Restrictions of character
objects know that they are characters. 

 A few formulations in the documentation concerning character tables have been
improved slightly. 
\item  Up to now, \texttt{IsPGroup} (\textbf{Reference: IsPGroup}) has rarely been set. Now many basic operations such as \texttt{SylowSubgroup} (\textbf{Reference: SylowSubgroup}) set this attribute on the returned result. 
\item  Computing an enumerator for a semigroup required too much time because it used
all elements instead of the given generators. [Reported by Manuel Delgado] 
\item  Avoid potential error message when working with automorphism groups. 
\item  Fixed wrong page references in manual indices. 
\item  Make \texttt{MutableCopyMat} an operation and install the former function which does call \texttt{List} (\textbf{Reference: Lists}) with \texttt{ShallowCopy} (\textbf{Reference: ShallowCopy}) the default method for lists. Also use this in a few appropriate places. 
\item  An old DEC compiler doesn't like C preprocessor directives that are preceded
by whitespace. Removed such whitespace. [Reported by Chris Wensley] 
\end{itemize}
 New or improved functionality: 
\begin{itemize}
\item  The primitive groups library has been extended to degree 2499. 
\item  New operation \texttt{Remove} (\textbf{Reference: Remove}) and extended functionality of \texttt{Add} (\textbf{Reference: Add}) with an optional argument giving the position of the insertion. They are based
on an efficient kernel function \texttt{COPY{\textunderscore}LIST{\textunderscore}ENTRIES}. 
\item  Added fast kernel implementation of Tarjan's algorithm for strongly connected
components of a directed graph. 
\item  Now \texttt{IsProbablyPrimeInt} (\textbf{Reference: IsProbablyPrimeInt}) can be used with larger numbers. (Made internal function \texttt{TraceModQF} non-recursive.) 
\item  A new operation \texttt{PadicValuation} (\textbf{Reference: PadicValuation}) and a corresponding method for rationals. 
\item  A new operation \texttt{PartialFactorization} (\textbf{Reference: PartialFactorization}) has been added, and a corresponding method for integers has been installed.
This method allows one to specify the amount of work to be spent on looking
for factors. 
\item  The generators of full s. c. algebras can now be accessed with the dot
operator. [Reported by Marcus Bishop] 
\item  New Conway polynomials computed by Kate Minola, John Bray, Richard Parker. 
\item  A new attribute \texttt{EpimorphismFromFreeGroup} (\textbf{Reference: EpimorphismFromFreeGroup}). The code has been written by Alexander Hulpke. 
\item  The functions \texttt{Lambda} (\textbf{Reference: Lambda}), \texttt{Phi} (\textbf{Reference: Phi}), \texttt{Sigma} (\textbf{Reference: Sigma}), and \texttt{Tau} (\textbf{Reference: Tau}) have been turned into operations, to admit the installation of methods for
arguments other than integers. 
\item  Up to now, one could assign only lists with \texttt{InstallFlushableValue} (\textbf{Reference: InstallFlushableValue}). Now also records are admitted. 
\item  \texttt{InstallMethod} (\textbf{Reference: InstallMethod}) now admits entering a list of strings instead of a list of required filters.
Each such string must evaluate to a filter when used as the argument of \texttt{EvalString} (\textbf{Reference: EvalString}). The advantage of this variant is that these strings are used to compose an
info string (which is shown by \texttt{ApplicableMethod}) that reflects exactly the required filters. 
\item  In test files that are read with \texttt{ReadTest} (\textbf{Reference: ReadTest}), the assertion level is set to 2 between \texttt{START{\textunderscore}TEST} and \texttt{STOP{\textunderscore}TEST}. This may result in runtimes for the tests that are substantially longer than
the usual runtimes with default assertion level 0. In particular this is the
reason why some of the standard test files require more time in \textsf{GAP} 4.4.6 than in \textsf{GAP} 4.4.5. 
\item  Some very basic functionality for floats. 
\end{itemize}
 }

  
\section{\textcolor{Chapter }{\textsf{GAP} 4.4 Update 7 (March 2006)}}\label{fix447}
\logpage{[ 5, 6, 0 ]}
\hyperdef{L}{X7B8B22C77DC970F8}{}
{
  New or improved functionality: 
\begin{itemize}
\item  The \texttt{Display} (\textbf{Reference: Display}) functionality for character tables has been extended by addition of an option
to show power maps and centralizer orders in a format similar to that used in
the ATLAS. Furthermore the options handling is now hierarchical, in order to
admit more flexible overloading. 
\item  For the function \texttt{LowIndexSubgroupsFpGroup} (\textbf{Reference: LowIndexSubgroupsFpGroup}), there is now an iterator variant \texttt{LowIndexSubgroupsFpGroupIterator}. [Suggested (and based on code contributed) by Michael Hartley] 
\item  Semigroup functionality in \textsf{GAP} has been improved and extended. Green's relations are now stored differently,
making the system more amenable to new methods for computing these relations
in special cases. It is now possible to calculate Green's classes etc. without
computing the entire semigroup or necessarily loading the package \textsf{MONOID}. Furthermore, the Froidure-Pin algorithm has now been implemented in \textsf{GAP}. 
\item  Functionality for creating free products of any list of groups for which a
finite presentation can be determined had been added. This function returns a
finitely presented group. This functionality includes the \texttt{Embedding} operation. As an application of this new code a specialized direct product
operation has been added for finitely presented groups which returns a
finitely presented group. This application includes \texttt{Embedding} and \texttt{Projection} functionality. 
\item  Some new Straight Line Program (SLP) functionality has been added. The new
functions take given SLPs and create new ones by restricting to a subset of
the results, or to an intermediate result or by calculating the product of the
results of two SLPs. 
\item  New code has been added to allow group elements with memory; that is, they
store automatically how they were derived from some given set of generators.
Note that there is not yet documentation for this functionality, but some
packages already use it. 
\item  New code has been added to handle matrices and vectors in such a way that they
do not change their representation in a generic manner. 
\item  The \texttt{Irr} (\textbf{Reference: Irr}) method for $p$-solvable $p$-modular Brauer tables now keeps the order of the irreducibles in the ordinary
table. 
\item  \textsf{GAP} can now handle any finite field for which the Conway polynomial is known or
can be computed. 
\item  New Conway polynomials provided by John Bray and Kate Minola have been added. 
\item  The \texttt{ReadTest} (\textbf{Reference: ReadTest}) methods for strings (filenames) and streams now automatically set the screen
width (see \texttt{SizeScreen} (\textbf{Reference: SizeScreen})) to 80 before the tests, and reset it afterwards. 
\item  Now a few more checks are done during the \texttt{configure} phase of compiling for future use of some I/O functions of the C-library in a
package. Also the path to the \textsf{GAP} binaries for the \textsf{GAP} compiler is now handled via autoconf. Finally, now \texttt{autoconf} version 2.59 is used. 
\end{itemize}
 Fixed bugs which could produce wrong results: 
\begin{itemize}
\item  Some technical errors in the functions for compressed vectors and matrices
which could lead to corruption of internal data structures and so to crashes
or conceivably to wrong results. [Reported by Roman Schmied] 
\item  A potential problem in the generic method for the undocumented operation \texttt{DirectFactorsOfGroup}: It was silently assumed that \texttt{NormalSubgroups} (\textbf{Reference: NormalSubgroups}) delivers the trivial subgroup as first and the whole group as last entry of
the resulting list. 
\item  The code for sublists of compressed vectors created by \texttt{vec\texttt{\symbol{123}}range\texttt{\symbol{125}}} may write one byte beyond the space allocated for the new vector, overwriting
part of the next object in the workspace. Thanks to Jack Schmidt for narrowing
down the problem. 
\item  Given a class function object of value zero, an \texttt{Arithmetic Operations for Class Functions} (\textbf{Reference: Arithmetic Operations for Class Functions}) method for a class function erroneously did not return \texttt{fail}. [Reported by Jack Schmidt] 
\item  The \texttt{Arithmetic Operations for Class Functions} (\textbf{Reference: Arithmetic Operations for Class Functions}) method for a class function erroneously returned a finite number if one of the
values was nonreal, not a cyclotomic integer, and had norm 1. 
\item  Two missing perfect groups were added, and the permutation degree lowered on
the perfect groups with the largest degrees. [Reported by Jack Schmidt] 
\item  When a character table was displayed with \texttt{Printing Character Tables} (\textbf{Reference: Printing Character Tables}), the centralizer order displayed for the first class shown was not correct if
it did not involve all prime divisors of the group. [Reported by Jack Schmidt] 
\item  The first argument of the function \texttt{VectorSpace} (\textbf{Reference: VectorSpace}) must be a field. This is checked from now on. [Reported by Laurent Bartholdi] 
\item  Up to now, it was possible to create a group object from a semigroup of
cyclotomics using \texttt{AsGroup} (\textbf{Reference: AsGroup}), although groups of cyclotomics are not admissible. [Reported by Alexander
Konovalov] 
\item  The documentation of \texttt{CharacteristicPolynomial(F,mat)} was ambiguous if \texttt{FieldOfMatrix(mat) {\textless}= F {\textless} DefaultFieldOfMatrix(mat)}. In particular, the result was representation dependent. This was fixed by
introducing a second field which specifies the vector space which mat acts
upon. [Reported by Jack Schmidt] 
\item  \texttt{AssociatedReesMatrixSemigroupOfDClass} produced an incorrect sandwich matrix for the semigroup created. This matrix
is an attribute set when creating the Rees matrix semigroup but is not used
for creating the semigroup. The incorrect result was returned when \texttt{SandwichMatrix} was called. [Reported by Nelson Silva and Joao Araujo] 
\item  The literal \texttt{"compiled"} was given an incorrect length. The kernel was then unable to find compiled
library code as the search path was incorrect. Also the documentation example
had an error in the path used to invoke the \texttt{gac} compiler. 
\item  The twisting group in a generic wreath product might have had intransitive
action. [Reported by Laurent Bartholdi] 
\item  There was an arithmetic bug in the polynomial reduction code. 
\end{itemize}
 Fixed bugs which could lead to crashes: 
\begin{itemize}
\item  Bug 1 in the list of fixed bugs which could lead to wrong results could also
potentially lead to crashes. 
\end{itemize}
 Other fixed bugs: 
\begin{itemize}
\item  The matrices of invariant forms stored as values of the attributes \texttt{InvariantBilinearForm} (\textbf{Reference: InvariantBilinearForm}), \texttt{InvariantQuadraticForm} (\textbf{Reference: InvariantQuadraticForm}), and \texttt{InvariantSesquilinearForm} (\textbf{Reference: InvariantSesquilinearForm}), for matrix groups over finite fields, are now in the (compressed) format
returned by \texttt{ImmutableMatrix} (\textbf{Reference: ImmutableMatrix}). 
\item  \texttt{String} now returns an immutable string, by making a copy before changing the
argument. 
\item  \texttt{permutation\texttt{\symbol{94}}0} and \texttt{permutation\texttt{\symbol{94}}1} were not handled with special code in the kernel, hence were very slow for big
permutations. [Reported by Max Neunh{\"o}ffer] 
\item  Added code to cache the induced pcgs for an arbitrary parent pcgs. (This code
was formerly part of the \textsf{CRISP} package.) 
\item  This fix consists of numerous changes to improve support for direct products,
including: - new methods for \texttt{PcgsElementaryAbelianSeries}, \texttt{PcgsChiefSeries}, \texttt{ExponentsOfPcElement}, \texttt{DepthOfPcElement} for direct products - fixed \texttt{EnumeratorOfPcgs} to test for membership first - new methods for membership test in groups which
have an induced pcgs - added \texttt{GroupOfPcgs} attribute to pcgs in various methods - fixed declarations of \texttt{PcgsElementaryAbelianSeries}, \texttt{PcgsChiefSeries} (the declared argument was a pcgs, not a group) [Reported by Roman Schmied] 
\item  Corrected a term ordering problem encountered by the basis construction code
for finite dimensional vector spaces of multivariate rational functions.
[Reported by Jan Draisma] 
\item  When the factor of a finite dimensional group ring by an ideal was formed, a
method intended for free algebras modulo relations was used, and the returned
factor algebra could be used for (almost) nothing. [Reported by Heiko
Dietrich] 
\item  Up to now, \texttt{PowerMap} (\textbf{Reference: PowerMap}) ran into an error when one asked for the n-th power map where n was not a
small integer. This happened in some \textsf{GAP} library functions if the exponent of the character table in question was not a
small integer. 
\item  Up to now, the test whether a finite field element was contained in a group of
finite field elements ran into an error if the element was not in the field
generated by the group elements. [Reported by Heiko Dietrich] 
\item  Conjugacy classes of natural (special) linear groups are now always returned
with trivial class first. 
\item  Up to now, it could happen that \texttt{CheckFixedPoints} (\textbf{Reference: CheckFixedPoints}) reduced an entry in its second argument to a list containing only one integer
but did not replace the list by that integer; according to the conventions,
this replacement should be done. 
\item  The functions \texttt{PrintTo} and \texttt{AppendTo} did not work correctly for streams. [Reported by Marco Costantini] 
\item  The function \texttt{Basis} did not return a value when it was called with the argument \texttt{Rationals}. [Reported by Klaus Lux] 
\item  For certain matrix groups, the function \texttt{StructureDescription} raised an error message. The reason for this was that a trivial method for \texttt{IsGeneralLinearGroup} for matrix groups in \texttt{lib/grpmat.gi} which is ranked higher than the nontrivial method for generic groups in \texttt{lib/grpnames.gi} called the operation \texttt{IsNaturalGL}, for which there was no nontrivial method available. [Reported by Nilo de
Roock] 
\item  Action on sets of length 1 was not correctly handled. [Reported by Mathieu
Dutour] 
\item  Now \texttt{WriteByte} admits writing zero characters to all streams. [Reported by Marco Costantini] 
\item  The conjugacy test for subgroups tests for elementary abelian regular normal
subgroup (EARNS) conjugacy. The fix will catch this in the case that the
second group has no EARNS. [Reported by Andrew Johnson] 
\item  So far, the UNIX installation didn't result in a correct gap.sh if the
installation path contained space characters. Now it should handle this case
correctly, as well as other unusual characters in path names (except for
double quotes). 
\end{itemize}
 }

  
\section{\textcolor{Chapter }{\textsf{GAP} 4.4 Update 8 (September 2006)}}\label{fix448}
\logpage{[ 5, 7, 0 ]}
\hyperdef{L}{X7B61699179F631E0}{}
{
  New or improved functionality: 
\begin{itemize}
\item  A function \texttt{Positions} (\textbf{Reference: Positions}) with underlying operation \texttt{PositionsOp}, which returns the list of all positions at which a given object appears in a
given list. 
\item  \texttt{LogFFE} (\textbf{Reference: LogFFE}) now returns \texttt{fail} when the element is not a power of the base. 
\item  It is now allowed to continue long integers, strings or identifiers by ending
a line with a backslash or with a backslash and carriage return character. So,
files with \textsf{GAP} code and DOS/Windows-style line breaks are now valid input on all
architectures. 
\item  The command line for starting the session and the system environment are now
available in \texttt{GAPInfo.SystemCommandLine} and \texttt{GAPInfo.SystemEnvironment}. 
\item  Names of all bound global variables and all component names are available on \textsf{GAP} level. 
\item  Added a few new Conway polynomials computed by Kate Minola and John Bray. 
\item  There is a new concept of \emph{random sources}, see \texttt{IsRandomSource} (\textbf{Reference: IsRandomSource}), which provides random number generators which are independent of each other.
There is kernel code for the Mersenne twister random number generator (based
on the code by Makoto Matsumoto distributed at \href{http://www.math.sci.hiroshima-u.ac.jp/~m-mat/MT/emt.html} {\texttt{http://www.math.sci.hiroshima-u.ac.jp/\texttt{\symbol{126}}m-mat/MT/emt.html}}). It provides fast 32-bit pseudorandom integers with a period of length $2^{19937}-1$ and a 623-dimensional equidistribution. The library methods for random
elements of lists and for random (long) integers are using the Mersenne
twister now.

 
\item  In line editing mode (usual input mode without -n option) in lines starting
with \texttt{gap{\textgreater} }, \texttt{{\textgreater} } or \texttt{brk{\textgreater} } this beginning part is immediately removed. This is a convenient feature that
allows one to cut and paste input lines from other sessions or from manual
examples into the current session. 
\end{itemize}
 Fixed bugs which could produce wrong results: 
\begin{itemize}
\item  The function \texttt{Decomposition} (\textbf{Reference: Decomposition}) returned coefficient vectors also in certain situations where in fact no
decomposition exists. This happened only if the matrix entered as the first
argument contained irrational values and a row in the matrix entered as the
second argument did not respect the algebraic conjugacy relations between the
columns of the first argument. So there was no problem for the usual cases
that the two matrices are integral or that they are lists of Brauer
characters. [Reported by J{\"u}rgen M{\"u}ller] 
\item  PC group homomorphisms can claim a wrong kernel after composition. [Reported
by Serge Bouc] 
\item  The return value of \texttt{OctaveAlgebra} (\textbf{Reference: OctaveAlgebra}) had an inconsistent defining structure constants table for the case of
coefficients fields not containing the integer zero. [Reported by G{\a'a}bor
Nagy] 
\item  The manual guarantees that a conjugator automorphism has a conjugating element
in the group if possible. This was not guaranteed. 
\item  \texttt{StabChain} (\textbf{Reference: StabChain (for a group (and a record))}) for symmetric groups gave a wrong result if fixed points were prescribed for
base. 
\item  Contrary to what is documented the function \texttt{POW{\textunderscore}OBJ{\textunderscore}INT} returned an immutable result for \texttt{POW{\textunderscore}OBJ{\textunderscore}INT(m,1)} for a mutable object \texttt{m}. This is triggered by the code \texttt{m\texttt{\symbol{94}}1}. 
\item  \texttt{PseudoRandom} (\textbf{Reference: PseudoRandom}) for a group had a problem if the group had lots of equal generators. The
produced elements were extremely poorly distributed in that case. This is now
fixed for the case that elements of the group can easily be sorted. 
\item  Fixed the bug that the type of a boolean list (see \texttt{More about Boolean Lists} (\textbf{Reference: More about Boolean Lists})) was computed wrongly: The type previously had \texttt{IS{\textunderscore}PLIST{\textunderscore}REP} instead of \texttt{IS{\textunderscore}BLIST{\textunderscore}REP} in its filter list. 
\item  \texttt{Orbits} (\textbf{Reference: Orbits}) did not respect a special \texttt{PositionCanonical} (\textbf{Reference: PositionCanonical}) method for right transversals. [Reported by Steve Costenoble] 
\item  Wrong results for \texttt{GcdInt} (\textbf{Reference: GcdInt}) for some arguments on 64 bit systems only. [Reported by Robert Morse] 
\item  When prescribing a subgroup to be included, the low index algorithm for fp
groups sometimes returned subgroups which are in fact conjugate. (No subgroups
are missing.) [Reported by Ignaz Soroko] 
\end{itemize}
 Fixed bugs which could lead to crashes: 
\begin{itemize}
\item  The command line option \texttt{-x} allowed arguments {\textgreater} 256 which can then result in internal buffers
overflowing. Now bigger numbers in the argument are equivalent to \texttt{-x 256}.  [Reported by Michael Hartley] 
\end{itemize}
 Other fixed bugs: 
\begin{itemize}
\item  Two special methods for the operation \texttt{CompositionMapping2} (\textbf{Reference: CompositionMapping2}) were not correct, such that composing (and multiplying) certain group
homomorphisms did not work. [Reported by Peter Mayr] 
\item  In the definition of \texttt{FrobeniusCharacterValue} (\textbf{Reference: FrobeniusCharacterValue}), it had been stated erroneously that the value must lie in the field of $p^n$-th roots of unity; the correct condition is that the value must lie in the
field of $(p^n-1)$-th roots of unity. [Reported by Jack Schmidt] 
\item  The function \texttt{DirectProduct} (\textbf{Reference: DirectProduct}) failed when one of the factors was known to be infinite. 
\item  For a linear action homomorphism \texttt{PreImageElm} was very slow because there was no good method to check for injectivity, which
is needed for nearly all good methods for \texttt{PreImageElm}. This change adds such a new method for \texttt{IsInjective}. [Reported by Akos Seress] 
\item  Rare errors in the complement routine for permutation groups. 
\item  Blocks code now uses jellyfish-style random elements to avoid bad Schreier
trees. 
\item  A method for \texttt{IsPolycyclicGroup} (\textbf{Reference: IsPolycyclicGroup}) has been added. Such a method was missing so far. 
\item  Corrected \texttt{EpimorphismSchurCover} (\textbf{Reference: EpimorphismSchurCover}) to handle the trivial group correctly. Added new methods that follow
immediately from computing the Schur Cover of a group. The attribute \texttt{Epicentre} (\textbf{Reference: Epicentre}), the operations \texttt{NonabelianExteriorSquare} (\textbf{Reference: NonabelianExteriorSquare}) and \texttt{EpimorphismNonabelianExteriorSquare} (\textbf{Reference: EpimorphismNonabelianExteriorSquare}), and the property \texttt{IsCentralFactor} (\textbf{Reference: IsCentralFactor}) are added to the library with documentation and references. 
\item  Display the correct expression in a call stack trace if an operation was
called somewhere up due to the evaluation of a unary or binary operation. 
\item  Made \texttt{StripMemory} an operation rather than a global function. Added \texttt{ForgetMemory} operation. 
\item  Adjust things slightly to make later conversion to new vectors/matrices
easier. Nothing of this should be visible. 
\item  Corrected some details in the documentation of the \textsf{GAP} language. [Reported by Alexander Konovalov] 
\item  Now \texttt{PositionSorted} (\textbf{Reference: PositionSorted}) is much faster on long mutable plain lists. (The former operation is
substituted by a function and a new operation \texttt{PositionSortedOp}.) [Reported by Silviu Radu] 
\item  Now it is possible to switch repeated warnings off when working with iterative
polynomial rings. 
\end{itemize}
 }

  
\section{\textcolor{Chapter }{\textsf{GAP} 4.4 Update 9 (November 2006)}}\label{fix449}
\logpage{[ 5, 8, 0 ]}
\hyperdef{L}{X81B223647C9BD9B9}{}
{
  Fixed bugs which could produce wrong results: 
\begin{itemize}
\item  The methods of \texttt{ReadByte} (\textbf{Reference: ReadByte}) for reading from files or terminals returned wrong results for characters in
the range \texttt{[128..255]}. [Reported by Yevgen Muntyan] 
\end{itemize}
 Other fixed bugs: 
\begin{itemize}
\item  A method for the operation \texttt{PseudoRandom} (\textbf{Reference: PseudoRandom}) did not succeed. 
\item  A fix for \texttt{Orbits} with a set of points as a seed. 
\item  Added a generic method such that \texttt{Positions} (\textbf{Reference: Positions}) works with all types of lists. 
\item  Fixed a problem in choosing the prime in the Dixon-Schneider algorithm.
[Reported by Toshio Sumi] 
\end{itemize}
 New or improved functionality: 
\begin{itemize}
\item  \texttt{ReducedOrdinary} was used in the manual, but was not documented, being a synonym for the
documented \texttt{ReducedCharacters}. Changed manual examples to use the latter form. [Reported by Vahid
Dabbaghian] 
\end{itemize}
 }

  
\section{\textcolor{Chapter }{\textsf{GAP} 4.4 Update 10 (October 2007)}}\label{fix4410}
\logpage{[ 5, 9, 0 ]}
\hyperdef{L}{X7BB4EC1F7B130FD5}{}
{
  New or improved functionality: 
\begin{itemize}
\item  Files in the \texttt{cnf} directory of the \textsf{GAP} distribution are now archived as binary files. Now \textsf{GAP} can be installed with UNIX or with WINDOWS style line breaks on any system and
should work without problems. 
\item  Since large finite fields are available, some restrictions in the code for
computing irreducible modules over finite fields are no longer necessary.
(They had been introduced in order to give better error messages.) 
\item  Made PositionSublist faster in case the search string does not contain
repetitive patterns. 
\item  The function \texttt{MakeImmutable} now returns its argument. 
\item  Dynamically loaded modules now work on Mac OS X. As a consequence, this allows
to work with the Browse, EDIM and IO packages on Mac OS X. 
\item  Introduced \texttt{ViewObj} and \texttt{PrintObj} methods for algebraic number fields. Made them applicable to \texttt{AlgebraicExtension} by adding the property \texttt{IsNumberField} in the infinite field case. 
\item  The function \texttt{CharacterTableRegular} (\textbf{Reference: CharacterTableRegular}) is documented now. 
\item  The function \texttt{ScalarProduct} (\textbf{Reference: ScalarProduct (for characters)}) now accepts also Brauer characters as arguments. 
\item  The function \texttt{QuaternionAlgebra} (\textbf{Reference: QuaternionAlgebra}) now accepts also a list of field elements instead of a field. Also, now the
comparison of return values (w.r.t. equality, containment) yields \texttt{true} if the parameters coincide and the ground fields fit. 
\item  The function \texttt{RemoveCharacters} (\textbf{Reference: RemoveCharacters}) is now documented. 
\item  Lists in \textsf{GAP} sometimes occupy memory for possible additional entries. Now plain lists and
strings read by \textsf{GAP} and the lists returned by \texttt{List} (\textbf{Reference: Lists}) only occupy the memory they really need. For more details see the
documentation of the new function \texttt{EmptyPlist} (\textbf{Reference: EmptyPlist}). 
\item  There are some new Conway polynomials in characteristic 2 and 3 provided by
Kate Minola. 
\item  A new operation \texttt{MemoryUsage} determines the memory usage in bytes of an object and all its subobjects. It
does not consider families and types but handles arbitrary self-referential
structures of objects. 
\end{itemize}
 Fixed bugs which could produce wrong results: 
\begin{itemize}
\item  When forming the semidirect product of a matrix group with a vector space over
a non-prime field the embedding of the vector space gave a wrong result.
[Reported by anvita21] 
\item  DefaultRing failed for constant polynomials over nonprime fields. [Reported by
Stefan Kohl] 
\item  The method in ffeconway.gi that gets coefficients WRT to the canonical basis
of the field from the representation is only correct if the basis is over the
prime field. Added a TryNextMethod if this is not the case. [Reported by Alla
Detinko] 
\item  Creating a large ({\textgreater}$2^{16}$) field over a non-prime subfield went completely wrong. [Reported by Jack
Schmidt, from Alla Detinko] 
\item  A method for Coefficients for Conway polynomial FFEs didn't check that the
basis provided was the canonical basis of the RIGHT field. [Reported by
Bettina Eick] 
\item  An elementary abelian series was calculated wrongly. [Reported by N. Sieben] 
\item  Orbits on sets of transformations failed. 
\item  Wrong methods for \texttt{GeneratorsOfRing} (\textbf{Reference: GeneratorsOfRing}) and \texttt{GeneratorsOfRingWithOne} (\textbf{Reference: GeneratorsOfRingWithOne}) have been removed. These methods were based on the assumption that one can
obtain a set of ring generators by taking the union of a known set of field
generators, the set of the inverses of these field generators
and{\nobreakspace}\texttt{\symbol{123}}1\texttt{\symbol{125}}. 
\item  The name of a group of order 117600 and degree 50 was incorrect in the \texttt{Primitive Permutation Groups} (\textbf{Reference: Primitive Permutation Groups}) Primitive Permutation Groups library. In particular, a group was wrongly
labelled as PGL(2, 49). 
\item  There was a possible error in \texttt{SubgroupsSolvableGroup} when computing subgroups within a subgroup. 
\item  An error in 2-Cohomology computation for pc groups was fixed. 
\item  \texttt{IsConjugate} used normality in a wrong supergroup 
\end{itemize}
 Fixed bugs which could lead to crashes: 
\begin{itemize}
\item  \textsf{GAP} crashed when the \texttt{PATH} environment variable was not set. [Reported by Robert F. Morse] 
\item  \textsf{GAP} could crash when started with option \texttt{-x 1}. Now the number of columns is initialized with at least{\nobreakspace}2.
[Reported by Robert F. Morse] 
\item  After loading a saved workspace \textsf{GAP} crashed when one tried to slice a compressed vector over a field with 2
{\textless} q {\textless}= 256 elements, which had already existed in the
saved workspace. [Reported by Laurent Bartholdi] 
\item  \texttt{FFECONWAY.WriteOverSmallestCommonField} tripped up when the common field is smaller than the field over which some of
the vector elements are written, because it did a test based on the degree of
the element, not the field it is written over. [Reported by Thomas Breuer] 
\item  Fixed the following error: When an FFE in the Conway polynomial representation
actually lied in a field that is handled in the internal representation (eg $GF(3)$) and you tried to write it over a bigger field that is ALSO handled
internally (eg $GF(9)$) you got an element written over the larger field, but in the Conway
polynomial representation, which is forbidden. [Reported by Jack Schmidt] 
\item  Attempting to compress a vector containing elements of a small finite field
represented as elements of a bigger (external) field caused a segfault.
[Reported by Edmund Robertson] 
\item  \textsf{GAP} crashed when \texttt{BlistList} was called with a range and a list containing large integers or non-integers.
[Reported by Laurent Bartholdi] 
\item  \textsf{GAP} no longer crashes when \texttt{OnTuples} is called with a list that contains holes. [Reported by Thomas Breuer] 
\end{itemize}
 Other fixed bugs: 
\begin{itemize}
\item  \texttt{Socle} for the trivial group could produce an error message. 
\item  \texttt{DirectoryContents} (\textbf{Reference: DirectoryContents}) ran into an error for immutable strings without trailing slash as argument.
[Reported by Thomas Breuer] 
\item  The functions \texttt{IsInjective} (\textbf{Reference: IsInjective}) and \texttt{IsSingleValued} (\textbf{Reference: IsSingleValued}) did not work for general linear mappings with trivial (pre)image. [Reported by
Alper Odabas] 
\item  Creating an enumerator for a prime field with more than 65536 elements ran
into an infinite recursion. [Reported by Akos Seress] 
\item  The performance of \texttt{List}, \texttt{Filtered}, \texttt{Number}, \texttt{ForAll} and \texttt{ForAny} if applied to non-internally represented lists was improved. Also the
performance of iterators for lists was slightly improved. 
\item  Finite field elements now know that they can be sorted easily which improves
performance in certain lookups. 
\item  A method for \texttt{IsSubset} (\textbf{Reference: IsSubset}) was missing for the case that exactly one argument is an inhomogeneous list.
[Reported by Laurent Bartholdi] 
\item  Long integers in expressions are now printed (was not yet implemented).
[Reported by Thomas Breuer] 
\item  Fixed kernel function for printing records. 
\item  New C library interfaces (e.g., to ncurses in the \textsf{Browse} package) need some more memory to be allocated with \texttt{malloc}. The default value of \textsf{GAP} \texttt{-a} option is now \texttt{2m{\textgreater}}. 
\item  Avoid warnings about pointer types by newer gcc compilers. 
\item  \texttt{IsBound(l[pos])} was failing for a large integer \mbox{\texttt{\mdseries\slshape pos}} only when coded (e.g. in a loop or function body). 
\item  \texttt{ZmodpZObj} is now a synonym for \texttt{ZmodnZObj} such that from now on such objects print in a way that can be read back into \textsf{GAP}. 
\item  The outdated note that binary streams are not yet implemented has been
removed. 
\end{itemize}
 }

  
\section{\textcolor{Chapter }{\textsf{GAP} 4.4 Update 11 (December 2008)}}\label{fix4411}
\logpage{[ 5, 10, 0 ]}
\hyperdef{L}{X7E4C6820809F2384}{}
{
  Fixed bugs which could produce wrong results: 
\begin{itemize}
\item  \texttt{MemoryUsage} (\textbf{Reference: MemoryUsage}) on objects with no subobjects left them in the cache and thus reported 0 in
subsequent calls to MemoryUsage for the same object. [Reported by Stefan Kohl] 
\item  \texttt{Irr} (\textbf{Reference: Irr}) might be missing characters. [Reported by Angel del Rio] 
\item  Up to now, it was allowed to call the function \texttt{FullMatrixAlgebraCentralizer} (\textbf{Reference: FullMatrixAlgebraCentralizer}) with a field and a list of matrices such that the entries of the matrices were
not contained in the field; in this situation, the result did not fit to the
documentation. Now the entries of the matrices are required to lie in the
field, if not then an error is signaled. 
\item  For those finite fields that are regarded as field extensions over non-prime
fields (one can construct such fields with \texttt{AsField} (\textbf{Reference: AsField})), the function \texttt{DefiningPolynomial} (\textbf{Reference: DefiningPolynomial}) erroneously returned a polynomial w.r.t. the extension of the prime field.
[Reported by Stefan Kohl] 
\item  Since the release of \textsf{GAP} 4.4.10, the return values of the function \texttt{QuaternionAlgebra} (\textbf{Reference: QuaternionAlgebra}) were not consistent w.r.t. the attribute \texttt{GeneratorsOfAlgebra} (\textbf{Reference: GeneratorsOfAlgebra}); the returned list could have length four or five. Now always the list of
elements of the canonical basis is returned. 
\item  \texttt{MonomialGrevlexOrdering} (\textbf{Reference: MonomialGrevlexOrdering}) calculated a wrong ordering in certain cases. [Reported by Paul Smith] 
\item  The (\textsf{GAP} kernel) method for the operation \texttt{IntersectSet} (\textbf{Reference: IntersectSet}) for ranges had two bugs, which could yield a result range with either too few
or too many elements. As a consequence, for example the \texttt{Intersection} (\textbf{Reference: Intersection}) results for ranges could be wrong. [Reported by Matthew Fayers] 
\item  Fixed a bug in the short-form display of elements of larger finite fields, a
bug in some cross-field conversions and some inefficiencies and a missing
method in the \texttt{LogFFE} (\textbf{Reference: LogFFE}) code. [Reported by Jia Huang] 
\item  In rare cases \texttt{SmithNormalFormIntegerMatTransforms} (\textbf{Reference: SmithNormalFormIntegerMatTransforms}) returned a wrong normal form (the version without transforming matrices did
not have this problem). This is fixed. [Reported by Alexander Hulpke] 
\item  The variant of the function \texttt{StraightLineProgram} (\textbf{Reference: StraightLineProgram (for a list of lines (and the number of generators))}) that takes a string as its first argument returned wrong results if the last
character of this string was a closing bracket. 
\item  The code for central series in a permutation group used too tight a bound and
thus falsely return a nilpotent permutation group as non-nilpotent. 
\end{itemize}
 Fixed bugs which could lead to crashes: 
\begin{itemize}
\item  Under certain circumstances the kernel code for position in blists would
access a memory location just after the end of the blist. If this location was
not accessible, a crash could result. This was corrected and the code was
cleaned up. [Reported by Alexander Hulpke] 
\end{itemize}
 Other fixed bugs: 
\begin{itemize}
\item  The function \texttt{IsomorphismTypeInfoFiniteSimpleGroup} (\textbf{Reference: IsomorphismTypeInfoFiniteSimpleGroup}) can be called with a positive integer instead of a group, and then returns
information about the simple group(s) of this order. (This feature is
currently undocumented.) For the argument 1, however, it ran into an infinite
loop. 
\item  A lookup in an empty dictionary entered a break loop. Now returns \texttt{fail}. [Reported by Laurent Bartholdi] 
\item  The c++ keyword \texttt{and} can no longer be used as a macro parameter in the kernel. [Reported by Paul
Smith] 
\item  The operation \texttt{KernelOfMultiplicativeGeneralMapping} (\textbf{Reference: KernelOfMultiplicativeGeneralMapping}) has methods designed to handle maps between permutation groups in a two-step
approach, but did not reliably trigger the second step. This has now been
fixed, preventing a slow infinite loop repeating the first step. This was
normally only seen as part of a larger calculation. 
\item  There were two methods for the operation \texttt{Intersection2} (\textbf{Reference: Intersection2}) which have implicitly assumed that finiteness of a collection can always be
decided. Now, these methods check for \texttt{IsFinite} (\textbf{Reference: IsFinite}) and \texttt{CanComputeSize} (\textbf{Reference: CanComputeSize}) prior to calling \texttt{IsFinite} (\textbf{Reference: IsFinite}). 
\item  Made error message in case of corrupted help book information (manual.six
file) shorter and more informative. [Reported by Alexander Hulpke] 
\item  \textsf{GAP} cannot call methods with more than six arguments. Now the functions \texttt{NewOperation} (\textbf{Reference: NewOperation}), \texttt{DeclareOperation} (\textbf{Reference: DeclareOperation}), and \texttt{InstallMethod} (\textbf{Reference: InstallMethod}) signal an error if one attempts to declare an operation or to install a method
with more than six arguments. 
\item  Up to now, \texttt{IsOne} (\textbf{Reference: IsOne}) had a special method for general mappings, which was much worse than the
generic method; this special method has now been removed. 
\item  When printing elements of an algebraic extension parentheses around
coefficients were missing. [Reported by Maxim Hendriks] 
\end{itemize}
 New or improved functionality: 
\begin{itemize}
\item  Make dynamic loading of modules possible on CYGWIN using a DLL based approach.
Also move to using autoconf version 2.61. 
\item  One can now call \texttt{Basis} (\textbf{Reference: Basis}), \texttt{Iterator} (\textbf{Reference: Iterator}) etc. with the return value of the function \texttt{AlgebraicExtension} (\textbf{Reference: AlgebraicExtension}). 
\item  The function \texttt{FrobeniusCharacterValue} (\textbf{Reference: FrobeniusCharacterValue}) returned \texttt{fail} for results that require a finite field with more than 65536 elements.
Meanwhile \textsf{GAP} can handle larger finite fields, so this restriction was removed. (It is still
possible that \texttt{FrobeniusCharacterValue} (\textbf{Reference: FrobeniusCharacterValue}) returns \texttt{fail}.) 
\item  Methods for testing membership in general linear groups and special linear
groups over the integers have been added. 
\item  Methods for \texttt{String} (\textbf{Reference: String}) and \texttt{ViewString} for full row modules have been added. Further, a default method for \texttt{IsRowModule} (\textbf{Reference: IsRowModule}) has been added, which returns \texttt{false} for objects which are not free left modules. 
\item  A \texttt{ViewString} method for objects with name has been added. 
\item  The method for \texttt{View} (\textbf{Reference: View}) for polynomial rings has been improved, and methods for \texttt{String} (\textbf{Reference: String}) and \texttt{ViewString} for polynomial rings have been added. 
\item  \texttt{Binomial} (\textbf{Reference: Binomial}) now works with huge \texttt{n}. 
\item  The function \texttt{InducedClassFunctionsByFusionMap} (\textbf{Reference: InducedClassFunctionsByFusionMap}) is now documented. 
\item  The return values of the function \texttt{QuaternionAlgebra} (\textbf{Reference: QuaternionAlgebra}) now store that they are division rings (if optional parameters are given then
of course ths depends on these parameters). 
\end{itemize}
 }

  
\section{\textcolor{Chapter }{\textsf{GAP} 4.4 Update 12 (December 2008)}}\label{fix4412}
\logpage{[ 5, 11, 0 ]}
\hyperdef{L}{X7F5BBE877AAA4BEE}{}
{
  Fixed bugs which could lead to crashes: 
\begin{itemize}
\item  A bug whereby leaving an incomplete statement on a line (for instance typing
while and then return) when prompt colouring was in use could lead to \textsf{GAP} crashing. 
\end{itemize}
 Other fixed bugs: 
\begin{itemize}
\item  A bug which made the command-line editor unusable in a 64-bit version of \textsf{GAP} on Mac{\nobreakspace}OS{\nobreakspace}X. 
\end{itemize}
 }

 }

    
\chapter{\textcolor{Chapter }{Changes from Earlier Versions}}\label{Changes from Earlier Versions}
\logpage{[ 6, 0, 0 ]}
\hyperdef{L}{X7F5DE9997CF43125}{}
{
   
\section{\textcolor{Chapter }{Changes between \textsf{GAP} 4.3 and \textsf{GAP} 4.4}}\label{ChangesGAP43toGAP44}
\logpage{[ 6, 1, 0 ]}
\hyperdef{L}{X7DBACE2286A43A31}{}
{
  The main changes between \textsf{GAP} 4.3 and \textsf{GAP} 4.4 are:  
\subsection{\textcolor{Chapter }{Potentially Incompatible Changes}}\label{Potentially Incompatible Changes}
\logpage{[ 6, 1, 1 ]}
\hyperdef{L}{X7C5AC61F824086D6}{}
{
  
\begin{itemize}
\item  The mechanism for the loading of Packages has changed to allow easier updates
independent of main \textsf{GAP} releases. Packages require a file \texttt{PackageInfo.g} now. The new \texttt{PackageInfo.g} files are available for all packages with the new version of GAP (see  \textbf{Example: PackageInfo.g for a GAP package}). 
\item  \texttt{IsSimpleGroup} (\textbf{Reference: IsSimpleGroup}) returns false now for the trivial group. 
\item  \texttt{PrimeBlocks} (\textbf{Reference: PrimeBlocks}): The output format has changed. 
\item  Division rings (see \texttt{IsDivisionRing} (\textbf{Reference: IsDivisionRing})) are now implemented as \texttt{IsRingWithOne} (\textbf{Reference: IsRingWithOne}). 
\item  \texttt{DirectSumOfAlgebras} (\textbf{Reference: DirectSumOfAlgebras (for two algebras)}): $p$-th power maps are compatible with the input now. 
\item  The print order for polynomials has been changed. 
\end{itemize}
 These changes are, in some respects, departures from our policy of maintaining
upward compatibility of documented functions between releases. In the first
case, we felt that the old behavior was sufficiently inconsistent, illogical,
and impossible to document that we had no alternative but to change it. In the
case of the package interface, the change was necessary to introduce new
functionality. The planned and phased removal of a few unnecessary functions
or synonyms is needed to avoid becoming buried in ``legacy'' interfaces, but we remain committed to our policy of maintaining upward
compatibility whenever sensibly possible. 

 
\begin{itemize}
\item  Groebner Bases: 

 Buchberger's algorithm to compute Groebner Bases has been implemented in GAP.
(A. Hulpke) 
\item  For large scale Groebner Basis computations there also is an interface to the
Singular system available in the \href{http://www.gap-system.org/Packages/singular.html} {\textsf{Singular}} package. (M. Costantini and W. de Graaf) 
\item  New methods for factorizing polynomials over algebraic extensions of the
rationals have been implemented in GAP. (A. Hulpke) 
\item  For more functionality to compute with algebraic number fields there is an
interface to the Kant system available in the \href{http://www.gap-system.org/Packages/alnuth.html} {\textsf{Alnuth}} package. (B. Assmann and B. Eick) 
\item  A new functionality to compute the minimal normal subgroups of a finite group,
as well as its socle, has been installed. (B. H{\"o}fling) 
\item  A fast method for recognizing whether a permutation group is symmetric or
alternating is available now (A. Seress) 
\item  A method for computing the Galois group of a rational polynomial is available
again. (A. Hulpke) 
\item  The algorithm for \texttt{BrauerCharacterValue} (\textbf{Reference: BrauerCharacterValue}) has been extended to the case where the splitting field is not supported in \textsf{GAP}. (T. Breuer) 
\item  Brauer tables of direct products can now be constructed from the known Brauer
tables of the direct factors. (T. Breuer) 
\item  Basic support for vector spaces of rational functions and of uea elements is
available now in \textsf{GAP}. (T. Breuer and W. de Graaf) 
\item  Various new functions for computations with integer matrices are available,
such as methods for computing normal forms of integer matrices as well as
nullspaces or solutions systems of equations. (W. Nickel and F. G{\"a}hler) 
\end{itemize}
 }

  
\subsection{\textcolor{Chapter }{New Packages}}\label{New Packages}
\logpage{[ 6, 1, 2 ]}
\hyperdef{L}{X7D9702E3815BE9FB}{}
{
  The following new Packages have been accepted. 

 
\begin{itemize}
\item  \href{http://www.gap-system.org/Packages/alnuth.html} { \textsf{Alnuth}: Algebraic Number Theory and an interface to the Kant system. } By B. Assmann and B. Eick. 
\item  \href{http://www.gap-system.org/Packages/laguna.html} { \textsf{LAGUNA}: Computing with Lie Algebras and Units of Group Algebras. } By V. Bovdi, A. Konovalov, R. Rossmanith, C. Schneider. 
\item  \href{http://www.gap-system.org/Packages/nq.html} { \textsf{NQ}: The ANU Nilpotent Quotient Algorithm. } By W. Nickel. 
\item  \href{http://www.gap-system.org/Packages/kbmag.html} { \textsf{KBMAG}: Knuth-Bendix for Monoids and Groups. } By D. Holt. 
\item  \href{http://www.gap-system.org/Packages/polycyclic.html} { \textsf{Polycyclic}: Computation with polycyclic groups. } By B. Eick and W. Nickel. 
\item  \href{http://www.gap-system.org/Packages/quagroup.html} { \textsf{QuaGroup}: Computing with Quantized Enveloping Algebras. } By W. de Graaf. 
\end{itemize}
 }

  
\subsection{\textcolor{Chapter }{Performance Enhancements}}\label{Performance Enhancements}
\logpage{[ 6, 1, 3 ]}
\hyperdef{L}{X7DA8709A850E772D}{}
{
  
\begin{itemize}
\item  The computation of irreducible representations and irreducible characters
using the Baum-Clausen algorithm and the implementation of the Dixon-Schneider
algorithm have been speeded up. 
\item  The algorithm for \texttt{PossibleClassFusions} (\textbf{Reference: PossibleClassFusions}) has been changed: the efficiency is improved and a new criterion is used. The
algorithm for \texttt{PossibleFusionsCharTableTom} (\textbf{Reference: PossibleFusionsCharTableTom}) has been speeded up. The method for \texttt{PrimeBlocks} (\textbf{Reference: PrimeBlocks}) has been improved following a suggestion of H. Pahlings. 
\item  New improved methods for normalizer and subgroup conjugation in $S_n$ have been installed and new improved methods for \texttt{IsNaturalSymmetricGroup} (\textbf{Reference: IsNaturalSymmetricGroup}) and \texttt{IsNaturalAlternatingGroup} (\textbf{Reference: IsNaturalAlternatingGroup}) have been implemented. These improve the available methods when groups of
large degrees are given. 
\item  The partition split method used in the permutation backtrack is now in the
kernel. Transversal computations in large permutation groups are improved.
Homomorphisms from free groups into permutation groups now give substantially
shorter words for preimages. 
\item  The membership test in \texttt{SP} (\textbf{Reference: Sp (for dimension and field size)}) and \texttt{SU} (\textbf{Reference: SU}) groups has been improved using the invariant forms underlying these groups. 
\item  An improvement for the cyclic extension method for the computation of subgroup
lattices has been implemented. 
\item  A better method for \texttt{MinimalPolynomial} (\textbf{Reference: MinimalPolynomial}) for finite field matrices has been implemented. 
\item  The display has changed and the arithmetic of multivariate polynomials has
been improved. 
\item  The \texttt{LogMod} (\textbf{Reference: LogMod}) function now uses Pollard's rho method combined with the Pohlig/Hellmann
approach. 
\item  Various functions for sets and lists have been improved following suggestions
of L. Teirlinck. These include: \texttt{Sort} (\textbf{Reference: Sort}), \texttt{Sortex} (\textbf{Reference: Sortex}), \texttt{SortParallel} (\textbf{Reference: SortParallel}), \texttt{SortingPerm} (\textbf{Reference: SortingPerm}), \texttt{NrArrangements} (\textbf{Reference: NrArrangements}). 
\item  The methods for \texttt{StructureConstantsTable} (\textbf{Reference: StructureConstantsTable}) and \texttt{GapInputSCTable} (\textbf{Reference: GapInputSCTable}) have been improved in the case of a known (anti-) symmetry following a
suggestion of M. Costantini. 
\end{itemize}
 

 The improvements listed in this Section have been implemented by T. Breuer and
A. Hulpke. }

  
\subsection{\textcolor{Chapter }{New Programming and User Features}}\label{New Programming and User Features}
\logpage{[ 6, 1, 4 ]}
\hyperdef{L}{X83E1A0D87D3D1EAF}{}
{
  
\begin{itemize}
\item  The 2GB limit for workspace size has been removed and version numbers for
saved workspaces have been introduced. (S. Linton and B. H{\"o}fling) 
\item  The limit on the total number of types created in a session has been removed.
(S. Linton) 
\item  There is a new mechanism for loading packages available. Packages need a file \texttt{PackageInfo.g} now. (T. Breuer and F. L{\"u}beck; see  \textbf{Example: PackageInfo.g for a GAP package}). 
\end{itemize}
 

 Finally, as always, a number of bugs have been fixed. This release thus
incorporates the contents of all the bug fixes which were released for \textsf{GAP} 4.3. It also fixes a number of bugs discovered since the last bug fix. }

 }

  
\section{\textcolor{Chapter }{Earlier Changes}}\label{Earlier Changes}
\logpage{[ 6, 2, 0 ]}
\hyperdef{L}{X8205C966783E6F37}{}
{
  The most important changes between \textsf{GAP} 4.2 and \textsf{GAP} 4.3 were: 

 
\begin{itemize}
\item  The performance of several routines has been substantially improved. 
\item  The functionality in the areas of finitely presented groups, Schur covers and
the calculation of representations has been extended. 
\item  The data libraries of transitive groups, finite integral matrix groups,
character tables and tables of marks have been extended. 
\item  The Windows installation has been simplified for the case where you are
installing \textsf{GAP} in its standard location. 
\item  Many bugs have been fixed. 

 
\end{itemize}
 

 The most important changes between \textsf{GAP} 4.1 and \textsf{GAP} 4.2 were: 

 
\begin{itemize}
\item  A much extended and improved library of small groups as well as associated \texttt{IdGroup} (\textbf{Reference: IdGroup}) routines. 
\item  The primitive groups library has been made more independent of the rest of \textsf{GAP}, some errors were corrected. 
\item  New (and often much faster) infrastructure for orbit computation, based on a
general ``dictionary'' abstraction. 
\item  New functionality for dealing with representations of algebras, and in
particular for semisimple Lie algebras. 
\item  New functionality for binary relations on arbitrary sets, magmas and
semigroups. 
\item  Bidirectional streams, allowing an external process to be started and then
controlled ``interactively'' by \textsf{GAP} 
\item  A prototype implementation of algorithms using general subgroup chains. 
\item  Changes in the behavior of vectors over small finite fields. 
\item  A fifth book ``New features for Developers'' has been added to the \textsf{GAP} manual. 
\item  Numerous bug fixes and performance improvements 
\end{itemize}
 

 The changes between the final release of \textsf{GAP} 3 (version 3.4.4) and \textsf{GAP} 4 are wide-ranging. The general philosophy of the changes is two-fold.
Firstly, many assumptions in the design of \textsf{GAP} 3 revealed its authors' primary interest in group theory, and indeed in finite
group theory. Although much of the \textsf{GAP} 4 library is concerned with groups, the basic design now allows extension to
other algebraic structures, as witnessed by the inclusion of substantial
bodies of algorithms for computation with semigroups and Lie algebras.
Secondly, as the scale of the system, and the number of people using and
contributing to it has grown, some aspects of the underlying system have
proved to be restricting, and these have been improved as part of
comprehensive re-engineering of the system. This has included the new method
selection system, which underpins the library, and a new, much more flexible, \textsf{GAP} package interface. 

 Details of these changes can be found in the document ``Migrating to GAP 4'' available at the \textsf{GAP} website, see \href{http://www.gap-system.org/Gap3/migratedoc.pdf} {\texttt{http://www.gap-system.org/Gap3/migratedoc.pdf}}. 

 It is perhaps worth mentioning a few points here. 

 Firstly, much remains unchanged, from the perspective of the mathematical
user: 

 
\begin{itemize}
\item  The syntax of that part of the \textsf{GAP} language that most users need for investigating mathematical problems. 

 
\item  The great majority of function names. 

 
\item  Data libraries and the access to them. 
\end{itemize}
 

 A number of visible aspects have changed: 

 
\begin{itemize}
\item  Some function names that need finer specifications now that there are more
structures available in \textsf{GAP}. 
\item  The access to information already obtained about a mathematical structure. In \textsf{GAP}{\nobreakspace}3 such information about a group could be looked up by directly
inspecting the group record, whereas in \textsf{GAP}{\nobreakspace}4 functions must be used to access such information. 
\end{itemize}
 

 Behind the scenes, much has changed: 

 
\begin{itemize}
\item  A new kernel, with improvements in memory management and in the language
interpreter, as well as new features such as saving of workspaces and the
possibility of compilation of \textsf{GAP} code into C. 
\item  A new structure to the library, based upon a new type and method selection
system, which is able to support a broader range of algebraic computation and
to make the structure of the library simpler and more modular. 
\item  New and faster algorithms in many mathematical areas. 
\item  Data structures and algorithms for new mathematical objects, such as algebras
and semigroups. 
\item  A new and more flexible structure for the \textsf{GAP} installation and documentation, which means, for example, that a \textsf{GAP} package and its documentation can be installed and be fully usable without any
changes to the \textsf{GAP} system. 
\end{itemize}
 

 Very few features of \textsf{GAP}{\nobreakspace}3 are not yet available in \textsf{GAP}{\nobreakspace}4. 

 
\begin{itemize}
\item  Not all of the \textsf{GAP}{\nobreakspace}3 packages have yet been converted for use with \textsf{GAP}{\nobreakspace}4. 
\item  The library of crystallographic groups which was present in \textsf{GAP}{\nobreakspace}3 is now part of a \textsf{GAP}{\nobreakspace}4 package \href{http://www.gap-system.org/Packages/crystcat.html} { \textsf{CrystCat} } by V. Felsch and F. G{\"a}hler. 
\end{itemize}
 }

 }

    \def\indexname{Index\logpage{[ "Ind", 0, 0 ]}
\hyperdef{L}{X83A0356F839C696F}{}
}

\cleardoublepage
\phantomsection
\addcontentsline{toc}{chapter}{Index}


\printindex

\newpage
\immediate\write\pagenrlog{["End"], \arabic{page}];}
\immediate\closeout\pagenrlog
\end{document}
